\documentclass[10pt,twoside,twocolumn]{book}
\usepackage[bg-letter]{lib/rpg-book} % Options: bg-a4, bg-letter, bg-full, bg-print, bg-none.
\usepackage[polish]{babel}
\usepackage[utf8]{inputenc}
\usepackage{hyperref}
\usepackage{multicol}
\usepackage{multirow}
\usepackage{lipsum}
\usepackage{tabularx}

\title{Musimy nazwać kampanię}
\date{\today}
\author{The Dungeon Mistress}
\author{Vivien}
\author{Yeleda}
\author{Calion}
\author{Garret}

% Start document
\begin{document}
\fontfamily{ppl}\selectfont % Set text font
\frontmatter

\maketitle
\begin{multicols}{2}
\tableofcontents
\end{multicols}

% Your content goes here
\mainmatter

\chapter{Opis świata}

\section{Historia}

\paragraph{}
Królestwa przez setki lat były pochłonięte gigantyczną wojną na wielu frontach.
Ze względu na równowagę sił żadne z państw nie było w stanie uzyskać znaczącej przewagi, a wszelkie sojusze między państwami trwały stosunkowo krótko.
Wszystko zmieniło się, gdy doszło do tajemniczej katastrofy na wschód od królestw, która dodatkowo rezonowała w bogatych w magię terenach An'Gammarny i Doliny Konwalii.
Po katastrofie w królestwach zastano ogromne zniszczenia, a po tym gigantyczną suszę i głód.
To w końcu zmusiło przywódców do pokoju i sojuszu, aby mogli przeżyć.
Sojusz z czasem uniemożliwił handel na niespotykaną dotąd skalę, a to, do powstania Gildii Kupców.
Gildia jest obecnie jedną z najsilniejszych sił politycznych i finansowych królestwa.

\section{Gildia Kupców}
\paragraph{}
Ściśle powiązana gigantyczna siatka kupców bankierów itp.
W praktyce to oni utrzymują pokój w królestwach dzięki gigantycznym wpływom na bardzo wysokich szczeblach oraz, oczywiście, gigantycznym pieniądzom.

\section{Konfraterie\\magów}
\paragraph{}
Bardzo luźne zgrupowania magów w różnych miastach królestw.
Większość magów szkoliło się w jednym z ośrodków Konfraterii, ponieważ daje to dostęp do bibliotek, miejsc do bezpiecznych eksperymentów oraz potencjalnych nauczycieli.
Większość magów z dumą deklaruje przynależność, do którejś z konfraterii, jako dowód, że mają porządne wykształcenie.
Najbardziej prestiżową konfraterią jest Latająca Wyspa Muria, lewitująca nad wybrzeżem Croridu.

\chapter{An'Gamarrna}

\paragraph{}
Państwo ludzi stworzone na ruinach dawnego Imperium, obecnie rządzone przez teokratyczny kult Bahamuta, do którego musi należeć nawet król.
Pańśtwo jest częścią sojuszniczych królestw.

\section{Geografia}
\paragraph{}
An'Gamarrna obejmuje zatokę pomiędzy Doliną Konwalii a Croridem i Eneaorem.
Platynowa zatoka nie umożliwia ze względu na niestabilne prądy przez istnienie latającej wyspy Muria, co powoduje bardzo napięte stosunki między An'Gamarrną a Croridem.
Obecną stolicą państwa jest święte miasto Glenrowan.
Innymi dużymi ośrodkami są dawna stolica Imperium, twierdza Bleihar (znajdująca się w górach na północy) oraz port Thelassa (około dzień drogi od Glenrowan nad brzegiem Platynowej Zatoki).
Klimat An'Gamarrny jest umiarkowany, z okazjolanymi przebłyskami niestabilnej pogody w miejscach gdzie kapłani nie kontrolują dobrze granic między planami.

\section{Polityka}
\paragraph{}
An'Gammarna rządzona jest przez króla, który aby odziedziczyć swoje królestwo musi zostać paladynem Bahamuta.
Obecnym królem jest Lorne III, który wziął za małżonkę Zephę, arcykapłankę Bahamuta.
Zepha jest młodszą od nieo o kilkanaście lat półelfką, dała mu dwóch synów.
Wiele osób uważa, że to ona naprawdę rządzi królestwem.
Wszelkie kulty czczące dobrych i neutralnych bogów cieszą się dużym poważaniem, kulty chaotyczne mogą być natomiast traktowane z pełną rezerwą.

\section{Historia}
\paragraph{}
An'Gammarna wg legend została założona przez zbuntowanych niewolników podczas schyłku Imperium.
Niewolnicy pierwotnie przejęli twierdzę Bleihar, a później objawił im się platynowy smok i polecił stworzyć święte miasto Glenrowan.
Ponieważ teren współczesnej An'Gammarny, poza twierdzą Bleihar, znajdował się ciągle w rękach elfów - wybuchła wojna.
Zakończyła się ona zwycięstwem ludzi i zniszczeniem elfiej stolicy Eleander.

\chapter{Vicovaro}

\paragraph{}
Państwo niziołków stworzone w miejscu kręgu druidzkiego specjalizującego się z kontroli pogody oraz magii wody. 
Przez Vicovaro przepływa większość handlu zewnętrznego w Królestwach i jest półoficjalną siedzibą Gildii Kupców.
Państwo jest częścią Sojuszniczych Królestw.

\section{Geografia}
\paragraph{}
Vicovaro jest najbardziej na zachód wysuniętym państwem Sojuszniczych Królestw. 
Od najbliższego wschodniego sąsiada, An'Gammarny, oddzielona jest Doliną Konwalii. 
Poza miastem portowym Vicovaro w państwie nie było żadnych innych miast (co formalnie czyniło z Vicovaro miasto-państwo), do czasu kiedy 200 lat temu kult Ioun wybudował Sjorden. 
W Vicovaro znajduje się największy port w Królestwach, w którym druidzi utrzymują permanentnie korzystną dla żeglarzy pogodę.

\section{Polityka}
\paragraph{}
Miastem rządzi diuk z klanu niziołków Olkyn, potomek Adana Olkyna, oryginalnego założyciela miasta Vicovaro. 
Aktualny diuk Dregras Olkyn jest jedynie marionetką, faktyczna władza znajduje się w rękach Arcydruidki Eryn oraz Thundy, przywódcy Gildii Kupców. 
Oboje nie przepadają za sobą nawzajem, ale prowadzą niechętną współpracę. 
Krąg druidzki pozostaje niezmiennie znacząca siłą polityczną w Vicovaro, ze względu na zależność miasta od ich magii. 
Te powiązania sprawiają, że krąg Vicovaro jest traktowany z pewną wrogością przez inne kręgi.

\section{Historia}
\paragraph{}
Zanim powstało Vicovaro na wybrzeżu znajdował się krąg druidzki. 
Wiele stuleci temu przywódca kręgu: druid Adan ściągnął do nadmorskiego siedliska kręgu cały swój klan. 
Klan stworzył port handlowy Vicovaro. 
Stopniowo coraz więcej niziołków  z okolicznych dolin zaczęło ściągać do portu, miasto zaczęło się rozrastać i zyskiwać na znaczeniu.
Port stał się łakomym kąskiem dla okolicznych państw ze względu na bogactwo i strategiczną lokalizację: konflikty wybuchały głównie z An'Gammarną i Wolnymi Wyspami Kye. 
Vicovaro zawsze jednak dało radę odeprzeć napastników dzięki swojej silnej flocie i magii druidzkej. 
Konflikt z An'Gammarna udało się w znacznym stopniu załagodzić po tym jak utworzono Sojusznicze Królestwa i udostępniono ziemię na budowę Sjorden kultowi Ioun. 
Ataki ze strony Wysp Kye ciągle się zdarzają, ale słabo zorganizowane wyspy nie maja szans ze współczesny Vicovaro.

\onecolumn
\large
\chapter{Opowiadanie}
\section*{Kozie górki}
\paragraph{}
Dzień miał się ku końcowi na szczęście w Kozich Górkach była gospoda i do której uderzyła w pierwszej kolejności.
Zimne piwo to coś na co ma się ochotę po tak ciepłym i długim dniu. 
Poza tym karczmy są kopalniami informacji. Tam nie musisz pytać, by dostać odpowiedź. 
Ktoś prędzej, czy później wpadnie zdyszany oznajmiając że kogoś innego zabrakło. 
Usadowiła się na wolnym miejscu pod ścianą na przeciw wejścia. 
Zdjęła kaptur i kiwnęła na barmana by podał jej piwa.
Gdy dostała napitek rozejrzała się po wnętrzu. 
Pod ścianą siedział księżycowy elf. 
Przymknęła oczy i westchnęła. 
Spod przymkniętych powiek przyjrzała mu się  uważniej. 
Szaty miał czyste i z kosztownego materiału. 
Włosy pięknie splecione opadały na wyprostowane plecy srebrnymi pasmami. 
Wypielęgnowane dłonie obejmowały oszroniony (!) kufel piwa. 
Czyżby mag? 
Cóż robiłby tu w tej zapadłej wiosce? 
Nie przypuszczała by szukał zaginionych. 
Tacy jak on poszukują jedynie potęgi w wiedzy, lub w praktyce zawartej.

\paragraph{}
Drzwi znów skrzypnęły i zdało się, że nikt nie wszedł. 
Jednak po chwili przy barze usiadł niziołek o wzroście około metra we wzorzystych szatach, których kolor zakrywała gruba warstwa kurzu. 
Jednak jego najbardziej charakterystyczną cechą była burza rudych loków. 
Ktoś taki nie schowa się w tłumie, o nie. 
Z twarzy bił blask szczerego uśmiechu, a w oczach czaił się błysk. 
Poszukiwacz przygód. 
I to taki wyposzczony. 
Taki, który dawno niczego nie przeżył. 
Karczmarzowi też się gęba cieszyła. 
Przyjezdni oznaczają zarobek. 
Bo gdzież indziej mieliby się podziać?

\paragraph{}
Dłuższa chwila minęła na zapełnianiu się karczmy. 
Rolnicy schodzili z pól, pasterze z pastwisk. 
Wszyscy ciągnęli do karczmy by odrobinę wytchnąć po pracowitym dniu. 
Stoliki zapełniały się ludźmi, ale rozmowy były prowadzone po cichu. 
Brakowało wesołego gwaru zwyczajowego dla tego typu miejsc. 
Gdy drzwi ponownie się otwarły ktoś z głębi sali gwizdnął donośnie, tak że i Vivien wyraźnie podniosła głowę. 
Nieprzeciętnej urody ludzka kobieta stanęła na środku na szeroko rozstawionych nogach z rękami opartymi na biodrach. 
Zdaje się, że otwarła te nieszczęsne drzwi kopniakiem. 
Rozejrzała się po wnętrzu, a jej mina zdradzała, że zrezygnowała z jakiegoś pomysłu. 
Usiadła przy narożniku baru. 
Jeśli wspomnieć przesądy to akurat ona nie powinna się obawiać staropanieństwa. 
Z drugiej strony jej szelmowski uśmiech sugerował, że małżeństwo to będzie jej ostatni przystanek.
Następny na pewno wejdzie smok. 
Idealnie pasowałby do całej tej dziwnej zbieraniny. 
Intuicja podpowiadała jej, że teraz powinno się było coś wydarzyć. 
Coś co zdoła tą zbieraninę wykorzystać.

\paragraph{}
Nie trzeba było długo czekać. 
Do środka niczym małe tornado wpadł młody chłopak, który w rękach ściskał małą szkatułkę. 
Wzrok miał rozbiegany i cały się trząsł.\newline
\indent - Ludzie! Pomóżcie! Tutaj straszy a ja muszę się stąd wydostać!\newline
Podbiegł do lady i oparł się oń jedną ręką niedaleko tej pięknej ludzkiej kobiety, która przyszła jako ostatnia. 
Niziołek też siedział przy barze i nie pozostał obojętny.\newline
\indent - Co straszy? Powiedz coś więcej.\newline
\indent - Zombie i cienie nie dają ludziom żyć. Tutaj w okolicy jest ich pełno! Nie mogę tu zostać...\newline
Księżycowy elf opróżnił kufel i z elfią gracją podszedł również do baru.
\indent - Odprowadzę cię, tylko powiedz gdzie chcesz się udać.\newline
\indent - Muszę dotrzeć do świątyni Ioun do kleryka, który mnie uczył. Wielu kleryków zaginęło w tych okolicach, nie chcę do nich dołączyć!\newline
Vivien nadstawiła uszu. Czyż nie po to tutaj zawędrowała? Podniosła się z miejsca i również podeszła do baru.
\indent - Jak długo to trwa? Ilu kleryków zaginęło?\newline
\indent - Niedawno, maksymalnie miesiąc. Jednak dużo naszych nie wróciło do świątyni w mieście. Wielu pielgrzymuje po okolicach, ale zawsze wracają. Myślę, że mój nauczyciel będzie więcej wiedzieć. Proszę o pomoc. Potrzebuję eskorty właśnie do miasta. Do Sioden jest około ośmiu godzin marszu stąd.\newline
\indent - Teraz już za późno na podróże, młody człowieku. - powiedział srebrnowłosy elf. - Wyruszymy z samego rana, jeśli pozwolisz.\newline
\indent - T-tak...\newline
\indent - Ja też pójdę. Chętnie dowiem się czegoś na temat tych zniknięć. - szybko wtrąciła Vivien.\newline
\indent - Ja też pójdę. - zgłosił się niziołek.\newline
\indent - I ja. - dopowiedziała ludzka kobieta.\newline
\indent - Mili państwo o świcie wyruszamy! - wykrzyknął i pobiegł na piętro gospody w kierunku gościnnych pokoi.\newline
Vivien patrzyła jeszcze chwilę za młodym chłopakiem gdy usłyszała charakterystyczny aksamitny głos elfiego mężczyzny.
\indent - Skoro mamy podróżować razem pozwólcie, że się przedstawię: jestem Calion.\newline
\indent - Garret Eartopple. - Niziołek podał całe nazwisko.\newline
\indent - Yeleda - dodała kobieta\newline
\indent - Vivien - przedstawiła się też.\newline
Nie zdążyli zamienić więcej słów gdy z wnętrza karczmy podeszła do nich wieśniaczka o dojrzałym wyrazie oczu i pomarszczonej skórze. Jakby miała już jakiś etap życia za sobą.\newline
\indent - Drodzy poszukiwacze przygód wybaczcie, że się wtrącam. Słyszałam, że pójdziecie do Sioden. Tydzień temu wyruszył tam mój syn Baellor i dotychczas nie wrócił. Popytajcie proszę o niego gdy już tam będziecie.\newline
Vivien poczuła ściśnięcie serca. Przerwane więzi rodzinne to coś, czego nie mogła zostawić swemu losowi.\newline
\indent - Dobrze. Sprawdzimy powiedz tylko jak wyglądał?\newline
\indent - Baellor ma już 30 lat. Jest wysoki, szczupły o jasnych włosach i niebieskich oczach.\newline
\indent - Zajmiemy się tym. - odpowiedziała machinalnie choć jeszcze nie wiedziała jak potoczą się ich dalsze losy.\newline
Wieśniaczka wyszła z gospody jak i większość gości. Po tym gwałtownym wydarzeniu wielu odeszła ochota do picia. Usiedli razem przy barze i spojrzeli po sobie.\newline
\indent - Nie. Nie rozumiem. Jak na tym pustkowiu może ktoś w ogóle zaginąć? Tu nic nie ma. - pierwsza odezwała się Yeleda.\newline
\indent - Przepraszam na chwilę, ale muszę rozejrzeć się po okolicy. - powiedział Calion i podniósł się.\newline
\indent - Pójdę z tobą. - Jeśli wierzyć pogłoskom nie powinno się chodzić po zmroku samemu.\newline
Wyszli przed budynek i pierwsze co rzuciło im się w oczy to to, że wieś była kompletnie opustoszała. Nikt nie przechadzał się między domkami. Noc była ciepła, a nikt nie siedział na ławkach ciesząc się pięknem wieczoru.
\indent - Jak tu pusto. - zauważyła.\newline
\indent - Mało kto pali światło. Dziwne, jakby się czegoś obawiali.\newline
\indent - Choć kusi mnie zbadanie obecności intruzów w dolinie wydaje mi się, że to nie najmądrzejszy pomysł. Nie uważasz?\newline
\indent - Cienie i zombie nie biorą się znikąd. To nie są istoty które zamieszkują jakieś okolice. One się pojawiają i zazwyczaj wiąże się to z niepoprawnym korzystaniem z magii.\newline
\indent - Nic stąd teraz nie wyciągniemy. Wracajmy.\newline
Weszli z powrotem do karczmy a Yeleda i Garret siedzieli przy jednym stole. Dosiedli się do nich i zamówili jeszcze po piwie. Calion od razu schłodził swoje.\newline
\indent - ...odszedłem bo czułem, że to nie to. Że mogę ćwiczyć moje mnisie umiejętności w prawdziwym życiu na trakcie.\newline
\indent - Łotrostwo nigdy mnie nie zawiodło i mówię ci, że odrzucanie bogactw to czysta głupota. Gdyby przeszkadzały ci jakieś ciężkie monety chętnie je przygarnę.\newline
\indent - A wy? Czym się zajmujecie? - spytał Garret ignorując zaczepki łotrzycy.\newline
\indent - Zgłębiam wiedzę magiczną już od wielu lat i powiem wam, że nie zamierzam przestać.\newline
\indent - Musisz dużo umieć, skoro nie masz przy sobie ni szpilki... - mruknęła Yeleda. Calion tylko się uśmiechnął.\newline
\indent - Szpilkę ma pewnie jego płaszcz. - Vivien uśmiechnęła się. Chciała się poczuć częścią grupy i przyznała się do bycia Druidem oraz tego, że długo już wędruje po świecie, ale nigdzie nie zagrzała miejsca na dłużej. Bała się jednak podać dokładną liczbę lat.\newline
\indent - Rozmawialiśmy z karczmarzem jednak nie powiedział nam niczego więcej niż już wiemy. Ponieważ jest już późno idę w kimę. Trzeba mieć siły na jutro. - Yeleda ziewnęła i przeciągnęła się po czym ruszyła w kierunku pokoi gościnnych. Reszta poszła za jej przykładem.

\paragraph{}
Nie pospali długo, ale Vivien nie potrzebowała tyle snu co reszta ras. Calion oczywiście też. Pobudka była gwałtowna i nieznosząca sprzeciwu, bowiem w powietrze wyleciała czwarta część karczmy razem z dachem. Wybuch i wstrząs były tak skuteczne, że wszyscy nocujący wybiegli na korytarz. Okazało się, że tym razem nocowała tylko ich czwórka. Panował półmrok charakterystyczny dla świtu. Yeleda jęczała, że nic nie widzi.\newline
\indent - Eee... Viv... co ty masz na szyi? - spytał Calion.\newline
\indent - O do kurwy... - wypsnęło jej się na widok ładnego naszyjnika z zielonym kamieniem. Chciała podnieść go do oczu, ale jej ręka trafiła w powietrze i zaklęła jeszcze raz. - Ej, wy też je macie...\newline
\indent - Pozwólcie, że je zbadam - powiedział Calion. - To mi zajmie chwilę.\newline
\indent - Ja pójdę zobaczyć miejsce katastrofy - zaoferowała się, wiedziona ciekawością jak i chęcią niesienia ewentualnej pomocy.\newline
\indent - Idę z tobą, Viv. - Yeleda chwyciła ją za połę płaszcza. Ruszyły śladem nadpalonego drewna. W tym miejscu zabrakło kawałka karczmy. Wybuch musiał nastąpić w tym pokoju, gdyż drzwi były wywalone, a w oczy pierwsze wpadło zwęglone ciało.\newline
\indent - Uuu... tego akurat nie chciałam widzieć. - Yeleda skrzywiła się. - Chodź przeszukajmy pokój.\newline
\indent - Tu nic nie ma. Wygląda, jakby coś w pobliżu łóżka spowodowało wybuch. Ten człowiek jest zupełnie spalony. Wielki ogień? Dziwne. Ech nie da się podejść dalej, podłoga jest nadwątlona eksplozją. - Vivien głośno myślała.\newline
\indent - Ha! Znalazłam 9 sztuk złota!\newline
\indent - Cztery i pół chciałaś powiedzieć. - Vivien obruszyła się. - Gdyby nie ja w ogóle byś tutaj nie dotarła.\newline
\indent - No dobra... Masz. Starą szmatę też chcesz?\newline
Podzieliły łup na dwie nie zamierzając dzielić się tym odkryciem. To nie było ważne dla sprawy. Wróciły do chłopaków i spytały o naszyjniki.\newline
\indent - Jest w tym jakaś mroczna magia, konkretnie nekromancja. Amulety istnieją naprawdę mimo, że nie jesteśmy w stanie ich dotknąć.\newline
\indent - To świetnie. Tam w pokoju nic nie było. Eksplozja najpewniej nastąpiła z jakiegoś punktu przy łóżku na którym były spalone doszczętnie zwłoki. Ciężko oszacować do kogo należały ale wydaje mi się, że w karczmie nie było nikogo oprócz nas i tego chłopaczka, który chciał byśmy go eskortowali.\newline
\indent - Intrygują mnie ślady eksplozji, pójdę obejrzę karczmę z zewnątrz. Może wybuch zostawił jakieś magiczne ślady. - Viv zaoferowała się wiedziona ciekawością.\newline
\indent - Ja przeszukam resztę pokoi.\newline
\indent - Ja zejdę na dół. Może karczmarz jest na dole.\newline
Ich trójka zeszła piętro niżej a niedaleko od schodów przy kontuarze leżało ciało karczmarza. 
Po wstępnych oględzinach nie udało im się ustalić przyczyny śmierci. 
Wydawało się, że umarł ze strachu.
Vivien udała się na zewnątrz obejrzeć ślady po katastrofie. 
Spalenizna nie pachniała magią, ani nie zostawiła szczególnych śladów. 
Eksplozja usunęła fragment dachu i ściany. 
Vivien widziała wnętrza pokoi zza ułamanych ścian. 
Podłoga ledwie się trzymała i w każdej chwili mogła się zawalić. 
Dziewczyna rozejrzała się też po bliższej okolicy ale nic nie znalazła. 
To był czysto fizyczny wybuch, acz przyczyny mogły być magiczne.
Gdy wróciła do towarzyszy, Yeleda też już była na dole dowiedziała się od chłopaków, że eksplozję spowodował najprawdopodobniej ten sam kamień, który posiadają w tych tajemniczych wisiorach. 
Jego odłamki Garret zauważył przez dziurę w suficie składziku, na podłodze pokoju młodego kapłana.\newline
\indent - Zajebiście. Mamy na szyjach przenośne bomby! - Yeleda tym stwierdzeniem podniosła wszystkim morale.\newline
\indent - Jak tam łupienie pozostałych pokoi?\newline
\indent - Nic. - mruknęła i spojrzała w podłogę. - Połamałam wytrychy i uderzyłam się łomem w rękę a Garret potłukł się gdy skakał po parapetach.\newline
\indent - Hm... to może rozejrzymy się po okolicy? - Zaproponowała Vivien.\newline
\indent - Albo udamy się od razu do Sioden - odpowiedział Calion. - I tak mieliśmy eskortować tam kapłana. Może w świątyni tej bogini udzielą nam więcej informacji co do tych kłopotliwych naszyjników.

\paragraph{}
- Patrzcie jaka wymarła ta wioska... Co prawda wczesna godzina, a nikogo nie ma na zewnątrz. Przecież to pasterze i rolnicy.\newline
\indent - Faktycznie, dziwne. - przytaknęła Vivien. Czy nie powinniśmy powiadomić sołtysa o śmierci karczmarza? Czemu nikt nie panikuje po wybuchu?\newline
\indent - Ta druga wielka chata to pewnie jego dom. Chodźmy tam.\newline
Podeszli do domu w którym było ciemno i cicho. 
Po chwili Yeleda usłyszała jakby głuche łupnięcia, a Calion odsunął się od drzwi. 
Yeleda nacisnęła klamkę i stanęli twarzą w twarz z zombie. 
Monstrum zamachnęło się pięścią ale nie trafiło nikogo. 
Vivien uderzyła adrenalina w żyły. 
To plugastwo drwi z cudu tworzenia samym swoim jestestwem. 
Zamachnęła się nań swym kosturem Shillegah. 
Pozostała trójka szybko dołączyła do walki. 
Garret imponował jej swoimi szybkimi atakami, Yeleda niesamowitym uśmiechem losu, a Calion... cóż on niczego nie potrzebował. 
Jego magiczne pociski zmieniały rzeczywistość. 
Dobrze było mieć go po swojej stronie. 
Zwłaszcza, że zombie nie miał z nimi szans. 
Czy był sołtysem czy nie, nie miał przy sobie wielkiego majątku Yeleda dokładnie to sprawdziła. 
Niesamowite, że dla niej te pieniądze były tak ważne.\newline
\indent - Myślę, że nie ma sensu dalej przeszukiwać wioski. - Calion udawał, że nie widzi poczynań Yeledy. - Intrygują mnie te kapliczki Ioun. Kapłani często je odwiedzali - może one przyniosą nam jakieś odpowiedzi.\newline
\indent - Też już o tym myślałam - zgodziła się Vivien. \newline 
Ruszyli we czwórkę do małego ołtarzyku. Calion zbadał ją pod kątem obecnej magii, Yeleda pod kątem pułapek, jednak poza wypalonymi świecami niczego nie znaleźli. Kapliczka była po prostu zaniedbana.\newline
\indent - Pozostaje nam iść do Sioden. Mimo, że eskortowany odszedł już do lepszego świata to nasza jedyna poszlaka. Tam była świątynia Ioun - jeśli ktoś się w niej ostał może udzieli nam informacji o naszej niewygodnej biżuterii.\newline

\section*{Siodden}
\paragraph{}
Kilka godzin podróży upłynęło spokojnie i zakończyło się w bramie miasta, gdzie strażnicy zachowywali się niecodziennie ostrożnie. 
Mimo jakichś niewyjaśnionych konsultacji postanowili ich wpuścić i nie niepokoili więcej.
Świątynia Ioun była skromnym, smukłym budynkiem umiejscowionym niedaleko murów miasta. 
Wchodziło się prostym korytarzem do wysokiej sali oświetlonej prostymi, również wysokimi oknami. 
Pod dachem było widać balkony drugiego piętra. 
Na wysokości oczu mieli jednak poprzewracane, prosto zbijane drewniane ławy i smukłą ciemną postać czytającą jakiś papier na przeciwległej ścianie. 
W tej pustej przestrzeni echo niosło się ładnie nie było więc mowy o przekradnięciu się niezauważonym.
Mężczyzna przedstawił się imieniem Keldin i twierdził że jest sługą świątyni. 
Potwierdził zniknięcia kapłanów i dziwnym błyskiem w oku zareagował na ich amulety. 
Podał im imię Amaryka, jednego z ważniejszych kapłanów, który miał w pobliżu Sioden posiadłość. 
Bez zagłębiania się w szczegóły poradził, by to tam się udali po rozwiązanie zagadki naszyjników. 
Pozwolił im też odetchnąć w świątyni po wyczerpującym dniu.

\paragraph{}
A nie czuli się najlepiej, bo walka z zombiakiami i cieniami, przeszukiwanie wioski, te dziwne amulety... To było bardzo dużo. Poczuła niemal, że przeżyła więcej niż dotychczas razem wzięte.
Nie.
Nie więcej.
Ale jakby bardziej.
Podniosła głowę i wyjrzała przez okno sali w której się ulokowali na rozgwieżdżone niebo. 
Światła miasta poukrywały najmniejszych obserwatorów ale Vivien wiedziała o ich obecności. 
I czuła, że ją ostrzegają. 
Zerknęła na Caliona, który skrupulatnie szykował się do transu. 
Pokręciła głową. 
Też była elfem i bez względu na osobiste odczucia było to atutem drużyny. 
W przyszłości niewykluczone, że nie będą mogli spać tak beztrosko.

\paragraph{}
Nad ranem obudziła się z dziwnym bólem mięśni. 
Rozciągnęła się i rozruszała stawy. 
Ból może i minął, ale czuła że nie będzie w stanie dać dziś z siebie wszystkiego. 
Rozejrzała się po sali - Yeleda i Garret też mieli nietęgie miny, a Calion zniknął. 
Ziewnęła i jęła ogarniać się do drogi. 
Dziś mieli zbadać posiadłość tego nikczemnika. 
Tylko bogowie wiedzą co może ich tam spotkać. 
Postanowiła też przemilczeć swoją niedyspozycję. 
Nie może się okazać najsłabszym ogniwem.

\paragraph{}
Keldin wyprowadził gotową drużynę z miasta i poprowadził do pobliskiej posiadłości. 
Mruknął słowo w nieznanym Vivien języku, dzięki któremu dezaktywował otaczającą dom barierę.
\indent - Proszę. Droga wolna i życzę powodzenia - uśmiechnął się tajemniczo, a Vivien znów dostrzegła błysk w jego oku.

\paragraph{}
Nie spodziewali się by oprócz bariery napotkali jakieś zabezpieczenia w tej spokojnej okolicy. 
Dom wyglądał najzupełniej zwyczajnie. 
Prosta bryła z szerokimi wrotami oraz licznymi oknami. 
Podeszli do wrót i nacisnęli klamkę. 
Drzwi ustąpiły posłusznie i Vivien wraz z Garretem pierwsi przeszli próg, a za nimi Yeleda i Calion. 
Bogato wystrojone wnętrze bardzo przypadło do gustu łotrzycy. 
Z kolei Garret pierwszy zwrócił uwagę na posąg pięknej kobiety stojący w środku hallu.
\indent - To musi być Ioun - powiedział Calion.\newline
\indent - Padnij! - zawołał Garret spod posągu, do którego stóp zdołał już dotrzeć. \newline
Wszyscy schylili się, ale Vivien znów poczuła rwący ból wszystkich mięśni, a potem nieprzyjemne ukłucie w ramię. Okrzyk przestrachu lub bólu wyrwał się ze wszystkich gardeł.\newline
\indent - Musimy być ostrożniejsi - warknęła Yeleda patrząc na Garreta. \newline
Wszyscy wyrwali sobie strzałki z miejsc, w które oberwali. Na szczęście rany nie paprały się, a ból powoli ustępował. Musiały być zatrute bo na policzki wystąpił im niezdrowy kolor zieleni pomieszanej z żółcią.\newline
\indent - Żeby znów nie strzelać w całą drużynę idę przeszukać pokoje po prawej stronie. - po tym oświadczeniu znikła za drzwiami i tyle ją widzieli.\newline
\indent - Będzie kradła - mruknął Calion.\newline
\indent - Zobaczymy czy jest co. - odpowiedziała druidka i wybrała lewą stronę. Garret poszedł za nią.\newline
Nie było co kraść. 
Pokoje zwykłego przeznaczenia zawierały meble i regały zastawione bibelotami. 
Na ścianach wisiały obrazy i gobeliny przedstawiające sceny obyczajowe z życia zwykłych ludzi oraz z życia magów. 
Brak obecności pułapek sam z siebie sugerował, że nie trafią na nic cennego. 
Wrócili do korytarza w którym czekał Calion i Yeleda z dziwną latarnią przytroczoną do paska.
\indent - Co to jest? - spytała Vivien\newline
\indent - To latarnia - odpowiedziała Yeleda.\newline
\indent - Magiczna - dodał Calion. Najwyraźniej zdołał ją już zbadać.\newline
Drużyna udała się schodami na górę, mimo że nie zbadali jeszcze wyjścia na tyły. 
Piętro w kształcie litery „U” na swych krańcach kończyło się oknami zajmującymi całą ścianę i zdradzającymi widoki na poważnie zarośnięty ogród.
U podstaw litery znaleźli drzwi po prawej stronie skryte jakimś dziwnym iskrzącym obłokiem oraz zwykłe drzwi po lewej.\newline
\indent - Ten dziwny obłok to magia, której nie potrafię rozpracować - powiedział Calion. - Jeśli nawet nas nie zrani to na pewno nie przepuści.\newline
\indent - Rozejrzę się na zewnątrz - zaproponował Garret.\newline
\indent - Nie idź tam sam. - zaoponowała Vivien.\newline
\indent - Tylko przez okno. Nie chcę by coś nas zaskoczyło gdy będziemy rozbrajać drzwi.\newline
\indent - To ja zajrzę do tych zwykłych po lewej.\newline
Nie były zaś takie zwykłe. 
Czuła, że tu nie będzie tak bezpiecznie jak na dole. 
Z obserwacji zwyczajów ludzi zdołała już wywnioskować, że cenniejsze rzeczy trzymają dalej od wejścia. 
Podeszła do drzwi i obmacała je dokładnie. 
Nasłuchiwała uważnie gdy ostukiwała futrynę.
\indent - Ach jest! - szepnęła triumfalnie gdy odnalazła pułapkę.\newline
Uchyliła drzwi na tyle, by móc się zmieścić w szparze i przekroczyła rozciągniętą linkę. Otwarcie drzwi na oścież przerwałoby żyłkę i najpewniej wypuściło w jej kierunku kolejne zatrute strzałki.
Znalazła się sypialni z wielkim, bogato zdobionym małżeńskim łożem. 
Ach po trudach podróży chciałoby się zatopić w miękkiej rozkoszy jaką mógł oferować ten mebel. 
Podrapała się w kark i gwałtowny skręt łokcia rozpromienił ból na długość całej ręki.
No tak. 
Nie mieli na to czasu, tykające bomby które mieli na szyjach szybko przywróciły ją do rzeczywistości.
\indent - Viv co tu masz? - zakrzyknęła Yeleda, która zjawiła się przy drzwiach.\newline
\indent - Uważaj na rozciągniętą linkę! To pułapka - odkrzyknęła szybko.\newline
\indent - Ach czaję. Zaraz ją rozbroję.\newline
\indent - I nic nie mam. Te same zakryte obłokiem magii drzwi, co w korytarzu. I to łóżko - uśmiechnęła się pod nosem.\newline
\indent - Wracajmy więc do Caliona może wykombinował jak się przedrzeć przez tamte.

\paragraph{}
\indent - Mam pomysł. - Calion wyglądał jakby oświecił go jakiś głos z góry. - Podaj mi latarnię, którą znalazłaś w gabinecie.\newline
\indent - Lepiej powiedz co mam z nią zrobić.\newline
\indent - ... miej mnie w opiece - szepnął tak, że tylko Vivien to słyszała. - Oświetl jej światłem wnękę drzwi. Powinny się same otworzyć.\newline
\indent - Garret podejdź tu do nas - Viv zdoła zawołać mnicha zanim drzwi stanęły otworem. \newline
Odkryły bogato zdobiony korytarz wyłożony purpurowym miękkim dywanem. 
Ściany pokryte były chabrową farbą oraz złotymi wzorami. 
Amaryk  żył niemal jak król. 
Z korytarza odchodziły kolejne drzwi.
\indent - Nie rozdzielajmy się i przeszukajmy wszystko po kolei - zasugerowała Vivien, po czym otworzyła najbliższe drzwi. - Ach!\newline
Jej okrzyk powiedział drużynie, że za drzwiami czekały na nich trzy cienie. 
Ruszyła do walki wręcz wprawiając w obroty swój kostur. 
Zawtórował jej Garret, który przyskoczył do wysokości mniej więcej kolan przeciwników. 
Calion poszedł po rozum do głowy i za pomocą magicznej ręki odchylił żaluzje. 
Promienie słoneczne które wpadły do wnętrza osłabiły zdolności bojowe cieni, a i pokazały drużynie, że znaleźli się w wielkiej bibliotece. 
Ponadto w odleglejszym rogu pokoju stał kamienny posąg jakiejś postaci. 
Yeleda wystrzeliła z łuku chcąc oszczędzić sobie walki wręcz, ale strzała chybiła i wbiła się posągowi w czoło. 
Magia Vivien jest kapryśna jak pogoda i raz wezwana niszczy wszystko na swej drodze musiała więc polegać na swoich umiejętnościach walki wręcz.
Calion nie miał jednak takich ograniczeń i szybko poczuli swąd palonego cienia.\newline
\indent - O matko, co to było?! - Yeleda łapała oddech.\newline
\indent - To był znak, że jesteśmy blisko jakichś skarbów - dodał Garret.

\twocolumn
\normalsize
\chapter{Kwiatki z sesji}

% \section{First Section}
% \lipsum[1] % filler text
% 
% \subsection{a subsection}
% \subsubsection{a subsubsection}
% 


\begin{rpg-quotebox}{Wewnętrzne rozterki wegan}
   \textit{Po zakończonej walce z zaklętymi krzewami.}\\
   
   \begin{tabularx}{\columnwidth}{lX}
      \textbf{Calion:} & Vivien, czy Ciebie nie boli zabójstwo tych paprotek? Przecież ich życie jest Ci bliskie.\\
      \textbf{Garret:} & O nie, ja mam na rękach jego liście!
   \end{tabularx}
\end{rpg-quotebox}


\begin{rpg-quotebox}{O mistycznym ognisku}
   \textit{Walka z cieniami, w której decydującą rolę odegrały płonące pociski Caliona. Po rozwianiu ostatniego z cieni:}\\
   
   \begin{tabularx}{\columnwidth}{lX}
      \textbf{Vivien :} & Czuję swąd palonego cienia!\\
   \end{tabularx}
\end{rpg-quotebox}


\section*{6 maja 2017}


\begin{rpg-quotebox}{Niemęskość}
   \textit{Uwiąd prącia jako następna klątwa z felernego amuletu.}
\end{rpg-quotebox}


\begin{rpg-quotebox}{Niefortunne przejęzyczenie}
   \textit{Calion padł ofiarą zaklętego posążka pantery, co spowodowało, że kilka godzin spędził zamieniony w wielkiego kota. Po rozwianiu klątwy wydarzyła się taka rozmowa:}\\
   
   \begin{tabularx}{\columnwidth}{lX}
      \textbf{Garret:} & Calion, potrzebuję pomocnej łapy... - tfu! - dłoni. \\
      \textbf{MG:} & Calion, rzucaj na wkurwienie. \\
   \end{tabularx}
\end{rpg-quotebox}


\begin{rpg-quotebox}{Oparzenia trzeciego stopnia}
   \begin{tabularx}{\columnwidth}{lX}
      \textbf{Matheus:} & Kasia, ile Ty masz inteligencji? Chyba zbyt dużo.\\
      \textbf{Madzia:} & Ha, jaki pocisk! Rzucałeś na trafienie?\\
   \end{tabularx}
\end{rpg-quotebox}


\begin{rpg-quotebox}{Rozmowy wysokiej wagi}
   \begin{tabularx}{\columnwidth}{lX}
      \multicolumn{2}{l}{\textit{Do Garreta:}}\\
      \textbf{Yeleda :} & Ile ważysz? \\
      \textbf{Garret:} & 50kg.\\
      \textbf{Mateusz:} & Ale w grze...\\
   \end{tabularx}
\end{rpg-quotebox}


\begin{rpg-quotebox}{Nie było telemarku}
   \begin{tabularx}{\columnwidth}{lX}
      \textbf{Yeleda:} & Garet jest w objęciach podłogi po lądowaniu na ryju.
   \end{tabularx}
\end{rpg-quotebox}


\begin{rpg-quotebox}{Pojawiają się niejakie myśli}
   \textit{Walka z cienistym smokiem zakończyła się finałem, w którym jego czaszka eksplodowała.}\\
   
   \begin{tabularx}{\columnwidth}{lX}
      \textbf{Garret:} & Mam twarz cieniem myśli skalaną.\\
   \end{tabularx}
\end{rpg-quotebox}


\begin{rpg-quotebox}{Jędrne negocjacje}
   \begin{tabularx}{\columnwidth}{lX}
      \textbf{Jubiler:} & Dam Ci za to 100 sztuk złota. \\
      \textbf{Yeleda:} & Zwariowałeś?! To jest warte conajmniej 150!\\
      \textbf{Jubiler:} & Cóż, jestem jedynym skupującym w okolicy, nie masz wyboru.\\
      \textbf{Yeleda:} & A jak pokażę cycki?
   \end{tabularx}
\end{rpg-quotebox}
    

\begin{rpg-quotebox}{Pan nie jest zadowolony}
   \textit{Tawerna w której zatrzymała się drużyna była zaopatrzona jedynie w jabłka i ich przetwory. Gustujący w piwie niziołek był co najmniej rozczarowany.}\\
   
   \begin{tabularx}{\columnwidth}{lX}
      \multicolumn{2}{l}{\textit{Pocieszającym tonem do Garreta:}}\\
      
      \textbf{Karczmarz:} & Mamy też jabłka w spirytusie.\\
   \end{tabularx}
\end{rpg-quotebox}


\begin{rpg-quotebox}{Sodoma i Gomora}
   \textit{MG opisuje hulanki, swawole i tańce (goście utworzyli korowód) oraz co robi drużyna wchodzącemu do karczmy Calionowi.}\\
   
   \begin{tabularx}{\columnwidth}{lX}
      \textbf{Łukasz:} & Dobrze, że Vivien jest w wężu, a nie wąż w niej. \\
   \end{tabularx}
\end{rpg-quotebox}


\begin{rpg-quotebox}{Dbajmy o czystość powietrza}
   \textit{Garret wybrał się na wieczorny spacer.}\\
   
   \begin{tabularx}{\columnwidth}{lX}
      \textbf{MG:} & Co robisz, gdy się wietrzysz?\\
      \textbf{Łukasz:} & Zamieniam się w wunderbaum i wieszam się w najbliższym golfie. \\
   \end{tabularx}
\end{rpg-quotebox}


\begin{rpg-quotebox}{Skutki przedawkowania hentai}
   \textit{Yeleda wpadła w pułapkę, w której żywe pnącza unieruchomiły ją od pasa w dół.}\\
   
   \begin{tabularx}{\columnwidth}{lX}
      \textbf{Garret:} & Gotuj sie na gwałt! \\
   \end{tabularx}
\end{rpg-quotebox}


\begin{rpg-quotebox}{Roślinna ośmiornica}
   \begin{tabularx}{\columnwidth}{lX}
      \textbf{Yeleda :} & Ile jeszcze tych macek?\\
      \textbf{MG:} & To jest mackarium\\
      \multicolumn{2}{l}{\textit{Vivien używa sztuczki Guidance.}}\\
      \textbf{Vivien:} & Macam się.\\
   \end{tabularx}
\end{rpg-quotebox}


\section*{29 maja 2017}

\section*{8 kwietnia 2017}


\begin{rpg-quotebox}{Jesteśmy skażeni elektrycznością}
   \textit{Podczas wchodzenia do mrocznej jaskini.}\\
   
   \begin{tabularx}{\columnwidth}{lX}
      \textbf{Vivien:} & Jak Garret włączy pochodnię...\\
   \end{tabularx}
\end{rpg-quotebox}

    
\begin{rpg-quotebox}{Nie ten klimat}
   \textit{W odpowiedzi na radioaktywne obrażenia kultystów.}\\

   \begin{tabularx}{\columnwidth}{lX}
      \textbf{Vivien:} & Skąd oni się urwali?\\
      \textbf{Garret:} & Z Fallouta.\\
   \end{tabularx}
\end{rpg-quotebox}

   
\begin{rpg-quotebox}{Nie czyń bezmyślnie!}
   \textit{Garret poszedł na pierwszy ogień przywitać się z kultystami, najpewniej wrogo nastawionymi.}\\
   
   \begin{tabularx}{\columnwidth}{lX}
      \multicolumn{2}{l}{\textit{}}\\
      
      \textbf{Kultyści:} & Czego tu szukasz?\\
      \textbf{Garret:} & Nie przemyślałem tego.\\
   \end{tabularx}
\end{rpg-quotebox}


\begin{rpg-quotebox}{Czas snajperów}
   \textit{Yeleda wystrzeliła zabójczą strzałę z krzaków trafiając jednego z kultystów w głowę. Ten padł jak długi nieprzytomny.}\\
   
   \begin{tabularx}{\columnwidth}{lX}
      \textbf{Garret:} & Jednemu z kultystów świetny pomysł strzelił do głowy.\\
      \multicolumn{2}{X}{\textit{Kultysta - nieudany rzut obronny na śmierć}}\\
      \textbf{MG:} & Jest nieprzytomny. Jego nieprzytomność sięga zenitu.\\
   \end{tabularx}
\end{rpg-quotebox}


\begin{rpg-quotebox}{Wątpliwe zamiary}
   \textit{Yeleda została trafiona zaklęciem i padła nieprzytomna.}\\
   
   \begin{tabularx}{\columnwidth}{lX}
      \textbf{Garret:} & Gdzie ona leży?\\
      \textbf{MG:} & W krzakach.\\
      \textbf{Garret:} & Czeka na okazję.\\
   \end{tabularx}
\end{rpg-quotebox}


\begin{rpg-quotebox}{Walcz z głową...}
   \textit{Kruk Caliona włączył się do bitwy. Kultysta postanowił obrać go na cel.}\\
   
   \begin{tabularx}{\columnwidth}{lX}
      \textbf{MG:} & Kultysta atakuje Kruka z pałki.\\
   \end{tabularx}
\end{rpg-quotebox}


\begin{rpg-quotebox}{O tym jak ważny jest dokładny opis otoczenia}
   \textit{Wygraliśmy potyczkę. Yeleda postanowiła przeszukać ciała.}\\
   
   \begin{tabularx}{\columnwidth}{lX}
      \textbf{MG:} & Znajdujesz cztery pałki.\\
      \textbf{Yeleda:} & Ale jakie?\\
      \textbf{MG:} & Takie do walenia.\\
      \textbf{Garret:} & Nadal nie sprecyzowałaś.\\
   \end{tabularx}
\end{rpg-quotebox}


\begin{rpg-quotebox}{Żart ciągnie się dalej}
   \textit{Vivien podbiegła do Velory z zamiarem przywalenia jej kosturem. }\\
   
   \begin{tabularx}{\columnwidth}{lX}
      \textbf{Garret:} & Grzmocisz ją drągiem?\\
   \end{tabularx}
\end{rpg-quotebox}


\begin{rpg-quotebox}{Mistrzowska aranżacja wnętrz}
   \textit{W odpowiedzi na to, jak konstrukt krasnoludzkiej zbroi postanowił nas zaatakować.}\\
   
   \begin{tabularx}{\columnwidth}{lX}
      \textbf{Garret:} & Latający miecz, gryzący krzak, żyjąca zbroja - what the fuck?"\\
   \end{tabularx}
\end{rpg-quotebox}


\begin{rpg-quotebox}{Przywar ciężko się wyzbyć}
   \textit{Walka z cieniami. Yeledzie nie wyszedł atak i miecz wyleciał jej z ręki.}\\
   
   \begin{tabularx}{\columnwidth}{lX}
      \multicolumn{2}{l}{\textit{Do Yeledy}}\\
      
      \textbf{Vivien:} & Upuściłaś miecz.\\
      \textbf{Garret:} & Cień nie jest tobą zainteresowany.\\
   \end{tabularx}
\end{rpg-quotebox}


\begin{rpg-quotebox}{Głos w sprawie tłumienia odruchów}
   \textit{Vivien ogłuszyła kapłanów i wyrzuciła ich ciała na odległość za pomocą czaru Thunderwave. Była wśród nich kapłanka, która przywołała iluzoryczny miecz. W efekcie miecz zniknął.}\\
   
   \begin{tabularx}{\columnwidth}{lX}
      \multicolumn{2}{l}{\textit{Do Yeledy}}\\
      
      \textbf{Garret:} & Miecz znika, nie podnosimy go - bo go nie ma.\\
   \end{tabularx}
\end{rpg-quotebox}


      
\begin{rpg-quotebox}{Kocim żartom nie było końca}
   \textit{Calion wziął w ręce figurkę tygrysa i w się niego zamienił. Zwyczajny tygrys pojawił się na miejscu maga drużyny.}\\

   \begin{tabularx}{\columnwidth}{lX}
      \textbf{MG:} & Calion masz na tyle inteligencji, że wiesz, że nie musisz ich atakować.\\
      \textbf{Yeleda:} & Kici kici kici!\\
      \textbf{Calion:} & Warczę na Yeledę.\\
   \end{tabularx}
   ~\newline\newline
   \begin{tabularx}{\columnwidth}{lX}
      \textbf{Garret:} & Jestem mały - mogę na nim jeździć.\\
      \textbf{Calion:} & Spróbuj, to cię rozszarpię.\\
   \end{tabularx}
   ~\newline\newline
   \begin{tabularx}{\columnwidth}{lX}
      \multicolumn{2}{X}{\textit{Fantazjując na temat tego co by było, gdyby Vivien zamieniła się teraz w panterę.}}\\
      
      \textbf{Yeleda:} & Czy ona i tygrys mogliby...\\
      \textbf{Vivien:} & Mam nadzieję, że jest kastrowany.\\
   \end{tabularx}
\end{rpg-quotebox}


\begin{rpg-quotebox}{Kocia retrospekcja}
   \textit{Yeleda chciała otworzyć jakiś zamek. Vivien oparła jej rękę na ramieniu by pomóc jej sztuczką Guidance.}\\
   
   \begin{tabularx}{\columnwidth}{lX}
      \textbf{Yeleda:} & O! Znów mnie głaszczesz.\\
      \multicolumn{2}{l}{\textit{Cicho do Yeledy.}}\\
      \textbf{Vivien:} & A gdy głaskałam Caliona miał takie aksamitne futro....\\
   \end{tabularx}
\end{rpg-quotebox}


\begin{rpg-quotebox}{Powerleveling}
   \textit{Po zakończonej walce z kultystami.}\\
   
   \begin{tabularx}{\columnwidth}{lX}
      \textbf{Yeleda:} & Kopej EXPA!
   \end{tabularx}
\end{rpg-quotebox}


\begin{rpg-quotebox}{O tym jak Vivien zaufała w pełni swoim bogom}
   \textit{Vivien bada mechanicznie ukryte drzwi. Po chwili odnajduje ukryty przycisk i jak gdyby nigdy nic naciska go.}\\
   
   \begin{tabularx}{\columnwidth}{lX}
      \textbf{MG:} & Rzucaj na zręczność.\\
      \multicolumn{2}{l}{\textit{Rzut k20 - 20}}\\
      \textbf{MG:} & Z zakamarków drzwi wystrzela zatruta strzałka w twoim kierunku, ale CUDEM robisz unik i    pocisk leci przez całą salę między głowami drużyny i wbija się w naprzeciwległą ścianę.\\
      \multicolumn{2}{l}{\textit{Do oszołomionej drużyny.}}\\
      \textbf{Vivien:} & Ups!\\
   \end{tabularx}
\end{rpg-quotebox}

\section*{11 czerwca 2017}



\begin{rpg-quotebox}{O niepomyślnych wiatrach}
   \textit{Drużyna została zaatakowana przez żywiołaki powietrza w sali prób.}\\\newline
   \begin{tabularx}{\columnwidth}{lX}
      \textbf{MG:} & Vivien i Caliona odrzuciło.\\
      \textbf{Yeleda:} & Najwyraźniej z tej dziury bardzo źle pachnie.
   \end{tabularx}
\end{rpg-quotebox}


\begin{rpg-quotebox}{Konflikt interesów}
   \begin{tabularx}{\columnwidth}{lX}
      \textbf{Calion:} & Lecz Viv, bo do dupy nakopię.\\
      \multicolumn{2}{X}{\textit{Z urażonym głosem}}\\
      \textbf{Vivien:} & Ja leczę...\\
      \multicolumn{2}{X}{\textit{Calion się uśmiecha.}}\\
      \textbf{Vivien:} & ... ale siebie.\\
      \multicolumn{2}{X}{\textit{Mina Caliona przybiera wściekłe rysy.}}
   \end{tabularx}
\end{rpg-quotebox}


\begin{rpg-quotebox}{A oczy jego otwarły się}
   \begin{tabularx}{\columnwidth}{lX}
      \textbf{Garret:} & Przyglądam się dziurze\\
      \multicolumn{2}{X}{\textit{Rezultat rzutu k20 - 20.}}\\
      \textbf{MG:} & Sens istnienia stanął przed tobą otworem\\
   \end{tabularx}
\end{rpg-quotebox}


\begin{rpg-quotebox}{Chwilowa nieobecność}
   \begin{tabularx}{\columnwidth}{lX}
      \textbf{Calion:} & Viv myślisz to co ja?\\
      \textbf{Vivien:} & Nie, ja nic nie myślę.\\
   \end{tabularx}
\end{rpg-quotebox}


\begin{rpg-quotebox}{Paradoks}
   \textit{Garret wykonuje piękny skok do wody.}\\
   \newline
   \begin{tabularx}{\columnwidth}{lX}
      \textbf{Garret:} & Osiągnąłem dno.\\
   \end{tabularx}
\end{rpg-quotebox}


\begin{rpg-quotebox}{Fatalne przejęczenie}
   \begin{tabularx}{\columnwidth}{lX}
      \textbf{Garret:} & Pomagam Vivien wziąć Yeledę pod wodą... - yyy! - pod wodę!\\
   \end{tabularx}
\end{rpg-quotebox}


\begin{rpg-quotebox}{Definitywny brak fornifilii}
   \begin{tabularx}{\columnwidth}{lX}
      \textbf{Yeleda:} & Nie pieprzę się z drzwiami.\\
   \end{tabularx}
\end{rpg-quotebox}


\begin{rpg-quotebox}{O nie taki wiatr chodziło}
   \textit{Calion używa czaru podmuchu wiatru. }\\
   \newline
   \begin{tabularx}{\columnwidth}{lX}
      \textbf{Garret:} & Calion robi magiczny pierd\\
   \end{tabularx}
\end{rpg-quotebox}


\begin{rpg-quotebox}{Ładne wybrnięcie}
   \textit{Drużyna odrzuciła propozycję wiedźmy by wydostać się z tuneli.}\\
   
   \begin{tabularx}{\columnwidth}{lX}
      \textbf{Yeleda:} & Czyli nara.\\
      \textbf{Calion:} & Nara czyli co?\\
      \textbf{Yeleda:} & Nara-żamy życie.\\
   \end{tabularx}
\end{rpg-quotebox}


\begin{rpg-quotebox}{Nagły problem obuwniczy}
   \begin{tabularx}{\columnwidth}{lX}
      \multicolumn{2}{l}{\textit{Do Yeledy:}}\\
      
      \textbf{MG:} & Weszłaś w śluz więc otrzymujesz 6 pkt. obrażeń od kwasu. I nie masz buta.\\
   \end{tabularx}
\end{rpg-quotebox}


\begin{rpg-quotebox}{Jak praca zespołowa czasem się nie udaje}
   \textit{Garret, Yeleda i Vivien postanowili przeskoczyć plamę śluzu, przy czym Yeleda i Vivien miały pecha i wyryły się na twarz. NA KONIEC Calion użył czaru, by usunąć śluz i przejść bez szwanku...}
\end{rpg-quotebox}


\begin{rpg-quotebox}{Pamiętaj aby nie narzekać w obecności MG na rozwój przygody}
   \textit{Po jednej z sesji Łukasz narzekał na ogrom magicznych pułapek i niewielką ilość pułapek ściśle mechanicznych. Niemal na początku kolejnej sesji, już jako Garret, potknął się o linę i uruchomił zapadnię w podłodze. Drużyna musiała wykonać test zręczności. Vivien i Yeleda w porę odskoczyły, a Calion zamortyzował upadek. MG nie pozwoliła jednak Garretowi nic zrobić, bo to on uruchomił pułapkę.}\\
   \newline
   \begin{tabularx}{\columnwidth}{lX}
      \textbf{MG:} & Ty masz na tej linie automatycznie porażkę!\\
      \textbf{Garret:} & Ale ja mam refleksy!\\
      \textbf{Yeleda:} & Chyba refluksy...\\
      \textbf{Calion:} & Chciałeś niemagiczne pułapki, patafianie. To masz!\\
   \end{tabularx}
\end{rpg-quotebox}


\begin{rpg-quotebox}{O dziwach matki natury}
   \begin{tabularx}{\columnwidth}{lX}
      \textbf{Yeleda:} & Czy pająki strzelają?\\
      \textbf{Garret:} & Tak, liną z dupy.\\
   \end{tabularx}
\end{rpg-quotebox}


\begin{rpg-quotebox}{Kiedy chwalebna rola zaczyna się przejadać}
   \textit{Jakiś potwór poczynił pogrom wśród drużyny.}\\
   \newline
   \begin{tabularx}{\columnwidth}{lX}
      \textbf{Vivien:} & Ja pierdolę. Znowu muszę leczyć.\\
   \end{tabularx}
\end{rpg-quotebox}


\begin{rpg-quotebox}{Każdy orze jak może}
   \textit{Drużyna chciała wejść do otworu w podłodze, tak by jak najmniej się poturbować. Yeleda użyła latającego płaszcza, wzięła pod pachy Caliona, a Vivien przemieniona w pająka usiadła jej na ramieniu. Garret z mnisią gracją wskoczył samodzielnie za nimi.}\\
\end{rpg-quotebox}


\begin{rpg-quotebox}{Antyornityzm}
   \textit{Po walce, w której Antaramo, kruk-chowaniec Caliona okazał się niezwykle pomocny.}\\
   \newline
   \begin{tabularx}{\columnwidth}{lX}
      \textbf{MG:} & Usmażyłeś pająka. Muszę znaleźć jakiegoś potwora, który zjada kruki.\\
   \end{tabularx}
\end{rpg-quotebox}


\begin{rpg-quotebox}{O tym jak brutalna siła stała się mototrem napędowym strachu}
   \textit{Drużyna została zaskoczona widokiem potężnej Banshee za otwartymi drzwiami w podziemiach. Calion instynktownie zaatakował ją zaklęciem Firebolt, ale niewiele wskórał.}\\
   \begin{tabularx}{\columnwidth}{lX}
      \textbf{MG:} & Jest odporna na ogień.\\
      \textbf{Garret:} & To ona jest z azbestu?!\\
   \end{tabularx}
   \textit{Garret postanowił ją załatwić pięściami i wyrzucił pod rząd trzy naturalne 20, co zabrało banshee jakieś $\frac{2}{3}$ życia. Drużyna zyskała punkt inspiracji.}\\
   \textit{Calion postanowił ją dobić magicznym pociskiem.}\\
   \begin{tabularx}{\columnwidth}{lX}
      \textbf{Calion:} & Lekko chujowo, tylko 7 punktów obrażeń.\\
      \textbf{MG:} & Zabijasz ją.\\
   \end{tabularx}
\end{rpg-quotebox}


\begin{rpg-quotebox}{Moralność panien D-n-dulskich}
   \textit{W pokoju razem z banshee było 6 kultystów, którzy srali po gaciach na widok Garreta po tym jak załatwił ich banshee. Byli tak przerażeni, że nie byli w stanie mówić. Jednak w końcu drużyna wyciągnęła z nich, że tak do końca nie wiedzieli co robią. Garret sprzeciwił się zabijaniu ich, bo: Naiwniactwo jest okolicznością łagodzącą.}\\
   
   \begin{tabularx}{\columnwidth}{lX}
      \textbf{Vivien:} & Wisi mi ich los.\\
      \textbf{Yeleda:} & Mi wisi bardziej niż tobie.
   \end{tabularx}
\end{rpg-quotebox}


\begin{rpg-quotebox}{Dobrze strzeżona spiżarnia}
   \textit{Drużyna została przeniesiona za pomocą portalu do czegoś co wyglądało jak spiżarnia.}\\
   
   \begin{tabularx}{\columnwidth}{lX}
      \textbf{MG:} & Znajdujecie konfitury, piwo, książki i bibeloty\\
      \textbf{Yeleda:} & Szukam pułapek.
   \end{tabularx}
\end{rpg-quotebox}


\begin{rpg-quotebox}{Pomoc poprzez molestowanie}
   \textit{Calion próbuje sobie przypomnieć w jakim mieście wylądowali.}\\
   
   \begin{tabularx}{\columnwidth}{lX}
      \multicolumn{2}{l}{\textit{Vivien korzysta ze sztuczki Guidance, aby mu pomóc.}}\\
      
      \textbf{MG:} & Calion, czujesz, że ktoś Cię maca.\\
   \end{tabularx}
\end{rpg-quotebox}

\section*{24 czerwca 2017}

\begin{rpg-quotebox}{O grupach etnicznych}
   \begin{tabularx}{\columnwidth}{lX}
      \textbf{MG:} & Niziołki nie są rasą mniejszościową.\\
      \textbf{Viv:} & Chyba niższościową.\\
      \textbf{Garret:} & Czuję się poniżony.
   \end{tabularx}
\end{rpg-quotebox}


\begin{rpg-quotebox}{Nieukrywane stereotypy}
   \begin{tabularx}{\columnwidth}{lX}
      \textbf{MG:} & Widzicie wnętrze karczmy wypełnione resztkami po imprezie. Za barem nikogo nie widać.\\
      \textbf{Yeleda:} & Pewnie niziołek go prowadzi.\\
      \textbf{Garret:} & Podchodzę do baru i wołam karczmarza.\\
      \textbf{MG:} & Z zaplecza polerując szklankę wychodzi niziołek...\\
   \end{tabularx}
\end{rpg-quotebox}


\begin{rpg-quotebox}{Strach nie sięga wysoko}
   \begin{tabularx}{\columnwidth}{lX}
      \textbf{Yeleda:} & Mam czkawkę.\\
      \textbf{Vivien:} & Wystraszcie ją!\\
      \textbf{Garret:} & Uaaaa!\\
      \textbf{Yeleda:} & Tak... Niziołku...\\
      \textbf{Vivien:} & Nawrzeszczał Ci na kolana.\\
   \end{tabularx}
\end{rpg-quotebox}


\begin{rpg-quotebox}{O tym jak MG nie wytrzymał głupiego pytania}
   \begin{tabularx}{\columnwidth}{lX}
      \textbf{MG :} & Barmanowi mózg wyeksmitował z głowy.\\
   \end{tabularx}
\end{rpg-quotebox}


\begin{rpg-quotebox}{O charakterach}
   \begin{tabularx}{\columnwidth}{lX}
      \textbf{Calion:} & Garret jest praworządny dobry, a ja dobry. Razem jesteśmy praworządni głupi.\\
   \end{tabularx}
\end{rpg-quotebox}


\begin{rpg-quotebox}{Bieda aż piszczy}
   \textit{Gracze znajdują się w ekskluzywnym sklepie prowadzonym przez drakonkę.} \\

   \begin{tabularx}{\columnwidth}{lX}
      \textbf{Yeleda:} & W sumie kupiłabym mapę, gdyby była jedna w rozsądnej cenie.\\
      \textbf{Garret:} & Czyli drakońskie ceny odpadają.\\
   \end{tabularx}
\end{rpg-quotebox}


\begin{rpg-quotebox}{Globalizacja dociera i do Zapomnianych Krain}
   \textit{Niecodziennie o racjach żywnościowych} \\

   \begin{tabularx}{\columnwidth}{lX}
      \textbf{MG:} & Te za 1 szt. srebra to taka średniowieczna zupka chińska.\\
   \end{tabularx}
\end{rpg-quotebox}


\begin{rpg-quotebox}{O tym jak w żyłach Rosjan i niziołków płynie ta sama krew}
   \textit{Po udanym teście na akrobatykę:}\\

   \begin{tabularx}{\columnwidth}{lX}
      \textbf{MG:} & Udaje ci się doskoczyć do parapetu i wykonać słowiański przykuc.\\
   \end{tabularx}
\end{rpg-quotebox}


\begin{rpg-quotebox}{O tym jak naturalne 20 pomaga w schadzkach w ciemnych alejkach}
   \textit{Garret przeczuwał, że był śledzony. Chcąc zmylić pościg uznał, że schowa się w ciemnej alei i z zaskoczenia zaatakuje potencjalnego złoczyńcę.}\\

   \begin{tabularx}{\columnwidth}{lX}
      \textbf{MG:} & Udało ci się powalić młodą kobietę.\\
      \textbf{Yeleda:} & Ty ogierze!\\
   \end{tabularx}
\end{rpg-quotebox}


\section*{15 lipca 2017}


\begin{rpg-quotebox}{O oczekiwaniach w sprawie korupcji}
   \begin{tabularx}{\columnwidth}{lX}
      \textbf{Matheus:} & Żarcie dla MGa $ \rightarrow $ +10 do wszystkich rzutów.
   \end{tabularx}
\end{rpg-quotebox}


\begin{rpg-quotebox}{O braku administracji państwowej}
   \textit{Gracze rozważają pomysł podróży poza gościńcem. Pytanie zostało skierowane do MG.}\\

   \begin{tabularx}{\columnwidth}{lX}
      \textbf{Calion:} & Czy Angamarna ma Lasy Państwowe?\\
   \end{tabularx}
\end{rpg-quotebox}


\begin{rpg-quotebox}{O przesadzaniu w obmyślaniu przykrywki}
   \textit{Vivien wymyśliła przykrywkę - jest studentką magii naturalnej i pisze pracę na temat szkodliwego wpływu kornika na populacje drzew w okolicznych lasach. }\\

   \begin{tabularx}{\columnwidth}{lX}
      \textbf{Vivien:} & Czemu drużyna się po lesie szlaja? Czy szaleje tam kornik drukarz?\\
   \end{tabularx}
\end{rpg-quotebox}


\begin{rpg-quotebox}{Kiedy survival staje się prawdziwą walką o przetrwanie}
   \begin{tabularx}{\columnwidth}{lX}
      \textbf{MG :} & Zaczynacie KRĄŻYĆ po lesie pod światłym przewodnictwem waszego druida.\\
   \end{tabularx}
\end{rpg-quotebox}



\begin{rpg-quotebox}{Problemy dendro-andrologiczne}
   \textit{W reakcji na napotkany szlak rozsypanych kawałków drewna w runie leśnym.}\\

   \begin{tabularx}{\columnwidth}{lX}
      \textbf{Garret:} & Albo to jest duże...\\
      \textbf{Vivien:} & Albo mamy do czynienia z łysiejącym Entem.\\
   \end{tabularx}
\end{rpg-quotebox}


\begin{rpg-quotebox}{Rozterki ekologiczne}
   \textit{Pytanie skierowane do MG na temat napotkanego potwora:}

   \begin{tabularx}{\columnwidth}{lX}
      \textbf{Calion:} & Czy czerwony smok jest pod ochroną? \\
   \end{tabularx}
\end{rpg-quotebox}


\begin{rpg-quotebox}{O tym jak trudne potyczki czasem nie bywają trudne...}
   \textit{Yeleda wykorzystała atak z zaskoczenia i wycelowała z łuku w oko smoka. Los chciał by k20 pokazała 20 i smok tak mocno oberwał w łeb że został niemal ogłuszony i oślepł.}
\end{rpg-quotebox}


\begin{rpg-quotebox}{... i nawet MG potrafi być zaskoczony obrotem spraw}
   \textit{Smok zajęty był rozrywaniem skrzyni więc drużyna miała przewagę atakiem z zaskoczenia. Po niewiarygodnym trafieniu z łuku, do akcji wkroczył Calion.}\\

   \begin{tabularx}{\columnwidth}{lX}
      \textbf{MG:} & Z krzaków wyskoczył opętany mag, który dobił smoka zanim ten choćby beknął.\\
   \end{tabularx}
\end{rpg-quotebox}


\begin{rpg-quotebox}{W przypływie podziwu}
   \textit{O obrocie spraw po niewiarygodnie szybkim rozprawieniu się z małym smokiem:} \\
   
   \begin{tabularx}{\columnwidth}{lX}
      \textbf{Vivien:} & Jesteście cud miód drużyną!\\
   \end{tabularx}
\end{rpg-quotebox}


\begin{rpg-quotebox}{Trafne podsumowanie potyczki}
   \textit{Po kratycznym trafieniu Yeledy.}\\

   \begin{tabularx}{\columnwidth}{lX}
      \textbf{Garett:} & One shot one kill no luck pure skill.\\
   \end{tabularx}
\end{rpg-quotebox}


\begin{rpg-quotebox}{Żart dobrze pociągnięty}
   \textit{BN przyłapał drużynę na zwycięstwie ze smokiem i spytał co się tu właściwie stało. Vivien pozostaje w roli studentki badającej drzewostan lasów Doliny Konwalii:} \\

   \begin{tabularx}{\columnwidth}{lX}
      \textbf{:} & To był wyjątkowo duży kornik. Bardzo szkodliwy!\\
   \end{tabularx}
\end{rpg-quotebox}


\begin{rpg-quotebox}{Obawy o zdrowy sen}
   \begin{tabularx}{\columnwidth}{lX}
      \multicolumn{2}{l}{\textit{Na temat odpoczynku w lesie:}} \\
      \textbf{Calion:} & Vincent powiedział, że smok miał brata.\\
      \textbf{Vivien:} & My mamy Yeledę.
   \end{tabularx}
\end{rpg-quotebox}


\begin{rpg-quotebox}{Nieformalne przedstawienie}
   \textit{Po przygodzie z bardzo szybkim uporaniem się z banshee, mnich uznał, że przedstawi się nowonapotkanemu BN-mu, Vincentowi w następujący sposób.}\\

   \begin{tabularx}{\columnwidth}{lX}
      \multicolumn{2}{l}{\textit{Wyciąga w górę pięść.}}\\
      \textbf{Garret:} & Jestem oficjalnym eksterminatorem banshee. \\
   \end{tabularx}
\end{rpg-quotebox}


\begin{rpg-quotebox}{O tym jak tłumaczyć niecodzienne talenty}
   \begin{tabularx}{\columnwidth}{lX}
      \multicolumn{2}{l}{\textit{Do MG.}}\\
      \textbf{Yeleda:} & Co trzeba mieć by oprawić smoka?\\
      \textbf{Calion:} & Backstory.
   \end{tabularx}
\end{rpg-quotebox}


\begin{rpg-quotebox}{O tym\, że skrywamy ukryty talent garbarniczy}
   \textit{Po oprawieniu truchła smoka:}\\
   
   \begin{tabularx}{\columnwidth}{lX}
      \textbf{MG:} & Udało się wam z niego wyłuskać jakieś 20kg łusek.\\
   \end{tabularx}
\end{rpg-quotebox}


\begin{rpg-quotebox}{Przypraw swoje życie}
   \textit{Calion użył stuczki Prestidigitacji by oczyścić nas z resztek wnętrzności smoka. Ta sztuczka pozwala się osuszyć, oczyścić czy choćby poprawić jakość jedzenia i to bez płacenia!}\\ 
   
   \begin{tabularx}{\columnwidth}{lX}
      \textbf{Garret:} & A teraz jeszcze szczypta magii!\\
   \end{tabularx}
\end{rpg-quotebox}


\begin{rpg-quotebox}{Niecodzienne nazewnictwo}
   \begin{tabularx}{\columnwidth}{lX}
      \textbf{Calion:} & Definicja portalu - slot na bramę.\\
   \end{tabularx}
\end{rpg-quotebox}


\begin{rpg-quotebox}{Rzut to nie wszystko}
   \textit{Calion miał przejść test wiedzy by sobie przypomnieć co nieco o ruinach które eksplorowaliśmy.}\\

   \begin{tabularx}{\columnwidth}{lX}
      \multicolumn{2}{l}{\textit{Rzut k20: 4}} \\ 
      \textbf{Calion:} & W sumie 11.\\
      \textbf{Vivien:} & Dodatkowo wykorzystuję swoją sztuczkę guidance i dokładam 4 do sumy rezultatu.\\
      \textbf{Calion:} & Wypadło 4 więc mamy wynik 15.\\
   \end{tabularx}
\end{rpg-quotebox}


\begin{rpg-quotebox}{O nieuczciwych przePISach}
   \textit{Callion i Garret wyrzucili 16 na inicjatywie i mieli doń ten sam bonus. Kolejność rozstrzygnął rzut kostką, a wygrany Calion otrzymał program 16+.}
\end{rpg-quotebox}


\begin{rpg-quotebox}{Paradoksalnie dobre podsumowanie sytuacji}
   \begin{tabularx}{\columnwidth}{lX}
      \textbf{MG:} & Dwa mocno nadszarpnięte życiem zombiaki wam zostały.\\
   \end{tabularx}
\end{rpg-quotebox}


\begin{rpg-quotebox}{O rozmowie na dwa fronty na temat szczęśliwych kostek}
   \begin{tabularx}{\columnwidth}{lX}
      \multicolumn{2}{l}{\textit{Do Kasi:}} \\
      \textbf{Madzia:} & Tu mamy cały asortyment broni przeciwko naszemu MG. \\
      \multicolumn{2}{l}{\textit{Do Kingi:}} \\
      \textbf{Madzia:} & Tego nie słyszałaś.
   \end{tabularx}
\end{rpg-quotebox}


\begin{rpg-quotebox}{O chęci mordoterapii...}
   \begin{tabularx}{\columnwidth}{lX}
      \multicolumn{2}{X}{\textit{Po desperackiej walce ze smokiem, MG uśmiecha się złowieszczo.}}\\
      \textbf{Calion:} & Ile trzeba było szczęścia by to przeżyć? \\
      \textbf{MG:} & DUŻO.\\
   \end{tabularx}
\end{rpg-quotebox}


\begin{rpg-quotebox}{Jak wszedłeś między wrony...}
   \begin{tabularx}{\columnwidth}{lX}
      \multicolumn{2}{l}{\textit{Gracze aktywowali pułapkę.}}\\
      \textbf{MG:} & Wszyscy wykonajcie rzut obronny na zręczność.\\
      \textbf{Calion:} & Ja nie byłem blisko drzwi.\\
      \textbf{MG:} & WSZYSCY!\\
   \end{tabularx}
\end{rpg-quotebox}


\begin{rpg-quotebox}{Niziołek z wozu...}
   \begin{tabularx}{\columnwidth}{lX}
      \textbf{Yeleda:} & To jak mi Garret pomoże to wyciągniemy wóz.\\
      \textbf{Garret:} & Wsiadam na wóz!\\
   \end{tabularx}
\end{rpg-quotebox}


\begin{rpg-quotebox}{Celowe przejęczyzenie}
   \begin{tabularx}{\columnwidth}{lX}
      \multicolumn{2}{l}{\textit{Garret wykorzystuje umiejętność Medyk.}}\\
      \textbf{Garret:} & Roztaczam nad chętnymi opiekę.\\
      \textbf{Yeleda:} & Ja jestem hentai.
   \end{tabularx}
\end{rpg-quotebox}

\begin{rpg-quotebox}{Tym razem pudło}
   \begin{tabularx}{\columnwidth}{lX}
      \textbf{MG:} & Rzucaj na trafienie\\
      \multicolumn{2}{l}{\textit{Kostka przelatuje obok}} \\
      \textbf{Vivien:} & Nie trafiłeś w wieżę.\\
   \end{tabularx}
\end{rpg-quotebox}

\begin{rpg-quotebox}{Fortuna kołem się toczy}
   \textit{Po nieudanym teście trafienia Yeleda myślała, że jej tura dobiegła końca.}\\

   \vskip-0.05cm
   \begin{tabularx}{\columnwidth}{lX}
      \textbf{MG:} & Ale miałaś przewagę, rzucaj jeszcze raz.\\
   \end{tabularx}

   \vskip0.15cm
   \textit{Yeleda wyrzuciła 20.}\\

   \vskip-0.05cm
   \begin{tabularx}{\columnwidth}{lX}
      \textbf{MG:} & Mogłam tego nie mówić...\\
   \end{tabularx}
\end{rpg-quotebox}


\begin{rpg-quotebox}{Zemsta faraonów}
   \begin{tabularx}{\columnwidth}{lX}
      \textbf{Calion:} & Używam magicznego pocisku z 2 poziomu.\\ 
   \end{tabularx}

   \vskip0.15cm
   \textit{Rzuca czterema kościami k4.} \\
   
   \begin{tabularx}{\columnwidth}{lX}
      \textbf{Vivien:} & Co to ma być??\\
      \textbf{Garret:} & To jest kurwa Egipt.
   \end{tabularx}
\end{rpg-quotebox}


\begin{rpg-quotebox}{O przemyślanym podarku}
   \textit{Wdzięczny BN obdarował graczy podarkami. Yeledzie przypadł w udziale zaklęty sztylet.}\\
   
   \begin{tabularx}{\columnwidth}{lX}
      \textbf{Calion:} & Skąd wiedział, że jest łotrzykiem?\\
      \textbf{MG:} & Ma broń na wierzchu.\\
   \end{tabularx}
   \newline

   \textit{Komentując przytroczone, powieszone i pochowane elementy uzbrojenia Yeledy} \\
   
   \begin{tabularx}{\columnwidth}{lX}
      \textbf{Calion:} & To chodząca zbrojownia!\\
   \end{tabularx}
\end{rpg-quotebox}


\section*{22 lipca 2017}


\begin{rpg-quotebox}{Wysokość spojrzenia ma znaczenie}
   \textit{Garret wyrzucił najwyższy wynik testu na percepcję.} \\

   \begin{tabularx}{\columnwidth}{lX}
      \textbf{MG:} & Tylko niziołek ma oczy.\\
   \end{tabularx}
\end{rpg-quotebox}


\begin{rpg-quotebox}{O ki(j) chodzi?}
   \begin{tabularx}{\columnwidth}{lX}
      \textbf{MG:} & Jesteś piętro niżej więc cała twoja akcja to ruch.\\
      \textbf{Garret:} & Chyba, że użyję punkt Ki.\\
      \textbf{MG:} & Pff...\\
   \end{tabularx}
\end{rpg-quotebox}

\begin{rpg-quotebox}{Jak światło stało się ich przewodnikiem}
   \textit{Przeciwnikowi atak wcześniej wypadł miecz z ręki więc została mu tylko kusza. Udało mu ukryć się w cieniu, jednak nie udało mu się wywieźć w pole Vivien. Ta rzuciła sztuczkę ,,światło'' na jego kuszę i poinstruowała drużynę by strzelali w stronę światła.}
\end{rpg-quotebox}


\begin{rpg-quotebox}{Słaniające się przejęzyczenie}
   \begin{tabularx}{\columnwidth}{lX}
      \textbf{MG:} & Przywaliłeś mu dość mocno i jest mocno poturbowany, ale stoi na ostatnich nogach.\\
   \end{tabularx}
\end{rpg-quotebox}


\begin{rpg-quotebox}{Logika stomatologiczna}
   \textit{Zapytana o lokalizację zakładnika:} \\

   \begin{tabularx}{\columnwidth}{lX}
      \textbf{Vivien:} & Yeleda pilnuje swojego nowego adoratora na dole.\\
      \textbf{Yeleda:} & On nie ma zębów!\\
      \textbf{Vivien:} & Przynajmniej nie ugryzie.\\
   \end{tabularx}
\end{rpg-quotebox}


\begin{rpg-quotebox}{O problemach moralnych}
   \begin{tabularx}{\columnwidth}{lX}
      \textbf{Calion:} & A to gdzie jest ta magiczna zbroja?\\
      \textbf{MG:} & Na gościu.\\
      \textbf{Vivien:} & On jeszcze żyje, więc to nie będzie ograbianie zwłok.\\
      \textbf{Calion :} & Tylko żywych...\\
   \end{tabularx}
\end{rpg-quotebox}


\begin{rpg-quotebox}{O braku spostrzegawczości}
   \begin{tabularx}{\columnwidth}{lX}
      \textbf{Zakładnik:} & Powiedziałem wam wszystko. Rozwiążcie mnie i pozwólcie odejść.\\
      \textbf{Yeleda:} & Już jesteś rozwiązany, paciuloku.\\
   \end{tabularx}
\end{rpg-quotebox}


\section*{6 sierpnia 2017}

% \begin{rpg-commentbox}{rpg-commentbox name}
% 	you can add some comments using this box
% \end{rpg-commentbox}

% \begin{rpg-warnbox}{Vivien:}
%    Coś mi tutaj śmierdziało. Wiedziałam, że coś z tym jest nie tak...
% \end{rpg-warnbox}

% \begin{rpg-suggestionbox}{rpg-suggestionbox name}
% 	you can add some suggestion using this box
% \end{rpg-suggestionbox}

\begin{rpg-quotebox}{Problemy niskich}
   \begin{tabularx}{\columnwidth}{lX}
      \textbf{Garret:} & Mam 93 cm wzrostu. \\
      \textbf{Vivien:} & Nie możesz wejść do makro.
   \end{tabularx}
\end{rpg-quotebox}


\begin{rpg-quotebox}{Poranna nagość}
   \textit{Yeleda do Vivien na temat nagości Garreta po wstaniu z łóżka}\\

   \begin{tabularx}{\columnwidth}{lX}
      \textbf{Yeleda:} & Przy twoim wzroście nie możesz widzieć jego przyrodzenia. \\
      \textbf{Vivien:} & Zależy czy ma długi nos...
   \end{tabularx}
\end{rpg-quotebox}


\begin{rpg-quotebox}{O szpiegach wśród niższych ras}
   \begin{tabularx}{\columnwidth}{lX}
      \textbf{Garret:} & Czy ta krasnoludzica ma brodę? \\
      \textbf{MG:}     & Nie. \\
      \textbf{Garret:} & Czyli to niziołek w przebraniu!
   \end{tabularx}
\end{rpg-quotebox}


\begin{rpg-quotebox}{O niespójnościach wypoczynku}
   \textit{MG przyzwał takich wrogów że drużyna zaliczyła TPK. Po jakimś czasie uratowała ich członkini Paktu i życzyła szczęścia. Po odzyskaniu przytomności:} \\

   \begin{tabularx}{\columnwidth}{lX}
      \textbf{Garret:} & Czemu mamy 1 HP? Czy to się nie liczyło jako długi odpoczynek?
   \end{tabularx}
\end{rpg-quotebox}


\begin{rpg-quotebox}{Krótko o szybkości}
   \begin{tabularx}{\columnwidth}{lX}
      \textbf{Yeleda:} & Yay! Mam 19 inicjatywy! \\
      \textbf{Calion, Garret:} & Bitch please! My mamy 20 i 21.
   \end{tabularx}
\end{rpg-quotebox}


\begin{rpg-quotebox}{Wiedza o aranżacji przestrzennej}
   \begin{tabularx}{\columnwidth}{lX}
      \textbf{Calion:} & Identyfikuję mimika! \\
      \textbf{MG:} & Kurwa, jaka to wiedza? \\
      \textbf{Garret:} & Wystrój wnętrz\\
   \end{tabularx}
\end{rpg-quotebox}


\begin{rpg-quotebox}{Misteria poruszania się w rzędzie}
   \textit{Drużyna próbowała skryć się w cieniu i tylko Yeleda przeszła test. Przejście było wąskie, więc ustawili się w szeregu w kolejności: Yeleda, Garret, Vivien, Calion.} \\

   \begin{tabularx}{\columnwidth}{lX}
      \textbf{MG:} & Garret myśli że idzie pierwszy... \\
   \end{tabularx}
\end{rpg-quotebox}


\begin{rpg-quotebox}{Z czarodziejami nie będziesz się nudził}
   \textit{Strażnik posłany po mistrza zakonu nie wracał tak długo, że drużyna zaczęła grać w kalambury przy użyciu sztuczek Caliona.} \\

   \begin{tabularx}{\columnwidth}{lX}
      \textbf{Yeleda:} Co oni tam produkują tego mistrza?! 
   \end{tabularx}
\end{rpg-quotebox}


\begin{rpg-quotebox}{Nie-gołąb pocztowy}
   \begin{tabularx}{\columnwidth}{lX}
      \textbf{Calion:} & Daję krukowi list do przekazania komuś na uniwersytecie. \\
      \textbf{Vivien} & Kruk pocztowy.\\
   \end{tabularx}
\end{rpg-quotebox}


\begin{rpg-quotebox}{Szczerość to podstawa}
   \begin{tabularx}{\columnwidth}{lX}
      \textbf{MG jako Siel:} & Pójście tam na pełnej kur...tyzanie jest złym pomysłem. \\
   \end{tabularx}
\end{rpg-quotebox}


\begin{rpg-quotebox}{Problemy obuwnicze}
   \begin{tabularx}{\columnwidth}{lX}
      \textbf{MG:} & To zupełnie inna para kaloszy. \\
      \textbf{Yeleda:} & Co to kalosze? \\
      \textbf{Garret:} & To zupełnie inna para trzewików odpornych na wilgoć.
   \end{tabularx}
\end{rpg-quotebox}


\begin{rpg-quotebox}{Te wybuchy i te rozbłyski...}
   \textit{W wyniku oszołomienia widowiskowością rzuconego właśnie zaklęcia.} \\

   \begin{tabularx}{\columnwidth}{lX}
      \textbf{Yeleda:} & Co ona z siebie wydała? \\
      \textbf{Vivien:} & Ostatniego slota.
   \end{tabularx}
\end{rpg-quotebox}


\begin{rpg-quotebox}{Chłodna rozprawa}
   \textit{MG wcielaja się w rolę Siel i przygotowuje zaklęcie lodowych noży.} \\

   \begin{tabularx}{\columnwidth}{lX}
      \textbf{Siel:} & Dobra, a nóż się coś trafi.\\
      \textbf{Calion:} & Też używam lodowych noży na kapłanach.\\
   \end{tabularx}
   \newline

   \textit{Vivien również postanowiła rzucić zaklęcie lodowych ostrzy.}\\

   \begin{tabularx}{\columnwidth}{lX}
      \textbf{Vivien:} & Witajcie w lodówce.\\
   \end{tabularx}
\end{rpg-quotebox}

\section*{6 sierpnia 2017}

\begin{rpg-quotebox}{Nie na żarty...}
   \begin{tabularx}{\columnwidth}{lX}
      \multicolumn{2}{l}{\textit{Do Łukasza, przed rozpoczęciem:}}\\
      \textbf{Madzia:} & Cały czas żresz, coś byś pomyślał też!\\
   \end{tabularx}
\end{rpg-quotebox}

\begin{rpg-quotebox}{Nie szata zdobi człowieka}
   \begin{tabularx}{\columnwidth}{lX}
      \textbf{MG:} & Wstajecie, jest trochę lepsza pogoda. Co robicie?\\
      \multicolumn{2}{l}{\textit{Do Łukasza:}}\\
      \textbf{Madzia:} & Jesteś ubrany? Rzuć na to.\\
      \textbf{Matheus:} & A to jest test na zręczność czy inteligencję?\\
   \end{tabularx}
\end{rpg-quotebox}

\begin{rpg-quotebox}{O wyborach dróg nie tylko życiowych}
   \textit{Drużyna ustala czy podążać leśnym traktem, czy skorzystać z gościńca.}\\
   
   \begin{tabularx}{\columnwidth}{lX}
      \textbf{Yeleda:} & Ja chce lasem!\\
      \textbf{Vivien:} & Będziesz się bić z każdym smokiem o legowisko?\\
      \textbf{Yeleda:} & Tak!\\
   \end{tabularx}
\end{rpg-quotebox}

\begin{rpg-quotebox}{O obsadzaniu ról nieodpowiednimi osobami}
   \textit{Drużyna wybrała wychowaną w mieście Yeledę by prowadziła ich przez leśne ostępy. W pewnym momencie usłyszeli wilcze warczenie.}\\
   
   \begin{tabularx}{\columnwidth}{lX}
      \multicolumn{2}{l}{\textit{Uprzedzając MG w jej roli:}}\\
      \textbf{Łukasz:} & Macie wrażenie, że wilki są wszędzie.\\
      \multicolumn{2}{l}{\textit{MG zaczęła się podejrzanie uśmiechać.}}\\
      \textbf{Łukasz:} & Nie! Nie powiedziałem tego!\\
   \end{tabularx}
\end{rpg-quotebox}

\begin{rpg-quotebox}{Wcale nie dobra zmiana}
   \textit{Drużyna stwierdziła, że Yeleda się nie sprawdziła i wybrali wychowanego w klasztorze mnicha na przewodnika. Ukrytych pod kopułą zdybały ogry, które po chwili przyprowadziły ogra przywódcę.}\\

   \begin{tabularx}{\columnwidth}{lX}
      \textbf{MG:} & Garret awansował do programu "Ogr +".\\
   \end{tabularx}
\end{rpg-quotebox}

\begin{rpg-quotebox}{Niektórych rzeczy nie można odzobaczyć}
   \textit{Drużyna walczy z ogrami w kompletnej ciemności. Garret, znacząco niższy od ogrów, ma ograniczone pole widzenia.}\\

   \begin{tabularx}{\columnwidth}{lX}
      \textbf{Garret:} & Szukam słabych punktów. Jest ciemno to chuj widzę.\\
   \end{tabularx}
\end{rpg-quotebox}

\begin{rpg-quotebox}{Kocie zabawy}
   \textit{Calion rozmawiał z duchem elfki, a reszta drużyny czekała w krzakach. Vivien jako pantera, skryta w cieniu Yeleda i Garret.}\\

   \begin{tabularx}{\columnwidth}{lX}
      \textbf{Vivien:} & Poluję na frędzle twojego płaszcza.\\
      \textbf{Yeleda:} & Ale ja nie mam frędzli.\\
      \textbf{Vivien:} & Już masz.\\
   \end{tabularx}
\end{rpg-quotebox}

\begin{rpg-quotebox}{O sposobach walki z cieczą}
   \begin{tabularx}{\columnwidth}{lX}
	   \textit{Walka z żywiołakami wody.}\\
	   \textbf{Garret:} & Próbuję osuszyć go ręcznikiem!\\
   \end{tabularx}
\end{rpg-quotebox}

\begin{rpg-quotebox}{Ręce\, które krzywdzą}
   \textit{Na pytanie czy żywiołaki są odporne na obrażenia fizyczne.}\\
   
   \begin{tabularx}{\columnwidth}{lX}
	   \textbf{MG:} & Garret, czy Twoje pięści są magiczne?\\
	   \textbf{Garret:} & Jeszcze nie...\\
   \end{tabularx}
\end{rpg-quotebox}

\begin{rpg-quotebox}{Raz na wozie\, raz pod wozem}
   \textit{Żywiołak obrał na cel Garreta.}\\
   
   \begin{tabularx}{\columnwidth}{lX}
      \textbf{Garret:} & Ciekawe, czy ci się to uda...\\
      \textbf{MG:} & 24 na trafienie.\\
      \textbf{Garret:} &... skurwysynu...\\
   \end{tabularx}
\end{rpg-quotebox}

\begin{rpg-quotebox}{Susze bywają uciążliwe}
   \textit{O stanie walki z żywiołakami:}\\
   
   \begin{tabularx}{\columnwidth}{lX}
      \textbf{MG:} & Wysuszyliście w ponad połowie jednego żywiołaka.\\
   \end{tabularx}
\end{rpg-quotebox}

\begin{rpg-quotebox}{Akwarium pogrzebowe}
   \textit{Jeden z żywiołaków wchłonął Garreta w swoje wodne trzewia.}\\
   
   \begin{tabularx}{\columnwidth}{lX}
      \textbf{MG:} & Gwoli uściślenia, jak będziecie bić żywiołaka Garret też oberwie.\\
      \textbf{Vivien:} & Dobra jebać, potem się go wskrzesi w końcu znam medycynę.\\
      \multicolumn{2}{l}{\textit{Szykując potężny czar:}}\\
      \textbf{Calion:} & No, skoro nasz druid tak mówi to wybacz bracie\\
   \end{tabularx}
\end{rpg-quotebox}

\begin{rpg-quotebox}{Katastrofa w przestworzach}
   \textit{Drużyna zjeżdża tunelem w dół, który kończy się lądowaniem z dwóch metrów na gołym kamieniu. Calion jechał pierwszy i postanowił rzucić na siebie czar powolnego spadania.}\\
   
   \begin{tabularx}{\columnwidth}{lX}
      \textbf{MG:} & Jesteś pewien że chcesz go rzucić?\\
      \textbf{Calion:} & Oczywiście! Tam są dwa metry!\\
      \textbf{MG:} & OK. Po chwili na pełnej kurwie nadlatuje Garret.\\
   \end{tabularx}
\end{rpg-quotebox}

\begin{rpg-quotebox}{Sztuka dyplomacji z potworami w praktyce}
   \begin{tabularx}{\columnwidth}{lX}
      \multicolumn{2}{l}{\textit{W języku Sylvan}}\\
      \textbf{Viv:} & Odejdź stąd Encie, a nic ci nie zrobimy. Nie jesteśmy wrogami.\\
      \textbf{Ent:} & Nie.\\
   \end{tabularx}
\end{rpg-quotebox}

\begin{rpg-quotebox}{Coś w nim pękło}
   \textit{Duch elfa zabił Vivien. Po tym Garret wyrzucił pod rząd 5 trafień krytycznych.}\\

   \begin{tabularx}{\columnwidth}{lX}
      \textbf{MG:} & Co tu sie właśnie odbyło??\\
      \textbf{Matheus:} & Jakby nie patrzeć to ten elf zabił jego kobietę...\\
   \end{tabularx}
\end{rpg-quotebox}

\section*{4 listopada 2017}

\begin{rpg-quotebox}{O bogatym bukiecie zapachowym}
   \begin{tabularx}{\columnwidth}{lX}
      \textbf{MG:} & Wchodzisz i uderza cię zapach klasycznego lochu.\\
      \textbf{Yeleda:} & Że czym dostałam w łeb?\\
   \end{tabularx}
\end{rpg-quotebox}

\begin{rpg-quotebox}{Kwestia niedopowiedzenia}
   \begin{tabularx}{\columnwidth}{lX}
      \textbf{MG:} & Widzisz zabite skrzynie.\\
      \textbf{Yeleda:} & Czym je zabili?\\
      \textbf{MG:} & Gwoździami!\\
   \end{tabularx}
\end{rpg-quotebox}

\begin{rpg-quotebox}{Z cyklu \emph{grzechy popełnione ustami}}
   \begin{tabularx}{\columnwidth}{lX}
      \textbf{Yeleda:} & Calion, chodź tu - tu jest ten elf którego zaciukaliśmy.\\
      \textbf{Drużyna:} & Nie zaciukaliśmy!\\
      \textbf{Yeleda:} & No to ten, którego biliśmy!\\
      \textbf{Drużyna:} & Nie biliśmy!\\
      \textbf{Yeleda:} & No to ten, przed którym uciekaliśmy!\\
      \textbf{Drużyna:} & Nie uciekaliśmy!\\
      \textbf{Yeleda:} & ... to ten właśnie!\\
   \end{tabularx}
\end{rpg-quotebox}

\begin{rpg-quotebox}{Trudne sprawy}
   \begin{tabularx}{\columnwidth}{lX}
      \textbf{Calion:} & Czemu ja? Viv też jest elfem!\\
      \textbf{Yeleda:} & Ale ty masz więcej życia.\\
      \multicolumn{2}{l}{\textit{Do MG}}\\
      \textbf{Calion:} & Czemu pytałaś na głos?\\
   \end{tabularx}
\end{rpg-quotebox}

\begin{rpg-quotebox}{Ale fizykę to trzeba szanować}
   \begin{tabularx}{\columnwidth}{lX}
      \textbf{Garret:} & Chcę rzucić w upiora granatem.\\
      \multicolumn{2}{l}{\textit{Po chwili namysłu}}\\
      \textbf{Garret:} & Czy to jest ściana nośna?\\
   \end{tabularx}
\end{rpg-quotebox}

\begin{rpg-quotebox}{Szukanie dziury w całym}
   \begin{tabularx}{\columnwidth}{lX}
      \textbf{MG:} & Upiory są odporne na obrażenia fizyczne i część magicznych, ale podatne na ataki od srebrnych broni.\\
      \textbf{Garret:} & Zaciskam pięsci na srebrnych monetach.\\
   \end{tabularx}
\end{rpg-quotebox}

\begin{rpg-quotebox}{Obawa o utratę integralości}
   \textit{Garret musiał się wydostać z płonącego okręgu, do którego wcześniej niezgrabnie wskoczył. Rzut poszedł dobrze.}\\

   \begin{tabularx}{\columnwidth}{lX}
      \textbf{MG:} & Tym razem artystycznie wyskoczyłeś z pomiędzy płomieni.\\
      \textbf{Vivien:} & W między czasie piekąc kiełbaskę.\\
      \textbf{Garret:} & Swoją?!\\
   \end{tabularx}
\end{rpg-quotebox}

\begin{rpg-quotebox}{Zagrożenie pożarowe}
   \begin{tabularx}{\columnwidth}{lX}
      \textbf{Garret:} & Okładam śpiworem ogień w celu ugaszenia.\\
      \textbf{Yeleda:} & Okładam ogień Garretem okładającym ogień śpiworem.\\
   \end{tabularx}
\end{rpg-quotebox}

\begin{rpg-quotebox}{Pszczółki i kwiatuszki}
   \textit{Drużyna zyskała 10 pkt doświadczenia za wyczerpującą dyskusję nt. edukacji seksualnej mnichów.}
\end{rpg-quotebox}

\begin{rpg-quotebox}{Stereotypom nie ma końca}
   \begin{tabularx}{\columnwidth}{lX}
      \multicolumn{2}{l}{\textit{Samokrytycznie:}}\\
      \textbf{Yeleda:} & Tylko końcówki włosów mam blond. Mózg się jeszcze broni!\\
   \end{tabularx}
\end{rpg-quotebox}

\begin{rpg-quotebox}{Nieczyste zagrywki}
   \begin{tabularx}{\columnwidth}{lX}
      \textbf{MG:} & Garret ma niezłą krzepę - zadaje 16 pkt. obrażeń z piąchy.\\
      \textbf{Garret:} & Jeszcze ze srebrnikami - lewy judaszowy!\\
   \end{tabularx}
\end{rpg-quotebox}

\begin{rpg-quotebox}{O niespójnej chęci zysku}
   \textit{Drużyna odnalazła magazyn z zadania i postanowiła go nie ograbiać, by nie narażać się Kultowi. Gdy już wyszli na zewnątrz i nie było mowy o powrocie:}\\
   
   \begin{tabularx}{\columnwidth}{lX}
      \textbf{Garret:} & W sumie to mistrz powiedział, że możemy sobie stamtąd zabrać co chcemy.\\
      \multicolumn{2}{l}{\textit{Unisono}}\\
      \textbf{Yeleda:} & \multirow{ 3}{*}{Teraz mi to mówisz?!}\\
      \textbf{Vivien:} & \\
      \textbf{Calion:} & \\
   \end{tabularx}
\end{rpg-quotebox}

\begin{rpg-quotebox}{O tym jak pozwolono zadecydować wcale-nie-ślepemu losowi}
   \textit{Drużyna rzucała kośćmi którą opcję drogi objąć i rzucała nimi tak długo aż wypadło to co chcieli.}
\end{rpg-quotebox}

\begin{rpg-quotebox}{O chłodnej kalkulacji}
   \begin{tabularx}{\columnwidth}{lX}
      \textbf{Vivien:} & Moja propozycja jest taka: Idźmy do Doliny Konwalii. Tam mamy pewność, że Dolina rozwali i nas i Pakt. We wszystkich innych opcjach Pakt ma przewagę.\\
      \textbf{Drużyna:} & Zgoda!\\
   \end{tabularx}
\end{rpg-quotebox}

\begin{rpg-quotebox}{O dziwnych decyzjach}
   \begin{tabularx}{\columnwidth}{lX}
      \textbf{Yeleda:} & Czy mogę się ukryć na środku gościńca?\\
      \textbf{Garret:} & Tak - obsyp się żwirem i udawaj drogę.\\
   \end{tabularx}
\end{rpg-quotebox}

\begin{rpg-quotebox}{O tym jak za zły żart można spłonąć ze wstydu}
   \begin{tabularx}{\columnwidth}{lX}
      \textbf{Yeleda:} & Przepraszam! Mają panowie ognia? Zapaliłabym coś.\\
      \textbf{BN-i:} & Co chcesz palić? Nie, nie mamy.\\
      \textbf{Garret:} & Karczmę na przykład.\\
   \end{tabularx}
\end{rpg-quotebox}

\begin{rpg-quotebox}{O nagłym zwrocie akcji}
   \begin{tabularx}{\columnwidth}{lX}
      \textbf{Yeleda:} & Okrążam karczmę tak szybko, że teraz to ja ich śledzę.\\
   \end{tabularx}
\end{rpg-quotebox}

\begin{rpg-quotebox}{O realnej ocenie własnych umiejętności}
   \begin{tabularx}{\columnwidth}{lX}
      \textbf{MG:} & W karczmie jest 10 najemników i dwa słoneczne elfy - magowie. Rzućcie na inicjatywę.\\
      \multicolumn{2}{l}{\textit{Poza postacią:}}\\
      \textbf{Garret:} & Czy mamy robić nowe postaci?\\
      \textbf{Vivien:} & Czy mam dostateczną wolę przeżycia, żeby wrócić jako duch?\\
   \end{tabularx}
\end{rpg-quotebox}

\begin{rpg-quotebox}{O tym jak reklamy nie znają granic}
   \begin{tabularx}{\columnwidth}{lX}
      \textbf{MG:} & Myślę, że karczmarka ma większy problem z 11 trupami.\\
      \textbf{Vivien:} & Kiepska reklama..\\
      \textbf{Garret:} & Nie do końca - pomyśl o takim szyldzie: "U nas zaliczysz zgona!", "U nas zalejesz się w trupa!"\\
   \end{tabularx}
\end{rpg-quotebox}

\section*{25 listopada 2017}

\begin{rpg-quotebox}{O smutku wywołanym brakiem odpowiedniego sprzętu}
   \begin{tabularx}{\columnwidth}{lX}
      \multicolumn{2}{l}{\textit{Poza postacią:}}\\
      \textbf{Vivien:} & Mamy już 5. poziom, a ty jeszcze nic nie uwarzyłeś!\\
      \textbf{Garret:} & Z rzeczy do warzenia piwa mam tylko umiejętność.\\
   \end{tabularx}
\end{rpg-quotebox}

\begin{rpg-quotebox}{Zaufanie to podstawa}
   \begin{tabularx}{\columnwidth}{lX}
      \textbf{Calion:} & Czyli wszyscy w transie, a tylko Yeleda śpi-śpi.\\
      \textbf{Vivien:} & Czyli jedyna osoba, która mogłaby ci wbić nóż w plecy gdy śpisz.\\
      \textbf{Yeleda:} & Point!\\
   \end{tabularx}
\end{rpg-quotebox}

\begin{rpg-quotebox}{O tym jak nowoczesne technologie wcale nie są takie mityczne}
   \begin{tabularx}{\columnwidth}{lX}
      \textbf{Calion:} & Rozświetlam swój sztylet.\\
      \textbf{Garret:} & Pnącza trzeba było rozświetlić!\\
      \multicolumn{2}{l}{\textit{Poza postacią:}}\\
      \textbf{Vivien:} & Byłby światłowód.\\
   \end{tabularx}
\end{rpg-quotebox}

\begin{rpg-quotebox}{Jak chcesz to zakończyć?}
   \begin{tabularx}{\columnwidth}{lX}
      \textbf{Garret:} & Skaczę na potwora.\\
      \textbf{MG:} & Jak chcesz na niego skoczyć? Na niego stricte, czy walnąć i wylądować obok?\\
      \textbf{Vivien:} & Czy chcesz w nim skończyć?\\
      \textbf{Garret:} & Z pięścią w jego ryju.\\
   \end{tabularx}
\end{rpg-quotebox}

\begin{rpg-quotebox}{Wyklepie się}
   \begin{tabularx}{\columnwidth}{lX}
      \textbf{MG:} & Potwór po ataku Garreta ma do wysokości ud solidne wgniecenia w korze.\\
   \end{tabularx}
\end{rpg-quotebox}

\begin{rpg-quotebox}{Anatomia anatomii nierówna}
   \textit{Vivien zmieniła się w tygrysa i chciała powalić potwora, ale zamiast tego ten ją ogłuszył potężnym uderzeniem.}\\

   \begin{tabularx}{\columnwidth}{lX}
      \textbf{Garret:} & Czy medycyna liczy się też jako weterynaria?\\
   \end{tabularx}
\end{rpg-quotebox}

\begin{rpg-quotebox}{Konflikt interesów}
   \textit{Yeleda dobiła atakujące pnącze zręcznym strzałem z ukrycia.}\\

   \begin{tabularx}{\columnwidth}{lX}
      \textbf{Calion:} & Hey! Bullshit!\\
      \textbf{Garret:} & Komu ty kibicujesz?\\
   \end{tabularx}
\end{rpg-quotebox}

\begin{rpg-quotebox}{Przy planowaniu należy rozważyć wszystkie okoliczności}
   \begin{tabularx}{\columnwidth}{lX}
      \textbf{Calion:} & Ale moment, co my chcemy zrobić?\\
      \textbf{Garret:} & Owinę się liną, wyjdę przez okno i zobaczę czy z innej strony widać miasto.\\
      \textbf{Yeleda:} & Czy to jest jedyne okno na tym poziomie?\\
      \multicolumn{2}{l}{\textit{Z pokrętnym uśmiechem:}}\\
      \textbf{MG:} & ...nie.\\
   \end{tabularx}
\end{rpg-quotebox}

\begin{rpg-quotebox}{Jeśli coś wydaje się głupie ale działa\, to nie jest głupie}
   \textit{Drużyna zastanawiała się czy przebijać się przez kolejne piętra zarośniętej latarni morskiej czy wybrać szybszy sposób.}\\

   \begin{tabularx}{\columnwidth}{lX}
      \textbf{Calion:} & Proponuję zejść na dół przez pnącza, a jak coś to wrócimy na górę i wyskoczymy przez okno.\\
   \end{tabularx}
\end{rpg-quotebox}

\begin{rpg-quotebox}{O kiepskiej współpracy}
   \textit{Vivien rozmawiała z BN-em (druidaem), którego odratowali z latarni. Rozmowa polegała na wymienianiu informacji na temat panującej inwazji oraz planów co do dalszych ruchów.}\\
   
   \begin{tabularx}{\columnwidth}{lX}
      \multicolumn{2}{l}{\textit{W tle, do MG:}}\\
      \textbf{Garret:} & Podpaliliśmy pnącza!\\
   \end{tabularx}
\end{rpg-quotebox}

\begin{rpg-quotebox}{Zawsze to jakieś uczucie}
   \begin{tabularx}{\columnwidth}{lX}
      \textbf{Vivien:} & Leczę Garreta z 3-ego slota!\\
      \textbf{MG:} & To jest miłość!\\
      \textbf{Calion:} & \multirow{2}{*}{Ona przecież mówiła, że nie legnie z niziołkiem.}\\
      \textbf{Yeleda:} & \\
      \textbf{Vivien:} & Mogę go leczyć, nie muszę całować.\\
   \end{tabularx}
\end{rpg-quotebox}

% \begin{rpg-quotebox}{}
%    \textit{}\\
%    
%    \begin{tabularx}{\columnwidth}{lX}
%       \multicolumn{2}{l}{\textit{}}\\
%       
%       \textbf{:} & \\
%    \end{tabularx}
% \end{rpg-quotebox}
DM: Garret po oberwaniu kamieniem od Giganta ma na czole odbity stopień.
% \begin{rpg-quotebox}{}
%    \textit{}\\
%    
%    \begin{tabularx}{\columnwidth}{lX}
%       \multicolumn{2}{l}{\textit{}}\\
%       
%       \textbf{:} & \\
%    \end{tabularx}
% \end{rpg-quotebox}
Drużyna zdecydowała zejść do kanałów.
DM: ścieków będzie tak po kolana, Garret będzie miał do pasa
% \begin{rpg-quotebox}{}
%    \textit{}\\
%    
%    \begin{tabularx}{\columnwidth}{lX}
%       \multicolumn{2}{l}{\textit{}}\\
%       
%       \textbf{:} & \\
%    \end{tabularx}
% \end{rpg-quotebox}
DM ma problem z wymawianiem literki R.
MG: Tu był magazyn przemycanych dóbr.
Yeleda: dupy przemycali - rude, blondynki co tylko chcesz
% \begin{rpg-quotebox}{}
%    \textit{}\\
%    
%    \begin{tabularx}{\columnwidth}{lX}
%       \multicolumn{2}{l}{\textit{}}\\
%       
%       \textbf{:} & \\
%    \end{tabularx}
% \end{rpg-quotebox}
Korytarz zagradzał potwór spaghetti w wersji ciernistych lian. Drużyna zastanawiała się czy z nim walczyć, czy iść do celu okrężną drogą.
Yeleda: ok czyli bitka! Ukrywam się!
% \begin{rpg-quotebox}{}
%    \textit{}\\
%    
%    \begin{tabularx}{\columnwidth}{lX}
%       \multicolumn{2}{l}{\textit{}}\\
%       
%       \textbf{:} & \\
%    \end{tabularx}
% \end{rpg-quotebox}
Potwór był dość mocnym przeciwnikiem i powalił Garreta tam gdzie stał - czyli w śmierdzącym ścieku
MG: jesteś nieprzytomny i w dodatku topisz się w brei
Garret: jestem mały i zwinny, na pewno dryfuję
MG: twarzą w dół
% \begin{rpg-quotebox}{}
%    \textit{}\\
%    
%    \begin{tabularx}{\columnwidth}{lX}
%       \multicolumn{2}{l}{\textit{}}\\
%       
%       \textbf{:} & \\
%    \end{tabularx}
% \end{rpg-quotebox}
Calion miał do wyboru dobić potwora lub ustabilizować wykrwawiającego się Garreta i narazić się na okazjonalny atak potwora. Jego dobry charakter wybrał za niego:
Calion: Stabilizuję Garreta umiejętnością
Medycyna K20: 1
MG: Dobiłeś go.
Po chwili
MG: Dobra, niech będzie. Już się nie wykrwawia.
% \begin{rpg-quotebox}{}
%    \textit{}\\
%    
%    \begin{tabularx}{\columnwidth}{lX}
%       \multicolumn{2}{l}{\textit{}}\\
%       
%       \textbf{:} & \\
%    \end{tabularx}
% \end{rpg-quotebox}
MG: Garret wspina się po drabinie niczym rącza łania.
% \begin{rpg-quotebox}{}
%    \textit{}\\
%    
%    \begin{tabularx}{\columnwidth}{lX}
%       \multicolumn{2}{l}{\textit{}}\\
%       
%       \textbf{:} & \\
%    \end{tabularx}
% \end{rpg-quotebox}
Vivien (OOC do wkurzających ją NPC-ów): już dawno zrobiłabym wam z kutasów pnącza
Garret: Pnącia! Kutaliany!
% \begin{rpg-quotebox}{}
%    \textit{}\\
%    
%    \begin{tabularx}{\columnwidth}{lX}
%       \multicolumn{2}{l}{\textit{}}\\
%       
%       \textbf{:} & \\
%    \end{tabularx}
% \end{rpg-quotebox}
Przewidywana obserwacja straży miejskiej, która puściła drużynę wolno, ale obiecała, że nie spuści z nich oka: Ta banda wkradła się do kanałów, przedarła się przez ich śmierdzące wnętrza, pobiła potwory z pnączy tylko po to by włamać się do spalonej kamienicy i urządzić sobie w niej melinę! Nie. Oni nie stanowią zagrożenia dla miasta.
% \begin{rpg-quotebox}{}
%    \textit{}\\
%    
%    \begin{tabularx}{\columnwidth}{lX}
%       \multicolumn{2}{l}{\textit{}}\\
%       
%       \textbf{:} & \\
%    \end{tabularx}
% \end{rpg-quotebox}



\section*{9 grudnia 2017}

% \begin{rpg-quotebox}{}
%    \textit{}\\
%    
%    \begin{tabularx}{\columnwidth}{lX}
%       \multicolumn{2}{l}{\textit{}}\\
%       
%       \textbf{:} & \\
%    \end{tabularx}
% \end{rpg-quotebox}
Vivien: Nie wchodzę w HTH bo Calion coś kombinuje.

% \begin{rpg-quotebox}{}
%    \textit{}\\
%    
%    \begin{tabularx}{\columnwidth}{lX}
%       \multicolumn{2}{l}{\textit{}}\\
%       
%       \textbf{:} & \\
%    \end{tabularx}
% \end{rpg-quotebox}
Waląc kosturem w drzwi pokoju w którym śpi Yeleda
Vivien: To nie jest alfabet morse'a to alfabet desperata.

% \begin{rpg-quotebox}{}
%    \textit{}\\
%    
%    \begin{tabularx}{\columnwidth}{lX}
%       \multicolumn{2}{l}{\textit{}}\\
%       
%       \textbf{:} & \\
%    \end{tabularx}
% \end{rpg-quotebox}
MG: Słyszycie stukanie.
Yeleda: Ukrywam się!
Calion: Ukrywam się!
Vivien: Ukrywam się! 
Garret: Ukrywam się!

% \begin{rpg-quotebox}{}
%    \textit{}\\
%    
%    \begin{tabularx}{\columnwidth}{lX}
%       \multicolumn{2}{l}{\textit{}}\\
%       
%       \textbf{:} & \\
%    \end{tabularx}
% \end{rpg-quotebox}
MG: Rzuć na trafienie.
Calion: Lekko pół chujowo.
MG: Nie trafiasz...

% \begin{rpg-quotebox}{}
%    \textit{}\\
%    
%    \begin{tabularx}{\columnwidth}{lX}
%       \multicolumn{2}{l}{\textit{}}\\
%       
%       \textbf{:} & \\
%    \end{tabularx}
% \end{rpg-quotebox}
Calion: Kruk wkurwia.
Garret: To on jeszcze żyje?

% \begin{rpg-quotebox}{}
%    \textit{}\\
%    
%    \begin{tabularx}{\columnwidth}{lX}
%       \multicolumn{2}{l}{\textit{}}\\
%       
%       \textbf{:} & \\
%    \end{tabularx}
% \end{rpg-quotebox}
Vivien: Ten druid chyba żałuje, że Cię zaatakował kijem.
Garret: Zaraz mu go wsadzę... w oczodół.

% \begin{rpg-quotebox}{}
%    \textit{}\\
%    
%    \begin{tabularx}{\columnwidth}{lX}
%       \multicolumn{2}{l}{\textit{}}\\
%       
%       \textbf{:} & \\
%    \end{tabularx}
% \end{rpg-quotebox}
Drużyna błądziła w kanałach i Garret postanowił wybrać kierunek na ślepy traf.
MG: Czujesz bryzę.
Garret: Bryzę czy bryzg?

% \begin{rpg-quotebox}{}
%    \textit{}\\
%    
%    \begin{tabularx}{\columnwidth}{lX}
%       \multicolumn{2}{l}{\textit{}}\\
%       
%       \textbf{:} & \\
%    \end{tabularx}
% \end{rpg-quotebox}
Calion: Viv ile ty masz lat?
Vivien: 190 z hakiem.
Garret: Och! To już jesteś pełnoletnia.

% \begin{rpg-quotebox}{}
%    \textit{}\\
%    
%    \begin{tabularx}{\columnwidth}{lX}
%       \multicolumn{2}{l}{\textit{}}\\
%       
%       \textbf{:} & \\
%    \end{tabularx}
% \end{rpg-quotebox}
MG: Ptak skacząc po okolicy nadziewa się na ukrytą Vivien.
Vivien: Wszedł we mnie?

% \begin{rpg-quotebox}{}
%    \textit{}\\
%    
%    \begin{tabularx}{\columnwidth}{lX}
%       \multicolumn{2}{l}{\textit{}}\\
%       
%       \textbf{:} & \\
%    \end{tabularx}
% \end{rpg-quotebox}
MG: Garret w wyniku szarpnięcia spada z klifu i wisi tylko na rękach wzdłuż urwiska.
Vivien: Odgrywam scenę z Króla Lwa.
Calion, Yeleda, Garret: Nie!

% \begin{rpg-quotebox}{}
%    \textit{}\\
%    
%    \begin{tabularx}{\columnwidth}{lX}
%       \multicolumn{2}{l}{\textit{}}\\
%       
%       \textbf{:} & \\
%    \end{tabularx}
% \end{rpg-quotebox}
MG: Widzicie dużo ptaków przelatujących obok waszej kryjówki
Calion: Ja się stresuję jako gracz.

% \begin{rpg-quotebox}{}
%    \textit{}\\
%    
%    \begin{tabularx}{\columnwidth}{lX}
%       \multicolumn{2}{l}{\textit{}}\\
%       
%       \textbf{:} & \\
%    \end{tabularx}
% \end{rpg-quotebox}
Calion: Czy widzę coś innego iż mewy?
MG: Orła.

% \begin{rpg-quotebox}{}
%    \textit{}\\
%    
%    \begin{tabularx}{\columnwidth}{lX}
%       \multicolumn{2}{l}{\textit{}}\\
%       
%       \textbf{:} & \\
%    \end{tabularx}
% \end{rpg-quotebox}
MG: Gryfy mają bardzo dobry wzrok.
Vivien: Sokoli?
MG: Gryfi.

% \begin{rpg-quotebox}{}
%    \textit{}\\
%    
%    \begin{tabularx}{\columnwidth}{lX}
%       \multicolumn{2}{l}{\textit{}}\\
%       
%       \textbf{:} & \\
%    \end{tabularx}
% \end{rpg-quotebox}
Garret: To będzie dziwne, że nagle w środku nocy do gospody wchodzi grupa ludzi.
Vivien: A co? Nikt normalny nie śpi za dnia w gnieździe gryfów?

% \begin{rpg-quotebox}{}
%    \textit{}\\
%    
%    \begin{tabularx}{\columnwidth}{lX}
%       \multicolumn{2}{l}{\textit{}}\\
%       
%       \textbf{:} & \\
%    \end{tabularx}
% \end{rpg-quotebox}
Garret: Ciekawe jakie owoce są w Feywild? Np. jakie jagody?
Vivien: Takie, że jak nie wyplujesz pestki to umrzesz.
Garret: A jak wyjdą inną stroną?
Vivien: Jagody, nie czopki!

\section*{5 stycznia 2018}

\begin{rpg-quotebox}{Tyle możliwości}
   \begin{tabularx}{\columnwidth}{lX}
      \textbf{MG:} Wchodzisz do korytarza. Widzisz dwóch mężczyzn i 10 drzwi.\\
   \end{tabularx}
\end{rpg-quotebox}

\begin{rpg-quotebox}{O tym jak niewielki wzrost jednak jest do czegoś przydatny}
   \begin{tabularx}{\columnwidth}{lX}
      \textbf{MG:} Nadlatuje w twoim kierunku 15 małych strzałek, ale jesteś niski i unikasz obrażeń.\\
      \textbf{Vivien:} Jesteś poniżej linii strzału.\\
      \textbf{MG:} Ale powyżej ST.\\
   \end{tabularx}
\end{rpg-quotebox}

\begin{rpg-quotebox}{No i cały misterny plan...}
   \begin{tabularx}{\columnwidth}{lX}
      \textbf{MG:} Yeleda, Ty nawigujesz.\\
      \textbf{K20:} 2\\
      \textbf{Yeleda:} Do dupy na raki my poszli.\\
   \end{tabularx}
\end{rpg-quotebox}

\begin{rpg-quotebox}{O konsekwencjach porażki}
   \textit{Calion podbiegł do uzbrojonego, wodnego żywiołaka z wyczarowaną magiczną bronią, ale jego atak nie trafił}\\
   
   \begin{tabularx}{\columnwidth}{lX}
      \textbf{MG:} Czy coś jeszcze robisz?\\
      \textbf{Calion:} Wspominam całe swoje życie.\\
   \end{tabularx}
\end{rpg-quotebox}

\begin{rpg-quotebox}{O liczbach, które przeraziły}
   \textit{Yeleda zadała żywiołakowi strzałem z łuku 11 punktów obrażeń.}\\

   \begin{tabularx}{\columnwidth}{lX}
      \textbf{MG:} Brawo! Żywiołak zszedł do liczby dwucyfrowej.\\
   \end{tabularx}
\end{rpg-quotebox}

\begin{rpg-quotebox}{O braku empatii uzdrowicieli}
   \begin{tabularx}{\columnwidth}{lX}
      \textbf{Garret:} Regeneruję sobie 18 punktów życia.\\
      \textbf{Vivien:} Jednego do leczenia mniej.\\
   \end{tabularx}
\end{rpg-quotebox}

\begin{rpg-quotebox}{Z innej perspektywy już nie taki niski}
   \begin{tabularx}{\columnwidth}{lX}
      \textit{Leżąc na ziemi:}\\
      \textbf{Yeleda:} Wstaję i ukrywam się.\\
      \textbf{MG:} W tych warunkach jesteś się w stanie schować za Garretem.\\
   \end{tabularx}
\end{rpg-quotebox}

\begin{rpg-quotebox}{O krępującej sytuacji}
   \begin{tabularx}{\columnwidth}{lX}
      \textbf{Garret:} Związuję go.\\
      \textbf{Yeleda:} Przyglądam się mu.\\
      \textit{Vivien rzuca Guidance na Yeledę.}\\
      \textbf{Calion:} Mamy związanego faceta i dwie laski, które się macają.\\
   \end{tabularx}
\end{rpg-quotebox}

\begin{rpg-quotebox}{O wielkich powodach do obaw}
   \begin{tabularx}{\columnwidth}{lX}
      \textbf{Garret:} Skoro nie chce mówić, trzeba go będzie utopić w kałuży.\\
      \textbf{MG:} Rzuć na zastraszenie.\\
      \textbf{K20:} 20\\
   \end{tabularx}
\end{rpg-quotebox}

\begin{rpg-quotebox}{Zgonospis}
   \begin{tabularx}{\columnwidth}{lX}
      \textit{Do BN-ego:}\\
      \textbf{Yeleda:} Jak chcesz umrzeć? Sztylet, miecz, łuk?\\
      \textbf{MG:} Taaak... wyciągnij menu.\\
   \end{tabularx}
\end{rpg-quotebox}

\begin{rpg-quotebox}{O niskim szacunku\, jakim darzy się niziołków}
   \begin{tabularx}{\columnwidth}{lX}
      \textbf{Yeleda:} Ja zmienię wizerunek, ale wy też powinniście się byli przebrać. Ale co zrobić z niziołkiem?\\
      \textbf{Vivien:} Przebierzemy go za dziecko, wsadzimy do wózka i tyle.\\
   \end{tabularx}
\end{rpg-quotebox}




% \begin{rpg-quotebox}{}
%    \textit{}\\
%    
%    \begin{tabularx}{\columnwidth}{lX}
%       \multicolumn{2}{l}{\textit{}}\\
%       
%       \textbf{:} & \\
%    \end{tabularx}
% \end{rpg-quotebox}








% \begin{rpg-paperbox}{rpg-paperbox}
% 	you can add some text here
% \end{rpg-paperbox}
% 
% %\newpage % Acts as columbreak because of twocolumn option; for pagebreak use \clearpage
% 
% \section{Some tables}
% \header{default rpg-table (2 column)}
% \begin{rpg-table}
%    	\textbf{Table head 1}  & \textbf{Table head 2} \\
%    	Some value  & Some value \\
%    	Some value  & Some value \\
%    	Some value  & Some value
% \end{rpg-table}
% 
% % For more columns, you can say \begin{rpg-table}[your options here].
% % For instance, if you wanted three columns, you could say
% % \begin{rpg-table}[XXX]. The usual host of tabular parameters are
% % aailable as well.
% \header{rpg-table with more columns}
% \begin{rpg-table}[XXX]
%     \textbf{Table head 1}  & \textbf{Table head 2} & \textbf{Table head 3}\\
%    	Some value  & Some value & Some value\\
%    	Some value  & Some value & Some value\\
%    	Some value  & Some value & Some value
% \end{rpg-table}
% 
% \header{default rpg-table2 (2 column)}
% \begin{rpg-table2}
%    	\textbf{Table head 1}  & \textbf{Table head 2} \\
%    	Some value  & Some value \\
%    	Some value  & Some value \\
%    	Some value  & Some value
% \end{rpg-table2}
% 
% 
% \section{List}
% \begin{rpg-list}
%     \item first list item
%     \item second list item
% \end{rpg-list}
% 
% \section{Monster}
% % You can optionally not include the background by saying
% %\begin{rpg-monsterboxnobg}{monsterboxnob}
% \begin{rpg-monsterbox}{rpg-monsterbox}
% 	\textit{Small metasyntatic variable (golbinoid), neutral evil}\\
% 	\rpghline
% 	\basics[%
% 	armorclass = 12,
% 	hitpoints  = 16 (3d8 + 3),
% 	speed      = 50 t
% 	]
% 	\rpghline
% 	\stats[ % This stat command will autocomplete the modifier for you
%     STR = 12, 
%     DEX = 7
% 	]
% 	\rpghline
% 	\details[%
% 	% If you want to use commas in these sections, enclose the
% 	% description in braces.
% 	% I'm so sorry.
% 	languages = {Common Lisp, Erlang},
% 	]
% 	\rpghline \\[1mm]
% 	\begin{rpg-monsteraction}[rpg-monsteraction]
% 		This Monster has some serious superpowers!
% 	\end{rpg-monsteraction}
% 
% 	\rpgmonstersection{rpgmonstersection}
% 	\begin{rpg-monsteraction}[rpm-monsteraction]
% 		This one can generate tremendous amounts of text! Though only when it wants to.
% 	\end{rpg-monsteraction}
% 
% 	\begin{rpg-monsteraction}[rpg-monsteraction]
%     See, here he goes again! Yet more text.
% 	\end{rpg-monsteraction}
% \end{rpg-monsterbox}
% 
% \chapter{Chapter name}
% 
% % End document
\end{document}
