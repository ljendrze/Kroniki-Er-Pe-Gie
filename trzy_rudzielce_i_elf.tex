\documentclass[10pt,twoside,twocolumn]{book}
\usepackage[bg-letter]{lib/rpg-book} % Options: bg-a4, bg-letter, bg-full, bg-print, bg-none.
\usepackage[polish]{babel}
\usepackage[utf8]{inputenc}
\usepackage{hyperref}
\usepackage{multicol}
\usepackage{multirow}
\usepackage{lipsum}
\usepackage{tabularx}

\title{Trzy Rudzielce i Elf}
\date{\today}
\author{The Dungeon Mistress, Vivien, Yeleda, Calion, Garret}

% Start document
\begin{document}
\fontfamily{ppl}\selectfont % Set text font
\frontmatter

\maketitle
\begin{multicols}{2}
\tableofcontents
\end{multicols}

% Your content goes here
\mainmatter

\chapter{Ildrim - kontynent imperium}

\paragraph{}
Ciężko opisać kontynent jako całość, gdy nawet najbardziej odważni i potężni podróżnicy rzadko opuszczają jego wybrzeża.
Gdy nawet potężni mistrzowie magii z trudem sięgają umysłem poza jego granice. 
Wyobraźmy sobie masę lądu otoczoną ze wszystkich stron oceanem.
I to nie takim zwykłym, oceanem kierowanym starożytną magią z Wysokich Planów.
Oceanem chcącym pożreć, pochłonąć nieszczęsny ląd.
Taki jest nasz Ildrim, zwany też Kontynentem Imperium.
Jeśli nic się nie zmieni, bądź zmieni się zbyt wiele, zostaniemy wszyscy pogrążeni pod wodą, skała i magią.
Ponieważ Ildrimem władają moce poza naszą kontrolą.

\paragraph{}
Śmiercionośny ocean wpływa na naszą egzystencję w jeszcze jeden sposób: niemal kompletnie odcina nas od innych cywilizacji.
O ile pas wody w pobliżu wybrzeży w większości jest stosunkowo spokojny, głębokie wody są niemal zupełnie niedostępne.
Jedynie druidzi z \hyperref[Vicovaro]{Vicovaro}, elfowie z Vorath i „wolni żeglarze” z Wysp Kye są w stanie próbować swoich sił z żywiołem.

\paragraph{}
Kontynent zatem jest całym światem dla praktycznie wszystkich jego mieszkańców, zapewne z  wyjątkiem najprostszych rolników, których pojęcie obejmuje co najwyżej sąsiednie miasto.
I nie powinno dziwić, że przez milenia Ildrim zaznał niewiele spokoju.

\paragraph{}
Większość kontynentu zajmują \hyperref[SojuszniczeKrolestwa]{Sojusznicze Królestwa}: od górskich twierdz \hyperref[AnGammarna]{An’Gammarny} do spustoszonych iglic Yunait.
To te siedem państw jest odpowiedzialnych za ostatnie wojny i niepokoje na kontynencie.
Jednak po stuleciach walk \hyperref[SojuszniczeKrolestwa]{Sojusznicze Królestwa} utrzymywane są w pokoju przez \hyperref[GildiaKupcow]{Gildię Kupców}.
Z punktu widzenia \hyperref[GildiaKupcow]{Gildii} pokój zostanie zachowany.
Niezależnie do ceny.
Państwa te pozostają, przynajmniej we własnym mniemaniu, ostoją cywilizacji i dziedzicami Imperium.
Poza elfim państwem-miastem Vorath są jedyną cywilizacją na powierzchni Planu Materialnego nie pozostająca w ukryciu.

\paragraph{}
Prawda, jak to ma w zwyczaju, jest nieco bardziej skomplikowana.
Jeśli ktoś zdoła przebyć zdradzieckie północne góry \hyperref[AnGammarna]{An’Gammarny} i \hyperref[Vicovaro]{Vicovaro} dotrze do mroźnych pustkowi.
Tam na pokrytych wiecznym lodowcem wybrzeżach znajduje się owiane legendami państwo.
Niewielu wie o nim cokolwiek pewnego, chociaż większość dzieci w \hyperref[SojuszniczeKrolestwa]{Sojuszniczych Królestwach} słyszało przed snem baśnie o magicznych miastach pośród lodu.
Jeszcze dalej na zachód znajduje się sporych rozmiarów wyspa, ukryta pośród mgieł.
Ta wyspa to Vorath, ostatnie miasto elfów na Planie Materialnym, a w każdym razie ostatnie na tym kontynencie.

\paragraph{}
Od południowo-zachodniej strony Ildrimu znajduje się Platynowa Zatoka.
Stanowi ona wybrzeże \hyperref[AnGammarna]{An’Gammarny}, Croridu oraz Eneadoru.
Wody zatoki, są wyjątkowo zdradliwe, nawet jak na warunki kontynentu.
Powodem jest Latająca Wyspa Thor Ar Muria, gdzie znajduje się największa na kontynencie akademia magii arkanicznej.
Na obszarze Platynowej Zatoki znajdują się żeglowne przesmyki, dostępne dla doświadczonych śmiałków, jednak większość wypełniona jest wirami i wiatrami o nieprzewidywalnym, magicznym źródle.

\paragraph{}
Na wschodzie znajduje się niezbadana i niebezpieczna kraina zwana Pustkowiem Kataklizmu.
Nieliczne jego fragmenty są zamieszkane przez pionierów z Domen Krasnoludów.
Choć niewiele osób o tym wie, w północno-zachodniej części Pustkowia znajduje się Przystań: miasto uciekinierów z Podmroku.
Dawni mieszkańcy tego miasta przemierzają także te nieprzyjazne tereny jako nomadzi.
Niektórzy twierdzą, że wschodnie wybrzeża Ildrimu otaczają spokoje wody, którymi można przepłynąć do innych kontynentów. Nikomu jednak nie udało się do nich dotrzeć.

\paragraph{}
Nie są to jednak jedynie tereny na lądzie dotknięte niszczącą magią.
Na wschód od \hyperref[Vicovaro]{Vicovaro} i w \hyperref[AnGammarna]{An’Gammarnie} znajdują się ruiny dawnego Imperium.
Jakiekolwiek magiczne eksperymenty przeprowadzali ich mieszkańcy, ich moc przesiąknęła kamienie, ziemię i drzewa.
W samej \hyperref[AnGammarna]{An’Gammarnie} klerycy i paladyni zdołali opanować nienaturalny żywioł, ale Pomiędzy \hyperref[AnGammarna]{An’Gammarną}, a \hyperref[Vicovaro]{Vicovaro} pozostaje puszcza, kompletnie niestabilna i nienadająca się do zamieszkania zwana Doliną Konwalii.

\paragraph{}
Jest wiele teorii skąd wzięła się aż tak duża magiczna niestabilność występująca na kontynencie, a także czemu jest jedno miejsce praktycznie kompletnie od nich wolne: leżący na zachodzie archipelag Wysp Kye.
Wyspy stanowią przytań dla wielu, którzy przystani poszukują, ale niekoniecznie na nią zasługują.
Zamieszkujący je „wolni żeglarze” prowadzą nieskończone spory z Gildią Kupców chcących dostępu do spokojnych wód i dóbr archipelagu.
Z różnymi skutkami.
Wśród plotek i legend związanych z wyspami znajdują się labirynty podziemi wypełnionych skarbami i starożytnymi tajemnicami.
Niestety niewielu badaczy miało okazję poznać te wypełnione piratami i najemnikami wyspy.

\paragraph{}
Zanim powstały aktualne państwa większość północnej i zachodniej części kontynentu zajmowało Imperium, powszechnie uznawana za państwo elfów zanim opuściły one Plan Materialny.
Imperium najprawdopodobniej zostało zniszczone przez pierwszy na kontynencie magiczny kataklizm.
Możliwe nawet, że go wywołało, doprowadzając Ildrim do tego czym jest teraz.
Praktycznie nie ma organizacji, która obejmowałaby swoim zasięgiem cały kontynent.
Pewne powiązania ma tak zwana \hyperref[KonfrateriaMagow]{Konfrateria Magów}.
Nie jest to formalna organizacja, a raczej ogólne określenie na wszystkich formalnie wyszkolonych magów arkanicznych.
W ramach \hyperref[KonfrateriaMagow]{Konfraterii} funkcjonują Strażnicy Arkanum, stosunkowo niewielka grupa rozproszona po całym Ildrimie.
Tutaj można już mówić o zorganizowaniu, ale dotyczy to stosunkowo niewielkiej, dobrze ukrytej grupy magów.
Większość członków \hyperref[KonfrateriaMagow]{Konfraterii} nie ma pojęcia o ich istnieniu.
Na podobnych zasadach można traktować Kręgi Druidów -  o ile poszczególne Kręgi są rozproszone po całym kontynencie, to nie stanowią jednolitej organizacji.

\label{SojuszniczeKrolestwa}
\chapter{Sojusznicze Królestwa}

\paragraph{}
Można by napisać książkę, a nawet tuzin książek o różnych teoriach co do powstania państw, które stworzyły Sojusznicze Królestwa.
Jedyne co można z pewnością napisać to, że Imperium upadło i na kontynencie powstał chaos.
Na północy elfy zaczęły walczyć z ludźmi i ich sojusznikami, co zakończyło się zwycięstwem ludzi i opuszczeniem Planu Materialnego przez elfy.
Państwa na południu nagle straciły groźnego przeciwnika.

\paragraph{}
Ci mieszkańcy kontynentu którzy nie odsunęli się w trudnodostępne regiony i plany utworzyli siedem państw.
Od północy na południe są to: \hyperref[Vicovaro]{Vicovaro}, \hyperref[AnGammarna]{An’Gammarna}, Crorid, Sothedora, Eneador, Chacan i Yunait.
Eneador, Chacan i Yunait nigdy nie były częścią Imperium, chociaż Eneador był z nim w sojuszu od początku swojego istnienia, by uniknąć podboju (formalnie był ówcześnie częścią Domen Krasnoludów).
Pozostałe państwa powstały na ruinach po początkowym chaosie, w szczególności \hyperref[AnGammarna]{An’Gammarna} i Crorid.

\paragraph{}
O ile obecnie królestwa są mieszaniną różnych ras (z wyjątkiem Yunait, które pozostaje niemal całkowicie zamieszkane przez drakonów) pierwotnie były dużo bardziej podzielone.
\hyperref[Vicovaro]{Vicovaro} było państwem niziołków, \hyperref[AnGammarna]{An'Gammarna} i Sothedora były państwami ludzi z Imperium, Chacan zaś niezalezną cywilizacją ludzi spoza Imperium.
Crorid i Eneador były koloniami krasnoludów i gnomów, przy czym Crorid został wcześniej całkowicie podbity przez Imperium.

\paragraph{}
Mniejsze lub większe potyczki między królestwami rozpoczęły się niemal od razu po upadku Imperium.
Trwały stulecia, ograniczają rozwój i handel.
Żadne z państw nie mogło uzyskać dłuższej przewagi.
Wojny zapewne trwałyby nadal gdyby nie drugi kataklizm.
Nie ma pewności czy był echem zniszczenia Imperium, czy czymś zupełnie nowym.
Pewne jest to, że magiczna eksplozja wstrząsnęła pustkowiami przy wschodniej granicy Yunait, niemal całkowicie je niszcząc. 
Magia rozproszyła się po Ildrimie jak kręgi na wodzie.
W końcu dotarła do imperialnych ruin w Dolinie Konwalii i \hyperref[AnGammarna]{An’Gammarnie}.
Magia zawsze była w nich intensywna, ale to sprawiło, że rozpętała się magiczna burza.
I kompletny chaos.
Kiedy pierwsze skutki się uspokoiły nastąpiły te bardziej długotrwałe.
Cały kontynent objęła katastrofalna susza.
Powoli zdano sobie sprawę, że królestwa nie przetrwają pojedynczo i współpraca jest jedyną szansą na przeżycie.
Tak powstał Sojusz.

\paragraph{}
Wraz z Sojuszem powstała \hyperref[GildiaKupcow]{Gildia Kupców}.
Ta prosta nazwa nie oddaje ogromu organizacji która przez lata oplotła Królestwa siecią kontaktów, szpiegów i intryg.
I która praktycznie rządzi we wszystkich siedmiu państwach.
Załagodzenie kataklizmu i jego efektów zajęło lata, ale przyniosło dodatkowe profity: lepszą komunikację, szlaki handlowe, a co za tym idzie pieniądze.
Nie chcąc tracić zysków \hyperref[GildiaKupcow]{Gildia} skupiła się na swoim nowym najważniejszym celu - utrzymać pokój w Sojuszniczych Królestwach za wszelką cenę.

\paragraph{}
Przez stulecia pokój został utrzymany.
Jednak silne oddziaływanie \hyperref[GildiaKupcow]{Gildii} sprawia, że sytuacja polityczna staje się coraz bardziej niestabilna, a wysiłki \hyperref[GildiaKupcow]{Gildii} by utrzymać status quo coraz bardziej brutalne.
Obecnie \hyperref[Vicovaro]{Vicovaro} jest niemal całkowicie kontrolowane przez \hyperref[GildiaKupcow]{Gildię} (pomijajac atak przez wrogo nastawioną grupę druidów).
Udało im się także się w dużym stopniu zdestabilizować władzę w Croridzie.
Pozostałe państwa zachowuja względnie niezależną władzę, ale to także zaczyna sie powoli zmieniać.

\chapter{An'Gamarrna}

\paragraph{}
Państwo ludzi stworzone na ruinach dawnego Imperium, obecnie rządzone przez teokratyczny kult Bahamuta, do którego musi należeć nawet król.
Pańśtwo jest częścią Sojuszniczych Królestw.

\section{Geografia}
\paragraph{}
An'Gamarrna obejmuje zatokę pomiędzy Doliną Konwalii a Croridem i Eneaorem.
Platynowa zatoka nie umożliwia ze względu na niestabilne prądy przez istnienie latającej wyspy Muria, co powoduje bardzo napięte stosunki między An'Gamarrną a Croridem.
Obecną stolicą państwa jest święte miasto Glenrowan.
Innymi dużymi ośrodkami są dawna stolica Imperium, twierdza Bleihar (znajdująca się w górach na północy) oraz port Thelassa (około dzień drogi od Glenrowan nad brzegiem Platynowej Zatoki).
Klimat An'Gamarrny jest umiarkowany, z okazjolanymi przebłyskami niestabilnej pogody w miejscach gdzie kapłani nie kontrolują dobrze granic między planami.

\section{Polityka}
\paragraph{}
An'Gammarna rządzona jest przez króla, który aby odziedziczyć swoje królestwo musi zostać paladynem Bahamuta.
Obecnym królem jest Lorne III, który wziął za małżonkę Zephę, arcykapłankę Bahamuta.
Zepha jest młodszą od nieo o kilkanaście lat półelfką, dała mu dwóch synów.
Wiele osób uważa, że to ona naprawdę rządzi królestwem.
Wszelkie kulty czczące dobrych i neutralnych bogów cieszą się dużym poważaniem, kulty chaotyczne mogą być natomiast traktowane z pełną rezerwą.

\section{Historia}
\paragraph{}
An'Gammarna wg legend została założona przez zbuntowanych niewolników podczas schyłku Imperium.
Niewolnicy pierwotnie przejęli twierdzę Bleihar, a później objawił im się platynowy smok i polecił stworzyć święte miasto Glenrowan.
Ponieważ teren współczesnej An'Gammarny, poza twierdzą Bleihar, znajdował się ciągle w rękach elfów - wybuchła wojna.
Zakończyła się ona zwycięstwem ludzi i zniszczeniem elfiej stolicy Eleander.

\label{Vicovaro}
\chapter{Vicovaro}

\paragraph{}
Państwo niziołków stworzone w miejscu kręgu druidzkiego specjalizującego się z kontroli pogody oraz magii wody. 
Przez \hyperref[GildiaKupcow]{Vicovaro} przepływa większość handlu zewnętrznego w \hyperref[SojuszniczeKrolestwa]{Królestwach} i jest półoficjalną siedzibą \hyperref[GildiaKupcow]{Gildii Kupców}.
Państwo jest częścią \hyperref[SojuszniczeKrolestwa]{Sojuszniczych Królestw}.

\section{Geografia}
\paragraph{}
Vicovaro jest najbardziej na zachód wysuniętym państwem \hyperref[SojuszniczeKrolestwa]{Sojuszniczych Królestw}.
Od najbliższego wschodniego sąsiada, \hyperref[AnGammarna]{An'Gammarny}, oddzielona jest Doliną Konwalii. 
Poza miastem portowym Vicovaro w państwie nie było żadnych innych miast (co formalnie czyniło z Vicovaro miasto-państwo), do czasu kiedy 200 lat temu \hyperref[KultIoun]{kult Ioun} wybudował Sjorden. 
W Vicovaro znajduje się największy port w \hyperref[SojuszniczeKrolestwa]{Królestwach}, w którym druidzi utrzymują permanentnie korzystną dla żeglarzy pogodę.

\section{Polityka}
\paragraph{}
Miastem rządzi diuk z klanu niziołków Olkyn, potomek Adana Olkyna, oryginalnego założyciela miasta Vicovaro. 
Aktualny diuk Dregras Olkyn jest jedynie marionetką, faktyczna władza znajduje się w rękach Arcydruidki Eryn oraz Thundy, przywódcy \hyperref[GildiaKupcow]{Gildii Kupców}.
Oboje nie przepadają za sobą nawzajem, ale prowadzą niechętną współpracę. 
Krąg druidzki pozostaje niezmiennie znacząca siłą polityczną w Vicovaro, ze względu na zależność miasta od ich magii. 
Te powiązania sprawiają, że krąg Vicovaro jest traktowany z pewną wrogością przez inne kręgi.

\section{Historia}
\paragraph{}
Zanim powstało Vicovaro na wybrzeżu znajdował się krąg druidzki. 
Wiele stuleci temu przywódca kręgu: druid Adan ściągnął do nadmorskiego siedliska kręgu cały swój klan. 
Klan stworzył port handlowy Vicovaro. 
Stopniowo coraz więcej niziołków  z okolicznych dolin zaczęło ściągać do portu, miasto zaczęło się rozrastać i zyskiwać na znaczeniu.
Port stał się łakomym kąskiem dla okolicznych państw ze względu na bogactwo i strategiczną lokalizację: konflikty wybuchały głównie z \hyperref[AnGammarna]{An'Gammarną} i Wolnymi Wyspami Kye. 
Vicovaro zawsze jednak dało radę odeprzeć napastników dzięki swojej silnej flocie i magii druidzkej. 
Konflikt z \hyperref[AnGammarna]{An'Gammarna} udało się w znacznym stopniu załagodzić po tym jak utworzono \hyperref[SojuszniczeKrolestwa]{Sojusznicze Królestwa} i udostępniono ziemię na budowę Sjorden \hyperref[KultIoun]{kultowi Ioun}.
Ataki ze strony Wysp Kye ciągle się zdarzają, ale słabo zorganizowane wyspy nie maja szans ze współczesny Vicovaro.

\chapter{Organizacje}

\label{GildiaKupcow}
\section{Gildia Kupców}
\paragraph{}
Ściśle powiązana gigantyczna siatka kupców bankierów.
Gildia Kupców jest, rzecz jasna, najpotężniejszą organizacją obejmującą swoim zasięgiem \hyperref[SojuszniczeKrolestwa]{Sojusznicze Królestwa}.
Każdy kto chce zajmować się handlem na dłuższą metę musi zostac członkiem Gildii, albo spotkaja go “nieprzyjemnie konsekwencje” wszelakiego rodzaju.
Gildia (nieco mniej oficjalnie) zajmuje się także polityką i dyplomacją.
W praktyce to oni utrzymują pokój w królestwach dzięki gigantycznym wpływom na bardzo wysokich szczeblach oraz, oczywiście, gigantycznym pieniądzom.
Tradycyjną siedzibą Gildii jest \hyperref[Vicovaro]{Vicovaro}, największy port hyperref[SojuszniczeKrolestwa]{Sojuszniczych Królestw} i ich jedyne połaczenie z wyspami Kye (i dalej).

\label{KonfraterieMagow}
\section[Konfraterie Magów]{Konfraterie\\Magów}
\paragraph{}
Bardzo luźne zgrupowania magów w różnych miastach królestw.
Większość magów szkoliło się w jednym z ośrodków Konfraterii, ponieważ daje to dostęp do bibliotek, miejsc do bezpiecznych eksperymentów oraz potencjalnych nauczycieli.
Większość magów z dumą deklaruje przynależność, do którejś z konfraterii, jako dowód, że mają porządne wykształcenie.
Najbardziej prestiżową konfraterią jest Latająca Wyspa Muria, lewitująca nad wybrzeżem Croridu.

\label{Varjossa}
\section{Varjossa}
\paragraph{}
Niewielu poza wysoko postawionymi członkami \hyperref[GildiaKupcow]{Gildii} wie o innej organizacji mającej koneksje w całych \hyperref[SojuczniczeKrolestwa]{Królestwach}, a nawet poza nimi: Varjossa, zwana nieco błędnie Gildią Zabójców.
Gdziekolwiek podróżnicy stykają się z gildiami zabójców, złodziei itp. prawie na pewno stoi za tym Varjossa.
Nawet jeśli członkowie najniższej rangi sami o tym nie wiedzą.
Wyżej postawionych członków mozna rozpoznać po tatuażu w kształcie oka, najczęściej z tyłu szyi.
Varjossa ściśle współpracuje z \hyperref[GildiaKupcow]{Gildią Kupców}, ale obie organizacje podkreślaja swoją niezależność.

\label{KultIoun}
\section{Sekretny Kult Ioun}
\paragraph{}
O ile wiele mniejszych i większych organizacji religijnych działa w \hyperref[SojuszniczeKrolestwa]{Królestwach} na uwagę zasługuje Sekretny Kult Ioun.
Stanowi on ,,dyskretne'' ramie instytucji religijnych w \hyperref[SojuszniczeKrolestwa]{Królestwach}.
Zajmują się zbieraniem i zabezpieczaniem wiedzy, szczególnie tej niebezpiecznej.
Poza tym działają, kiedy potrzeba bardziej dyskretnego podejścia niż wysłanie typowego paladyna.
Posiadają nieco szemraną reputację i niezbyt "przyjacielskich" członków, ale koniec końców, posiadają szacunek większości władz religijnych w \hyperref[SojuszniczeKrolestwa]{Królestwach}.


\onecolumn
\large
\chapter{Opowiadanie}
\section*{Kozie górki}
\paragraph{}
Dzień miał się ku końcowi na szczęście w Kozich Górkach była gospoda i do której uderzyła w pierwszej kolejności.
Zimne piwo to coś na co ma się ochotę po tak ciepłym i długim dniu. 
Poza tym karczmy są kopalniami informacji. Tam nie musisz pytać, by dostać odpowiedź. 
Ktoś prędzej, czy później wpadnie zdyszany oznajmiając że kogoś innego zabrakło. 
Usadowiła się na wolnym miejscu pod ścianą na przeciw wejścia. 
Zdjęła kaptur i kiwnęła na barmana by podał jej piwa.
Gdy dostała napitek rozejrzała się po wnętrzu. 
Pod ścianą siedział księżycowy elf. 
Przymknęła oczy i westchnęła. 
Spod przymkniętych powiek przyjrzała mu się  uważniej. 
Szaty miał czyste i z kosztownego materiału. 
Włosy pięknie splecione opadały na wyprostowane plecy srebrnymi pasmami. 
Wypielęgnowane dłonie obejmowały oszroniony (!) kufel piwa. 
Czyżby mag? 
Cóż robiłby tu w tej zapadłej wiosce? 
Nie przypuszczała by szukał zaginionych. 
Tacy jak on poszukują jedynie potęgi w wiedzy, lub w praktyce zawartej.

\paragraph{}
Drzwi znów skrzypnęły i zdało się, że nikt nie wszedł. 
Jednak po chwili przy barze usiadł niziołek o wzroście około metra we wzorzystych szatach, których kolor zakrywała gruba warstwa kurzu. 
Jednak jego najbardziej charakterystyczną cechą była burza rudych loków. 
Ktoś taki nie schowa się w tłumie, o nie. 
Z twarzy bił blask szczerego uśmiechu, a w oczach czaił się błysk. 
Poszukiwacz przygód. 
I to taki wyposzczony. 
Taki, który dawno niczego nie przeżył. 
Karczmarzowi też się gęba cieszyła. 
Przyjezdni oznaczają zarobek. 
Bo gdzież indziej mieliby się podziać?

\paragraph{}
Dłuższa chwila minęła na zapełnianiu się karczmy. 
Rolnicy schodzili z pól, pasterze z pastwisk. 
Wszyscy ciągnęli do karczmy by odrobinę wytchnąć po pracowitym dniu. 
Stoliki zapełniały się ludźmi, ale rozmowy były prowadzone po cichu. 
Brakowało wesołego gwaru zwyczajowego dla tego typu miejsc. 
Gdy drzwi ponownie się otwarły ktoś z głębi sali gwizdnął donośnie, tak że i Vivien wyraźnie podniosła głowę. 
Nieprzeciętnej urody ludzka kobieta stanęła na środku na szeroko rozstawionych nogach z rękami opartymi na biodrach. 
Zdaje się, że otwarła te nieszczęsne drzwi kopniakiem. 
Rozejrzała się po wnętrzu, a jej mina zdradzała, że zrezygnowała z jakiegoś pomysłu. 
Usiadła przy narożniku baru. 
Jeśli wspomnieć przesądy to akurat ona nie powinna się obawiać staropanieństwa. 
Z drugiej strony jej szelmowski uśmiech sugerował, że małżeństwo to będzie jej ostatni przystanek.
Następny na pewno wejdzie smok. 
Idealnie pasowałby do całej tej dziwnej zbieraniny. 
Intuicja podpowiadała jej, że teraz powinno się było coś wydarzyć. 
Coś co zdoła tą zbieraninę wykorzystać.

\paragraph{}
Nie trzeba było długo czekać. 
Do środka niczym małe tornado wpadł młody chłopak, który w rękach ściskał małą szkatułkę. 
Wzrok miał rozbiegany i cały się trząsł.\newline
\indent - Ludzie! Pomóżcie! Tutaj straszy a ja muszę się stąd wydostać!\newline
Podbiegł do lady i oparł się oń jedną ręką niedaleko tej pięknej ludzkiej kobiety, która przyszła jako ostatnia. 
Niziołek też siedział przy barze i nie pozostał obojętny.\newline
\indent - Co straszy? Powiedz coś więcej.\newline
\indent - Zombie i cienie nie dają ludziom żyć. Tutaj w okolicy jest ich pełno! Nie mogę tu zostać...\newline
Księżycowy elf opróżnił kufel i z elfią gracją podszedł również do baru.
\indent - Odprowadzę cię, tylko powiedz gdzie chcesz się udać.\newline
\indent - Muszę dotrzeć do świątyni Ioun do kleryka, który mnie uczył. Wielu kleryków zaginęło w tych okolicach, nie chcę do nich dołączyć!\newline
Vivien nadstawiła uszu. Czyż nie po to tutaj zawędrowała? Podniosła się z miejsca i również podeszła do baru.
\indent - Jak długo to trwa? Ilu kleryków zaginęło?\newline
\indent - Niedawno, maksymalnie miesiąc. Jednak dużo naszych nie wróciło do świątyni w mieście. Wielu pielgrzymuje po okolicach, ale zawsze wracają. Myślę, że mój nauczyciel będzie więcej wiedzieć. Proszę o pomoc. Potrzebuję eskorty właśnie do miasta. Do Sioden jest około ośmiu godzin marszu stąd.\newline
\indent - Teraz już za późno na podróże, młody człowieku. - powiedział srebrnowłosy elf. - Wyruszymy z samego rana, jeśli pozwolisz.\newline
\indent - T-tak...\newline
\indent - Ja też pójdę. Chętnie dowiem się czegoś na temat tych zniknięć. - szybko wtrąciła Vivien.\newline
\indent - Ja też pójdę. - zgłosił się niziołek.\newline
\indent - I ja. - dopowiedziała ludzka kobieta.\newline
\indent - Mili państwo o świcie wyruszamy! - wykrzyknął i pobiegł na piętro gospody w kierunku gościnnych pokoi.\newline
Vivien patrzyła jeszcze chwilę za młodym chłopakiem gdy usłyszała charakterystyczny aksamitny głos elfiego mężczyzny.
\indent - Skoro mamy podróżować razem pozwólcie, że się przedstawię: jestem Calion.\newline
\indent - Garret Eartopple. - Niziołek podał całe nazwisko.\newline
\indent - Yeleda - dodała kobieta\newline
\indent - Vivien - przedstawiła się też.\newline
Nie zdążyli zamienić więcej słów gdy z wnętrza karczmy podeszła do nich wieśniaczka o dojrzałym wyrazie oczu i pomarszczonej skórze. Jakby miała już jakiś etap życia za sobą.\newline
\indent - Drodzy poszukiwacze przygód wybaczcie, że się wtrącam. Słyszałam, że pójdziecie do Sioden. Tydzień temu wyruszył tam mój syn Baellor i dotychczas nie wrócił. Popytajcie proszę o niego gdy już tam będziecie.\newline
Vivien poczuła ściśnięcie serca. Przerwane więzi rodzinne to coś, czego nie mogła zostawić swemu losowi.\newline
\indent - Dobrze. Sprawdzimy powiedz tylko jak wyglądał?\newline
\indent - Baellor ma już 30 lat. Jest wysoki, szczupły o jasnych włosach i niebieskich oczach.\newline
\indent - Zajmiemy się tym. - odpowiedziała machinalnie choć jeszcze nie wiedziała jak potoczą się ich dalsze losy.\newline
Wieśniaczka wyszła z gospody jak i większość gości. Po tym gwałtownym wydarzeniu wielu odeszła ochota do picia. Usiedli razem przy barze i spojrzeli po sobie.\newline
\indent - Nie. Nie rozumiem. Jak na tym pustkowiu może ktoś w ogóle zaginąć? Tu nic nie ma. - pierwsza odezwała się Yeleda.\newline
\indent - Przepraszam na chwilę, ale muszę rozejrzeć się po okolicy. - powiedział Calion i podniósł się.\newline
\indent - Pójdę z tobą. - Jeśli wierzyć pogłoskom nie powinno się chodzić po zmroku samemu.\newline
Wyszli przed budynek i pierwsze co rzuciło im się w oczy to to, że wieś była kompletnie opustoszała. Nikt nie przechadzał się między domkami. Noc była ciepła, a nikt nie siedział na ławkach ciesząc się pięknem wieczoru.
\indent - Jak tu pusto. - zauważyła.\newline
\indent - Mało kto pali światło. Dziwne, jakby się czegoś obawiali.\newline
\indent - Choć kusi mnie zbadanie obecności intruzów w dolinie wydaje mi się, że to nie najmądrzejszy pomysł. Nie uważasz?\newline
\indent - Cienie i zombie nie biorą się znikąd. To nie są istoty które zamieszkują jakieś okolice. One się pojawiają i zazwyczaj wiąże się to z niepoprawnym korzystaniem z magii.\newline
\indent - Nic stąd teraz nie wyciągniemy. Wracajmy.\newline
Weszli z powrotem do karczmy a Yeleda i Garret siedzieli przy jednym stole. Dosiedli się do nich i zamówili jeszcze po piwie. Calion od razu schłodził swoje.\newline
\indent - ...odszedłem bo czułem, że to nie to. Że mogę ćwiczyć moje mnisie umiejętności w prawdziwym życiu na trakcie.\newline
\indent - Łotrostwo nigdy mnie nie zawiodło i mówię ci, że odrzucanie bogactw to czysta głupota. Gdyby przeszkadzały ci jakieś ciężkie monety chętnie je przygarnę.\newline
\indent - A wy? Czym się zajmujecie? - spytał Garret ignorując zaczepki łotrzycy.\newline
\indent - Zgłębiam wiedzę magiczną już od wielu lat i powiem wam, że nie zamierzam przestać.\newline
\indent - Musisz dużo umieć, skoro nie masz przy sobie ni szpilki... - mruknęła Yeleda. Calion tylko się uśmiechnął.\newline
\indent - Szpilkę ma pewnie jego płaszcz. - Vivien uśmiechnęła się. Chciała się poczuć częścią grupy i przyznała się do bycia Druidem oraz tego, że długo już wędruje po świecie, ale nigdzie nie zagrzała miejsca na dłużej. Bała się jednak podać dokładną liczbę lat.\newline
\indent - Rozmawialiśmy z karczmarzem jednak nie powiedział nam niczego więcej niż już wiemy. Ponieważ jest już późno idę w kimę. Trzeba mieć siły na jutro. - Yeleda ziewnęła i przeciągnęła się po czym ruszyła w kierunku pokoi gościnnych. Reszta poszła za jej przykładem.

\paragraph{}
Nie pospali długo, ale Vivien nie potrzebowała tyle snu co reszta ras. Calion oczywiście też. Pobudka była gwałtowna i nieznosząca sprzeciwu, bowiem w powietrze wyleciała czwarta część karczmy razem z dachem. Wybuch i wstrząs były tak skuteczne, że wszyscy nocujący wybiegli na korytarz. Okazało się, że tym razem nocowała tylko ich czwórka. Panował półmrok charakterystyczny dla świtu. Yeleda jęczała, że nic nie widzi.\newline
\indent - Eee... Viv... co ty masz na szyi? - spytał Calion.\newline
\indent - O do kurwy... - wypsnęło jej się na widok ładnego naszyjnika z zielonym kamieniem. Chciała podnieść go do oczu, ale jej ręka trafiła w powietrze i zaklęła jeszcze raz. - Ej, wy też je macie...\newline
\indent - Pozwólcie, że je zbadam - powiedział Calion. - To mi zajmie chwilę.\newline
\indent - Ja pójdę zobaczyć miejsce katastrofy - zaoferowała się, wiedziona ciekawością jak i chęcią niesienia ewentualnej pomocy.\newline
\indent - Idę z tobą, Viv. - Yeleda chwyciła ją za połę płaszcza. Ruszyły śladem nadpalonego drewna. W tym miejscu zabrakło kawałka karczmy. Wybuch musiał nastąpić w tym pokoju, gdyż drzwi były wywalone, a w oczy pierwsze wpadło zwęglone ciało.\newline
\indent - Uuu... tego akurat nie chciałam widzieć. - Yeleda skrzywiła się. - Chodź przeszukajmy pokój.\newline
\indent - Tu nic nie ma. Wygląda, jakby coś w pobliżu łóżka spowodowało wybuch. Ten człowiek jest zupełnie spalony. Wielki ogień? Dziwne. Ech nie da się podejść dalej, podłoga jest nadwątlona eksplozją. - Vivien głośno myślała.\newline
\indent - Ha! Znalazłam 9 sztuk złota!\newline
\indent - Cztery i pół chciałaś powiedzieć. - Vivien obruszyła się. - Gdyby nie ja w ogóle byś tutaj nie dotarła.\newline
\indent - No dobra... Masz. Starą szmatę też chcesz?\newline
Podzieliły łup na dwie nie zamierzając dzielić się tym odkryciem. To nie było ważne dla sprawy. Wróciły do chłopaków i spytały o naszyjniki.\newline
\indent - Jest w tym jakaś mroczna magia, konkretnie nekromancja. Amulety istnieją naprawdę mimo, że nie jesteśmy w stanie ich dotknąć.\newline
\indent - To świetnie. Tam w pokoju nic nie było. Eksplozja najpewniej nastąpiła z jakiegoś punktu przy łóżku na którym były spalone doszczętnie zwłoki. Ciężko oszacować do kogo należały ale wydaje mi się, że w karczmie nie było nikogo oprócz nas i tego chłopaczka, który chciał byśmy go eskortowali.\newline
\indent - Intrygują mnie ślady eksplozji, pójdę obejrzę karczmę z zewnątrz. Może wybuch zostawił jakieś magiczne ślady. - Viv zaoferowała się wiedziona ciekawością.\newline
\indent - Ja przeszukam resztę pokoi.\newline
\indent - Ja zejdę na dół. Może karczmarz jest na dole.\newline
Ich trójka zeszła piętro niżej a niedaleko od schodów przy kontuarze leżało ciało karczmarza. 
Po wstępnych oględzinach nie udało im się ustalić przyczyny śmierci. 
Wydawało się, że umarł ze strachu.
Vivien udała się na zewnątrz obejrzeć ślady po katastrofie. 
Spalenizna nie pachniała magią, ani nie zostawiła szczególnych śladów. 
Eksplozja usunęła fragment dachu i ściany. 
Vivien widziała wnętrza pokoi zza ułamanych ścian. 
Podłoga ledwie się trzymała i w każdej chwili mogła się zawalić. 
Dziewczyna rozejrzała się też po bliższej okolicy ale nic nie znalazła. 
To był czysto fizyczny wybuch, acz przyczyny mogły być magiczne.
Gdy wróciła do towarzyszy, Yeleda też już była na dole dowiedziała się od chłopaków, że eksplozję spowodował najprawdopodobniej ten sam kamień, który posiadają w tych tajemniczych wisiorach. 
Jego odłamki Garret zauważył przez dziurę w suficie składziku, na podłodze pokoju młodego kapłana.\newline
\indent - Zajebiście. Mamy na szyjach przenośne bomby! - Yeleda tym stwierdzeniem podniosła wszystkim morale.\newline
\indent - Jak tam łupienie pozostałych pokoi?\newline
\indent - Nic. - mruknęła i spojrzała w podłogę. - Połamałam wytrychy i uderzyłam się łomem w rękę a Garret potłukł się gdy skakał po parapetach.\newline
\indent - Hm... to może rozejrzymy się po okolicy? - Zaproponowała Vivien.\newline
\indent - Albo udamy się od razu do Sioden - odpowiedział Calion. - I tak mieliśmy eskortować tam kapłana. Może w świątyni tej bogini udzielą nam więcej informacji co do tych kłopotliwych naszyjników.

\paragraph{}
- Patrzcie jaka wymarła ta wioska... Co prawda wczesna godzina, a nikogo nie ma na zewnątrz. Przecież to pasterze i rolnicy.\newline
\indent - Faktycznie, dziwne. - przytaknęła Vivien. Czy nie powinniśmy powiadomić sołtysa o śmierci karczmarza? Czemu nikt nie panikuje po wybuchu?\newline
\indent - Ta druga wielka chata to pewnie jego dom. Chodźmy tam.\newline
Podeszli do domu w którym było ciemno i cicho. 
Po chwili Yeleda usłyszała jakby głuche łupnięcia, a Calion odsunął się od drzwi. 
Yeleda nacisnęła klamkę i stanęli twarzą w twarz z zombie. 
Monstrum zamachnęło się pięścią ale nie trafiło nikogo. 
Vivien uderzyła adrenalina w żyły. 
To plugastwo drwi z cudu tworzenia samym swoim jestestwem. 
Zamachnęła się nań swym kosturem Shillegah. 
Pozostała trójka szybko dołączyła do walki. 
Garret imponował jej swoimi szybkimi atakami, Yeleda niesamowitym uśmiechem losu, a Calion... cóż on niczego nie potrzebował. 
Jego magiczne pociski zmieniały rzeczywistość. 
Dobrze było mieć go po swojej stronie. 
Zwłaszcza, że zombie nie miał z nimi szans. 
Czy był sołtysem czy nie, nie miał przy sobie wielkiego majątku Yeleda dokładnie to sprawdziła. 
Niesamowite, że dla niej te pieniądze były tak ważne.\newline
\indent - Myślę, że nie ma sensu dalej przeszukiwać wioski. - Calion udawał, że nie widzi poczynań Yeledy. - Intrygują mnie te kapliczki Ioun. Kapłani często je odwiedzali - może one przyniosą nam jakieś odpowiedzi.\newline
\indent - Też już o tym myślałam - zgodziła się Vivien. \newline 
Ruszyli we czwórkę do małego ołtarzyku. Calion zbadał ją pod kątem obecnej magii, Yeleda pod kątem pułapek, jednak poza wypalonymi świecami niczego nie znaleźli. Kapliczka była po prostu zaniedbana.\newline
\indent - Pozostaje nam iść do Sioden. Mimo, że eskortowany odszedł już do lepszego świata to nasza jedyna poszlaka. Tam była świątynia Ioun - jeśli ktoś się w niej ostał może udzieli nam informacji o naszej niewygodnej biżuterii.\newline

\section*{Siodden}
\paragraph{}
Kilka godzin podróży upłynęło spokojnie i zakończyło się w bramie miasta, gdzie strażnicy zachowywali się niecodziennie ostrożnie. 
Mimo jakichś niewyjaśnionych konsultacji postanowili ich wpuścić i nie niepokoili więcej.
Świątynia Ioun była skromnym, smukłym budynkiem umiejscowionym niedaleko murów miasta. 
Wchodziło się prostym korytarzem do wysokiej sali oświetlonej prostymi, również wysokimi oknami. 
Pod dachem było widać balkony drugiego piętra. 
Na wysokości oczu mieli jednak poprzewracane, prosto zbijane drewniane ławy i smukłą ciemną postać czytającą jakiś papier na przeciwległej ścianie. 
W tej pustej przestrzeni echo niosło się ładnie nie było więc mowy o przekradnięciu się niezauważonym.
Mężczyzna przedstawił się imieniem Keldin i twierdził że jest sługą świątyni. 
Potwierdził zniknięcia kapłanów i dziwnym błyskiem w oku zareagował na ich amulety. 
Podał im imię Amaryka, jednego z ważniejszych kapłanów, który miał w pobliżu Sioden posiadłość. 
Bez zagłębiania się w szczegóły poradził, by to tam się udali po rozwiązanie zagadki naszyjników. 
Pozwolił im też odetchnąć w świątyni po wyczerpującym dniu.

\paragraph{}
A nie czuli się najlepiej, bo walka z zombiakiami i cieniami, przeszukiwanie wioski, te dziwne amulety... To było bardzo dużo. Poczuła niemal, że przeżyła więcej niż dotychczas razem wzięte.
Nie.
Nie więcej.
Ale jakby bardziej.
Podniosła głowę i wyjrzała przez okno sali w której się ulokowali na rozgwieżdżone niebo. 
Światła miasta poukrywały najmniejszych obserwatorów ale Vivien wiedziała o ich obecności. 
I czuła, że ją ostrzegają. 
Zerknęła na Caliona, który skrupulatnie szykował się do transu. 
Pokręciła głową. 
Też była elfem i bez względu na osobiste odczucia było to atutem drużyny. 
W przyszłości niewykluczone, że nie będą mogli spać tak beztrosko.

\paragraph{}
Nad ranem obudziła się z dziwnym bólem mięśni. 
Rozciągnęła się i rozruszała stawy. 
Ból może i minął, ale czuła że nie będzie w stanie dać dziś z siebie wszystkiego. 
Rozejrzała się po sali - Yeleda i Garret też mieli nietęgie miny, a Calion zniknął. 
Ziewnęła i jęła ogarniać się do drogi. 
Dziś mieli zbadać posiadłość tego nikczemnika. 
Tylko bogowie wiedzą co może ich tam spotkać. 
Postanowiła też przemilczeć swoją niedyspozycję. 
Nie może się okazać najsłabszym ogniwem.

\paragraph{}
Keldin wyprowadził gotową drużynę z miasta i poprowadził do pobliskiej posiadłości. 
Mruknął słowo w nieznanym Vivien języku, dzięki któremu dezaktywował otaczającą dom barierę.
\indent - Proszę. Droga wolna i życzę powodzenia - uśmiechnął się tajemniczo, a Vivien znów dostrzegła błysk w jego oku.

\paragraph{}
Nie spodziewali się by oprócz bariery napotkali jakieś zabezpieczenia w tej spokojnej okolicy. 
Dom wyglądał najzupełniej zwyczajnie. 
Prosta bryła z szerokimi wrotami oraz licznymi oknami. 
Podeszli do wrót i nacisnęli klamkę. 
Drzwi ustąpiły posłusznie i Vivien wraz z Garretem pierwsi przeszli próg, a za nimi Yeleda i Calion. 
Bogato wystrojone wnętrze bardzo przypadło do gustu łotrzycy. 
Z kolei Garret pierwszy zwrócił uwagę na posąg pięknej kobiety stojący w środku hallu.
\indent - To musi być Ioun - powiedział Calion.\newline
\indent - Padnij! - zawołał Garret spod posągu, do którego stóp zdołał już dotrzeć. \newline
Wszyscy schylili się, ale Vivien znów poczuła rwący ból wszystkich mięśni, a potem nieprzyjemne ukłucie w ramię. Okrzyk przestrachu lub bólu wyrwał się ze wszystkich gardeł.\newline
\indent - Musimy być ostrożniejsi - warknęła Yeleda patrząc na Garreta. \newline
Wszyscy wyrwali sobie strzałki z miejsc, w które oberwali. Na szczęście rany nie paprały się, a ból powoli ustępował. Musiały być zatrute bo na policzki wystąpił im niezdrowy kolor zieleni pomieszanej z żółcią.\newline
\indent - Żeby znów nie strzelać w całą drużynę idę przeszukać pokoje po prawej stronie. - po tym oświadczeniu znikła za drzwiami i tyle ją widzieli.\newline
\indent - Będzie kradła - mruknął Calion.\newline
\indent - Zobaczymy czy jest co. - odpowiedziała druidka i wybrała lewą stronę. Garret poszedł za nią.\newline
Nie było co kraść. 
Pokoje zwykłego przeznaczenia zawierały meble i regały zastawione bibelotami. 
Na ścianach wisiały obrazy i gobeliny przedstawiające sceny obyczajowe z życia zwykłych ludzi oraz z życia magów. 
Brak obecności pułapek sam z siebie sugerował, że nie trafią na nic cennego. 
Wrócili do korytarza w którym czekał Calion i Yeleda z dziwną latarnią przytroczoną do paska.
\indent - Co to jest? - spytała Vivien\newline
\indent - To latarnia - odpowiedziała Yeleda.\newline
\indent - Magiczna - dodał Calion. Najwyraźniej zdołał ją już zbadać.\newline
Drużyna udała się schodami na górę, mimo że nie zbadali jeszcze wyjścia na tyły. 
Piętro w kształcie litery „U” na swych krańcach kończyło się oknami zajmującymi całą ścianę i zdradzającymi widoki na poważnie zarośnięty ogród.
U podstaw litery znaleźli drzwi po prawej stronie skryte jakimś dziwnym iskrzącym obłokiem oraz zwykłe drzwi po lewej.\newline
\indent - Ten dziwny obłok to magia, której nie potrafię rozpracować - powiedział Calion. - Jeśli nawet nas nie zrani to na pewno nie przepuści.\newline
\indent - Rozejrzę się na zewnątrz - zaproponował Garret.\newline
\indent - Nie idź tam sam. - zaoponowała Vivien.\newline
\indent - Tylko przez okno. Nie chcę by coś nas zaskoczyło gdy będziemy rozbrajać drzwi.\newline
\indent - To ja zajrzę do tych zwykłych po lewej.\newline
Nie były zaś takie zwykłe. 
Czuła, że tu nie będzie tak bezpiecznie jak na dole. 
Z obserwacji zwyczajów ludzi zdołała już wywnioskować, że cenniejsze rzeczy trzymają dalej od wejścia. 
Podeszła do drzwi i obmacała je dokładnie. 
Nasłuchiwała uważnie gdy ostukiwała futrynę.
\indent - Ach jest! - szepnęła triumfalnie gdy odnalazła pułapkę.\newline
Uchyliła drzwi na tyle, by móc się zmieścić w szparze i przekroczyła rozciągniętą linkę. Otwarcie drzwi na oścież przerwałoby żyłkę i najpewniej wypuściło w jej kierunku kolejne zatrute strzałki.
Znalazła się sypialni z wielkim, bogato zdobionym małżeńskim łożem. 
Ach po trudach podróży chciałoby się zatopić w miękkiej rozkoszy jaką mógł oferować ten mebel. 
Podrapała się w kark i gwałtowny skręt łokcia rozpromienił ból na długość całej ręki.
No tak. 
Nie mieli na to czasu, tykające bomby które mieli na szyjach szybko przywróciły ją do rzeczywistości.
\indent - Viv co tu masz? - zakrzyknęła Yeleda, która zjawiła się przy drzwiach.\newline
\indent - Uważaj na rozciągniętą linkę! To pułapka - odkrzyknęła szybko.\newline
\indent - Ach czaję. Zaraz ją rozbroję.\newline
\indent - I nic nie mam. Te same zakryte obłokiem magii drzwi, co w korytarzu. I to łóżko - uśmiechnęła się pod nosem.\newline
\indent - Wracajmy więc do Caliona może wykombinował jak się przedrzeć przez tamte.

\paragraph{}
\indent - Mam pomysł. - Calion wyglądał jakby oświecił go jakiś głos z góry. - Podaj mi latarnię, którą znalazłaś w gabinecie.\newline
\indent - Lepiej powiedz co mam z nią zrobić.\newline
\indent - ... miej mnie w opiece - szepnął tak, że tylko Vivien to słyszała. - Oświetl jej światłem wnękę drzwi. Powinny się same otworzyć.\newline
\indent - Garret podejdź tu do nas - Viv zdoła zawołać mnicha zanim drzwi stanęły otworem. \newline
Odkryły bogato zdobiony korytarz wyłożony purpurowym miękkim dywanem. 
Ściany pokryte były chabrową farbą oraz złotymi wzorami. 
Amaryk  żył niemal jak król. 
Z korytarza odchodziły kolejne drzwi.
\indent - Nie rozdzielajmy się i przeszukajmy wszystko po kolei - zasugerowała Vivien, po czym otworzyła najbliższe drzwi. - Ach!\newline
Jej okrzyk powiedział drużynie, że za drzwiami czekały na nich trzy cienie. 
Ruszyła do walki wręcz wprawiając w obroty swój kostur. 
Zawtórował jej Garret, który przyskoczył do wysokości mniej więcej kolan przeciwników. 
Calion poszedł po rozum do głowy i za pomocą magicznej ręki odchylił żaluzje. 
Promienie słoneczne które wpadły do wnętrza osłabiły zdolności bojowe cieni, a i pokazały drużynie, że znaleźli się w wielkiej bibliotece. 
Ponadto w odleglejszym rogu pokoju stał kamienny posąg jakiejś postaci. 
Yeleda wystrzeliła z łuku chcąc oszczędzić sobie walki wręcz, ale strzała chybiła i wbiła się posągowi w czoło. 
Magia Vivien jest kapryśna jak pogoda i raz wezwana niszczy wszystko na swej drodze musiała więc polegać na swoich umiejętnościach walki wręcz.
Calion nie miał jednak takich ograniczeń i szybko poczuli swąd palonego cienia.\newline
\indent - O matko, co to było?! - Yeleda łapała oddech.\newline
\indent - To był znak, że jesteśmy blisko jakichś skarbów - dodał Garret.

\twocolumn
\normalsize
\chapter{Historie postaci}

\section{Garret Eartopple}

\paragraph{}
Garret Eartopple pochodził z niewielkiej wioski zwanej Buccorough.
Niziołkowie zamieszkujący ten rejon wiedli spokojne, pełne radości z każdej najdrobniejszej chwili życie.
Trudnili się uprawą ziemi, hodowali niewielkie, z racji na swój rozmiar, zwierzęta takie jak kury i króliki.
Każdą wolną chwilę spędzali na wspólnych zabawach i festynach, gdzie snuli opowieści o życiu i swoich przodkach i tańczyli przy dźwiękach skocznej muzyki.
Przy każdej możliwej sposobności cieszyli się z wspólnej bliskości swoich rodzin.
Nigdy nie parali się sztuką wojenną i unikali wszelkich form konfliktu.
Oczywiście, jak to ma miejsce w każdym społeczeństwie, często mieli różne zdania i się kłócili, zwłaszcza kiedy podczas zabawy domowo warzone piwo lało się szerokimi strumieniami.
Kłótnie jednak zawsze kończyły się śmiechem, poklepywaniem po plecach i przyjacielskich uściskach w których często, upojeni alkoholem i zmęczeniem po skocznych podrygach, zasypiali i budzili się dopiero o brzasku z ciągłym uśmiechem na twarzy.

\paragraph{}
Zdarzenia, które zmieniły bieg życia naszego bohatera miały miejsce kiedy ten kończył właśnie cztery lata.
Garret wracał roześmiany wraz ze swoim ojcem i wielu innymi dorosłymi i znużonymi mężczyznami z pól.
Zbliżała się wieczorna pora, letnie słońce przygrzewało już nie tak silnie.
Młody Garret jak i reszta jego rówieśników spędziła cały czas na wesołej zabawie w polu, obserwując i ucząc się jak za kilka lat będą pomagać swoim starszym ziomkom przy pracy na roli.

\paragraph{}
Wróciwszy w obręb wioski, Garret podbiegł niezgrabnie w kierunku swojego domu i rzucił się w objęcia swojej matki.
Kończył jej właśnie opowiadać, na tyle na ile potrafił, jak spędził dzień na zabawie ze swoimi przyjaciółmi, kiedy ojciec również dotarł do domu i uściskał swoich najbliższych.
Alton i Callie, rodzice naszego bohatera spojrzeli na siebie czule i z ciepłym blaskiem w oczach.
Nigdy nie czuli się szczęśliwsi.
Zawsze czuli że byli dla siebie stworzeni i jedyne czego pragnęli to założyć rodzinę i spędzić ze sobą całe życie.
Utulili i ucałowali Garreta.
Był ich oczkiem w głowie, choć wiedzieli, że za niespełna pół roku urodzi się także ich druga pociecha, z czego bardzo się cieszyli.
Callie czuła, że będzie to córeczka, tak jak planowali, tak jak tego pragnęli – najpierw synek, później córeczka.

\paragraph{}
Wesołym chwilom nie było końca, szczęśliwa rodzina udała się na wspólny plac – dziś był dzień zabawy.
Cała wioska była w trakcie zbiorów, w tym roku wyjątkowo obfitych, co było kolejną okazją do świętowania i ogólnej wspólnej radości.
Stoły były suto zastawione pysznym jadłem, ciepłe piwo nęciło cudownym zapachem, a grajkowie już kończyli strojenie instrumentów.
Zabawa właśnie się rozpoczynała.
Śmiech i wesoły gwar nie miały końca, zapadł zmierzch, jednak bezchmurna noc i palące się ogniska nie zachęcały do zakończenia szczęśliwej zabawy.

\paragraph{}
Właśnie wtedy, dało się słyszeć tętent końskich kopyt.
Napastnikami byli ludzie ubrani w pancerze wykonane ze skóry i wzmacniane elementami metalu.
Była to banda spod znaku Rdzawego Ostrza - najemne zbiry nie będące pod niczyją komendą.
Zaatakowali zachęceni łatwością walki i ilością jedzenia, którymi mogli nażreć się do syta, bez potrzeby narażania się na jakiekolwiek ryzyko.
Bandyci byli jednak głodni nie tylko jedzenia.
Byli chorymi wyrzutkami o zwichrowanym umyśle i fakt doświadczenia nowych, wyuzdanych fantazji z filigranowymi i pucułowatymi kobietami obudził w nich apetyt na coś więcej.

\paragraph{}
Niziołkowie nie mogli mierzyć się z napastnikami.
Nie oddali jednak darmo swej skóry.
Mężczyźni chwycili za jedyne co mogło posłużyć im za broń – kosy, widły, grabie i cepy i stanęli naprzeciw napastnikom.
Wiedzieli, że muszą zrobić wszystko co w ich mocy, by spróbować odeprzeć napastników, jednak w obliczu przewagi rozmiaru, siły i doświadczenia w walce napastników, wiedzieli, że ich opór niewiele zmieni i wynik walki jest z góry przesądzony.
\emph{Ukryj dziecko!} krzyknął Alton.
Callie ze strachem w oczach pochwyciła Garreta i ruszyła pędem w stronę domu.
Jej oczom ukazała się borsucza nora.
Wsunęła tam małego Garreta i kazała mu pod żadnym pozorem nie wychodzić, ani nie wydawać żadnego odgłosu.
Mały Garret widząc przerażenie w oczach matki, podświadomie zrozumiał.
Callie odbiegła od nory, żeby nie dać znaku napastnikom o tym, że coś może się w niej znajdować.
Po odbiegnięciu niewielkiego dystansu od nory, została schwytana przez rosłego napastnika o twarzy nieskalanej myślą.
Odgłosy walki dochodzące z głównego placu zmieniły się.
W tym czasie wszyscy mężczyźni zostali już wycięci w pień – nie stanowili żadnej przeszkody dla napastników - dało się teraz słyszeć jedynie krzyki bólu i rozpaczy przeplatane przejmującym rykiem śmiechu i chorej satysfakcji napastników.
Callie wiedziała co ją spotka, jednak jej drobna postura nie pozwoliła stawić jakiegokolwiek znaczącego oporu.
Była tylko zabawką w rękach ogromnego zbójcy, który rycząc ze śmiechu zaspokajał swoje najbardziej wyuzdane i wypaczone żądze.
Jej krzyki i pojękiwania tylko zwabiły innych napastników, dla których zabrakło już innych kobiet, lub z nudów zaszlachtowali te, które już posiedli i wykorzystali.
Kilku rosłych drabów zabawiało się matką Garreta jak przedmiotem.
Na jej nieszczęście, jej ciało było silne, a świadomość nie mogła ulecieć, ze strachu o swoje ukryte dziecko.
Niestety, nora w której znajdował się mały Garret znajdowała się zbyt blisko miejsca kaźni jego matki.
Sparaliżowany strachem nie potrafił nawet zamknąć powiek i był świadkiem całego wypaczonego i dantejskiego przedstawienia.
Zbójcy zaspokoiwszy się, poderżnęli niewinnej Callie gardło i odrzucili na bok niczym szmacianą lalkę, którą się znudzili.

\paragraph{}
Całą noc i poranek napychali swoje wygłodniałe żołądki i kiedy uznali że nie ma już ani kropli piwa ani okrucha pieczystego, czy swojskiego chleba odjechali wymieniając się jeszcze niewybrednymi komentarzami na temat ostatniej nocy.
W ich spaczonych umysłach przednio się bawili.
Młody Garret cały ten czas siedział w bezruchu w borsuczej norze.
Strach powodował, że nie czuł poruszających się po nim owadów, które nie czuły, że chodzą po żywej istocie.

\paragraph{}
Mijały godziny, przeminął dzień i kolejna noc.
Następnego południa, na miejsce rzezi niewinnej społeczności przybli mnisi z odległego zakonu Ilmatera.
Wyruszyli oni na pielgrzymkę, aby nieść dobro napotkanym potrzebującym.
Widok zmaltretowanych ciał i pogromu jaki miał tutaj miejsce wstrząsnął nimi wszystkimi do żywego.
Poczuli się w obowiązku pochować wszystkich poległych.
Poczuli jednak w okolicy przepływ energii ki.
Podążając jej tropem znaleźli małego niziołka ukrytego w norze.
Choć mały Garret był przerażony, nawet się nie poruszył – trauma całkowicie go sparaliżowała.
Mnisi bez trudu wyciągnęli go z nory i przygarnęli.
Widok, który zobaczyli odczytali jako misję dla siebie.
Pochowawszy wszystkie ciała z należytym szacunkiem, udali się w drogę powrotną do klasztoru, aby przyjąć młodego Garreta pod swoją opiekę i obdarzyć go tym czym mogli – swoją wiedzą, nauką i ciepłem, które panowało między wszystkimi braćmi.

\paragraph{}
W drodze jak i po dotarciu do klasztoru młody Garret przez kilka miesięcy był całkowicie aspołeczny.
Nie potrafił przemówić, nie wychodził ze swojej celi, dopiero po jakimś czasie zaczął przyjmować jakiekolwiek pokarmy.
Ciepło i dobro jakim obdarzyli go tutejsi mnisi pozwoliło mu jednak znaleźć ukojenie i poradzić sobie z traumą jakiej doświadczył.
Poświęcił się całkowicie kształtowaniu swojego ciała i ducha, chcąc szerzyć dobro i pomoc innym, taką jakiej i on doświadczył od swoich braci.

\paragraph{}
W wieku 27 lat, dziękując za całe nauki i wszystko czego nauczył się od swoich braci zdecydował się opuścić mury klasztoru.
Udał się w własną pielgrzymkę, na którą pragnął poświęcić całe swoje życie stawiając sobie dwa cele.
Pierwszym było niesienie pomocy i dobra potrzebującym.
Stawanie w obronie słabszych, karanie ludzi podłych i złych, gnębiących innnych.

\paragraph{}
Drugi cel naszego bohatera został jednak skryty przed wszystkimi braćmi.
Był on głównym powodem opuszczenia zakonu.
Garret wiedział, że nie może spędzić w klasztorze ani chwili dłużej wiedząc że jego serce nie jest do końca czyste.
Żyłby wciąż w zakłamaniu, czego nie mógłby dopuścić wobec swoich braci, których tak pokochał i pragnął żyć z nimi w harmonii.
Jego duch nie mógł zaznać spokoju póki nie udoskonali on swoich sztuk walki wręcz i pełnej kontroli nad swoim ciałem i umysłem.
Dopiero wtedy, kiedy będzie w stanie kontrolować energię ki innych istot, nie spocznie, póki sam nie stanie oko w oko z całą bandą Rdzawego Ostrza i nie wymierzy im zasłużonej sprawiedliwości manewrując ich własną energią życiową tak, by czuli ból, który sami zadawali.
By czuli panikę, strach i przerażenie swoich ofiar.
By czuli każdą łamaną kość, by czuli każdy fragment ciętej i okaleczanej skóry, by czuli, jak ich narządy wewnętrzne przemieszczają się i rwą się na drobne kawałki.
By byli w pełni świadomi sposobu, w jaki sprawiedliwie będą konali w męczarniach i agonii.
Długo.
Powolnie.
W cierpieniu.

\twocolumn
\normalsize
\chapter{Kwiatki z sesji}

% \section{First Section}
% \lipsum[1] % filler text
% 
% \subsection{a subsection}
% \subsubsection{a subsubsection}
% 


\begin{rpg-quotebox}{Wewnętrzne rozterki wegan}
   \textit{Po zakończonej walce z zaklętymi krzewami.}\\
   
   \begin{tabularx}{\columnwidth}{lX}
      \textbf{Calion:} & Vivien, czy Ciebie nie boli zabójstwo tych paprotek? Przecież ich życie jest Ci bliskie.\\
      \textbf{Garret:} & O nie, ja mam na rękach jego liście!
   \end{tabularx}
\end{rpg-quotebox}


\begin{rpg-quotebox}{O mistycznym ognisku}
   \textit{Walka z cieniami, w której decydującą rolę odegrały płonące pociski Caliona. Po rozwianiu ostatniego z cieni:}\\
   
   \begin{tabularx}{\columnwidth}{lX}
      \textbf{Vivien :} & Czuję swąd palonego cienia!\\
   \end{tabularx}
\end{rpg-quotebox}


\section*{6 maja 2017}


\begin{rpg-quotebox}{Niemęskość}
   \textit{Uwiąd prącia jako następna klątwa z felernego amuletu.}
\end{rpg-quotebox}


\begin{rpg-quotebox}{Niefortunne przejęzyczenie}
   \textit{Calion padł ofiarą zaklętego posążka pantery, co spowodowało, że kilka godzin spędził zamieniony w wielkiego kota. Po rozwianiu klątwy wydarzyła się taka rozmowa:}\\
   
   \begin{tabularx}{\columnwidth}{lX}
      \textbf{Garret:} & Calion, potrzebuję pomocnej łapy... - tfu! - dłoni. \\
      \textbf{MG:} & Calion, rzucaj na wkurwienie. \\
   \end{tabularx}
\end{rpg-quotebox}


\begin{rpg-quotebox}{Oparzenia trzeciego stopnia}
   \begin{tabularx}{\columnwidth}{lX}
      \textbf{Matheus:} & Kasia, ile Ty masz inteligencji? Chyba zbyt dużo.\\
      \textbf{Madzia:} & Ha, jaki pocisk! Rzucałeś na trafienie?\\
   \end{tabularx}
\end{rpg-quotebox}


\begin{rpg-quotebox}{Rozmowy wysokiej wagi}
   \begin{tabularx}{\columnwidth}{lX}
      \multicolumn{2}{l}{\textit{Do Garreta:}}\\
      \textbf{Yeleda :} & Ile ważysz? \\
      \textbf{Garret:} & 50kg.\\
      \textbf{Mateusz:} & Ale w grze...\\
   \end{tabularx}
\end{rpg-quotebox}


\begin{rpg-quotebox}{Nie było telemarku}
   \begin{tabularx}{\columnwidth}{lX}
      \textbf{Yeleda:} & Garet jest w objęciach podłogi po lądowaniu na ryju.
   \end{tabularx}
\end{rpg-quotebox}


\begin{rpg-quotebox}{Pojawiają się niejakie myśli}
   \textit{Walka z cienistym smokiem zakończyła się finałem, w którym jego czaszka eksplodowała.}\\
   
   \begin{tabularx}{\columnwidth}{lX}
      \textbf{Garret:} & Mam twarz cieniem myśli skalaną.\\
   \end{tabularx}
\end{rpg-quotebox}


\begin{rpg-quotebox}{Jędrne negocjacje}
   \begin{tabularx}{\columnwidth}{lX}
      \textbf{Jubiler:} & Dam Ci za to 100 sztuk złota. \\
      \textbf{Yeleda:} & Zwariowałeś?! To jest warte conajmniej 150!\\
      \textbf{Jubiler:} & Cóż, jestem jedynym skupującym w okolicy, nie masz wyboru.\\
      \textbf{Yeleda:} & A jak pokażę cycki?
   \end{tabularx}
\end{rpg-quotebox}
    

\begin{rpg-quotebox}{Pan nie jest zadowolony}
   \textit{Tawerna w której zatrzymała się drużyna była zaopatrzona jedynie w jabłka i ich przetwory. Gustujący w piwie niziołek był co najmniej rozczarowany.}\\
   
   \begin{tabularx}{\columnwidth}{lX}
      \multicolumn{2}{l}{\textit{Pocieszającym tonem do Garreta:}}\\
      
      \textbf{Karczmarz:} & Mamy też jabłka w spirytusie.\\
   \end{tabularx}
\end{rpg-quotebox}


\begin{rpg-quotebox}{Sodoma i Gomora}
   \textit{MG opisuje hulanki, swawole i tańce (goście utworzyli korowód) oraz co robi drużyna wchodzącemu do karczmy Calionowi.}\\
   
   \begin{tabularx}{\columnwidth}{lX}
      \textbf{Łukasz:} & Dobrze, że Vivien jest w wężu, a nie wąż w niej. \\
   \end{tabularx}
\end{rpg-quotebox}


\begin{rpg-quotebox}{Dbajmy o czystość powietrza}
   \textit{Garret wybrał się na wieczorny spacer.}\\
   
   \begin{tabularx}{\columnwidth}{lX}
      \textbf{MG:} & Co robisz, gdy się wietrzysz?\\
      \textbf{Łukasz:} & Zamieniam się w wunderbaum i wieszam się w najbliższym golfie. \\
   \end{tabularx}
\end{rpg-quotebox}


\begin{rpg-quotebox}{Skutki przedawkowania hentai}
   \textit{Yeleda wpadła w pułapkę, w której żywe pnącza unieruchomiły ją od pasa w dół.}\\
   
   \begin{tabularx}{\columnwidth}{lX}
      \textbf{Garret:} & Gotuj sie na gwałt! \\
   \end{tabularx}
\end{rpg-quotebox}


\begin{rpg-quotebox}{Roślinna ośmiornica}
   \begin{tabularx}{\columnwidth}{lX}
      \textbf{Yeleda :} & Ile jeszcze tych macek?\\
      \textbf{MG:} & To jest mackarium\\
      \multicolumn{2}{l}{\textit{Vivien używa sztuczki Guidance.}}\\
      \textbf{Vivien:} & Macam się.\\
   \end{tabularx}
\end{rpg-quotebox}


\section*{29 maja 2017}

\section*{8 kwietnia 2017}


\begin{rpg-quotebox}{Jesteśmy skażeni elektrycznością}
   \textit{Podczas wchodzenia do mrocznej jaskini.}\\
   
   \begin{tabularx}{\columnwidth}{lX}
      \textbf{Vivien:} & Jak Garret włączy pochodnię...\\
   \end{tabularx}
\end{rpg-quotebox}

    
\begin{rpg-quotebox}{Nie ten klimat}
   \textit{W odpowiedzi na radioaktywne obrażenia kultystów.}\\

   \begin{tabularx}{\columnwidth}{lX}
      \textbf{Vivien:} & Skąd oni się urwali?\\
      \textbf{Garret:} & Z Fallouta.\\
   \end{tabularx}
\end{rpg-quotebox}

   
\begin{rpg-quotebox}{Nie czyń bezmyślnie!}
   \textit{Garret poszedł na pierwszy ogień przywitać się z kultystami, najpewniej wrogo nastawionymi.}\\
   
   \begin{tabularx}{\columnwidth}{lX}
      \multicolumn{2}{l}{\textit{}}\\
      
      \textbf{Kultyści:} & Czego tu szukasz?\\
      \textbf{Garret:} & Nie przemyślałem tego.\\
   \end{tabularx}
\end{rpg-quotebox}


\begin{rpg-quotebox}{Czas snajperów}
   \textit{Yeleda wystrzeliła zabójczą strzałę z krzaków trafiając jednego z kultystów w głowę. Ten padł jak długi nieprzytomny.}\\
   
   \begin{tabularx}{\columnwidth}{lX}
      \textbf{Garret:} & Jednemu z kultystów świetny pomysł strzelił do głowy.\\
      \multicolumn{2}{X}{\textit{Kultysta - nieudany rzut obronny na śmierć}}\\
      \textbf{MG:} & Jest nieprzytomny. Jego nieprzytomność sięga zenitu.\\
   \end{tabularx}
\end{rpg-quotebox}


\begin{rpg-quotebox}{Wątpliwe zamiary}
   \textit{Yeleda została trafiona zaklęciem i padła nieprzytomna.}\\
   
   \begin{tabularx}{\columnwidth}{lX}
      \textbf{Garret:} & Gdzie ona leży?\\
      \textbf{MG:} & W krzakach.\\
      \textbf{Garret:} & Czeka na okazję.\\
   \end{tabularx}
\end{rpg-quotebox}


\begin{rpg-quotebox}{Walcz z głową...}
   \textit{Kruk Caliona włączył się do bitwy. Kultysta postanowił obrać go na cel.}\\
   
   \begin{tabularx}{\columnwidth}{lX}
      \textbf{MG:} & Kultysta atakuje Kruka z pałki.\\
   \end{tabularx}
\end{rpg-quotebox}


\begin{rpg-quotebox}{O tym jak ważny jest dokładny opis otoczenia}
   \textit{Wygraliśmy potyczkę. Yeleda postanowiła przeszukać ciała.}\\
   
   \begin{tabularx}{\columnwidth}{lX}
      \textbf{MG:} & Znajdujesz cztery pałki.\\
      \textbf{Yeleda:} & Ale jakie?\\
      \textbf{MG:} & Takie do walenia.\\
      \textbf{Garret:} & Nadal nie sprecyzowałaś.\\
   \end{tabularx}
\end{rpg-quotebox}


\begin{rpg-quotebox}{Żart ciągnie się dalej}
   \textit{Vivien podbiegła do Velory z zamiarem przywalenia jej kosturem. }\\
   
   \begin{tabularx}{\columnwidth}{lX}
      \textbf{Garret:} & Grzmocisz ją drągiem?\\
   \end{tabularx}
\end{rpg-quotebox}


\begin{rpg-quotebox}{Mistrzowska aranżacja wnętrz}
   \textit{W odpowiedzi na to, jak konstrukt krasnoludzkiej zbroi postanowił nas zaatakować.}\\
   
   \begin{tabularx}{\columnwidth}{lX}
      \textbf{Garret:} & Latający miecz, gryzący krzak, żyjąca zbroja - what the fuck?"\\
   \end{tabularx}
\end{rpg-quotebox}


\begin{rpg-quotebox}{Przywar ciężko się wyzbyć}
   \textit{Walka z cieniami. Yeledzie nie wyszedł atak i miecz wyleciał jej z ręki.}\\
   
   \begin{tabularx}{\columnwidth}{lX}
      \multicolumn{2}{l}{\textit{Do Yeledy}}\\
      
      \textbf{Vivien:} & Upuściłaś miecz.\\
      \textbf{Garret:} & Cień nie jest tobą zainteresowany.\\
   \end{tabularx}
\end{rpg-quotebox}


\begin{rpg-quotebox}{Głos w sprawie tłumienia odruchów}
   \textit{Vivien ogłuszyła kapłanów i wyrzuciła ich ciała na odległość za pomocą czaru Thunderwave. Była wśród nich kapłanka, która przywołała iluzoryczny miecz. W efekcie miecz zniknął.}\\
   
   \begin{tabularx}{\columnwidth}{lX}
      \multicolumn{2}{l}{\textit{Do Yeledy}}\\
      
      \textbf{Garret:} & Miecz znika, nie podnosimy go - bo go nie ma.\\
   \end{tabularx}
\end{rpg-quotebox}


      
\begin{rpg-quotebox}{Kocim żartom nie było końca}
   \textit{Calion wziął w ręce figurkę tygrysa i w się niego zamienił. Zwyczajny tygrys pojawił się na miejscu maga drużyny.}\\

   \begin{tabularx}{\columnwidth}{lX}
      \textbf{MG:} & Calion masz na tyle inteligencji, że wiesz, że nie musisz ich atakować.\\
      \textbf{Yeleda:} & Kici kici kici!\\
      \textbf{Calion:} & Warczę na Yeledę.\\
   \end{tabularx}
   ~\newline\newline
   \begin{tabularx}{\columnwidth}{lX}
      \textbf{Garret:} & Jestem mały - mogę na nim jeździć.\\
      \textbf{Calion:} & Spróbuj, to cię rozszarpię.\\
   \end{tabularx}
   ~\newline\newline
   \begin{tabularx}{\columnwidth}{lX}
      \multicolumn{2}{X}{\textit{Fantazjując na temat tego co by było, gdyby Vivien zamieniła się teraz w panterę.}}\\
      
      \textbf{Yeleda:} & Czy ona i tygrys mogliby...\\
      \textbf{Vivien:} & Mam nadzieję, że jest kastrowany.\\
   \end{tabularx}
\end{rpg-quotebox}


\begin{rpg-quotebox}{Kocia retrospekcja}
   \textit{Yeleda chciała otworzyć jakiś zamek. Vivien oparła jej rękę na ramieniu by pomóc jej sztuczką Guidance.}\\
   
   \begin{tabularx}{\columnwidth}{lX}
      \textbf{Yeleda:} & O! Znów mnie głaszczesz.\\
      \multicolumn{2}{l}{\textit{Cicho do Yeledy.}}\\
      \textbf{Vivien:} & A gdy głaskałam Caliona miał takie aksamitne futro....\\
   \end{tabularx}
\end{rpg-quotebox}


\begin{rpg-quotebox}{Powerleveling}
   \textit{Po zakończonej walce z kultystami.}\\
   
   \begin{tabularx}{\columnwidth}{lX}
      \textbf{Yeleda:} & Kopej EXPA!
   \end{tabularx}
\end{rpg-quotebox}


\begin{rpg-quotebox}{O tym jak Vivien zaufała w pełni swoim bogom}
   \textit{Vivien bada mechanicznie ukryte drzwi. Po chwili odnajduje ukryty przycisk i jak gdyby nigdy nic naciska go.}\\
   
   \begin{tabularx}{\columnwidth}{lX}
      \textbf{MG:} & Rzucaj na zręczność.\\
      \includegraphics[scale=0.06]{img/d20.png}\textbf{:}& 20\\
      \textbf{MG:} & Z zakamarków drzwi wystrzela zatruta strzałka w twoim kierunku, ale CUDEM robisz unik i    pocisk leci przez całą salę między głowami drużyny i wbija się w naprzeciwległą ścianę.\\
      \multicolumn{2}{l}{\textit{Do oszołomionej drużyny.}}\\
      \textbf{Vivien:} & Ups!\\
   \end{tabularx}
\end{rpg-quotebox}

\section*{11 czerwca 2017}



\begin{rpg-quotebox}{O niepomyślnych wiatrach}
   \textit{Drużyna została zaatakowana przez żywiołaki powietrza w sali prób.}\\\newline
   \begin{tabularx}{\columnwidth}{lX}
      \textbf{MG:} & Vivien i Caliona odrzuciło.\\
      \textbf{Yeleda:} & Najwyraźniej z tej dziury bardzo źle pachnie.
   \end{tabularx}
\end{rpg-quotebox}


\begin{rpg-quotebox}{Konflikt interesów}
   \begin{tabularx}{\columnwidth}{lX}
      \textbf{Calion:} & Lecz Viv, bo do dupy nakopię.\\
      \multicolumn{2}{X}{\textit{Z urażonym głosem}}\\
      \textbf{Vivien:} & Ja leczę...\\
      \multicolumn{2}{X}{\textit{Calion się uśmiecha.}}\\
      \textbf{Vivien:} & ... ale siebie.\\
      \multicolumn{2}{X}{\textit{Mina Caliona przybiera wściekłe rysy.}}
   \end{tabularx}
\end{rpg-quotebox}


\begin{rpg-quotebox}{A oczy jego otwarły się}
   \begin{tabularx}{\columnwidth}{lX}
      \textbf{Garret:} & Przyglądam się dziurze\\
      \includegraphics[scale=0.06]{img/d20.png}\textbf{:}& 20\\
      \textbf{MG:} & Sens istnienia stanął przed tobą otworem\\
   \end{tabularx}
\end{rpg-quotebox}


\begin{rpg-quotebox}{Chwilowa nieobecność}
   \begin{tabularx}{\columnwidth}{lX}
      \textbf{Calion:} & Viv myślisz to co ja?\\
      \textbf{Vivien:} & Nie, ja nic nie myślę.\\
   \end{tabularx}
\end{rpg-quotebox}


\begin{rpg-quotebox}{Paradoks}
   \textit{Garret wykonuje piękny skok do wody.}\\
   \newline
   \begin{tabularx}{\columnwidth}{lX}
      \textbf{Garret:} & Osiągnąłem dno.\\
   \end{tabularx}
\end{rpg-quotebox}


\begin{rpg-quotebox}{Fatalne przejęczenie}
   \begin{tabularx}{\columnwidth}{lX}
      \textbf{Garret:} & Pomagam Vivien wziąć Yeledę pod wodą... - yyy! - pod wodę!\\
   \end{tabularx}
\end{rpg-quotebox}


\begin{rpg-quotebox}{Definitywny brak fornifilii}
   \begin{tabularx}{\columnwidth}{lX}
      \textbf{Yeleda:} & Nie pieprzę się z drzwiami.\\
   \end{tabularx}
\end{rpg-quotebox}


\begin{rpg-quotebox}{O nie taki wiatr chodziło}
   \textit{Calion używa czaru podmuchu wiatru. }\\
   \newline
   \begin{tabularx}{\columnwidth}{lX}
      \textbf{Garret:} & Calion robi magiczny pierd\\
   \end{tabularx}
\end{rpg-quotebox}


\begin{rpg-quotebox}{Ładne wybrnięcie}
   \textit{Drużyna odrzuciła propozycję wiedźmy by wydostać się z tuneli.}\\
   
   \begin{tabularx}{\columnwidth}{lX}
      \textbf{Yeleda:} & Czyli nara.\\
      \textbf{Calion:} & Nara czyli co?\\
      \textbf{Yeleda:} & Nara-żamy życie.\\
   \end{tabularx}
\end{rpg-quotebox}


\begin{rpg-quotebox}{Nagły problem obuwniczy}
   \begin{tabularx}{\columnwidth}{lX}
      \multicolumn{2}{l}{\textit{Do Yeledy:}}\\
      
      \textbf{MG:} & Weszłaś w śluz więc otrzymujesz 6 pkt. obrażeń od kwasu. I nie masz buta.\\
   \end{tabularx}
\end{rpg-quotebox}


\begin{rpg-quotebox}{Jak praca zespołowa czasem się nie udaje}
   \textit{Garret, Yeleda i Vivien postanowili przeskoczyć plamę śluzu, przy czym Yeleda i Vivien miały pecha i wyryły się na twarz. NA KONIEC Calion użył czaru, by usunąć śluz i przejść bez szwanku...}
\end{rpg-quotebox}


\begin{rpg-quotebox}{Pamiętaj aby nie narzekać w obecności MG na rozwój przygody}
   \textit{Po jednej z sesji Łukasz narzekał na ogrom magicznych pułapek i niewielką ilość pułapek ściśle mechanicznych. Niemal na początku kolejnej sesji, już jako Garret, potknął się o linę i uruchomił zapadnię w podłodze. Drużyna musiała wykonać test zręczności. Vivien i Yeleda w porę odskoczyły, a Calion zamortyzował upadek. MG nie pozwoliła jednak Garretowi nic zrobić, bo to on uruchomił pułapkę.}\\
   \newline
   \begin{tabularx}{\columnwidth}{lX}
      \textbf{MG:} & Ty masz na tej linie automatycznie porażkę!\\
      \textbf{Garret:} & Ale ja mam refleksy!\\
      \textbf{Yeleda:} & Chyba refluksy...\\
      \textbf{Calion:} & Chciałeś niemagiczne pułapki, patafianie. To masz!\\
   \end{tabularx}
\end{rpg-quotebox}


\begin{rpg-quotebox}{O dziwach matki natury}
   \begin{tabularx}{\columnwidth}{lX}
      \textbf{Yeleda:} & Czy pająki strzelają?\\
      \textbf{Garret:} & Tak, liną z dupy.\\
   \end{tabularx}
\end{rpg-quotebox}


\begin{rpg-quotebox}{Kiedy chwalebna rola zaczyna się przejadać}
   \textit{Jakiś potwór poczynił pogrom wśród drużyny.}\\
   \newline
   \begin{tabularx}{\columnwidth}{lX}
      \textbf{Vivien:} & Ja pierdolę. Znowu muszę leczyć.\\
   \end{tabularx}
\end{rpg-quotebox}


\begin{rpg-quotebox}{Każdy orze jak może}
   \textit{Drużyna chciała wejść do otworu w podłodze, tak by jak najmniej się poturbować. Yeleda użyła latającego płaszcza, wzięła pod pachy Caliona, a Vivien przemieniona w pająka usiadła jej na ramieniu. Garret z mnisią gracją wskoczył samodzielnie za nimi.}\\
\end{rpg-quotebox}


\begin{rpg-quotebox}{Antyornityzm}
   \textit{Po walce, w której Antaramo, kruk-chowaniec Caliona okazał się niezwykle pomocny.}\\
   \newline
   \begin{tabularx}{\columnwidth}{lX}
      \textbf{MG:} & Usmażyłeś pająka. Muszę znaleźć jakiegoś potwora, który zjada kruki.\\
   \end{tabularx}
\end{rpg-quotebox}


\begin{rpg-quotebox}{O tym jak brutalna siła stała się mototrem napędowym strachu}
   \textit{Drużyna została zaskoczona widokiem potężnej Banshee za otwartymi drzwiami w podziemiach. Calion instynktownie zaatakował ją zaklęciem Firebolt, ale niewiele wskórał.}\\
   \begin{tabularx}{\columnwidth}{lX}
      \textbf{MG:} & Jest odporna na ogień.\\
      \textbf{Garret:} & To ona jest z azbestu?!\\
   \end{tabularx}
   \textit{Garret postanowił ją załatwić pięściami i wyrzucił pod rząd trzy naturalne 20, co zabrało banshee jakieś $\frac{2}{3}$ życia. Drużyna zyskała punkt inspiracji.}\\
   \textit{Calion postanowił ją dobić magicznym pociskiem.}\\
   \begin{tabularx}{\columnwidth}{lX}
      \textbf{Calion:} & Lekko chujowo, tylko 7 punktów obrażeń.\\
      \textbf{MG:} & Zabijasz ją.\\
   \end{tabularx}
\end{rpg-quotebox}


\begin{rpg-quotebox}{Moralność panien D-n-dulskich}
   \textit{W pokoju razem z banshee było 6 kultystów, którzy srali po gaciach na widok Garreta po tym jak załatwił ich banshee. Byli tak przerażeni, że nie byli w stanie mówić. Jednak w końcu drużyna wyciągnęła z nich, że tak do końca nie wiedzieli co robią. Garret sprzeciwił się zabijaniu ich, bo: Naiwniactwo jest okolicznością łagodzącą.}\\
   
   \begin{tabularx}{\columnwidth}{lX}
      \textbf{Vivien:} & Wisi mi ich los.\\
      \textbf{Yeleda:} & Mi wisi bardziej niż tobie.
   \end{tabularx}
\end{rpg-quotebox}


\begin{rpg-quotebox}{Dobrze strzeżona spiżarnia}
   \textit{Drużyna została przeniesiona za pomocą portalu do czegoś co wyglądało jak spiżarnia.}\\
   
   \begin{tabularx}{\columnwidth}{lX}
      \textbf{MG:} & Znajdujecie konfitury, piwo, książki i bibeloty\\
      \textbf{Yeleda:} & Szukam pułapek.
   \end{tabularx}
\end{rpg-quotebox}


\begin{rpg-quotebox}{Pomoc poprzez molestowanie}
   \textit{Calion próbuje sobie przypomnieć w jakim mieście wylądowali.}\\
   
   \begin{tabularx}{\columnwidth}{lX}
      \multicolumn{2}{l}{\textit{Vivien korzysta ze sztuczki Guidance, aby mu pomóc.}}\\
      
      \textbf{MG:} & Calion, czujesz, że ktoś Cię maca.\\
   \end{tabularx}
\end{rpg-quotebox}

\section*{24 czerwca 2017}

\begin{rpg-quotebox}{O grupach etnicznych}
   \begin{tabularx}{\columnwidth}{lX}
      \textbf{MG:} & Niziołki nie są rasą mniejszościową.\\
      \textbf{Viv:} & Chyba niższościową.\\
      \textbf{Garret:} & Czuję się poniżony.
   \end{tabularx}
\end{rpg-quotebox}


\begin{rpg-quotebox}{Nieukrywane stereotypy}
   \begin{tabularx}{\columnwidth}{lX}
      \textbf{MG:} & Widzicie wnętrze karczmy wypełnione resztkami po imprezie. Za barem nikogo nie widać.\\
      \textbf{Yeleda:} & Pewnie niziołek go prowadzi.\\
      \textbf{Garret:} & Podchodzę do baru i wołam karczmarza.\\
      \textbf{MG:} & Z zaplecza polerując szklankę wychodzi niziołek...\\
   \end{tabularx}
\end{rpg-quotebox}


\begin{rpg-quotebox}{Strach nie sięga wysoko}
   \begin{tabularx}{\columnwidth}{lX}
      \textbf{Yeleda:} & Mam czkawkę.\\
      \textbf{Vivien:} & Wystraszcie ją!\\
      \textbf{Garret:} & Uaaaa!\\
      \textbf{Yeleda:} & Tak... Niziołku...\\
      \textbf{Vivien:} & Nawrzeszczał Ci na kolana.\\
   \end{tabularx}
\end{rpg-quotebox}


\begin{rpg-quotebox}{O tym jak MG nie wytrzymał głupiego pytania}
   \begin{tabularx}{\columnwidth}{lX}
      \textbf{MG :} & Barmanowi mózg wyeksmitował z głowy.\\
   \end{tabularx}
\end{rpg-quotebox}


\begin{rpg-quotebox}{O charakterach}
   \begin{tabularx}{\columnwidth}{lX}
      \textbf{Calion:} & Garret jest praworządny dobry, a ja dobry. Razem jesteśmy praworządni głupi.\\
   \end{tabularx}
\end{rpg-quotebox}


\begin{rpg-quotebox}{Bieda aż piszczy}
   \textit{Gracze znajdują się w ekskluzywnym sklepie prowadzonym przez drakonkę.} \\

   \begin{tabularx}{\columnwidth}{lX}
      \textbf{Yeleda:} & W sumie kupiłabym mapę, gdyby była jedna w rozsądnej cenie.\\
      \textbf{Garret:} & Czyli drakońskie ceny odpadają.\\
   \end{tabularx}
\end{rpg-quotebox}


\begin{rpg-quotebox}{Globalizacja dociera i do Zapomnianych Krain}
   \textit{Niecodziennie o racjach żywnościowych} \\

   \begin{tabularx}{\columnwidth}{lX}
      \textbf{MG:} & Te za 1 szt. srebra to taka średniowieczna zupka chińska.\\
   \end{tabularx}
\end{rpg-quotebox}


\begin{rpg-quotebox}{O tym jak w żyłach Rosjan i niziołków płynie ta sama krew}
   \textit{Po udanym teście na akrobatykę:}\\

   \begin{tabularx}{\columnwidth}{lX}
      \textbf{MG:} & Udaje ci się doskoczyć do parapetu i wykonać słowiański przykuc.\\
   \end{tabularx}
\end{rpg-quotebox}


\begin{rpg-quotebox}{O tym jak naturalne 20 pomaga w schadzkach w ciemnych alejkach}
   \textit{Garret przeczuwał, że był śledzony. Chcąc zmylić pościg uznał, że schowa się w ciemnej alei i z zaskoczenia zaatakuje potencjalnego złoczyńcę.}\\

   \begin{tabularx}{\columnwidth}{lX}
      \textbf{MG:} & Udało ci się powalić młodą kobietę.\\
      \textbf{Yeleda:} & Ty ogierze!\\
   \end{tabularx}
\end{rpg-quotebox}


\section*{15 lipca 2017}


\begin{rpg-quotebox}{O oczekiwaniach w sprawie korupcji}
   \begin{tabularx}{\columnwidth}{lX}
      \textbf{Matheus:} & Żarcie dla MGa $ \rightarrow $ +10 do wszystkich rzutów.
   \end{tabularx}
\end{rpg-quotebox}


\begin{rpg-quotebox}{O braku administracji państwowej}
   \textit{Gracze rozważają pomysł podróży poza gościńcem. Pytanie zostało skierowane do MG.}\\

   \begin{tabularx}{\columnwidth}{lX}
      \textbf{Calion:} & Czy Angamarna ma Lasy Państwowe?\\
   \end{tabularx}
\end{rpg-quotebox}


\begin{rpg-quotebox}{O przesadzaniu w obmyślaniu przykrywki}
   \textit{Vivien wymyśliła przykrywkę - jest studentką magii naturalnej i pisze pracę na temat szkodliwego wpływu kornika na populacje drzew w okolicznych lasach. }\\

   \begin{tabularx}{\columnwidth}{lX}
      \textbf{Vivien:} & Czemu drużyna się po lesie szlaja? Czy szaleje tam kornik drukarz?\\
   \end{tabularx}
\end{rpg-quotebox}


\begin{rpg-quotebox}{Kiedy survival staje się prawdziwą walką o przetrwanie}
   \begin{tabularx}{\columnwidth}{lX}
      \textbf{MG :} & Zaczynacie KRĄŻYĆ po lesie pod światłym przewodnictwem waszego druida.\\
   \end{tabularx}
\end{rpg-quotebox}



\begin{rpg-quotebox}{Problemy dendro-andrologiczne}
   \textit{W reakcji na napotkany szlak rozsypanych kawałków drewna w runie leśnym.}\\

   \begin{tabularx}{\columnwidth}{lX}
      \textbf{Garret:} & Albo to jest duże...\\
      \textbf{Vivien:} & Albo mamy do czynienia z łysiejącym Entem.\\
   \end{tabularx}
\end{rpg-quotebox}


\begin{rpg-quotebox}{Rozterki ekologiczne}
   \textit{Pytanie skierowane do MG na temat napotkanego potwora:}

   \begin{tabularx}{\columnwidth}{lX}
      \textbf{Calion:} & Czy czerwony smok jest pod ochroną? \\
   \end{tabularx}
\end{rpg-quotebox}


\begin{rpg-quotebox}{O tym jak trudne potyczki czasem nie bywają trudne...}
   \textit{Yeleda wykorzystała atak z zaskoczenia i wycelowała z łuku w oko smoka. Los chciał by k20 pokazała 20 i smok tak mocno oberwał w łeb że został niemal ogłuszony i oślepł.}
\end{rpg-quotebox}


\begin{rpg-quotebox}{... i nawet MG potrafi być zaskoczony obrotem spraw}
   \textit{Smok zajęty był rozrywaniem skrzyni więc drużyna miała przewagę atakiem z zaskoczenia. Po niewiarygodnym trafieniu z łuku, do akcji wkroczył Calion.}\\

   \begin{tabularx}{\columnwidth}{lX}
      \textbf{MG:} & Z krzaków wyskoczył opętany mag, który dobił smoka zanim ten choćby beknął.\\
   \end{tabularx}
\end{rpg-quotebox}


\begin{rpg-quotebox}{W przypływie podziwu}
   \textit{O obrocie spraw po niewiarygodnie szybkim rozprawieniu się z małym smokiem:} \\
   
   \begin{tabularx}{\columnwidth}{lX}
      \textbf{Vivien:} & Jesteście cud miód drużyną!\\
   \end{tabularx}
\end{rpg-quotebox}


\begin{rpg-quotebox}{Trafne podsumowanie potyczki}
   \textit{Po kratycznym trafieniu Yeledy.}\\

   \begin{tabularx}{\columnwidth}{lX}
      \textbf{Garett:} & One shot one kill no luck pure skill.\\
   \end{tabularx}
\end{rpg-quotebox}


\begin{rpg-quotebox}{Żart dobrze pociągnięty}
   \textit{BN przyłapał drużynę na zwycięstwie ze smokiem i spytał co się tu właściwie stało. Vivien pozostaje w roli studentki badającej drzewostan lasów Doliny Konwalii:} \\

   \begin{tabularx}{\columnwidth}{lX}
      \textbf{:} & To był wyjątkowo duży kornik. Bardzo szkodliwy!\\
   \end{tabularx}
\end{rpg-quotebox}


\begin{rpg-quotebox}{Obawy o zdrowy sen}
   \begin{tabularx}{\columnwidth}{lX}
      \multicolumn{2}{l}{\textit{Na temat odpoczynku w lesie:}} \\
      \textbf{Calion:} & Vincent powiedział, że smok miał brata.\\
      \textbf{Vivien:} & My mamy Yeledę.
   \end{tabularx}
\end{rpg-quotebox}


\begin{rpg-quotebox}{Nieformalne przedstawienie}
   \textit{Po przygodzie z bardzo szybkim uporaniem się z banshee, mnich uznał, że przedstawi się nowonapotkanemu BN-mu, Vincentowi w następujący sposób.}\\

   \begin{tabularx}{\columnwidth}{lX}
      \multicolumn{2}{l}{\textit{Wyciąga w górę pięść.}}\\
      \textbf{Garret:} & Jestem oficjalnym eksterminatorem banshee. \\
   \end{tabularx}
\end{rpg-quotebox}


\begin{rpg-quotebox}{O tym jak tłumaczyć niecodzienne talenty}
   \begin{tabularx}{\columnwidth}{lX}
      \multicolumn{2}{l}{\textit{Do MG.}}\\
      \textbf{Yeleda:} & Co trzeba mieć by oprawić smoka?\\
      \textbf{Calion:} & Backstory.
   \end{tabularx}
\end{rpg-quotebox}


\begin{rpg-quotebox}{O tym\, że skrywamy ukryty talent garbarniczy}
   \textit{Po oprawieniu truchła smoka:}\\
   
   \begin{tabularx}{\columnwidth}{lX}
      \textbf{MG:} & Udało się wam z niego wyłuskać jakieś 20kg łusek.\\
   \end{tabularx}
\end{rpg-quotebox}


\begin{rpg-quotebox}{Przypraw swoje życie}
   \textit{Calion użył stuczki Prestidigitacji by oczyścić nas z resztek wnętrzności smoka. Ta sztuczka pozwala się osuszyć, oczyścić czy choćby poprawić jakość jedzenia i to bez płacenia!}\\ 
   
   \begin{tabularx}{\columnwidth}{lX}
      \textbf{Garret:} & A teraz jeszcze szczypta magii!\\
   \end{tabularx}
\end{rpg-quotebox}


\begin{rpg-quotebox}{Niecodzienne nazewnictwo}
   \begin{tabularx}{\columnwidth}{lX}
      \textbf{Calion:} & Definicja portalu - slot na bramę.\\
   \end{tabularx}
\end{rpg-quotebox}


\begin{rpg-quotebox}{Rzut to nie wszystko}
   \textit{Calion miał przejść test wiedzy by sobie przypomnieć co nieco o ruinach które eksplorowaliśmy.}\\

   \begin{tabularx}{\columnwidth}{lX}
      \includegraphics[scale=0.06]{img/d20.png}\textbf{:}& 4\\
      \textbf{Calion:} & W sumie 11.\\
      \textbf{Vivien:} & Dodatkowo wykorzystuję swoją sztuczkę guidance i dokładam 4 do sumy rezultatu.\\
      \textbf{Calion:} & Wypadło 4 więc mamy wynik 15.\\
   \end{tabularx}
\end{rpg-quotebox}


\begin{rpg-quotebox}{O nieuczciwych przePISach}
   \textit{Callion i Garret wyrzucili 16 na inicjatywie i mieli doń ten sam bonus. Kolejność rozstrzygnął rzut kostką, a wygrany Calion otrzymał program 16+.}
\end{rpg-quotebox}


\begin{rpg-quotebox}{Paradoksalnie dobre podsumowanie sytuacji}
   \begin{tabularx}{\columnwidth}{lX}
      \textbf{MG:} & Dwa mocno nadszarpnięte życiem zombiaki wam zostały.\\
   \end{tabularx}
\end{rpg-quotebox}


\begin{rpg-quotebox}{O rozmowie na dwa fronty na temat szczęśliwych kostek}
   \begin{tabularx}{\columnwidth}{lX}
      \multicolumn{2}{l}{\textit{Do Kasi:}} \\
      \textbf{Madzia:} & Tu mamy cały asortyment broni przeciwko naszemu MG. \\
      \multicolumn{2}{l}{\textit{Do Kingi:}} \\
      \textbf{Madzia:} & Tego nie słyszałaś.
   \end{tabularx}
\end{rpg-quotebox}


\begin{rpg-quotebox}{O chęci mordoterapii...}
   \begin{tabularx}{\columnwidth}{lX}
      \multicolumn{2}{X}{\textit{Po desperackiej walce ze smokiem, MG uśmiecha się złowieszczo.}}\\
      \textbf{Calion:} & Ile trzeba było szczęścia by to przeżyć? \\
      \textbf{MG:} & DUŻO.\\
   \end{tabularx}
\end{rpg-quotebox}


\begin{rpg-quotebox}{Jak wszedłeś między wrony...}
   \begin{tabularx}{\columnwidth}{lX}
      \multicolumn{2}{l}{\textit{Gracze aktywowali pułapkę.}}\\
      \textbf{MG:} & Wszyscy wykonajcie rzut obronny na zręczność.\\
      \textbf{Calion:} & Ja nie byłem blisko drzwi.\\
      \textbf{MG:} & WSZYSCY!\\
   \end{tabularx}
\end{rpg-quotebox}


\begin{rpg-quotebox}{Niziołek z wozu...}
   \begin{tabularx}{\columnwidth}{lX}
      \textbf{Yeleda:} & To jak mi Garret pomoże to wyciągniemy wóz.\\
      \textbf{Garret:} & Wsiadam na wóz!\\
   \end{tabularx}
\end{rpg-quotebox}


\begin{rpg-quotebox}{Celowe przejęczyzenie}
   \begin{tabularx}{\columnwidth}{lX}
      \multicolumn{2}{l}{\textit{Garret wykorzystuje umiejętność Medyk.}}\\
      \textbf{Garret:} & Roztaczam nad chętnymi opiekę.\\
      \textbf{Yeleda:} & Ja jestem hentai.
   \end{tabularx}
\end{rpg-quotebox}

\begin{rpg-quotebox}{Tym razem pudło}
   \begin{tabularx}{\columnwidth}{lX}
      \textbf{MG:} & Rzucaj na trafienie\\
      \multicolumn{2}{l}{\textit{Kostka przelatuje obok}} \\
      \textbf{Vivien:} & Nie trafiłeś w wieżę.\\
   \end{tabularx}
\end{rpg-quotebox}

\begin{rpg-quotebox}{Fortuna kołem się toczy}
   \textit{Po nieudanym teście trafienia Yeleda myślała, że jej tura dobiegła końca.}\\

   \vskip-0.05cm
   \begin{tabularx}{\columnwidth}{lX}
      \textbf{MG:} & Ale miałaś przewagę, rzucaj jeszcze raz.\\
      \includegraphics[scale=0.06]{img/d20.png}\textbf{:}& 20\\
      \textbf{MG:} & Mogłam tego nie mówić...\\
   \end{tabularx}
\end{rpg-quotebox}


\begin{rpg-quotebox}{Zemsta faraonów}
   \begin{tabularx}{\columnwidth}{lX}
      \textbf{Calion:} & Używam magicznego pocisku z 2 poziomu.\\ 
   \end{tabularx}

   \vskip0.15cm
   \textit{Rzuca czterema kościami k4.} \\
   
   \begin{tabularx}{\columnwidth}{lX}
      \textbf{Vivien:} & Co to ma być??\\
      \textbf{Garret:} & To jest kurwa Egipt.
   \end{tabularx}
\end{rpg-quotebox}


\begin{rpg-quotebox}{O przemyślanym podarku}
   \textit{Wdzięczny BN obdarował graczy podarkami. Yeledzie przypadł w udziale zaklęty sztylet.}\\
   
   \begin{tabularx}{\columnwidth}{lX}
      \textbf{Calion:} & Skąd wiedział, że jest łotrzykiem?\\
      \textbf{MG:} & Ma broń na wierzchu.\\
   \end{tabularx}
   \newline

   \textit{Komentując przytroczone, powieszone i pochowane elementy uzbrojenia Yeledy} \\
   
   \begin{tabularx}{\columnwidth}{lX}
      \textbf{Calion:} & To chodząca zbrojownia!\\
   \end{tabularx}
\end{rpg-quotebox}


\section*{22 lipca 2017}


\begin{rpg-quotebox}{Wysokość spojrzenia ma znaczenie}
   \textit{Garret wyrzucił najwyższy wynik testu na percepcję.} \\

   \begin{tabularx}{\columnwidth}{lX}
      \textbf{MG:} & Tylko niziołek ma oczy.\\
   \end{tabularx}
\end{rpg-quotebox}


\begin{rpg-quotebox}{O ki(j) chodzi?}
   \begin{tabularx}{\columnwidth}{lX}
      \textbf{MG:} & Jesteś piętro niżej więc cała twoja akcja to ruch.\\
      \textbf{Garret:} & Chyba, że użyję punkt Ki.\\
      \textbf{MG:} & Pff...\\
   \end{tabularx}
\end{rpg-quotebox}

\begin{rpg-quotebox}{Jak światło stało się ich przewodnikiem}
   \textit{Przeciwnikowi atak wcześniej wypadł miecz z ręki więc została mu tylko kusza. Udało mu ukryć się w cieniu, jednak nie udało mu się wywieźć w pole Vivien. Ta rzuciła sztuczkę ,,światło'' na jego kuszę i poinstruowała drużynę by strzelali w stronę światła.}
\end{rpg-quotebox}


\begin{rpg-quotebox}{Słaniające się przejęzyczenie}
   \begin{tabularx}{\columnwidth}{lX}
      \textbf{MG:} & Przywaliłeś mu dość mocno i jest mocno poturbowany, ale stoi na ostatnich nogach.\\
   \end{tabularx}
\end{rpg-quotebox}


\begin{rpg-quotebox}{Logika stomatologiczna}
   \textit{Zapytana o lokalizację zakładnika:} \\

   \begin{tabularx}{\columnwidth}{lX}
      \textbf{Vivien:} & Yeleda pilnuje swojego nowego adoratora na dole.\\
      \textbf{Yeleda:} & On nie ma zębów!\\
      \textbf{Vivien:} & Przynajmniej nie ugryzie.\\
   \end{tabularx}
\end{rpg-quotebox}


\begin{rpg-quotebox}{O problemach moralnych}
   \begin{tabularx}{\columnwidth}{lX}
      \textbf{Calion:} & A to gdzie jest ta magiczna zbroja?\\
      \textbf{MG:} & Na gościu.\\
      \textbf{Vivien:} & On jeszcze żyje, więc to nie będzie ograbianie zwłok.\\
      \textbf{Calion :} & Tylko żywych...\\
   \end{tabularx}
\end{rpg-quotebox}


\begin{rpg-quotebox}{O braku spostrzegawczości}
   \begin{tabularx}{\columnwidth}{lX}
      \textbf{Zakładnik:} & Powiedziałem wam wszystko. Rozwiążcie mnie i pozwólcie odejść.\\
      \textbf{Yeleda:} & Już jesteś rozwiązany, paciuloku.\\
   \end{tabularx}
\end{rpg-quotebox}


\section*{6 sierpnia 2017}

\begin{rpg-quotebox}{Problemy niskich}
   \begin{tabularx}{\columnwidth}{lX}
      \textbf{Garret:} & Mam 93 cm wzrostu. \\
      \textbf{Vivien:} & Nie możesz wejść do makro.
   \end{tabularx}
\end{rpg-quotebox}


\begin{rpg-quotebox}{Poranna nagość}
   \textit{Yeleda do Vivien na temat nagości Garreta po wstaniu z łóżka}\\

   \begin{tabularx}{\columnwidth}{lX}
      \textbf{Yeleda:} & Przy twoim wzroście nie możesz widzieć jego przyrodzenia. \\
      \textbf{Vivien:} & Zależy czy ma długi nos...
   \end{tabularx}
\end{rpg-quotebox}


\begin{rpg-quotebox}{O szpiegach wśród niższych ras}
   \begin{tabularx}{\columnwidth}{lX}
      \textbf{Garret:} & Czy ta krasnoludzica ma brodę? \\
      \textbf{MG:}     & Nie. \\
      \textbf{Garret:} & Czyli to niziołek w przebraniu!
   \end{tabularx}
\end{rpg-quotebox}


\begin{rpg-quotebox}{O niespójnościach wypoczynku}
   \textit{MG przyzwał takich wrogów że drużyna zaliczyła TPK. Po jakimś czasie uratowała ich członkini Paktu i życzyła szczęścia. Po odzyskaniu przytomności:} \\

   \begin{tabularx}{\columnwidth}{lX}
      \textbf{Garret:} & Czemu mamy 1 HP? Czy to się nie liczyło jako długi odpoczynek?
   \end{tabularx}
\end{rpg-quotebox}


\begin{rpg-quotebox}{Krótko o szybkości}
   \begin{tabularx}{\columnwidth}{lX}
      \textbf{Yeleda:} & Yay! Mam 19 inicjatywy! \\
      \textbf{Calion, Garret:} & Bitch please! My mamy 20 i 21.
   \end{tabularx}
\end{rpg-quotebox}


\begin{rpg-quotebox}{Wiedza o aranżacji przestrzennej}
   \begin{tabularx}{\columnwidth}{lX}
      \textbf{Calion:} & Identyfikuję mimika! \\
      \textbf{MG:} & Kurwa, jaka to wiedza? \\
      \textbf{Garret:} & Wystrój wnętrz\\
   \end{tabularx}
\end{rpg-quotebox}


\begin{rpg-quotebox}{Misteria poruszania się w rzędzie}
   \textit{Drużyna próbowała skryć się w cieniu i tylko Yeleda przeszła test. Przejście było wąskie, więc ustawili się w szeregu w kolejności: Yeleda, Garret, Vivien, Calion.} \\

   \begin{tabularx}{\columnwidth}{lX}
      \textbf{MG:} & Garret myśli że idzie pierwszy... \\
   \end{tabularx}
\end{rpg-quotebox}


\begin{rpg-quotebox}{Z czarodziejami nie będziesz się nudził}
   \textit{Strażnik posłany po mistrza zakonu nie wracał tak długo, że drużyna zaczęła grać w kalambury przy użyciu sztuczek Caliona.} \\

   \begin{tabularx}{\columnwidth}{lX}
      \textbf{Yeleda:} Co oni tam produkują tego mistrza?! 
   \end{tabularx}
\end{rpg-quotebox}


\begin{rpg-quotebox}{Nie-gołąb pocztowy}
   \begin{tabularx}{\columnwidth}{lX}
      \textbf{Calion:} & Daję krukowi list do przekazania komuś na uniwersytecie. \\
      \textbf{Vivien} & Kruk pocztowy.\\
   \end{tabularx}
\end{rpg-quotebox}


\begin{rpg-quotebox}{Szczerość to podstawa}
   \begin{tabularx}{\columnwidth}{lX}
      \textbf{MG jako Siel:} & Pójście tam na pełnej kur...tyzanie jest złym pomysłem. \\
   \end{tabularx}
\end{rpg-quotebox}


\begin{rpg-quotebox}{Problemy obuwnicze}
   \begin{tabularx}{\columnwidth}{lX}
      \textbf{MG:} & To zupełnie inna para kaloszy. \\
      \textbf{Yeleda:} & Co to kalosze? \\
      \textbf{Garret:} & To zupełnie inna para trzewików odpornych na wilgoć.
   \end{tabularx}
\end{rpg-quotebox}


\begin{rpg-quotebox}{Te wybuchy i te rozbłyski...}
   \textit{W wyniku oszołomienia widowiskowością rzuconego właśnie zaklęcia.} \\

   \begin{tabularx}{\columnwidth}{lX}
      \textbf{Yeleda:} & Co ona z siebie wydała? \\
      \textbf{Vivien:} & Ostatniego slota.
   \end{tabularx}
\end{rpg-quotebox}


\begin{rpg-quotebox}{Chłodna rozprawa}
   \textit{MG wcielaja się w rolę Siel i przygotowuje zaklęcie lodowych noży.} \\

   \begin{tabularx}{\columnwidth}{lX}
      \textbf{Siel:} & Dobra, a nóż się coś trafi.\\
      \textbf{Calion:} & Też używam lodowych noży na kapłanach.\\
   \end{tabularx}
   \newline

   \textit{Vivien również postanowiła rzucić zaklęcie lodowych ostrzy.}\\

   \begin{tabularx}{\columnwidth}{lX}
      \textbf{Vivien:} & Witajcie w lodówce.\\
   \end{tabularx}
\end{rpg-quotebox}

\section*{6 sierpnia 2017}

\begin{rpg-quotebox}{Nie na żarty...}
   \begin{tabularx}{\columnwidth}{lX}
      \multicolumn{2}{l}{\textit{Do Łukasza, przed rozpoczęciem:}}\\
      \textbf{Madzia:} & Cały czas żresz, coś byś pomyślał też!\\
   \end{tabularx}
\end{rpg-quotebox}

\begin{rpg-quotebox}{Nie szata zdobi człowieka}
   \begin{tabularx}{\columnwidth}{lX}
      \textbf{MG:} & Wstajecie, jest trochę lepsza pogoda. Co robicie?\\
      \multicolumn{2}{l}{\textit{Do Łukasza:}}\\
      \textbf{Madzia:} & Jesteś ubrany? Rzuć na to.\\
      \textbf{Matheus:} & A to jest test na zręczność czy inteligencję?\\
   \end{tabularx}
\end{rpg-quotebox}

\begin{rpg-quotebox}{O wyborach dróg nie tylko życiowych}
   \textit{Drużyna ustala czy podążać leśnym traktem, czy skorzystać z gościńca.}\\
   
   \begin{tabularx}{\columnwidth}{lX}
      \textbf{Yeleda:} & Ja chce lasem!\\
      \textbf{Vivien:} & Będziesz się bić z każdym smokiem o legowisko?\\
      \textbf{Yeleda:} & Tak!\\
   \end{tabularx}
\end{rpg-quotebox}

\begin{rpg-quotebox}{O obsadzaniu ról nieodpowiednimi osobami}
   \textit{Drużyna wybrała wychowaną w mieście Yeledę by prowadziła ich przez leśne ostępy. W pewnym momencie usłyszeli wilcze warczenie.}\\
   
   \begin{tabularx}{\columnwidth}{lX}
      \multicolumn{2}{l}{\textit{Uprzedzając MG w jej roli:}}\\
      \textbf{Łukasz:} & Macie wrażenie, że wilki są wszędzie.\\
      \multicolumn{2}{l}{\textit{MG zaczęła się podejrzanie uśmiechać.}}\\
      \textbf{Łukasz:} & Nie! Nie powiedziałem tego!\\
   \end{tabularx}
\end{rpg-quotebox}

\begin{rpg-quotebox}{Wcale nie dobra zmiana}
   \textit{Drużyna stwierdziła, że Yeleda się nie sprawdziła i wybrali wychowanego w klasztorze mnicha na przewodnika. Ukrytych pod kopułą zdybały ogry, które po chwili przyprowadziły ogra przywódcę.}\\

   \begin{tabularx}{\columnwidth}{lX}
      \textbf{MG:} & Garret awansował do programu "Ogr +".\\
   \end{tabularx}
\end{rpg-quotebox}

\begin{rpg-quotebox}{Niektórych rzeczy nie można odzobaczyć}
   \textit{Drużyna walczy z ogrami w kompletnej ciemności. Garret, znacząco niższy od ogrów, ma ograniczone pole widzenia.}\\

   \begin{tabularx}{\columnwidth}{lX}
      \textbf{Garret:} & Szukam słabych punktów. Jest ciemno to chuj widzę.\\
   \end{tabularx}
\end{rpg-quotebox}

\begin{rpg-quotebox}{Kocie zabawy}
   \textit{Calion rozmawiał z duchem elfki, a reszta drużyny czekała w krzakach. Vivien jako pantera, skryta w cieniu Yeleda i Garret.}\\

   \begin{tabularx}{\columnwidth}{lX}
      \textbf{Vivien:} & Poluję na frędzle twojego płaszcza.\\
      \textbf{Yeleda:} & Ale ja nie mam frędzli.\\
      \textbf{Vivien:} & Już masz.\\
   \end{tabularx}
\end{rpg-quotebox}

\begin{rpg-quotebox}{O sposobach walki z cieczą}
   \begin{tabularx}{\columnwidth}{lX}
	   \textit{Walka z żywiołakami wody.}\\
	   \textbf{Garret:} & Próbuję osuszyć go ręcznikiem!\\
   \end{tabularx}
\end{rpg-quotebox}

\begin{rpg-quotebox}{Ręce\, które krzywdzą}
   \textit{Na pytanie czy żywiołaki są odporne na obrażenia fizyczne.}\\
   
   \begin{tabularx}{\columnwidth}{lX}
	   \textbf{MG:} & Garret, czy Twoje pięści są magiczne?\\
	   \textbf{Garret:} & Jeszcze nie...\\
   \end{tabularx}
\end{rpg-quotebox}

\begin{rpg-quotebox}{Raz na wozie\, raz pod wozem}
   \textit{Żywiołak obrał na cel Garreta.}\\
   
   \begin{tabularx}{\columnwidth}{lX}
      \textbf{Garret:} & Ciekawe, czy ci się to uda...\\
      \textbf{MG:} & 24 na trafienie.\\
      \textbf{Garret:} &... skurwysynu...\\
   \end{tabularx}
\end{rpg-quotebox}

\begin{rpg-quotebox}{Susze bywają uciążliwe}
   \textit{O stanie walki z żywiołakami:}\\
   
   \begin{tabularx}{\columnwidth}{lX}
      \textbf{MG:} & Wysuszyliście w ponad połowie jednego żywiołaka.\\
   \end{tabularx}
\end{rpg-quotebox}

\begin{rpg-quotebox}{Akwarium pogrzebowe}
   \textit{Jeden z żywiołaków wchłonął Garreta w swoje wodne trzewia.}\\
   
   \begin{tabularx}{\columnwidth}{lX}
      \textbf{MG:} & Gwoli uściślenia, jak będziecie bić żywiołaka Garret też oberwie.\\
      \textbf{Vivien:} & Dobra jebać, potem się go wskrzesi w końcu znam medycynę.\\
      \multicolumn{2}{l}{\textit{Szykując potężny czar:}}\\
      \textbf{Calion:} & No, skoro nasz druid tak mówi to wybacz bracie\\
   \end{tabularx}
\end{rpg-quotebox}

\begin{rpg-quotebox}{Katastrofa w przestworzach}
   \textit{Drużyna zjeżdża tunelem w dół, który kończy się lądowaniem z dwóch metrów na gołym kamieniu. Calion jechał pierwszy i postanowił rzucić na siebie czar powolnego spadania.}\\
   
   \begin{tabularx}{\columnwidth}{lX}
      \textbf{MG:} & Jesteś pewien że chcesz go rzucić?\\
      \textbf{Calion:} & Oczywiście! Tam są dwa metry!\\
      \textbf{MG:} & OK. Po chwili na pełnej kurwie nadlatuje Garret.\\
   \end{tabularx}
\end{rpg-quotebox}

\begin{rpg-quotebox}{Sztuka dyplomacji z potworami w praktyce}
   \begin{tabularx}{\columnwidth}{lX}
      \multicolumn{2}{l}{\textit{W języku Sylvan}}\\
      \textbf{Viv:} & Odejdź stąd Encie, a nic ci nie zrobimy. Nie jesteśmy wrogami.\\
      \textbf{Ent:} & Nie.\\
   \end{tabularx}
\end{rpg-quotebox}

\begin{rpg-quotebox}{Coś w nim pękło}
   \textit{Duch elfa zabił Vivien. Po tym Garret wyrzucił pod rząd 5 trafień krytycznych.}\\

   \begin{tabularx}{\columnwidth}{lX}
      \textbf{MG:} & Co tu sie właśnie odbyło??\\
      \textbf{Matheus:} & Jakby nie patrzeć to ten elf zabił jego kobietę...\\
   \end{tabularx}
\end{rpg-quotebox}

\section*{4 listopada 2017}

\begin{rpg-quotebox}{O bogatym bukiecie zapachowym}
   \begin{tabularx}{\columnwidth}{lX}
      \textbf{MG:} & Wchodzisz i uderza cię zapach klasycznego lochu.\\
      \textbf{Yeleda:} & Że czym dostałam w łeb?\\
   \end{tabularx}
\end{rpg-quotebox}

\begin{rpg-quotebox}{Kwestia niedopowiedzenia}
   \begin{tabularx}{\columnwidth}{lX}
      \textbf{MG:} & Widzisz zabite skrzynie.\\
      \textbf{Yeleda:} & Czym je zabili?\\
      \textbf{MG:} & Gwoździami!\\
   \end{tabularx}
\end{rpg-quotebox}

\begin{rpg-quotebox}{Z cyklu \emph{grzechy popełnione ustami}}
   \begin{tabularx}{\columnwidth}{lX}
      \textbf{Yeleda:} & Calion, chodź tu - tu jest ten elf którego zaciukaliśmy.\\
      \textbf{Drużyna:} & Nie zaciukaliśmy!\\
      \textbf{Yeleda:} & No to ten, którego biliśmy!\\
      \textbf{Drużyna:} & Nie biliśmy!\\
      \textbf{Yeleda:} & No to ten, przed którym uciekaliśmy!\\
      \textbf{Drużyna:} & Nie uciekaliśmy!\\
      \textbf{Yeleda:} & ... to ten właśnie!\\
   \end{tabularx}
\end{rpg-quotebox}

\begin{rpg-quotebox}{Trudne sprawy}
   \begin{tabularx}{\columnwidth}{lX}
      \textbf{Calion:} & Czemu ja? Viv też jest elfem!\\
      \textbf{Yeleda:} & Ale ty masz więcej życia.\\
      \multicolumn{2}{l}{\textit{Do MG}}\\
      \textbf{Calion:} & Czemu pytałaś na głos?\\
   \end{tabularx}
\end{rpg-quotebox}

\begin{rpg-quotebox}{Ale fizykę to trzeba szanować}
   \begin{tabularx}{\columnwidth}{lX}
      \textbf{Garret:} & Chcę rzucić w upiora granatem.\\
      \multicolumn{2}{l}{\textit{Po chwili namysłu}}\\
      \textbf{Garret:} & Czy to jest ściana nośna?\\
   \end{tabularx}
\end{rpg-quotebox}

\begin{rpg-quotebox}{Szukanie dziury w całym}
   \begin{tabularx}{\columnwidth}{lX}
      \textbf{MG:} & Upiory są odporne na obrażenia fizyczne i część magicznych, ale podatne na ataki od srebrnych broni.\\
      \textbf{Garret:} & Zaciskam pięsci na srebrnych monetach.\\
   \end{tabularx}
\end{rpg-quotebox}

\begin{rpg-quotebox}{Obawa o utratę integralości}
   \textit{Garret musiał się wydostać z płonącego okręgu, do którego wcześniej niezgrabnie wskoczył. Rzut poszedł dobrze.}\\

   \begin{tabularx}{\columnwidth}{lX}
      \textbf{MG:} & Tym razem artystycznie wyskoczyłeś z pomiędzy płomieni.\\
      \textbf{Vivien:} & W między czasie piekąc kiełbaskę.\\
      \textbf{Garret:} & Swoją?!\\
   \end{tabularx}
\end{rpg-quotebox}

\begin{rpg-quotebox}{Zagrożenie pożarowe}
   \begin{tabularx}{\columnwidth}{lX}
      \textbf{Garret:} & Okładam śpiworem ogień w celu ugaszenia.\\
      \textbf{Yeleda:} & Okładam ogień Garretem okładającym ogień śpiworem.\\
   \end{tabularx}
\end{rpg-quotebox}

\begin{rpg-quotebox}{Pszczółki i kwiatuszki}
   \textit{Drużyna zyskała 10 pkt doświadczenia za wyczerpującą dyskusję nt. edukacji seksualnej mnichów.}
\end{rpg-quotebox}

\begin{rpg-quotebox}{Stereotypom nie ma końca}
   \begin{tabularx}{\columnwidth}{lX}
      \multicolumn{2}{l}{\textit{Samokrytycznie:}}\\
      \textbf{Yeleda:} & Tylko końcówki włosów mam blond. Mózg się jeszcze broni!\\
   \end{tabularx}
\end{rpg-quotebox}

\begin{rpg-quotebox}{Nieczyste zagrywki}
   \begin{tabularx}{\columnwidth}{lX}
      \textbf{MG:} & Garret ma niezłą krzepę - zadaje 16 pkt. obrażeń z piąchy.\\
      \textbf{Garret:} & Jeszcze ze srebrnikami - lewy judaszowy!\\
   \end{tabularx}
\end{rpg-quotebox}

\begin{rpg-quotebox}{O niespójnej chęci zysku}
   \textit{Drużyna odnalazła magazyn z zadania i postanowiła go nie ograbiać, by nie narażać się Kultowi. Gdy już wyszli na zewnątrz i nie było mowy o powrocie:}\\
   
   \begin{tabularx}{\columnwidth}{lX}
      \textbf{Garret:} & W sumie to mistrz powiedział, że możemy sobie stamtąd zabrać co chcemy.\\
      \multicolumn{2}{l}{\textit{Unisono}}\\
      \textbf{Yeleda:} & \multirow{ 3}{*}{Teraz mi to mówisz?!}\\
      \textbf{Vivien:} & \\
      \textbf{Calion:} & \\
   \end{tabularx}
\end{rpg-quotebox}

\begin{rpg-quotebox}{O tym jak pozwolono zadecydować wcale-nie-ślepemu losowi}
   \textit{Drużyna rzucała kośćmi którą opcję drogi objąć i rzucała nimi tak długo aż wypadło to co chcieli.}
\end{rpg-quotebox}

\begin{rpg-quotebox}{O chłodnej kalkulacji}
   \begin{tabularx}{\columnwidth}{lX}
      \textbf{Vivien:} & Moja propozycja jest taka: Idźmy do Doliny Konwalii. Tam mamy pewność, że Dolina rozwali i nas i Pakt. We wszystkich innych opcjach Pakt ma przewagę.\\
      \textbf{Drużyna:} & Zgoda!\\
   \end{tabularx}
\end{rpg-quotebox}

\begin{rpg-quotebox}{O dziwnych decyzjach}
   \begin{tabularx}{\columnwidth}{lX}
      \textbf{Yeleda:} & Czy mogę się ukryć na środku gościńca?\\
      \textbf{Garret:} & Tak - obsyp się żwirem i udawaj drogę.\\
   \end{tabularx}
\end{rpg-quotebox}

\begin{rpg-quotebox}{O tym jak za zły żart można spłonąć ze wstydu}
   \begin{tabularx}{\columnwidth}{lX}
      \textbf{Yeleda:} & Przepraszam! Mają panowie ognia? Zapaliłabym coś.\\
      \textbf{BN-i:} & Co chcesz palić? Nie, nie mamy.\\
      \textbf{Garret:} & Karczmę na przykład.\\
   \end{tabularx}
\end{rpg-quotebox}

\begin{rpg-quotebox}{O nagłym zwrocie akcji}
   \begin{tabularx}{\columnwidth}{lX}
      \textbf{Yeleda:} & Okrążam karczmę tak szybko, że teraz to ja ich śledzę.\\
   \end{tabularx}
\end{rpg-quotebox}

\begin{rpg-quotebox}{O realnej ocenie własnych umiejętności}
   \begin{tabularx}{\columnwidth}{lX}
      \textbf{MG:} & W karczmie jest 10 najemników i dwa słoneczne elfy - magowie. Rzućcie na inicjatywę.\\
      \multicolumn{2}{l}{\textit{Poza postacią:}}\\
      \textbf{Garret:} & Czy mamy robić nowe postaci?\\
      \textbf{Vivien:} & Czy mam dostateczną wolę przeżycia, żeby wrócić jako duch?\\
   \end{tabularx}
\end{rpg-quotebox}

\begin{rpg-quotebox}{O tym jak reklamy nie znają granic}
   \begin{tabularx}{\columnwidth}{lX}
      \textbf{MG:} & Myślę, że karczmarka ma większy problem z 11 trupami.\\
      \textbf{Vivien:} & Kiepska reklama..\\
      \textbf{Garret:} & Nie do końca - pomyśl o takim szyldzie: "U nas zaliczysz zgona!", "U nas zalejesz się w trupa!"\\
   \end{tabularx}
\end{rpg-quotebox}

\section*{25 listopada 2017}

\begin{rpg-quotebox}{O smutku wywołanym brakiem odpowiedniego sprzętu}
   \begin{tabularx}{\columnwidth}{lX}
      \multicolumn{2}{l}{\textit{Poza postacią:}}\\
      \textbf{Vivien:} & Mamy już 5. poziom, a ty jeszcze nic nie uwarzyłeś!\\
      \textbf{Garret:} & Z rzeczy do warzenia piwa mam tylko umiejętność.\\
   \end{tabularx}
\end{rpg-quotebox}

\begin{rpg-quotebox}{Zaufanie to podstawa}
   \begin{tabularx}{\columnwidth}{lX}
      \textbf{Calion:} & Czyli wszyscy w transie, a tylko Yeleda śpi-śpi.\\
      \textbf{Vivien:} & Czyli jedyna osoba, która mogłaby ci wbić nóż w plecy gdy śpisz.\\
      \textbf{Yeleda:} & Point!\\
   \end{tabularx}
\end{rpg-quotebox}

\begin{rpg-quotebox}{O tym jak nowoczesne technologie wcale nie są takie mityczne}
   \begin{tabularx}{\columnwidth}{lX}
      \textbf{Calion:} & Rozświetlam swój sztylet.\\
      \textbf{Garret:} & Pnącza trzeba było rozświetlić!\\
      \multicolumn{2}{l}{\textit{Poza postacią:}}\\
      \textbf{Vivien:} & Byłby światłowód.\\
   \end{tabularx}
\end{rpg-quotebox}

\begin{rpg-quotebox}{Jak chcesz to zakończyć?}
   \begin{tabularx}{\columnwidth}{lX}
      \textbf{Garret:} & Skaczę na potwora.\\
      \textbf{MG:} & Jak chcesz na niego skoczyć? Na niego stricte, czy walnąć i wylądować obok?\\
      \textbf{Vivien:} & Czy chcesz w nim skończyć?\\
      \textbf{Garret:} & Z pięścią w jego ryju.\\
   \end{tabularx}
\end{rpg-quotebox}

\begin{rpg-quotebox}{Wyklepie się}
   \begin{tabularx}{\columnwidth}{lX}
      \textbf{MG:} & Potwór po ataku Garreta ma do wysokości ud solidne wgniecenia w korze.\\
   \end{tabularx}
\end{rpg-quotebox}

\begin{rpg-quotebox}{Anatomia anatomii nierówna}
   \textit{Vivien zmieniła się w tygrysa i chciała powalić potwora, ale zamiast tego ten ją ogłuszył potężnym uderzeniem.}\\

   \begin{tabularx}{\columnwidth}{lX}
      \textbf{Garret:} & Czy medycyna liczy się też jako weterynaria?\\
   \end{tabularx}
\end{rpg-quotebox}

\begin{rpg-quotebox}{Konflikt interesów}
   \textit{Yeleda dobiła atakujące pnącze zręcznym strzałem z ukrycia.}\\

   \begin{tabularx}{\columnwidth}{lX}
      \textbf{Calion:} & Hey! Bullshit!\\
      \textbf{Garret:} & Komu ty kibicujesz?\\
   \end{tabularx}
\end{rpg-quotebox}

\begin{rpg-quotebox}{Przy planowaniu należy rozważyć wszystkie okoliczności}
   \begin{tabularx}{\columnwidth}{lX}
      \textbf{Calion:} & Ale moment, co my chcemy zrobić?\\
      \textbf{Garret:} & Owinę się liną, wyjdę przez okno i zobaczę czy z innej strony widać miasto.\\
      \textbf{Yeleda:} & Czy to jest jedyne okno na tym poziomie?\\
      \multicolumn{2}{l}{\textit{Z pokrętnym uśmiechem:}}\\
      \textbf{MG:} & ...nie.\\
   \end{tabularx}
\end{rpg-quotebox}

\begin{rpg-quotebox}{Jeśli coś wydaje się głupie ale działa\, to nie jest głupie}
   \textit{Drużyna zastanawiała się czy przebijać się przez kolejne piętra zarośniętej latarni morskiej czy wybrać szybszy sposób.}\\

   \begin{tabularx}{\columnwidth}{lX}
      \textbf{Calion:} & Proponuję zejść na dół przez pnącza, a jak coś to wrócimy na górę i wyskoczymy przez okno.\\
   \end{tabularx}
\end{rpg-quotebox}

\begin{rpg-quotebox}{O kiepskiej współpracy}
   \textit{Vivien rozmawiała z BN-em (druidaem), którego odratowali z latarni. Rozmowa polegała na wymienianiu informacji na temat panującej inwazji oraz planów co do dalszych ruchów.}\\
   
   \begin{tabularx}{\columnwidth}{lX}
      \multicolumn{2}{l}{\textit{W tle, do MG:}}\\
      \textbf{Garret:} & Podpaliliśmy pnącza!\\
   \end{tabularx}
\end{rpg-quotebox}

\begin{rpg-quotebox}{Zawsze to jakieś uczucie}
   \begin{tabularx}{\columnwidth}{lX}
      \textbf{Vivien:} & Leczę Garreta z 3-ego slota!\\
      \textbf{MG:} & To jest miłość!\\
      \textbf{Calion:} & \multirow{2}{*}{Ona przecież mówiła, że nie legnie z niziołkiem.}\\
      \textbf{Yeleda:} & \\
      \textbf{Vivien:} & Mogę go leczyć, nie muszę całować.\\
   \end{tabularx}
\end{rpg-quotebox}

\begin{rpg-quotebox}{Grunt to umieć ruszyć głową}
   \begin{tabularx}{\columnwidth}{lX}
      \textbf{DM:} & Garret po oberwaniu kamieniem od Giganta ma na czole odbity stopień.\\
   \end{tabularx}
\end{rpg-quotebox}

\begin{rpg-quotebox}{Nierówny poziom}
   \textit{Drużyna zdecydowała zejść do kanałów.}\\
   
   \begin{tabularx}{\columnwidth}{lX}
      \textbf{MG:} & Ścieków będzie tak po kolana, Garret będzie miał do pasa.\\
   \end{tabularx}
\end{rpg-quotebox}

\begin{rpg-quotebox}{Dobła}
   \textit{MG ma problem z wymawianiem głoski ,,r''.}\\
   
   \begin{tabularx}{\columnwidth}{lX}
      \textbf{MG:} & Tu był magazyn przemycanych dóbr.\\
      \textbf{Yeleda:} & Dupy przemycali - rude, blondynki co tylko chcesz.\\
   \end{tabularx}
\end{rpg-quotebox}

\begin{rpg-quotebox}{Zawsze to jakaś taktyka}
   \textit{Korytarz zagradzał potwór wyglądający niczym plątanina ciernistych lian. Drużyna zastanawiała się czy z nim walczyć, czy iść do celu okrężną drogą.}\\

   \begin{tabularx}{\columnwidth}{lX}
      \textbf{Yeleda:} & Ok czyli bitka! Ukrywam się!\\
   \end{tabularx}
\end{rpg-quotebox}

\begin{rpg-quotebox}{Cuchnący i parszywy los}
   \textit{Potwór był dość mocnym przeciwnikiem i powalił Garreta tam gdzie stał - czyli w śmierdzącym ścieku.}\\
   
   \begin{tabularx}{\columnwidth}{lX}
      \textbf{MG:} & Jesteś nieprzytomny i w dodatku topisz się w brei.\\
      \textbf{Garret:} & Jestem mały i zwinny, na pewno dryfuję.\\
      \textbf{MG:} & Twarzą w dół.\\
   \end{tabularx}
\end{rpg-quotebox}

\begin{rpg-quotebox}{Chciał dobrze}
\textit{Calion miał do wyboru dobić potwora lub ustabilizować wykrwawiającego się Garreta i narazić się na okazjonalny atak potwora. Jego dobry charakter wybrał za niego:}\\
   
   \begin{tabularx}{\columnwidth}{lX}
      \textbf{Calion:} & Stabilizuję Garreta, korzystam z Medycyny.\\
      \includegraphics[scale=0.06]{img/d20.png}\textbf{:}& 1\\
      \textbf{MG:} & Dobiłeś go.\\
      \multicolumn{2}{l}{\textit{Po chwili:}}\\
      \textbf{MG:} & Dobra, niech będzie. Już się nie wykrwawia.\\
   \end{tabularx}
\end{rpg-quotebox}

\begin{rpg-quotebox}{Opis w pełni zgodny z naturą}
   \begin{tabularx}{\columnwidth}{lX}
   \textbf{MG:} & Garret wspina się po drabinie niczym rącza łania.\\
   \end{tabularx}
\end{rpg-quotebox}

\begin{rpg-quotebox}{Zabawy słowne}
   \begin{tabularx}{\columnwidth}{lX}
      \multicolumn{2}{l}{\textit{Poza postacią, do irytujących BN-ów:}}\\
      \textbf{Vivien:} & Już dawno zrobiłabym wam z kutasów pnącza.\\
      \textbf{Garret:} & Pnącia! Kutaliany!\\
   \end{tabularx}
\end{rpg-quotebox}

\begin{rpg-quotebox}{Logiczne wnioski}
   \textit{Obserwacja jednego ze strażników po tym jak puszczono drużynę wolno, ale ciągle mając na nich oko:}\\
   
   \begin{tabularx}{\columnwidth}{lX}
      \textbf{Strażnik:} & Ta banda wkradła się do kanałów, przedarła się przez ich śmierdzące wnętrza, pobiła potwory z pnączy tylko po to by włamać się do spalonej kamienicy i urządzić sobie w niej melinę. Nie, oni nie stanowią zagrożenia dla miasta.\\
   \end{tabularx}
\end{rpg-quotebox}


\section*{9 grudnia 2017}

\begin{rpg-quotebox}{O zaufaniu w drużynie}
   \begin{tabularx}{\columnwidth}{lX}
      \textbf{Vivien:} & Nie wchodzę do walki w zwarciu, bo Calion coś kombinuje.\\
   \end{tabularx}
\end{rpg-quotebox}

\begin{rpg-quotebox}{O sposobie komunikacji}
   \textit{Waląc kosturem w drzwi pokoju w którym śpi Yeleda.}\\

   \begin{tabularx}{\columnwidth}{lX}
      \textbf{Vivien:} &  To nie jest alfabet Morse'a to alfabet desperata.\\
   \end{tabularx}
\end{rpg-quotebox}

\begin{rpg-quotebox}{O tym jak wykształcił się instynkt przetrwania}
   \begin{tabularx}{\columnwidth}{lX}
      \textbf{MG:} & Słyszycie stukanie.\\
      \textbf{Yeleda:} & Ukrywam się!\\
      \textbf{Calion:} & Ukrywam się!\\
      \textbf{Vivien:} & Ukrywam się! \\
      \textbf{Garret:} & Ukrywam się!\\
   \end{tabularx}
\end{rpg-quotebox}


\begin{rpg-quotebox}{Podsumowanie rzutu}
   \begin{tabularx}{\columnwidth}{lX}
      \textbf{MG:} & Rzuć na trafienie.\\
      \textbf{Calion:} & Lekko pół chujowo.\\
      \textbf{MG:} & Nie trafiasz...\\
   \end{tabularx}
\end{rpg-quotebox}

\begin{rpg-quotebox}{O reakcji na długą nieobecność}
   \begin{tabularx}{\columnwidth}{lX}
      \textbf{Calion:} & Kruk wkurwia.\\
      \textbf{Garret:} & To on jeszcze żyje?\\
   \end{tabularx}
\end{rpg-quotebox}

\begin{rpg-quotebox}{O różnych miejscach w które słońce nie dociera}
   \begin{tabularx}{\columnwidth}{lX}
      \textbf{Vivien:} & Ten druid chyba żałuje, że Cię zaatakował kijem.\\
      \textbf{Garret:} & Zaraz mu go wsadzę... w oczodół.\\
   \end{tabularx}
\end{rpg-quotebox}

\begin{rpg-quotebox}{Szczegóły mają znaczenie}
   \textit{Drużyna błądziła w kanałach i Garret postanowił wybrać kierunek na ślepy traf.}\\

   \begin{tabularx}{\columnwidth}{lX}
      \textbf{MG:} & Czujesz bryzę.\\
      \textbf{Garret:} & Bryzę czy bryzg?\\
   \end{tabularx}
\end{rpg-quotebox}

\begin{rpg-quotebox}{O tym że nie zawsze wygląda się na tyle lat ile się ma}
   \begin{tabularx}{\columnwidth}{lX}
      \textbf{Calion:} & Viv ile ty masz lat?\\
      \textbf{Vivien:} & 190 z hakiem.\\
      \textbf{Garret:} & Och! To już jesteś pełnoletnia.\\
   \end{tabularx}
\end{rpg-quotebox}

\begin{rpg-quotebox}{O pomyłkach ornitologicznych}
   \begin{tabularx}{\columnwidth}{lX}
      \textbf{MG:} & Ptak skacząc po okolicy nadziewa się na ukrytą Vivien.\\
      \textbf{Vivien:} & Wszedł we mnie?\\
   \end{tabularx}
\end{rpg-quotebox}

\begin{rpg-quotebox}{O tym jak kreskówki mogą niszczyć życie}
   \begin{tabularx}{\columnwidth}{lX}
      \textbf{MG:} & Garret w wyniku szarpnięcia spada z klifu i wisi tylko na rękach wzdłuż urwiska.\\
      \textbf{Vivien:} & Odgrywam scenę z Króla Lwa.\\
      \multicolumn{2}{l}{\textit{Unisono:}}\\
      \textbf{Yeleda:} & \multirow{3}{*}{Nie!}\\
      \textbf{Calion:} & \\
      \textbf{Garret:} & \\
   \end{tabularx}
\end{rpg-quotebox}

\begin{rpg-quotebox}{A tym razem o filmach}
   \begin{tabularx}{\columnwidth}{lX}
      \textbf{MG:} & Widzicie dużo ptaków przelatujących obok waszej kryjówki.\\
      \textbf{Calion:} & Ja się stresuję jako gracz.\\
   \end{tabularx}
\end{rpg-quotebox}

\begin{rpg-quotebox}{O tym jak MG pokochała ornitologię}
   \begin{tabularx}{\columnwidth}{lX}
      \textbf{Calion:} & Czy widzę coś innego niż mewy?\\
      \textbf{MG:} & Orła.\\
   \end{tabularx}
\end{rpg-quotebox}

\begin{rpg-quotebox}{Może już wystarczy tych ptaków?}
   \begin{tabularx}{\columnwidth}{lX}
      \textbf{MG:} & Gryfy mają bardzo dobry wzrok.\\
      \textbf{Vivien:} & Sokoli?\\
      \textbf{MG:} & Gryfi.\\
   \end{tabularx}
\end{rpg-quotebox}

\begin{rpg-quotebox}{O normalnych miejscach na nocleg}
   \begin{tabularx}{\columnwidth}{lX}
      \textbf{Garret:} & To będzie dziwne, że nagle w środku nocy do gospody wchodzi grupa ludzi.\\
      \textbf{Vivien:} & A co? Nikt normalny nie śpi za dnia w gnieździe gryfów?\\
   \end{tabularx}
\end{rpg-quotebox}

\begin{rpg-quotebox}{O tym jak spotkała się geografia\, botanika\, gastrologia i proktologia}
   \begin{tabularx}{\columnwidth}{lX}
      \textbf{Garret:} & Ciekawe jakie owoce są w Feywild? Np. jakie jagody?\\
      \textbf{Vivien:} & Takie, że jak nie wyplujesz pestki to umrzesz.\\
      \textbf{Garret:} & A jak wyjdą inną stroną?\\
      \textbf{Vivien:} & Jagody, nie czopki!\\
   \end{tabularx}
\end{rpg-quotebox}

\section*{5 stycznia 2018}

\begin{rpg-quotebox}{Tyle możliwości}
   \begin{tabularx}{\columnwidth}{lX}
      \textbf{MG:} & Wchodzisz do korytarza. Widzisz dwóch mężczyzn i 10 drzwi.\\
   \end{tabularx}
\end{rpg-quotebox}

\begin{rpg-quotebox}{O tym jak niewielki wzrost jednak jest do czegoś przydatny}
   \begin{tabularx}{\columnwidth}{lX}
      \textbf{MG:} & Nadlatuje w twoim kierunku 15 małych strzałek, ale jesteś niski i unikasz obrażeń.\\
      \textbf{Vivien:} & Jesteś poniżej linii strzału.\\
      \textbf{MG:} & Ale powyżej ST.\\
   \end{tabularx}
\end{rpg-quotebox}

\begin{rpg-quotebox}{No i cały misterny plan...}
   \begin{tabularx}{\columnwidth}{lX}
      \textbf{MG:} & Yeleda, Ty nawigujesz.\\
      \includegraphics[scale=0.06]{img/d20.png}\textbf{:}& 2\\
      \textbf{Yeleda:} & Do dupy na raki my poszli.\\
   \end{tabularx}
\end{rpg-quotebox}

\begin{rpg-quotebox}{O konsekwencjach porażki}
   \textit{Calion podbiegł do uzbrojonego, wodnego żywiołaka z wyczarowaną magiczną bronią, ale jego atak nie trafił}\\
   
   \begin{tabularx}{\columnwidth}{lX}
      \textbf{MG:} & Czy coś jeszcze robisz?\\
      \textbf{Calion:} & Wspominam całe swoje życie.\\
   \end{tabularx}
\end{rpg-quotebox}

\begin{rpg-quotebox}{O liczbach\, które przeraziły}
   \textit{Yeleda zadała żywiołakowi strzałem z łuku 11 punktów obrażeń.}\\

   \begin{tabularx}{\columnwidth}{lX}
      \textbf{MG:} & Brawo! Żywiołak zszedł do liczby dwucyfrowej.\\
   \end{tabularx}
\end{rpg-quotebox}

\begin{rpg-quotebox}{O braku empatii uzdrowicieli}
   \begin{tabularx}{\columnwidth}{lX}
      \textbf{Garret:} & Regeneruję sobie 18 punktów życia.\\
      \textbf{Vivien:} & Jednego do leczenia mniej.\\
   \end{tabularx}
\end{rpg-quotebox}

\begin{rpg-quotebox}{Z innej perspektywy już nie taki niski}
   \begin{tabularx}{\columnwidth}{lX}
      \multicolumn{2}{l}{\textit{Leżąc na ziemi:}}\\
      \textbf{Yeleda:} & Wstaję i ukrywam się.\\
      \textbf{MG:} & W tych warunkach jesteś się w stanie schować za Garretem.\\
   \end{tabularx}
\end{rpg-quotebox}

\begin{rpg-quotebox}{O krępującej sytuacji}
   \begin{tabularx}{\columnwidth}{lX}
      \textbf{Garret:} & Związuję go.\\
      \textbf{Yeleda:} & Przyglądam się mu.\\
      \multicolumn{2}{l}{\textit{Vivien rzuca Guidance na Yeledę.}}\\
      \textbf{Calion:} & Mamy związanego faceta i dwie laski, które się macają.\\
   \end{tabularx}
\end{rpg-quotebox}

\begin{rpg-quotebox}{O wielkich powodach do obaw}
   \begin{tabularx}{\columnwidth}{lX}
      \textbf{Garret:} & Skoro nie chce mówić, trzeba go będzie utopić w kałuży.\\
      \textbf{MG:} & Rzuć na zastraszenie.\\
      \includegraphics[scale=0.06]{img/d20.png}\textbf{:}& 20\\
   \end{tabularx}
\end{rpg-quotebox}

\begin{rpg-quotebox}{Zgonospis}
   \begin{tabularx}{\columnwidth}{lX}
      \multicolumn{2}{l}{\textit{Do BN-ego:}}\\
      \textbf{Yeleda:} & Jak chcesz umrzeć? Sztylet, miecz, łuk?\\
      \textbf{MG:} & Taaak... wyciągnij menu.\\
   \end{tabularx}
\end{rpg-quotebox}

\begin{rpg-quotebox}{O niskim szacunku\, jakim darzy się niziołków}
   \begin{tabularx}{\columnwidth}{lX}
      \textbf{Yeleda:} & Ja zmienię wizerunek, ale wy też powinniście się byli przebrać. Ale co zrobić z niziołkiem?\\
      \textbf{Vivien:} & Przebierzemy go za dziecko, wsadzimy do wózka i tyle.\\
   \end{tabularx}
\end{rpg-quotebox}

\section*{14 stycznia 2018}

\begin{rpg-quotebox}{To nie jest jego mocna strona}
   \begin{tabularx}{\columnwidth}{lX}
      \includegraphics[scale=0.06]{img/d20.png} \textbf{Calion:} & 12\\
      \includegraphics[scale=0.06]{img/d20.png} \textbf{Yeleda:} & 20\\
      \includegraphics[scale=0.06]{img/d20.png} \textbf{Vivien:} & 21\\
      \includegraphics[scale=0.06]{img/d20.png} \textbf{Garret:} & 20\\
      \textbf{MG:} & Rozumiem, że Calion trochę mniej klekocze.\\
   \end{tabularx}
\end{rpg-quotebox}

\begin{rpg-quotebox}{Jej natomiast tak}
   \begin{tabularx}{\columnwidth}{lX}
      \textbf{MG:} & Yeleda w końcu dojdzie do tego, że będzie podwajać biegłość na skradanie.\\
      \textbf{Yeleda:} & Who's the master?\\
      \textbf{Vivien:} & Nikogo takiego nie widzę.\\
   \end{tabularx}
\end{rpg-quotebox}

\begin{rpg-quotebox}{O tym\, że łatwo jest upaść}
   \textit{Podczas schodzenia ze stromej ściany:}\\

   \begin{tabularx}{\columnwidth}{lX}
      \textbf{MG:} & Yeleda jest asekurowana przez Garreta i jakoś leci.\\
      \textbf{Yeleda:} & Dobrze, że nie ryjem w dół.\\
      \textbf{Calion:} & Yy! To moja robota!\\
   \end{tabularx}
\end{rpg-quotebox}

\begin{rpg-quotebox}{O niebywałej akrobatyce}
   \begin{tabularx}{\columnwidth}{lX}
      \textbf{Garret:} & Skaczę na najbliższą półkę.\\
      \includegraphics[scale=0.06]{img/d20.png}\textbf{:} & 20\\
      \textbf{MG:} & Lądujesz tak ślicznie, że harpia prawie bije brawo.\\
   \end{tabularx}
\end{rpg-quotebox}

\begin{rpg-quotebox}{Próba to czasem zbyt mało}
   \begin{tabularx}{\columnwidth}{lX}
      \textbf{Garret:} & Czy sięgnę ręką po ataku włócznią?\\
      \textbf{DM:} & Gdybyś próbował skoczyć, walnąć i wylądować to...\\
      \textbf{Calion:} & To brzmi groźnie, zwłaszcza, że padło słowo ,,próbował''.\\
   \end{tabularx}
\end{rpg-quotebox}

\begin{rpg-quotebox}{Nie o takie zagrożenie chodziło}
   \begin{tabularx}{\columnwidth}{lX}
      \textbf{Garret:} & Czy w tej wodzie jest coś podejrzanego?\\
      \textbf{Vivien:} & Bakteria e-coli.\\
   \end{tabularx}
\end{rpg-quotebox}

\begin{rpg-quotebox}{O wszędobylskiej ekologii}
   \begin{tabularx}{\columnwidth}{lX}
      \textbf{Calion:} & Czyli mamy kwaśne deszcze, lasy państwowe i co jeszcze??\\
   \end{tabularx}
\end{rpg-quotebox}

\begin{rpg-quotebox}{O tym jak przeznaczenia uniknąć się nie da}
   \begin{tabularx}{\columnwidth}{lX}
      \textbf{Garret:} &  Sprawdzam pobudki topielców.\\
      \includegraphics[scale=0.06]{img/d20.png}\textbf{:}& 1\\
      \textbf{Garret:} &  Przerzucam.\\
      \includegraphics[scale=0.06]{img/d20.png}\textbf{:}& 1\\
   \end{tabularx}
\end{rpg-quotebox}

\begin{rpg-quotebox}{Początki skoku o tyczce}
   \begin{tabularx}{\columnwidth}{lX}
      \textbf{MG:} & Drogę zastawia wam porzucony wóz.\\
      \textbf{Vivien:} & Wykonuję skok o kosturze w celu przeskoczenia przeszkody!\\
   \end{tabularx}
\end{rpg-quotebox}

\begin{rpg-quotebox}{Miłe złego początki}
   \textit{Kolejne rzuty na skradanie się.}\\

   \begin{tabularx}{\columnwidth}{lX}
      \includegraphics[scale=0.06]{img/d20.png}\textbf{:}& 20\\
      \includegraphics[scale=0.06]{img/d20.png}\textbf{:}& 18\\
      \includegraphics[scale=0.06]{img/d20.png}\textbf{:}& 12\\
      \includegraphics[scale=0.06]{img/d20.png}\textbf{:}& 10\\
      \textbf{Yeleda:} & Mamy tendencję spadkową.\\
   \end{tabularx}
\end{rpg-quotebox}

\begin{rpg-quotebox}{O braku przyzwyczajenia do magicznych spektakli}
   \textit{W karczmie w Zielonej Jabłoni widać ciągle zmieniających się druidów w ich zwierzęce formy.}\\

   \begin{tabularx}{\columnwidth}{lX}
      \textbf{Garret:} & Chyba nigdy się do tego nie przyzwyczaję.\\
      \textbf{Vivien:} & To muszę jeszcze nad tobą popracować.\\
   \end{tabularx}
\end{rpg-quotebox}

\begin{rpg-quotebox}{O klarownej i jasnej odpowiedzi}
   \begin{tabularx}{\columnwidth}{lX}
      \textbf{Yeleda:} & Calion, ty to powiedziałeś do mnie czy do niego?\\
      \textbf{Calion:} & Tak.\\
      \textbf{Vivien:} & High elf!\\
   \end{tabularx}
\end{rpg-quotebox}

\begin{rpg-quotebox}{O dziwnych żądaniach wiedźm}
   \begin{tabularx}{\columnwidth}{lX}
      \textbf{Garret:} & Czego wiedźmy mogłyby oczekiwać w zamian za pomoc?\\
      \textbf{DM:} & Najprawdopodobniej czegoś metafizycznego.\\
      \textbf{Yeleda:} & Przejścia na buddyzm.\\
   \end{tabularx}
\end{rpg-quotebox}

\begin{rpg-quotebox}{W sumie to jeden pies}
   \begin{tabularx}{\columnwidth}{lX}
      \textbf{MG:} & Wchodzicie do karczmy i czujecie zapach mokrego psa.\\
      \textbf{Yeleda:} & Wilczur czy jamnik?\\
   \end{tabularx}
\end{rpg-quotebox}

\begin{rpg-quotebox}{O zmianie sposobu walki}
   \begin{tabularx}{\columnwidth}{lX}
      \textbf{Garret:} & Chciałbym prostą włócznię.\\
      \textbf{Yeleda:} & CSI już cię nie rozpozna po śladzie pięści.\\
   \end{tabularx}
\end{rpg-quotebox}

\begin{rpg-quotebox}{O tym jak niewiele jest potrzeba do diametralnych różnic}
   \begin{tabularx}{\columnwidth}{lX}
      \textbf{Yeleda:} & Rzucam na trafienie z przewagą\\
      \includegraphics[scale=0.06]{img/d20.png},\includegraphics[scale=0.06]{img/d20.png}\textbf{:}& 20, 1\\
   \end{tabularx}
\end{rpg-quotebox}

\begin{rpg-quotebox}{O tym jak magowie zawsze preferują walkę na dystans}
   \begin{tabularx}{\columnwidth}{lX}
      \textbf{MG:} & Została jedna harpia z wbitą w ciało włócznią Garreta.\\
      \textbf{Calion:} & Rzucam w nią mieczem!\\
   \end{tabularx}
\end{rpg-quotebox}

\begin{rpg-quotebox}{O tym jak zbliżenia nie zawsze są satysfakcjonujące dla obydwu stron}
   \textit{Garret zabił przebitą włócznią harpię i gdy spadała zdołał wyszarpnąć z niej swoją broń.}\\

   \begin{tabularx}{\columnwidth}{lX}
      \textbf{Vivien:} & Najpierw ją nadziałeś, a potem się z niej wysunąłeś.\\
      \textbf{Garret:} & Została krytycznie spenetrowana.\\
   \end{tabularx}
\end{rpg-quotebox}

\section*{17 stycznia 2018}

\begin{rpg-quotebox}{I tak już nic nie uratuje}
   \textit{Rzucamy na percepcję.}\\

   \begin{tabularx}{\columnwidth}{lX}
      \textbf{Garret:} & To mam 3 na kostce plus...\\
      \textbf{Calion:} & Plus nieważne, next!\\
   \end{tabularx}
\end{rpg-quotebox}

\begin{rpg-quotebox}{Z braku laku...}
   \begin{tabularx}{\columnwidth}{lX}
      \multicolumn{2}{l}{\textit{W reakcji na brak strzałek:}}\\
      \textbf{Garret:} & Wyciągam pochodnię i rzucam tym patykiem w pnącza.\\
   \end{tabularx}
\end{rpg-quotebox}

\begin{rpg-quotebox}{Wunsz\!}
   \textit{Walczymy z jakimś niewidzialnym potworem na którego drużyna miała utrudnienie w ataku.}\\

   \begin{tabularx}{\columnwidth}{lX}
      \textbf{MG:} & Vivien, twoja kolej.\\
      \textbf{Vivien:} & Zamieniam się w boa dusiciela, który ma blindsight!\\
      \textbf{MG:} & Ekhem.... Widzisz żywiołaka powietrza.\\
      \textbf{Żywiołak Powietrza:} & WTF! Skąd tu ten wąż?!\\
   \end{tabularx}
\end{rpg-quotebox}

\begin{rpg-quotebox}{Bić\, czy nie bić - oto jest pytanie}
   \begin{tabularx}{\columnwidth}{lX}
      \textbf{Calion:} & Rzucam na trafienie magicznym mieczem z przewagą.\\
      \includegraphics[scale=0.06]{img/d20.png},\includegraphics[scale=0.06]{img/d20.png}\textbf{:}& 20, 1\\
      \textbf{DM:} & Calion wyczarowywuje miecz i w czasie gdy zamierza nim rzucić w potwora zaczyna się nagle zastanawiać, czy przemoc jest rozwiązaniem.\\
      \textbf{Vivien:} & Ja już nigdy nie powiem, że mam pecha.\\
   \end{tabularx}
\end{rpg-quotebox}

\begin{rpg-quotebox}{Nie pojawił się\, to i nie znika}
   \begin{tabularx}{\columnwidth}{lX}
      \textbf{Vivien:} & Czy niewidzialny żywiołak powietrza ujawnia się po śmierci?\\
      \textbf{MG:} & Nie.\\
      \textbf{Vivien:} & Czyli Yeleda nie widziała go ani za życia, ani po śmierci.\\
   \end{tabularx}
\end{rpg-quotebox}

\begin{rpg-quotebox}{O dennych żartach}
   \begin{tabularx}{\columnwidth}{lX}
      \textbf{MG:} & Ok, jesteście na etapie podrywania laski z wody.\\
      \textbf{Garret:} & Plus jest taki, że jest już mokra.\\
      \textbf{Vivien:} & Poziom żartów wynika z tego, że dochodzimy do 22:00.\\
   \end{tabularx}
\end{rpg-quotebox}

\begin{rpg-quotebox}{O porzuconych trudach}
   \begin{tabularx}{\columnwidth}{lX}
      \textbf{MG:} & Syrena odzywa się we wspólnym.\\
      \textbf{Calion:} & Pierdolę dalszy rytuał rozpoznawania języka.\\
   \end{tabularx}
\end{rpg-quotebox}

\begin{rpg-quotebox}{Głodnemu chleb na myśli}
   \begin{tabularx}{\columnwidth}{lX}
      \multicolumn{2}{l}{\textit{Do Garreta:}}\\
      \textbf{Syreny:} & Jesteś taki uroczy. Możemy Cię porwać?\\
      \textbf{Garret:} & Wybaczcie, ale jestem mnichem i składałem śluby czystości.\\
      \textbf{Calion:} & Eej wróć! Veto!\\
   \end{tabularx}
\end{rpg-quotebox}

\begin{rpg-quotebox}{Pion jaki jest każdy widzi}
   \begin{tabularx}{\columnwidth}{lX}
      \textbf{MG:} & Żywiołak wali cię w łeb, mam nadzieję że jeszcze stoisz.\\
      \textbf{Vivien:} & Syrena pokazała cycki więc tak.\\
   \end{tabularx}
\end{rpg-quotebox}

\begin{rpg-quotebox}{O tym jak zawsze przerywa się w najlepszych momentach}
   \begin{tabularx}{\columnwidth}{lX}
      \textbf{Yeleda:} & Chyba musimy się zbierać.\\
      \textbf{Garret:} & Ale zaczęło się robić interesująco!\\
      \textbf{Calion:} & Spójrzmy prawdzie w oczy - i tak nie zamoczysz.\\
   \end{tabularx}
\end{rpg-quotebox}


\section*{23 lutego 2018}

\begin{rpg-quotebox}{O niecodziennym podejściu do identyfikacji}
   \begin{tabularx}{\columnwidth}{lX}
      \textbf{MG:} & Garret robi małego maszka w celu sztachnięcia się mgłą, żeby ją zidentyfikować.\\
   \end{tabularx}
\end{rpg-quotebox}

\begin{rpg-quotebox}{O tym że gracze czasem są zagubieni jak dzieci}
   \begin{tabularx}{\columnwidth}{lX}
      \textbf{MG:} & Vivien łapie Yeledę za rączkę, Yeleda macha swoją. Kto ją łapie?\\
   \end{tabularx}
\end{rpg-quotebox}

\begin{rpg-quotebox}{O wiedzy posiadanej lecz nie do końca rozumianej}
   \begin{tabularx}{\columnwidth}{lX}
      \textbf{MG:} & Calion, Ty masz wiedzę na temat lokalizacji niektórych przejść do planu Feywild.\\
      \textbf{Calion:} & \multirow{4}{*}{Wow!}\\
      \textbf{Garret:} & \\
      \textbf{Vivien:} & \\
      \textbf{Yeleda:} & \\
      \textbf{MG:} & Vivien, Ty wiesz o takich przejściach do planów żywiołów.\\
      \textbf{Vivien:} & Aha, dobrze wiedzieć.\\
   \end{tabularx}
\end{rpg-quotebox}

\begin{rpg-quotebox}{O lapsusach}
   \begin{tabularx}{\columnwidth}{lX}
      \textbf{MG:} & Drugi druid rzuca w Ciebie truciskiem pocizny.\\
   \end{tabularx}
\end{rpg-quotebox}

\begin{rpg-quotebox}{O złym wyborze czasu}
   \begin{tabularx}{\columnwidth}{lX}
      \textbf{MG:} & Calion, twoja tura.\\
      \textbf{Calion:} & Wertuję książkę.\\
      \textbf{Vivien:} & Nie masz kiedy tego robić!?\\
   \end{tabularx}
\end{rpg-quotebox}

\begin{rpg-quotebox}{O braku odporności na zazdrość}
   \begin{tabularx}{\columnwidth}{lX}
      \textbf{MG:} & Dostajesz 22 pkt. obrażeń od trucizny.\\
      \textbf{Garret:} & Jestem odporny jako Osiadły Niziołek.\\
      \textbf{Calion:} & Bullshit!\\
   \end{tabularx}
\end{rpg-quotebox}

\section*{2 marca 2018}

\begin{rpg-quotebox}{}
   \textit{Calion miał wchodzić po linie asekurowany przez Garreta.}\\

   \begin{tabularx}{\columnwidth}{lX}
      \multicolumn{2}{l}{\textit{Do Garreta:}}\\
      \textbf{Vivien:} & Wciągnij Caliona nosem.\\
   \end{tabularx}
\end{rpg-quotebox}

\begin{rpg-quotebox}{O składzie chemicznym mnicha}
   \begin{tabularx}{\columnwidth}{lX}
      \textbf{MG:} & Garret ty się składasz w 10\% z mięśni i w 90\% z kopa.\\
   \end{tabularx}
\end{rpg-quotebox}

\begin{rpg-quotebox}{O perypetiach w nawigowaniu}
   \begin{tabularx}{\columnwidth}{lX}
      \textbf{MG:} & Idziecie w kierunku Zielonej Jabłoni, Vivien prowadzi i wszystko widzi. Nawet jeśli coś na was czyhało to w porę zdołała was schować w trawach lub za skałami.\\
      \textbf{Garret:} & Albo w grobie.\\
   \end{tabularx}
\end{rpg-quotebox}

\begin{rpg-quotebox}{}
   \begin{tabularx}{\columnwidth}{lX}
      \textbf{Calion:} & Jakie są obrażenia gdy ktoś nadepnie na druida pod postacią pająka?\\
   \end{tabularx}
\end{rpg-quotebox}

\begin{rpg-quotebox}{Z niejednego młyna mąkę... czy jakoś tak...}
   \textit{Taliesin urządził sobie siedzibę w młynie.}\\

   \begin{tabularx}{\columnwidth}{lX}
      \multicolumn{2}{l}{\textit{Do Taliesina:}}\\
      \textbf{Vivien:} & Widzę, że macie tu niezłe zamieszanie.\\
      \textbf{Yeleda:} & Macie tu niezły młyn.\\
   \end{tabularx}
\end{rpg-quotebox}

\begin{rpg-quotebox}{Jasełka i Smoki}
   \begin{tabularx}{\columnwidth}{lX}
      \textbf{MG:} & Nie było dla was miejsca w gospodzie.\\
      \textbf{Vivien:} & A w stajence?\\
      \textbf{Yeleda:} & Viv jako Maria, Calion jako Józef, Garret - Jezus.\\
      \textbf{Vivien:} & A ty co? Gwiazda Betlejemska?\\
   \end{tabularx}
\end{rpg-quotebox}

\begin{rpg-quotebox}{O niepokalanym poczęciu}
   \begin{tabularx}{\columnwidth}{lX}
\textbf{Vivien:} & Przyszła mi okropna myśl: mam z Calionem Garreta.\\
\textbf{Calion:} & Bullshit!\\
\textbf{DM:} & To był Bóg...\\
\textbf{Vivien:} & Ok, to mam Garreta z Obad-Haiem a Calion udaje że jest jego ojcem.\\
\textbf{Calion:} & Jeszcze gorzej...\\
   \end{tabularx}
\end{rpg-quotebox}

\begin{rpg-quotebox}{Potrzeba matką wynalazków}
\textit{W tle Yeleda dumała skąd wziąć strzały}\\

   \begin{tabularx}{\columnwidth}{lX}
   \multicolumn{2}{l}{\textit{Do Vivien:}}\\
      \textbf{Calion:} & Czy umiesz się zmienić  w  coś co ma pióra?\\
      \textbf{Vivien:} & Yyy... coś co nie lata.\\
      \textbf{MG:} & Struś?\\
      \textbf{Yeleda:} & Będziesz e-Mu.\\
      \textbf{Vivien:} & Ale zaraz po co wy to chcecie?\\
      \textbf{Yeleda:} & No na lotki do strzał!\\
      \textbf{Vivien:} & I chcesz mi dupę oskubać?? Chyba cię boli!\\
      \textbf{Calion:} & Ile to będzie punktów obrażeń?\\
      \textbf{MG:} & Jak już jakieś by były to z kopniaka w kierunku Yeledy.\\
      \textbf{Yeleda:} & Ale może się ukryję? Nie wiedziałaby skąd to?\\
      \textbf{Calion:} & Od dupy strony.\\
   \end{tabularx}
\end{rpg-quotebox}

\begin{rpg-quotebox}{O odporności na wiedzę}
   \textit{Calion uczy Vivien rytuałów ze znalezionej księgi, ale opornie to idzie.}\\

   \begin{tabularx}{\columnwidth}{lX}
      \textbf{MG:} & Uczycie się i nie wiadomo kto kogo chce bardziej zatłuc tą książką.\\
   \end{tabularx}
\end{rpg-quotebox}

\section*{18 kwietnia 2018}

\begin{rpg-quotebox}{Tonący brzytwy się chwyta}
   \begin{tabularx}{\columnwidth}{lX}
      \textbf{MG:} & Garret, ze swojej pozycji widzisz dwa lecące w ciebie oszczepy.\\
      \textbf{Vivien:} & Dmuchaj.\\
   \end{tabularx}
\end{rpg-quotebox}

\begin{rpg-quotebox}{Poduszka do igieł}
   \begin{tabularx}{\columnwidth}{lX}
      \textbf{MG:} & Rzut na trafienie dla pierwszego oszczepu - 22.\\
      \textbf{Garret:} & Nie trafia!\\
      \textbf{MG:} & To był żart?\\
      \textbf{Garret:} & Moje AC + poziom desperacji = 500.\\
   \end{tabularx}
\end{rpg-quotebox}

\begin{rpg-quotebox}{I cały misterny plan w pizdu...}
   \begin{tabularx}{\columnwidth}{lX}
      \textbf{Garret:} & Łapię pocisk i odrzucam w kierunku atakującego.\\
      \includegraphics[scale=0.06]{img/d20.png}\textbf{:}& 10\\
      \textbf{MG:} & Nie trafiasz i oszczep ląduje obok.\\
      \multicolumn{2}{l}{\textit{Do Garreta:}}\\
      \textbf{Yeleda:} & Czy ty jesteś świadom, że oddałeś mu broń?\\
   \end{tabularx}
\end{rpg-quotebox}

\begin{rpg-quotebox}{Sokole oko}
   \begin{tabularx}{\columnwidth}{lX}
      \textbf{MG:} & Tylko Yeleda widzi, że dwóch włóczników podbiega do drużyny od tyłu, nie spodziewali się jednak chmury dymu.\\
   \end{tabularx}
   ~\newline
   \textit{Yeleda strzela krytycznie w kierunku jednego z nich i zadaje 44 punkty obrażeń.}\\

   \begin{tabularx}{\columnwidth}{lX}
      \textbf{MG:} & Włócznik biegnie i myśli "kurwa co to za chmura?". Po czym jego mózg eksploduje od krytycznej strzały Yeledy.\\
   \end{tabularx}
\end{rpg-quotebox}


\begin{rpg-quotebox}{Mów mi Flash}
   \begin{tabularx}{\columnwidth}{lX}
      \textbf{MG:} & Ok, to dziewczyny idą na górę penetrować piętro a Garret oświadcza, że dochodzi najszybciej.\\
   \end{tabularx}
\end{rpg-quotebox}


\begin{rpg-quotebox}{Metaforyczne zdolności manualne}
   \textit{Na temat penetracji pokoi:}\\

   \begin{tabularx}{\columnwidth}{lX}
      \textbf{Vivien:} &  Yeleda, idziemy po facetów? Czy robimy to na własną rękę?\\
   \end{tabularx}
\end{rpg-quotebox}


\begin{rpg-quotebox}{O problemach z gotowością bojową}
   \textit{Garret próbował zaatakować potwora dwa razy przy użyciu włóczni i dwa razy nie trafił. Wtedy postanowił poprawić pięścią.}\\

   \begin{tabularx}{\columnwidth}{lX}
      \textbf{MG:} & Garret, nie chcę cię martwić ale gdy włócznia nie daje rady to pozostaje ręka.\\
   \end{tabularx}
\end{rpg-quotebox}


\section{18 maja 2018}


\begin{rpg-quotebox}{Gerofilia?}
   \begin{tabularx}{\columnwidth}{lX}
      \textbf{Vivien:} & Brzęczenie słyszeliśmy z prawej czy z lewej strony pokoju?\\
      \textbf{Calion:} & A co za różnica? Z boku!\\
      \textbf{Yeleda:} & Sam jesteś zbok.\\
      \textbf{Calion:} & Powiedziała ta, której starość kojarzy się z seksem.\\
   \end{tabularx}
   ~\newline

   \textit{(Drużyna rozwijała skrót SKS jako seks kurwa seks, zamiast starość)}\\
\end{rpg-quotebox}


\begin{rpg-quotebox}{Sole bardzo trzeźwiące}
   \begin{tabularx}{\columnwidth}{lX}
      \textbf{Garret:} & \multirow{2}{*}{Podchodzimy do zwłok.}\\
      \textbf{Calion:} & \\
      \textbf{MG:} & Rzućcie rzut obronny na wytrzymałość\\
      \includegraphics[scale=0.06]{img/d20.png}\textbf{(Garret):}& 6.\\
      \includegraphics[scale=0.06]{img/d20.png}\textbf{(Calion):}& 13.\\
      \textbf{MG:} & Garret odchodzi na bok i puszcza pięknego pawia, a Calion zdołał się przypatrzeć i musi rzucić na arkana.\\
      \includegraphics[scale=0.06]{img/d20.png}\textbf{(Calion):}& 31.\\
      \textbf{MG:} & Przypomina ci się, że doskonale znasz przyczynę tego zgonu i źródło smrodu. Wiesz, że smród wydziela się gdy ktoś próbuje wyssać z kogoś życie i zamienić je na moc magiczną.\\
      \textbf{Vivien:} & Calionowi smród przeczyścił mózg.\\
   \end{tabularx}
\end{rpg-quotebox}


\section{7 lipca 2018}


\begin{rpg-quotebox}{Strach nabyty}
   \textit{Vivien boi się klamek od kiedy Yeleda sprawdza wszystkie drzwi pod kątem pułapek.}\\
\end{rpg-quotebox}


\begin{rpg-quotebox}{Przygotowanie winno być poczynione przed faktem}
   \textit{Yeleda wyważyła drzwi i drużyna stanęła twarzą w twarz z żywiołakiem ziemi.}\\

   \begin{tabularx}{\columnwidth}{lX}
      \textbf{Calion:} & Kurna, ile ja mam spell slotów.\\
   \end{tabularx}
\end{rpg-quotebox}


\begin{rpg-quotebox}{Czy ktoś spodziewał się czegoś innego?}
   \begin{tabularx}{\columnwidth}{lX}
      \textbf{Calion:} & Czy ja wiem jakie to to ma słabe strony?\\
      \textbf{MG:} & Nie jest twoja tura.\\
      \textbf{Vivien:} & Czyli Garret wali w ciemno.\\
      \textbf{Calion:} & Zobaczymy czy go zmiażdżysz czy zgnieciesz.\\
   \end{tabularx}
\end{rpg-quotebox}


\begin{rpg-quotebox}{O słusznych wnioskach.}
   \textit{Calion oberwał od żywiołaka na tyle mocno, że stracił przytomność.}\\

   \begin{tabularx}{\columnwidth}{lX}
      \textbf{Vivien:} & Ok, czyja tura teraz.\\
      \textbf{Calion:} & Na pewno nie moja.\\
   \end{tabularx}
\end{rpg-quotebox}


\begin{rpg-quotebox}{Prawidła dekoracyjne}
   \begin{tabularx}{\columnwidth}{lX}
      \textbf{Vivien:} & Calion robi za dywan.\\
      \multicolumn{2}{l}{\textit{Do zmienionej w niedźwiedzia Vivien:}}\\
      \textbf{Calion:} & O ile wiem to niedźwiedzie zwykle robią za dywany.\\
   \end{tabularx}
\end{rpg-quotebox}


\begin{rpg-quotebox}{}
   \textit{Na polu bitwy została Yeleda i Vivien, a Garret i Calion zostali nieprzytomni.}\\

   \begin{tabularx}{\columnwidth}{lX}
      \textbf{Yeleda:} & A mówi się, że to mężczyźni to silna płeć.\\
      \textbf{Vivien:} & Niedźwiedzie to najsilniejsza płeć!\\
   \end{tabularx}
\end{rpg-quotebox}


\begin{rpg-quotebox}{Widocznie nie potrzeba}
   \textit{Calion i Garret wyrzucili 20 na rzucie obronnym na śmierć i wstali z 1 punktem życia.}\\

   \begin{tabularx}{\columnwidth}{lX}
      \textbf{Vivien:} & Chuj, nie leczę was.\\
   \end{tabularx}
\end{rpg-quotebox}


\begin{rpg-quotebox}{To nie ta legenda}
   \textit{Po bitwie Yeleda szukała bełtów, ale zgubiła dwa z nich.}\\

   \begin{tabularx}{\columnwidth}{lX}
      \textbf{MG:} & Zaginęły w kamieniu.\\
      \textbf{Vivien:} & Bełty w kamieniu, jak wyciągniesz to zostaniesz prawdziwą łotrzycą.\\
   \end{tabularx}
\end{rpg-quotebox}


\begin{rpg-quotebox}{Bo sen to najlepsze lekarstwo}
   \begin{tabularx}{\columnwidth}{lX}
      \textbf{Garret:} & Długi odpoczynek w D\&D to OIOM w prawdziwym życiu.\\
   \end{tabularx}
\end{rpg-quotebox}


\begin{rpg-quotebox}{Matka ostrzega}
   \begin{tabularx}{\columnwidth}{lX}
      \textbf{Tolena:} & Te istoty to Yuan-ti - czyli niegdyś przeklęci ludzie lub inne istoty humanoidalne.\\
      \multicolumn{2}{l}{\textit{Do Vivien:}}\\
      \textbf{Yeleda:} & Przestań się zmieniać w węża bo ci tak zostanie.\\
   \end{tabularx}
\end{rpg-quotebox}


\begin{rpg-quotebox}{Łożołom}
   \textit{Drużyna słyszała znajomy stuk kamieni o kamień zdradzający obecność kolejnych żywiołaków ziemi. Gdy zbliżyli się do klapy w podłodze dźwięk ustał.}\\

   \begin{tabularx}{\columnwidth}{lX}
      \textbf{Yeleda:} & Albo szły, albo się ruchały.\\
   \end{tabularx}
\end{rpg-quotebox}

% ================================================================================




% \begin{rpg-quotebox}{}
%    \textit{}\\
%    
%    \begin{tabularx}{\columnwidth}{lX}
%       \multicolumn{2}{l}{\textit{}}\\
%       
%       \textbf{:} & \\
%    \end{tabularx}
% \end{rpg-quotebox}








% \begin{rpg-paperbox}{rpg-paperbox}
% 	you can add some text here
% \end{rpg-paperbox}
% 
% %\newpage % Acts as columbreak because of twocolumn option; for pagebreak use \clearpage
% 
% \section{Some tables}
% \header{default rpg-table (2 column)}
% \begin{rpg-table}
%    	\textbf{Table head 1}  & \textbf{Table head 2} \\
%    	Some value  & Some value \\
%    	Some value  & Some value \\
%    	Some value  & Some value
% \end{rpg-table}
% 
% % For more columns, you can say \begin{rpg-table}[your options here].
% % For instance, if you wanted three columns, you could say
% % \begin{rpg-table}[XXX]. The usual host of tabular parameters are
% % aailable as well.
% \header{rpg-table with more columns}
% \begin{rpg-table}[XXX]
%     \textbf{Table head 1}  & \textbf{Table head 2} & \textbf{Table head 3}\\
%    	Some value  & Some value & Some value\\
%    	Some value  & Some value & Some value\\
%    	Some value  & Some value & Some value
% \end{rpg-table}
% 
% \header{default rpg-table2 (2 column)}
% \begin{rpg-table2}
%    	\textbf{Table head 1}  & \textbf{Table head 2} \\
%    	Some value  & Some value \\
%    	Some value  & Some value \\
%    	Some value  & Some value
% \end{rpg-table2}
% 
% 
% \section{List}
% \begin{rpg-list}
%     \item first list item
%     \item second list item
% \end{rpg-list}
% 
\chapter{Karty postaci}
% You can optionally not include the background by saying
%\begin{rpg-monsterboxnobg}{Garret Eartopple}
\begin{rpg-monsterbox}{Garret Eartopple}
   \textit{Stout halfling, lawful neutral}\\
   \rpghline
   \basics[%
      armorclass = 15 (+2),
      hitpoints  = 44 (3d8 + 3),
      speed      = 40 ft
   ]
   \rpghline
   \stats[ % This stat command will autocomplete the modifier for you
      STR = 12, 
      DEX = 17,
      CON = 14,
      INT = 8,
      WIS = 15,
      CHA = 10
   ]
   \rpghline
   \header{Saving Throws}
   \begin{rpg-table}[cXr]
      $\bigstar$ & Strength     & $+4$ \\
      $\bigstar$ & Dexteriety   & $+6$ \\
                 & Condition    & $+2$ \\
                 & Intelligence & $-1$ \\
                 & Wisdom       & $+2$ \\
                 & Charisma     & $ 0$ \\
   \end{rpg-table}
   \header{Skills}
   \begin{rpg-table}[cXr]
      $\bigstar$         & Acrobatics (Dex)      & $+6$ \\
                         & Animal Handling (Wis) & $+2$ \\
                         & Arcana (Int)          & $-1$ \\
      $\bigstar$         & Athletics (Str)       & $+4$ \\
                         & Deception (Cha)       & $ 0$ \\
                         & History (Int)         & $-1$ \\
                         & Insight (Wis)         & $+2$ \\
                         & Intimidation (Cha)    & $ 0$ \\
                         & Investigation (Int)   & $-1$ \\
      $\bigstar\bigstar$ & Medicine (Wis)        & $+8$ \\
                         & Nature (Int)          & $-1$ \\
                         & Perception (Wis)      & $+2$ \\
                         & Performance (Cha)     & $ 0$ \\
                         & Persuassion (Cha)     & $ 0$ \\
      $\bigstar$         & Religion (Int)        & $+2$ \\
                         & Sleight of Hand (Dex) & $+3$ \\
                         & Stealth (Dex)         & $+3$ \\
                         & Survival (Wis)        & $+2$ \\
   \end{rpg-table}
   \details[%
      % If you want to use commas in these sections, enclose the
      % description in braces.
      % I'm so sorry.
      languages = {Common, Dwarvish, Halfling},
      proficiencies = {Herbalism kit, Brewer\'s kit}
   ]
   \rpghline \\[1mm]

   \rpgmonstersection{Racial Traits}
   \paragraph{\emph{Lucky}}
   When you roll a 1 on an attack, ability check or saving throw, you can reroll the die and must use the new roll.
	
   \paragraph{\emph{Brave}}
   You have advantage on saving throws against being frigthened.

   \paragraph{\emph{Halfling Nimbleness}}
   You can move through the space of any creatire that is of size larger than yours.

   \paragraph{\emph{Stout Resilience}}
   You have advantage on saving throws against poison and you have resistance against poison damage.
   \newline ~
   \rpgmonstersection{Class Feats}
   \paragraph{\emph{Unarmoured Defense}}
   While you are wearing no armor and not wielding a shield, your AC equals: $10 + Dex~mod + Wis~mod$.

   \paragraph{\emph{Martial Arts}}
   Your practice of martial arts gives you mastery of combat styles tha use unarmed strikes and monk weapons, which are short swords and any simple weapons that don't have two-handed or heavy property.

   \hskip4mm You gain the following benefits while you are unarmed or wielding only monk weapons and you aren't wearing armor or wielding a shield:
   \begin{itemize}
      \item You can use Dexteriety insted of Strength for the attack and damage rolls of your unarmed strike and monk weapns.
      \item You can roll a d6 in place of the normal damage of your unarmed strike or monk weapon.
      \item When you use Attack action with an unarmed strike or a monk weapon on your turn, you can make one unarmed strike as a bonus action.
   \end{itemize}

   \paragraph{\emph{Ki}}
   Your training allows you to harness the mystic energy of ki.
   Your access to this energy is represented by a number of ki points.
   Your monk level determines the number of points you have.
   When you spend a ki point, it is unavailable until you finish a short or long rest.
   You must spend at least 30 minutes of the rest meditating to regain your ki points.

   \hskip4mm The saving throw required by your opponent to avoid some of the effects of a ki feature is calculated as follows:
   
   $DC = 8 + Proficiency + Wis~modifier$

   \paragraph{\emph{Flurry of Blows}}
   Immediately after you take the Attack action on your turn, you can spend 1 ki point to make two unarmed strikes as a bonus action.

   \paragraph{\emph{Patient Defense}}
   You can spend 1 ki point to take the Dodge action as a bonus action on your turn.

   \paragraph{\emph{Step of the Wind}}
   You can spend 1 ki point to take the Disengage or Dash action as a bonus action on your turn, and your jump distance is doubled for the turn.

   \paragraph{\emph{Unarmored Movement}}
   Your speed increases by 10 feet while you are not wearing armor or wielding a shield.

   \paragraph{\emph{Deflect Missiles}}
   You can use your reaction to deflect or catch the missile when you are hit by a ranged weapon attack.
   When you do so, the damage you take from the attack is reduced by:
   
   $1d10 + Dex~modifier + Monk~level$
   
   \hskip4mm If you reduce the damage to 0, you can catch the missile if it is small enough for you to hold in one hand and you have at least one hand free.
   If you catch a missile in this way, you can spend 1 ki point to make a ranged attack (20/60 ft) with the weapon or piece of ammunition you just caught, as part of the same reaction.
   You make this attack with proficiency, regardless of your weapon proficiencies, and the missile counts as a monk weapon for the attack.

   \paragraph{\emph{Slow Fall}}
   You can use your reaction when you fall to reduce any falling damage you take by an amount equal to five times your monk level.

   \paragraph{\emph{Extra Attack}}
   You can attack twice, instead of once, whenever you take the Attack action on your turn.

   \paragraph{\emph{Stunning Strike}}
   When you hit another creature with a melee weapon attack, you can spend 1 ki point to attempt a stunning strike.
   The target must succeed on a Constitution saving throw or be stunned until the end of your next turn.

   \paragraph{\emph{Ki-Empowered Strikes}}
   Your unarmed strikes count as magical for the purpose of overcoming resistance and immunity to non-magical attacks and damage.

   \paragraph{\emph{Evasion}}
   When you are subjected to an effect that allows you to make a Dexterity saving throw to take only half damage, you instead take no damage if you succeed on the saving throw, and only half damage if you fail.

   \paragraph{\emph{Stillness of Mind}}
   You can use your action to end one effect on yourself that is causing you to be charmed or frightened.

   \rpgmonstersection{Way of the Open hand}
   \paragraph{\emph{Open Hand Technique}}
   Whenever you hit a creature with one of the attacks granted by your \emph{Flurry of Blows}, you can impose one of the following effects on that target:
    \begin{itemize}
       \item It must succeed on a Dexterity saving throw or be knocked prone.
       \item It must make a Strength saving throw. If it fails, you can push it up to 15 feet away from you.
       \item It can't take reactions until the end of your next turn.
    \end{itemize}

   \paragraph{\emph{Wholeness of Body}}
   As an action, you can regain hit points equal to three times your monk level.
   You must finish a long rest before you can use this feature again.

   \rpgmonstersection{Other feats}
   \paragraph{\emph{Medic}}
   You master the physician's arts, gaining the following benefits:
   \begin{itemize}
      \item Increase your Wisdom score by 1, to a maximum of 20.
      \item You gain proficiency in the Medicine skill. If you are already proficient in the skill, you add double your proficiency bonus to checks you make with it.
      \item During a short rest, you can clean and bind the wounds of up to six willing beasts and humanoids. Make a DC 15 Medicine check for each creature. On a success, if a creature spends a Hit Die during this rest, that creature can forgo the roll and instead regain the maximum number of hit points the die can restore. A creature can do so only once per rest, regardless of how many Hit Dice it spends.
   \end{itemize}

\end{rpg-monsterbox}

\begin{rpg-monsterbox}{Vivien Ithwen}
   \textit{Wood elf, true neutral\\Druid level 7}\\
   \rpghline
   \basics[%
      armorclass = 14,
      hitpoints  = 39,
      speed      = 35 (+10) ft
   ]
   \rpghline
   \stats[ % This stat command will autocomplete the modifier for you
      STR = 12,
      DEX = 14,
      CON = 14,
      INT = 13,
      WIS = 17,
      CHA = 13
   ]
   \rpghline
   \header{Saving Throws}
   \begin{rpg-table}[cXr]
                 & Strength     & $+1$ \\
                 & Dexteriety   & $(\triangle)+2$ \\
                 & Constitution & $+2$ \\
      $\bigstar$ & Intelligence & $+4$ \\
      $\bigstar$ & Wisdom       & $+6$ \\
                 & Charisma     & $+1$ \\
   \end{rpg-table}
   \header{Skills}
   \begin{rpg-table}[cXr]
                 & Acrobatics (Dex)      & $+6$ \\
                 & Animal Handling (Wis) & $+2$ \\
      $\bigstar$ & Arcana (Int)          & $-1$ \\
                 & Athletics (Str)       & $+4$ \\
                 & Deception (Cha)       & $ 0$ \\
                 & History (Int)         & $-1$ \\
                 & Insight (Wis)         & $+2$ \\
                 & Intimidation (Cha)    & $ 0$ \\
                 & Investigation (Int)   & $-1$ \\
      $\bigstar$ & Medicine (Wis)        & $+8$ \\
                 & Nature (Int)          & $-1$ \\
      $\bigstar$ & Perception (Wis)      & $+2$ \\
                 & Performance (Cha)     & $ 0$ \\
                 & Persuassion (Cha)     & $ 0$ \\
      $\bigstar$ & Religion (Int)        & $+2$ \\
                 & Sleight of Hand (Dex) & $+3$ \\
                 & Stealth (Dex)         & $+3$ \\
      $\bigstar$ & Survival (Wis)        & $+2$ \\
   \end{rpg-table}
   \details[%
      % If you want to use commas in these sections, enclose the
      % description in braces.
      % I'm so sorry.
      languages = {Common, Sylvan, Elvish, Druidic},
      proficiencies = {Herbalism kit}
   ]
   \rpghline \\[1mm]

   \rpgmonstersection{Racial Traits}
   \paragraph{\emph{Darkvision}}
   You can see in dim light within 60 ft of you as if it were bright light, and in darkness as if it were dim light.
   You can't discern color in darkness, only shades of gray.

   \paragraph{\emph{Keen Senses}}
   You have proficiency in Perception.

   \paragraph{\emph{Fey Ancestry}}
   You have advantage on saves against being charmed, and magic can't put you to sleep.
   
   \paragraph{\emph{Trance}}
   Elves meditate deeply, remaining semiconscious, for 4 hours a day in place of sleeping.
   You gain the same benefit that a human does from 8 hours of sleep.

   \paragraph{\emph{Fleet of Foot}}
   Your base walking speed increases to 35 feet.
   
   \paragraph{\emph{Mask of the Wild}}
   You can attempt to hide even when you are only lightly obscured by foliage, heavy rain, falling snow, mist, and other natural phenomena.
   \newline ~

   \rpgmonstersection{Class Feats}
   \paragraph{\emph{Druidic}}
   You know Druidic, the secret language of druids.
   You can speak the language and use it to leave hidden messages.
   You and others who know this language automatically spot such a message.
   Others spot the message's presence with a successful DC 15 Wisdom (Perception) check but can't decipher it without magic.

   \paragraph{\emph{Spellcasting}}~
   \begin{rpg-table}[Xr]
      Cantrips known & 4 \\
      1 lvl slots    & 4 \\
      2 lvl slots    & 3 \\
      3 lvl slots    & 3 \\
      4 lvl slots    & 1 \\
      5 lvl slots    & - \\
      6 lvl slots    & - \\
      7 lvl slots    & - \\
      8 lvl slots    & - \\
      9 lvl slots    & - \\
   \end{rpg-table}
   To cast one of druid spells, you must expend a slot of the spell's level or higher.
   You regain all expended spell slots when you finish a long rest.

   \hskip4mm You prepare the list of druid spells that are available for you to cast, choosing from the druid spell list.
   When you do so, choose a number of druid spells equal to:
   \newline~
   $Wis~mod + Druid~lvl$ (minimum of one spell).
   \newline~
   \hskip4mm The spells must be of a level for which you have spell slots.

   \hskip4mm You can also change your list of prepared spells when you finish a long rest.

   \hskip4mm Wisdom is your spellcasting ability for your druid spells.
   You use your Wisdom whenever a spell refers to your spellcasting ability.
   In addition, you use your Wisdom modifier when setting the saving throw DC for a druid spell you cast and when making an attack roll with one.
   \newline~
   \begin{tabularx}{\columnwidth}{rcl}
      $DC$ & $=$ & $8 + Proficiency + Wis~mod$ \\
      $Spell~Attack~mod$ & $=$ & $Proficiency + Wis~mod$
   \end{tabularx}

   \paragraph{\emph{Ritual Casting}}
   You can cast a druid spell as a ritual if that spell has the ritual tag and you have the spell prepared.

   \paragraph{\emph{Spellcasting Focus}}
   You can use a druidic focus as a spellcasting focus for your druid spells.

   \paragraph{\emph{Wild Shape}}
   You can use your action to magically assume the shape of a beast that you have seen before.
   You can use this feature twice.
   You regain expended uses when you finish a short or long rest.

   \hskip4mm Your druid level determines the beasts you can transform into, as shown in the Beast Shapes table.
   \begin{rpg-table}[XXXr]
      Level & Max CR &                 Limitations & Example\\
        2nd &    1/4 & No flying or swimming speed & Wolf\\
        4th &    1/2 &             No flying speed & Crocodile\\
        8th &      1 &	                         - & Giant eagle
   \end{rpg-table}

   \hskip4mm You can stay in a beast shape for a number of hours equal to half your druid level (rounded down).
   You then revert to your normal form unless you expend another use of this feature.
   You can revert to your normal form earlier by using a bonus action on your turn.
   You automatically revert if you fall unconscious, drop to 0 hit points, or die.

   \hskip4mm While you are transformed, the following rules apply:
   \begin{itemize}
      \item Your game statistics are replaced by the statistics of the beast, but you retain your alignment, personality, and Intelligence, Wisdom, and Charisma scores.
      You also retain all of your skill and saving throw proficiencies, in addition to gaining those of the creature.
      If the creature has the same proficiency as you and the bonus is higher than yours, use the creature's bonus.
      If the creature has any legendary or lair actions, you can't use them.

      \item When you transform, you assume the beast's hit points and Hit Dice.
      When you revert to your normal form, you return to the number of hit points you had before you transformed.
      However, if you revert as a result of dropping to 0 hit points, any extra damage carries over to your normal form.
      As long as the excess damage doesn't reduce your normal form to 0 hit points, you aren't knocked unconscious.

       \item You can't cast spells, and your ability to speak or take any action that requires hands is limited to the capabilities of your beast form.
       Transforming doesn't break your concentration on a spell you've already cast, however, or prevent you from taking actions that are part of a spell that you've already cast.

       \item You retain the benefit of any features from your class, race, or other source and can use them if the new form is physically capable of doing so.
       However, you can't use any of your special senses, such as darkvision, unless your new form also has that sense.

      \item You choose whether your equipment falls to the ground in your space, merges into your new form, or is worn by it.
      Worn equipment functions as normal, but the GM decides whether it is practical for the new form to wear a piece of equipment.
      Your equipment doesn't change to match the new form, and any equipment that the new form can't wear must either fall to the ground or merge with it.
      Equipment that merges with the form has no effect until you leave the form.
   \end{itemize}

   \rpgmonstersection{Circle of the Moon}
   \paragraph{\emph{Combat Wild Shape}}
   When you choose this circle, you gain the ability to use Wild Shape on your turn as a bonus action, rather than as an action.

   \hskip4mm Additionally, while you are transformed by Wild Shape, you can use a bonus action to expend one spell slot to regain 1d8 hit points per level of the spell slot expended.

   \paragraph{\emph{Circle Forms}}
   You can transform into a beast with a challenge rating as high as your druid level divided by 3, rounded down.

   \paragraph{\emph{Primal Strike}}
   Your attacks in beast form count as magical for the purpose of overcoming resistance and immunity to nonmagical attacks and damage.

   \rpgmonstersection{Other feats}

   \paragraph{\emph{Magic Initiate}}
   You learn two cantrips of your choice from Cleric's spell list.

   \hskip4mm In addition, choose one 1st-level spell from that same list.
   You learn that spell and, using this feat, can cast it at its lowest level.
   Once you cast it in this way, you must finish a long rest before you can cast it in this way again.

   \hskip4mm Your spellcasting ability for these spells depends on the class you chose.

\end{rpg-monsterbox}



\subsection*{Control Flames}
\label{ControlFlames}
\begin{rpg-table}[Xr]
   Level        & Cantrip\\
   School       & Transmutation\\
   Casting Time & Action\\
   Range        & 60 ft\\
   Components   & S\\
   Duration     & Instantaneous/1 hour\\
\end{rpg-table}
\hskip4mm You choose nonmagical flame that you can see within range and that fits within a 5-foot cube.
You affect it in one of the following ways:
\begin{itemize}
   \item You instantaneously expand the flame 5 feet in one direction, provided that wood or other fuel is present in the new location.
   \item You instantaneously extinguish the flames within the cube.
   \item You double or halve the area of bright light and dim light cast by the flame, change its color, or both. The change lasts for 1 hour.
   \item You cause simple shapes--such as the vague form of a creature, an inanimate object, or a location--to appear within the flames and animate as you like. The shapes last for 1 hour.
\end{itemize}
\hskip4mm If you cast this spell multiple times, you can have up to three of its non-instantaneous effects active at a time, and you can dismiss such an effect as an action.

\subsection*{Create Bonfire}
\label{CreateBonfire}
\begin{rpg-table}[Xr]
   Level        & Cantrip\\
   School       & Conjuration\\
   Casting Time & Action\\
   Range        & 60 ft\\
   Components   & V, S\\
   Duration     & Concentration, up to 1 minute\\
\end{rpg-table}

You create a bonfire on ground that you can see within range.
Until the spell ends, the magic bonfire fills a 5-foot cube.
Any creature in the bonfire's space when you cast the spell must succeed on a Dexterity saving throw or take 1d8 fire damage.
A creature must also make the saving throw when it moves into the bonfire's space for the first time on a turn or ends its turn there.

The bonfire ignites flammable objects in its area that aren't being worn or carried.

The spell's damage increases by 1d8 when you reach 5th level (2d8), 11th level (3d8), and 17th level (4d8).

Druidcraft
Transmutation cantrip
Casting Time: 1 action
Range: 30 feet
Components: V, S
Duration: Instantaneous

Whispering to the spirits of nature, you create one of the following effects within range:

    You create a tiny, harmless sensory effect that predicts what the weather will be at your location for the next 24 hours. The effect might manifest as a golden orb for clear skies, a cloud for rain, falling snowflakes for snow, and so on. This effect persists for 1 round.
    You instantly make a flower blossom, a seed pod open, or a leaf bud bloom.
    You create an instantaneous, harmless sensory effect, such as falling leaves, a puff of wind, the sound of a small animal, or the faint odor of skunk. The effect must fit in a 5-foot cube.
    You instantly light or snuff out a candle, a torch, or a small campfire.

Frostbite
Evocation cantrip
Casting Time: 1 action
Range: 60 feet
Components: V, S
Duration: Instantaneous

You cause numbing frost to form on one creature that you can see within range. The target must make a Constitution saving throw. On a failed save, the target takes 1d6 cold damage, and it has disadvantage on the next weapon attack roll it makes before the end of its next turn.

The spell's damage increases by 1d6 when you reach 5th level (2d6), 11th level (3d6), and 17th level (4d6).
Guidance
Divination cantrip
Casting Time: 1 action
Range: Touch
Components: V, S
Duration: Concentration, up to 1 minute

You touch one willing creature. Once before the spell ends, the target can roll a d4 and add the number rolled to one ability check of its choice. It can roll the die before or after making the ability check. The spell then ends.
Gust
Transmutation cantrip
Casting Time: 1 action
Range: 30 feet
Components: V, S
Duration: Instantaneous

You seize the air and compel it to create one of the following effects at a point you can see within range:

    One Medium or smaller creature that you choose must succeed on a Strength saving throw or be pushed up to 5 feet away from you.

    You create a small blast of air capable of moving one object that is neither held nor carried and that weighs no more than 5 pounds. The object is pushed up to 10 feet away from you. It isn't pushed with enough force to cause damage.

    You create a harmless sensory affect using air, such as causing leaves to rustle, wind to slam shutters shut, or your clothing to ripple in a breeze.

Magic Stone
Transmutation cantrip
Casting Time: 1 bonus action
Range: Touch
Components: V, S
Duration: 1 minute

You touch one to three pebbles and imbue them with magic. You or someone else can make a ranged spell attack with one of the pebbles by throwing it or hurling it with a sling. If thrown, it has a range of 60 feet. If someone else attacks with the pebble, that attacker adds your spellcasting ability modifier, not the attacker's, to the attack roll. On a hit, the target takes bludgeoning damage equal to 1d6 + your spellcasting ability modifier. Hit or miss, the spell then ends on the stone.

If you cast this spell again, the spell ends early on any pebbles still affected by it.
Mending
Transmutation cantrip
Casting Time: 1 minute
Range: Touch
Components: V, S, M
Duration: Instantaneous

This spell repairs a single break or tear in an object you touch, such as a broken chain link, two halves of a broken key, a torn cloak, or a leaking wineskin. As long as the break or tear is no larger than 1 foot in any dimension, you mend it, leaving no trace of the former damage.

This spell can physically repair a magic item or construct, but the spell can't restore magic to such an object.
Mold Earth
Transmutation cantrip
Casting Time: 1 action
Range: 30 feet
Components: S
Duration: Instantaneous/1 hour

You choose a portion of dirt or stone that you can see within range and that fits within a 5-foot cube. You manipulate it in one of the following ways:

    If you target an area of loose earth, you can instantaneously excavate it, move it along the ground, and deposit it up to 5 feet away. This movement doesn't have enough force to cause damage.

    You cause shapes, colors, or both to appear on the dirt or stone, spelling out words, creating images, or shaping patterns. The changes last for 1 hour.

    If the dirt or stone you target is on the ground, you cause it to become difficult terrain. Alternatively, you can cause the ground to become normal terrain if it is already difficult terrain. This change lasts for 1 hour.

If you cast this spell multiple times, you can have no more than two of its non-instantaneous effects active at a time, and you can dismiss such an effect as an action.
Produce Flame
Conjuration cantrip
Casting Time: 1 action
Range: Self
Components: V, S
Duration: 10 minutes

A flickering flame appears in your hand. The flame remains there for the duration and harms neither you nor your equipment. The flame sheds bright light in a 10-foot radius and dim light for an additional 10 feet. The spell ends if you dismiss it as an action or if you cast it again.

You can also attack with the flame, although doing so ends the spell. When you cast this spell, or as an action on a later turn, you can hurl the flame at a creature within 30 feet of you. Make a ranged spell attack. On a hit, the target takes 1d8 fire damage.

This spell's damage increases by 1d8 when you reach 5th level (2d8), 11th level (3d8), and 17th level (4d8).
Resistance
Abjuration cantrip
Casting Time: 1 action
Range: Touch
Components: V, S, M
Duration: Concentration, up to 1 minute

You touch one willing creature. Once before the spell ends, the target can roll a d4 and add the number rolled to one saving throw of its choice. It can roll the die before or after making the saving throw. The spell then ends.
Shape Water
Transmutation cantrip
Casting Time: 1 action
Range: 30 feet
Components: S
Duration: Instantaneous/1 hour

You choose an area of water that you can see within range and that fits within a 5-foot cube. You manipulate it in one of the following ways:

    You instantaneously move or otherwise change the flow of the water as you direct, up to 5 feet in any direction. This movement doesn't have enough force to cause damage.

    You cause the water to form into simple shapes and animate at your direction. This change lasts for 1 hour.

    You change the water's color or opacity. The water must be changed in the same way throughout. This change lasts for 1 hour.

    You freeze the water, provided that there are no creatures in it. The water unfreezes in 1 hour.

If you cast this spell multiple times, you can have no more than two of its non-instantaneous effects active at a time, and you can dismiss such an effect as an action.
Shillelagh
Transmutation cantrip
Casting Time: 1 bonus action
Range: Touch
Components: V, S, M
Duration: 1 minute

The wood of a club or quarterstaff you are holding is imbued with nature's power. For the duration, you can use your spellcasting ability instead of Strength for the attack and damage rolls of melee attacks using that weapon, and the weapon's damage die becomes a d8. The weapon also becomes magical, if it isn't already. The spell ends if you cast it again or if you let go of the weapon.
Thorn Whip
Transmutation cantrip
Casting Time: 1 action
Range: 30 feet
Components: V, S, M
Duration: Instantaneous

You create a long, vine-like whip covered in thorns that lashes out at your command toward a creature in range. Make a melee spell attack against the target. If the attack hits, the creature takes 1d6 piercing damage, and if the creature is Large or smaller, you pull the creature up to 10 feet closer to you.

This spell’s damage increases by 1d6 when you reach 5th level (2d6), 11th level (3d6), and 17th level (4d6).
Thunderclap
Evocation cantrip
Casting Time: 1 action
Range: Self (5-foot radius)
Components: S
Duration: Instantaneous

You create a burst of thunderous sound that can be heard up to 100 feet away. Each creature within range, other than you, must succeed on a Constitution saving throw or take 1d6 thunder damage.

The spell's damage increases by 1d6 when you reach 5th level (2d6), 11th level (3d6), and 17th level (4d6).
% Absorb Elements
% 1st-level abjuration
% Casting Time: 1 action
% Range: Self
% Components: S
% Duration: 1 round
% 
% The spell captures some of the incoming energy, lessening its effect on you and storing it for your next melee attack. You have resistance to the triggering damage type until the start of your next turn. Also, the first time you hit with a melee attack on your next turn, the target takes an extra 1d6 damage of the triggering type, and the spell ends.
% 
% At higher levels:
% 
% When you cast this spell using a spell slot of 2nd level or higher, the extra damage increases by 1d6 for each slot level above 1st.
% Animal Friendship
% 1st-level enchantment
% Casting Time: 1 action
% Range: 30 feet
% Components: V, S, M
% Duration: 24 hours
% 
% This spell lets you convince a beast that you mean it no harm. Choose a beast that you can see within range. It must see and hear you. If the beast's Intelligence is 4 or higher, the spell fails. Otherwise, the beast must succeed on a Wisdom saving throw or be charmed by you for the spell's duration. If you or one of your companions harms the target, the spells ends.
% 
% At higher levels:
% 
% When you cast this spell using a spell slot of 2nd level or higher, you can affect one additional beast level above 1st.
% Beast Bond
% 1st-level divination
% Casting Time: 1 action
% Range: Touch
% Components: V, S, M
% Duration: Concentration, up to 10 minutes
% 
% You establish a telepathic link with one beast you touch that is friendly to you or charmed by you. The spell fails if the beast's Intelligence is 4 or higher. Until the spell ends, the link is active while you and the beast are within line of sight of each other. Through the link, the beast can understand your telepathic messages to it, and it can telepathically communicate simple emotions and concepts back to you. While the link is active, the beast gains advantage on attack rolls against any creature within 5 feet of you that you can see.
% Charm Person
% 1st-level enchantment
% Casting Time: 1 action
% Range: 30 feet
% Components: V, S
% Duration: 1 hour
% 
% You attempt to charm a humanoid you can see within range. It must make a Wisdom saving throw, and does so with advantage if you or your companions are fighting it. If it fails the saving throw, it is charmed by you until the spell ends or until you or your companions do anything harmful to it. The charmed creature regards you as a friendly acquaintance. When the spell ends, the creature knows it was charmed by you.
% 
% At higher levels:
% 
% When you cast this spell using a spell slot of 2nd level or higher, you can target one additional creature for each slot level above 1st. The creatures must be within 30 feet of each other when you target them.
% Create or Destroy Water
% 1st-level transmutation
% Casting Time: 1 action
% Range: 30 feet
% Components: V, S, M
% Duration: Instantaneous
% 
% You either create or destroy water.
% 
% Create Water. You create up to 10 gallons of clean water within range in an open container. Alternatively, the water falls as rain in a 30-foot cube within range, extinguishing exposed flames in the area.
% 
% Destroy Water. You destroy up to 10 gallons of water in an open container within range. Alternatively, you destroy fog in a 30-foot cube within range.
% 
% At higher levels:
% 
% When you cast this spell using a spell slot of 2nd level or higher, you create or destroy 10 additional gallons of water, or the size of the cube increases by 5 feet, for each slot level above 1st.
% Cure Wounds
% 1st-level evocation
% Casting Time: 1 action
% Range: Touch
% Components: V, S
% Duration: Instantaneous
% 
% A creature you touch regains a number of hit points equal to 1d8 + your spellcasting ability modifier. This spell has no effect on undead or constructs.
% 
% At higher levels:
% 
% When you cast this spell using a spell slot of 2nd level or higher, the healing increases by 1d8 for each slot level above 1st.
% Detect Magic
% 1st-level divination
% Casting Time: 1 action
% Range: Self
% Components: V, S
% Duration: Concentration, up to 10 minutes
% 
% For the duration, you sense the presence of magic within 30 feet of you. If you sense magic in this way, you can use your action to see a faint aura around any visible creature or object in the area that bears magic, and you learn its school of magic, if any.
% 
% The spell can penetrate most barriers, but it is blocked by 1 foot of stone, 1 inch of common metal, a thin sheet of lead, or 3 feet of wood or dirt.
% Detect Poison and Disease
% 1st-level divination
% Casting Time: 1 action
% Range: Self
% Components: V, S, M
% Duration: Concentration, up to 10 minutes
% 
% For the duration, you can sense the presence and location of poisons, poisonous creatures, and diseases within 30 feet of you. You also identify the kind of poison, poisonous creature, or disease in each case.
% 
% The spell can penetrate most barriers, but it is blocked by 1 foot of stone, 1 inch of common metal, a thin sheet of lead, or 3 feet of wood or dirt.
% Earth Tremor
% 1st-level evocation
% Casting Time: 1 action
% Range: Self (10-foot radius)
% Components: V, S
% Duration: Instantaneous
% 
% You cause a tremor in the ground within range. Each creature other than you in that area must make a Dexterity saving throw. On a failed save, a creature takes 1d6 bludgeoning damage and is knocked prone. If the ground in that area is loose earth or stone, it becomes difficult terrain until cleared, with each 5-foot-diameter portion requiring at least 1 minute to clear by hand.
% 
% At higher levels:
% 
% When you cast this spell using a spell slot of 2nd level or higher, the damage increases by 1d6 for each slot level above 1st.
% 
% Entangle
% 1st-level conjuration
% Casting Time: 1 action
% Range: 90 feet
% Components: V, S
% Duration: Concentration, up to 1 minute
% 
% Grasping weeds and vines sprout from the ground in a 20-foot square starting from a point within range. For the duration, these plants turn the ground in the area into difficult terrain.
% 
% A creature in the area when you cast the spell must succeed on a Strength saving throw or be restrained by the entangling plants until the spell ends. A creature restrained by the plants can use its action to make a Strength check against your spell save DC. On a success, it frees itself.
% 
% When the spell ends, the conjured plants wilt away.
% Faerie Fire
% 1st-level evocation
% Casting Time: 1 action
% Range: 60 feet
% Components: V
% Duration: Concentration, up to 1 minute
% 
% Each object in a 20-foot cube within range is outlined in blue, green, or violet light (your choice). Any creature in the area when the spell is cast is also outlined in light if it fails a Dexterity saving throw. For the duration, objects and affected creatures shed dim light in a 10-foot radius.
% 
% Any attack roll against an affected creature or object has advantage if the attacker can see it, and the affected creature or object can't benefit from being invisible.
% Fog Cloud
% 1st-level conjuration
% Casting Time: 1 action
% Range: 120 feet
% Components: V, S
% Duration: Concentration, up to 1 hour
% 
% You create a 20-foot-radius sphere of fog centered on a point within range. The sphere spreads around corners, and its area is heavily obscured. It lasts for the duration or until a wind of moderate or greater speed (at least 10 miles per hour) disperses it.
% 
% At higher levels:
% 
% When you cast this spell using a spell slot of 2nd level or higher, the radius of the fog increases by 20 feet for each slot level above 1st.
% Goodberry
% 1st-level transmutation
% Casting Time: 1 action
% Range: Touch
% Components: V, S, M
% Duration: Instantaneous
% 
% Up to ten berries appear in your hand and are infused with magic for the duration. A creature can use its action to eat one berry. Eating a berry restores 1 hit point, and the berry provides enough nourishment to sustain a creature for one day.
% 
% The berries lose their potency if they have not been consumed within 24 hours of the casting of this spell.
% Healing Word
% 1st-level evocation
% Casting Time: 1 bonus action
% Range: 60 feet
% Components: V
% Duration: Instantaneous
% 
% A creature of your choice that you can see within range regains hit points equal to 1d4 + your spellcasting ability modifier. This spell has no effect on undead or constructs.
% 
% At higher levels:
% 
% When you cast this spell using a spell slot of 2nd level or higher, the healing increases by 1d4 for each slot level above 1st.
% Ice Knife
% 1st-level conjuration
% Casting Time: 1 action
% Range: 60 feet
% Components: S, M
% Duration: Instantaneous
% 
% You create a shard of ice and fling it at one creature within range. Make a ranged spell attack against the target. On a hit, the target takes 1d10 piercing damage. Hit or miss, the shard then explodes. The target and each creature within 5 feet of it must succeed on a Dexterity saving throw or take 2d6 cold damage.
% 
% At higher levels:
% 
% When you cast this spell using a spell slot of 2nd level or higher, the cold damage increases by 1d6 for each slot level above 1st.
% Jump
% 1st-level transmutation
% Casting Time: 1 action
% Range: Touch
% Components: V, S, M
% Duration: 1 minute
% 
% You touch a creature. The creature's jump distance is tripled until the spell ends.
% Longstrider
% 1st-level transmutation
% Casting Time: 1 action
% Range: Touch
% Components: V, S, M
% Duration: 1 hour
% 
% You touch a creature. The target's speed increases by 10 feet until the spell ends.
% 
% At higher levels:
% 
% When you cast this spell using a spell slot of 2nd level or higher, you can target one additional creature for each slot level above 1st.
% Purify Food and Drink
% 1st-level transmutation
% Casting Time: 1 action
% Range: 10 feet
% Components: V, S
% Duration: Instantaneous
% 
% All nonmagical food and drink within a 5-foot-radius sphere centered on a point of your choice within range is purified and rendered free of poison and disease.
% Speak with Animals
% 1st-level divination
% Casting Time: 1 action
% Range: Self
% Components: V, S
% Duration: 10 minutes
% 
% You gain the ability to comprehend and verbally communicate with beasts for the duration. The knowledge and awareness of many beasts is limited by their intelligence, but at minimum, beasts can give you information about nearby locations and monsters, including whatever they can perceive or have perceived within the past day. You might be able to persuade a beast to perform a small favor for you, at the GM's discretion.
% Thunderwave
% 1st-level evocation
% Casting Time: 1 action
% Range: Self
% Components: V, S
% Duration: Instantaneous
% 
% A wave of thunderous force sweeps out from you. Each creature in a 15-foot cube originating from you must make a Constitution saving throw. On a failed save, a creature takes 2d8 thunder damage and is pushed 10 feet away from you. On a successful save, the creature takes half as much damage and isn't pushed.
% 
% In addition, unsecured objects that are completely within the area of effect are automatically pushed 10 feet away from you by the spell's effect, and the spell emits a thunderous boom audible out to 300 feet.
% 
% At higher levels:
% 
% When you cast this spell using a spell slot of 2nd level or higher, the damage increases by 1d8 for each slot level above 1st.
% Animal Messenger
% 2nd-level enchantment
% Casting Time: 1 action
% Range: 30 feet
% Components: V, S, M
% Duration: 24 hours
% 
% By means of this spell, you use an animal to deliver a message. Choose a Tiny beast you can see within range, such as a squirrel, a blue jay, or a bat. You specify a location, which you must have visited, and a recipient who matches a general description, such as "a man or woman dressed in the uniform of the town guard" or "a red-haired dwarf wearing a pointed hat." You also speak a message of up to twenty-five words. The target beast travels for the duration of the spell toward the specified location, covering about 50 miles per 24 hours for a flying messenger, or 25 miles for other animals.
% 
% When the messenger arrives, it delivers your message to the creature that you described, replicating the sound of your voice. The messenger speaks only to a creature matching the description you gave. If the messenger doesn't reach its destination before the spell ends, the message is lost, and the beast makes its way back to where you cast this spell.
% 
% At higher levels:
% 
% If you cast this spell using a spell slot of 3rd level or higher, the duration of the spell increases by 48 hours for each slot level above 2nd.
% Barkskin
% 2nd-level transmutation
% Casting Time: 1 action
% Range: Touch
% Components: V, S, M
% Duration: Concentration, up to 1 hour
% 
% You touch a willing creature. Until the spell ends, the target's skin has a rough, bark-like appearance, and the target's AC can't be less than 16, regardless of what kind of armor it is wearing.
% Beast Sense
% 2nd-level divination
% Casting Time: 1 action
% Range: Touch
% Components: S
% Duration: Concentration, up to 1 hour
% 
% You touch a willing beast. For the duration of the spell, you can use your action to see through the beast's eyes and hear what it hears, and continue to do so until you use your action to return to your normal senses. While perceiving through the beast's senses, you gain the benefits of any special senses possessed by that creature, though you are blinded and deafened to your own surroundings.
% Darkvision
% 2nd-level transmutation
% Casting Time: 1 action
% Range: Touch
% Components: V, S, M
% Duration: 8 hours
% 
% You touch a willing creature to grant it the ability to see in the dark. For the duration, that creature has darkvision out to a range of 60 feet.
% Dust Devil
% 2nd-level conjuration
% Casting Time: 1 action
% Range: 60 feet
% Components: V, S, M
% Duration: Concentration, up to 1 minute
% 
% Choose an unoccupied 5-foot cube of air that you can see within range. An elemental force that resembles a dust devil appears in the cube and lasts for the spell's duration.
% 
% Any creature that ends its turn within 5 feet of the dust devil must make a Strength saving throw. On a failed save, the creature takes 1d8 bludgeoning damage and is pushed 10 feet away. On a successful save, the creature takes half as much damage and isn't pushed.
% 
% As a bonus action, you can move the dust devil up to 30 feet in any direction. If the dust devil moves over sand, dust, loose dirt, or small gravel, it sucks up the material and forms a 10-foot-radius cloud of debris around itself that lasts until the start of your next turn. The cloud heavily obscures its area.
% 
% At higher levels:
% 
% When you cast this spell using a spell slot of 3rd level or higher, the damage increases by 1d8 for each slot level above 2nd.
% Earthbind
% 2nd-level transmutation
% Casting Time: 1 action
% Range: 300 feet
% Components: V
% Duration: Concentration, up to 1 minute
% 
% Choose one creature you can see within range. Yellow strips of magical energy loop around the creature. The target must succeed on a Strength saving throw, or its flying speed (if any) is reduced to 0 feet for the spell's duration. An airborne creature affected by this spell safely descends at 60 feet per round until it reaches the ground or the spell ends.
% Enhance Ability
% 2nd-level transmutation
% Casting Time: 1 action
% Range: Touch
% Components: V, S, M
% Duration: Concentration, up to 1 hour
% 
% You touch a creature and bestow upon it a magical enhancement. Choose one of the following effects; the target gains that effect until the spell ends.
% 
% Bear's Endurance. The target has advantage on Constitution checks. It also gains 2d6 temporary hit points, which are lost when the spell ends.
% 
% Bull's Strength. The target has advantage on Strength checks, and his or her carrying capacity doubles.
% 
% Cat's Grace. The target has advantage on Dexterity checks. It also doesn't take damage from falling 20 feet or less if it isn't incapacitated.
% 
% Eagle's Splendor. The target has advantage on Charisma checks.
% 
% Fox's Cunning. The target has advantage on Intelligence checks.
% 
% Owl's Wisdom. The target has advantage on Wisdom checks.
% 
% At higher levels:
% 
% When you cast this spell using a spell slot of 3rd level or higher, you can target one additional creature for each slot level above 2nd.
% Find Traps
% 2nd-level divination
% Casting Time: 1 action
% Range: 120 feet
% Components: V, S
% Duration: Instantaneous
% 
% You sense the presence of any trap within range that is within line of sight. A trap, for the purpose of this spell, includes anything that would inflict a sudden or unexpected effect you consider harmful or undesirable, which was specifically intended as such by its creator. Thus, the spell would sense an area affected by the alarm spell, a glyph of warding, or a mechanical pit trap, but it would not reveal a natural weakness in the floor, an unstable ceiling, or a hidden sinkhole.
% 
% This spell merely reveals that a trap is present. You don't learn the location of each trap, but you do learn the general nature of the danger posed by a trap you sense.
% Flame Blade
% 2nd-level evocation
% Casting Time: 1 bonus action
% Range: Self
% Components: V, S, M
% Duration: Concentration, up to 10 minutes
% 
% You evoke a fiery blade in your free hand. The blade is similar in size and shape to a scimitar, and it lasts for the duration. If you let go of the blade, it disappears, but you can evoke the blade again as a bonus action.
% 
% You can use your action to make a melee spell attack with the fiery blade. On a hit, the target takes 3d6 fire damage.
% 
% The flaming blade sheds bright light in a 10-foot radius and dim light for an additional 10 feet.
% 
% At higher levels:
% 
% When you cast this spell using a spell slot of 4th level or higher, the damage increases by 1d6 for every two slot levels above 2nd.
% Flaming Sphere
% 2nd-level conjuration
% Casting Time: 1 action
% Range: 60 feet
% Components: V, S, M
% Duration: Concentration, up to 1 minute
% 
% A 5-foot-diameter sphere of fire appears in an unoccupied space of your choice within range and lasts for the duration. Any creature that ends its turn within 5 feet of the sphere must make a Dexterity saving throw. The creature takes 2d6 fire damage on a failed save, or half as much damage on a successful one.
% 
% As a bonus action, you can move the sphere up to 30 feet. If you ram the sphere into a creature, that creature must make the saving throw against the sphere's damage, and the sphere stops moving this turn.
% 
% When you move the sphere, you can direct it over barriers up to 5 feet tall and jump it across pits up to 10 feet wide. The sphere ignites flammable objects not being worn or carried, and it sheds bright light in a 20-foot radius and dim light for an additional 20 feet.
% 
% At higher levels:
% 
% When you cast this spell using a spell slot of 3rd level or higher, the damage increases by 1d6 for each slot level above 2nd.
% Gust of Wind
% 2nd-level evocation
% Casting Time: 1 action
% Range: Self
% Components: V, S, M
% Duration: Concentration, up to 1 minute
% 
% A line of strong wind 60 feet long and 10 feet wide blasts from you in a direction you choose for the spell's duration. Each creature that starts its turn in the line must succeed on a Strength saving throw or be pushed 15 feet away from you in a direction following the line.
% 
% Any creature in the line must spend 2 feet of movement for every 1 foot it moves when moving closer to you.
% 
% The gust disperses gas or vapor, and it extinguishes candles, torches, and similar unprotected flames in the area. It causes protected flames, such as those of lanterns, to dance wildly and has a 50 percent chance to extinguish them.
% 
% As a bonus action on each of your turns before the spell ends, you can change the direction in which the line blasts from you.
% Heat Metal
% 2nd-level transmutation
% Casting Time: 1 action
% Range: 60 feet
% Components: V, S, M
% Duration: Concentration, up to 1 minute
% 
% Choose a manufactured metal object, such as a metal weapon or a suit of heavy or medium metal armor, that you can see within range. You cause the object to glow red-hot. Any creature in physical contact with the object takes 2d8 fire damage when you cast the spell. Until the spell ends, you can use a bonus action on each of your subsequent turns to cause this damage again.
% 
% If a creature is holding or wearing the object and takes the damage from it, the creature must succeed on a Constitution saving throw or drop the object if it can. If it doesn't drop the object, it has disadvantage on attack rolls and ability checks until the start of your next turn.
% 
% At higher levels:
% 
% When you cast this spell using a spell slot of 3rd level or higher, the damage increases by 1d8 for each slot level above 2nd.
% Hold Person
% 2nd-level enchantment
% Casting Time: 1 action
% Range: 60 feet
% Components: V, S, M
% Duration: Concentration, up to 1 minute
% 
% Choose a humanoid that you can see within range. The target must succeed on a Wisdom saving throw or be paralyzed for the duration. At the end of each of its turns, the target can make another Wisdom saving throw. On a success, the spell ends on the target.
% 
% At higher levels:
% 
% When you cast this spell using a spell slot of 3rd level or higher, you can target one additional humanoid for each slot level above 2nd. The humanoids must be within 30 feet of each other when you target them.
% Lesser Restoration
% 2nd-level abjuration
% Casting Time: 1 action
% Range: Touch
% Components: V, S
% Duration: Instantaneous
% 
% You touch a creature and can end either one disease or one condition afflicting it. The condition can be blinded, deafened, paralyzed, or poisoned.
% Locate Animals or Plants
% 2nd-level divination
% Casting Time: 1 action
% Range: Self
% Components: V, S, M
% Duration: Instantaneous
% 
% Describe or name a specific kind of beast or plant. Concentrating on the voice of nature in your surroundings, you learn the direction and distance to the closest creature or plant of that kind within 5 miles, if any are present.
% Locate Object
% 2nd-level divination
% Casting Time: 1 action
% Range: Self
% Components: V, S, M
% Duration: Concentration, up to 10 minutes
% 
% Describe or name an object that is familiar to you. You sense the direction to the object's location, as long as that object is within 1,000 feet of you. If the object is in motion, you know the direction of its movement.
% 
% The spell can locate a specific object known to you, as long as you have seen it up close--within 30 feet--at least once. Alternatively, the spell can locate the nearest object of a particular kind, such as a certain kind of apparel, jewelry, furniture, tool, or weapon.
% 
% This spell can't locate an object if any thickness of lead, even a thin sheet, blocks a direct path between you and the object.
% Moonbeam
% 2nd-level evocation
% Casting Time: 1 action
% Range: 120 feet
% Components: V, S, M
% Duration: Concentration, up to 1 minute
% 
% A silvery beam of pale light shines down in a 5-foot- radius, 40-foot-high cylinder centered on a point within range. Until the spell ends, dim light fills the cylinder.
% 
% When a creature enters the spell's area for the first time on a turn or starts its turn there, it is engulfed in ghostly flames that cause searing pain, and it must make a Constitution saving throw. It takes 2d10 radiant damage on a failed save, or half as much damage on a successful one.
% 
% A shapechanger makes its saving throw with disadvantage. If it fails, it also instantly reverts to its original form and can't assume a different form until it leaves the spell's light.
% 
% On each of your turns after you cast this spell, you can use an action to move the beam up to 60 feet in any direction.
% 
% At higher levels:
% 
% When you cast this spell using a spell slot of 3rd level or higher, the damage increases by 1d10 for each slot level above 2nd.
% Pass without Trace
% 2nd-level abjuration
% Casting Time: 1 action
% Range: Self
% Components: V, S, M
% Duration: Concentration, up to 1 hour
% 
% A veil of shadows and silence radiates from you, masking you and your companions from detection. For the duration, each creature you choose within 30 feet of you (including you) has a +10 bonus to Dexterity (Stealth) checks and can't be tracked except by magical means. A creature that receives this bonus leaves behind no tracks or other traces of its passage.
% Protection from Poison
% 2nd-level abjuration
% Casting Time: 1 action
% Range: Touch
% Components: V, S
% Duration: 1 hour
% 
% You touch a creature. If it is poisoned, you neutralize the poison. If more than one poison afflicts the target, you neutralize one poison that you know is present, or you neutralize one at random.
% 
% For the duration, the target has advantage on saving throws against being poisoned, and it has resistance to poison damage.
% Skywrite
% 2nd-level transmutation
% Casting Time: 1 action
% Range: Sight
% Components: V, S
% Duration: Concentration, up to 1 hour
% 
% You cause up to ten words to form in a part of the sky you can see. The words appear to be made of cloud and remain in place for the spell's duration. The words dissipate when the spell ends. A strong wind can disperse the clouds and end the spell early.
% Spike Growth
% 2nd-level transmutation
% Casting Time: 1 action
% Range: 150 feet
% Components: V, S, M
% Duration: Concentration, up to 10 minutes
% 
% The ground in a 20-foot radius centered on a point within range twists and sprouts hard spikes and thorns. The area becomes difficult terrain for the duration. When a creature moves into or within the area, it takes 2d4 piercing damage for every 5 feet it travels.
% 
% The transformation of the ground is camouflaged to look natural. Any creature that can't see the area at the time the spell is cast must make a Wisdom (Perception) check against your spell save DC to recognize the terrain as hazardous before entering it.
% Warding Wind
% 2nd-level evocation
% Casting Time: 1 action
% Range: Self
% Components: V
% Duration: Concentration, up to 10 minutes
% 
% A strong wind (20 miles per hour) blows around you in a 10-foot radius and moves with you, remaining centered on you. The wind lasts for the spell's duration.
% 
% The wind has the following effects:
% 
%     It deafens you and other creatures in its area.
%     It extinguishes unprotected flames in its area that are torch-sized or smaller.
%     It hedges out vapor, gas, and fog that can be dispersed by strong wind.
%     The area is difficult terrain for creatures other than you.
%     The attack rolls of ranged weapon attacks have disadvantage if the attacks pass in or out of the wind.
% 
% Call Lightning
% 3rd-level conjuration
% Casting Time: 1 action
% Range: 120 feet
% Components: V, S
% Duration: Concentration, up to 10 minutes
% 
% A storm cloud appears in the shape of a cylinder that is 10 feet tall with a 60-foot radius, centered on a point you can see 100 feet directly above you. The spell fails if you can't see a point in the air where the storm cloud could appear (for example, if you are in a room that can't accommodate the cloud).
% 
% When you cast the spell, choose a point you can see within range. A bolt of lightning flashes down from the cloud to that point. Each creature within 5 feet of that point must make a Dexterity saving throw. A creature takes 3d10 lightning damage on a failed save, or half as much damage on a successful one. On each of your turns until the spell ends, you can use your action to call down lightning in this way again, targeting the same point or a different one.
% If you are outdoors in stormy conditions when you cast this spell, the spell gives you control over the existing storm instead of creating a new one. Under such conditions, the spell's damage increases by 1d10.
% 
% At higher levels:
% 
% When you cast this spell using a spell slot of 4th or higher level, the damage increases by 1d10 for each slot level above 3rd.
% Conjure Animals
% 3rd-level conjuration
% Casting Time: 1 action
% Range: 60 feet
% Components: V, S
% Duration: Concentration, up to 1 hour
% 
% You summon fey spirits that take the form of beasts and appear in unoccupied spaces that you can see within range. Choose one of the following options for what appears:
% 
%     One beast of challenge rating 2 or lower
%     Two beasts of challenge rating 1 or lower
%     Four beasts of challenge rating 1/2 or lower
%     Eight beasts of challenge rating 1/4 or lower
% 
% 
% Each beast is also considered fey, and it disappears when it drops to 0 hit points or when the spell ends.
% 
% The summoned creatures are friendly to you and your companions. Roll initiative for the summoned creatures as a group, which has its own turns. They obey any verbal commands that you issue to them (no action required by you). If you don't issue any commands to them, they defend themselves from hostile creatures, but otherwise take no actions.
% 
% The GM has the creatures' statistics. Sample creatures can be found below.
% 
% At higher levels:
% 
% When you cast this spell using certain higher-level spell slots, you choose one of the summoning options above, and more creatures appear: twice as many with a 5th-level slot, three times as many with a 7th-level slot, and four times as many with a 9th-level slot.
% 
% Sample Creatures
% 
% Daylight
% 3rd-level evocation
% Casting Time: 1 action
% Range: 60 feet
% Components: V, S
% Duration: 1 hour
% 
% A 60-foot-radius sphere of light spreads out from a point you choose within range. The sphere is bright light and sheds dim light for an additional 60 feet.
% 
% If you chose a point on an object you are holding or one that isn't being worn or carried, the light shines from the object and moves with it. Completely covering the affected object with an opaque object, such as a bowl or a helm, blocks the light.
% 
% If any of this spell's area overlaps with an area of darkness created by a spell of 3rd level or lower, the spell that created the darkness is dispelled.
% Dispel Magic
% 3rd-level abjuration
% Casting Time: 1 action
% Range: 120 feet
% Components: V, S
% Duration: Instantaneous
% 
% Choose one creature, object, or magical effect within range. Any spell of 3rd level or lower on the target ends. For each spell of 4th level or higher on the target, make an ability check using your spellcasting ability. The DC equals 10 + the spell's level. On a successful check, the spell ends.
% 
% At higher levels:
% 
% When you cast this spell using a spell slot of 4th level or higher, you automatically end the effects of a spell on the target if the spell's level is equal to or less than the level of the spell slot you used.
% Erupting Earth
% 3rd-level transmutation
% Casting Time: 1 action
% Range: 120 feet
% Components: V, S, M
% Duration: Instantaneous
% 
% Choose a point you can see on the ground within range. A fountain of churned earth and stone erupts in a 20-foot cube centered on that point. Each creature in that area must make a Dexterity saving throw. A creature takes 3d12 bludgeoning damage on a failed save, or half as much damage on a successful one. Additionally, the ground in that area becomes difficult terrain until cleared. Each 5-foot-square portion of the area requires at least 1 minute to clear by hand.
% 
% At higher levels:
% 
% When you cast this spell using a spell slot of 4th level or higher, the damage increases by 1d12 for each slot level above 3rd.
% Feign Death
% 3rd-level necromancy
% Casting Time: 1 action
% Range: Touch
% Components: V, S, M
% Duration: 1 hour
% 
% You touch a willing creature and put it into a cataleptic state that is indistinguishable from death.
% 
% For the spell’s duration, or until you use an action to touch the target and dismiss the spell, the target appears dead to all outward inspection and to spells used to determine the target’s status. The target is blinded and incapacitated, and its speed drops to 0. The target has resistance to all damage except psychic damage. If the target is diseased or poisoned when you cast the spell, or becomes diseased or poisoned while under the spell’s effect, the disease and poison have no effect until the spell ends.
% Flame Arrows
% 3rd-level transmutation
% Casting Time: 1 action
% Range: Touch
% Components: V, S
% Duration: Concentration, up to 1 hour
% 
% You touch a quiver containing arrows or bolts. When a target is hit by a ranged weapon attack using a piece of ammunition drawn from the quiver, the target takes an extra 1d6 fire damage. The spell's magic ends on the piece of ammunition when it hits or misses, and the spell ends when twelve pieces of ammunition have been drawn from the quiver.
% 
% At higher levels:
% 
% When you cast this spell using a spell slot of 4th level or higher, the number of pieces of ammunition you can affect with this spell increases by two for each slot level above 3rd.
% Meld into Stone
% 3rd-level transmutation
% Casting Time: 1 action
% Range: Touch
% Components: V, S
% Duration: 8 hours
% 
% You step into a stone object or surface large enough to fully contain your body, melding yourself and all the equipment you carry with the stone for the duration. Using your movement, you step into the stone at a point you can touch. Nothing of your presence remains visible or otherwise detectable by nonmagical senses.
% 
% While merged with the stone, you can't see what occurs outside it, and any Wisdom (Perception) checks you make to hear sounds outside it are made with disadvantage. You remain aware of the passage of time and can cast spells on yourself while merged in the stone. You can use your movement to leave the stone where you entered it, which ends the spell. You otherwise can't move.
% 
% Minor physical damage to the stone doesn't harm you, but its partial destruction or a change in its shape (to the extent that you no longer fit within it) expels you and deals 6d6 bludgeoning damage to you. The stone's complete destruction (or transmutation into a different substance) expels you and deals 50 bludgeoning damage to you. If expelled, you fall prone in an unoccupied space closest to where you first entered.
% Plant Growth
% 3rd-level transmutation
% Casting Time: 1 action
% Range: 150 feet
% Components: V, S
% Duration: Instantaneous
% 
% This spell channels vitality into plants within a specific area. There are two possible uses for the spell, granting either immediate or long-term benefits.
% 
% If you cast this spell using 1 action, choose a point within range. All normal plants in a 100-foot radius centered on that point become thick and overgrown. A creature moving through the area must spend 4 feet of movement for every 1 foot it moves.
% 
% You can exclude one or more areas of any size within the spell's area from being affected.
% 
% If you cast this spell over 8 hours, you enrich the land. All plants in a half-mile radius centered on a point within range become enriched for 1 year. The plants yield twice the normal amount of food when harvested.
% Protection from Energy
% 3rd-level abjuration
% Casting Time: 1 action
% Range: Touch
% Components: V, S
% Duration: Concentration, up to 1 hour
% 
% For the duration, the willing creature you touch has resistance to one damage type of your choice: acid, cold, fire, lightning, or thunder.
% Sleet Storm
% 3rd-level conjuration
% Casting Time: 1 action
% Range: 150 feet
% Components: V, S, M
% Duration: Concentration, up to 1 minute
% 
% Until the spell ends, freezing rain and sleet fall in a 20-foot-tall cylinder with a 40-foot radius centered on a point you choose within range. The area is heavily obscured, and exposed flames in the area are doused.
% 
% The ground in the area is covered with slick ice, making it difficult terrain. When a creature enters the spell's area for the first time on a turn or starts its turn there, it must make a Dexterity saving throw. On a failed save, it falls prone.
% 
% If a creature is concentrating in the spell's area, the creature must make a successful Constitution saving throw against your spell save DC or lose concentration.
% Speak with Plants
% 3rd-level transmutation
% Casting Time: 1 action
% Range: Self
% Components: V, S
% Duration: 10 minutes
% 
% You imbue plants within 30 feet of you with limited sentience and animation, giving them the ability to communicate with you and follow your simple commands. You can question plants about events in the spell's area within the past day, gaining information about creatures that have passed, weather, and other circumstances.
% 
% You can also turn difficult terrain caused by plant growth (such as thickets and undergrowth) into ordinary terrain that lasts for the duration. Or you can turn ordinary terrain where plants are present into difficult terrain that lasts for the duration, causing vines and branches to hinder pursuers, for example.
% 
% Plants might be able to perform other tasks on your behalf, at the GM's discretion. The spell doesn't enable plants to uproot themselves and move about, but they can freely move branches, tendrils, and stalks.
% 
% If a plant creature is in the area, you can communicate with it as if you shared a common language, but you gain no magical ability to influence it.
% 
% This spell can cause the plants created by the entangle spell to release a restrained creature.
% Tidal Wave
% 3rd-level conjuration
% Casting Time: 1 action
% Range: 120 feet
% Components: V, S, M
% Duration: Instantaneous
% 
% You conjure up a wave of water that crashes down on an area within range. The area can be up to 30 feet long, up to 10 feet wide, and up to 10 feet tall. Each creature in that area must make a Dexterity saving throw. On a failed save, a creature takes 4d8 bludgeoning damage and is knocked prone. On a successful save, a creature takes half as much damage and isn't knocked prone. The water then spreads out across the ground in all directions, extinguishing unprotected flames in its area and within 30 feet of it, and then it vanishes.
% Wall of Water
% 3rd-level evocation
% Casting Time: 1 action
% Range: 60 feet
% Components: V, S, M
% Duration: Concentration, up to 10 minutes
% 
% You create a wall of water on the ground at a point you can see within range. You can make the wall up to 30 feet long, 10 feet high, and 1 foot thick, or you can make a ringed wall up to 20 feet in diameter, 20 feet high, and 1 foot thick. The wall vanishes when the spell ends. The wall's space is difficult terrain.
% 
% Any ranged weapon attack that enters the wall's space has disadvantage on the attack roll, and fire damage is halved if the fire effect passes through the wall to reach its target. Spells that deal cold damage that pass through the wall cause the area of the wall they pass through to freeze solid (at least a 5-foot-square section is frozen). Each 5-foot-square frozen section has AC 5 and 15 hit points. Reducing a frozen section to 0 hit points destroys it. When a section is destroyed, the wall's water doesn't fill it.
% Water Breathing
% 3rd-level transmutation
% Casting Time: 1 action
% Range: 30 feet
% Components: V, S, M
% Duration: 24 hours
% 
% This spell grants up to ten willing creatures you can see within range the ability to breathe underwater until the spell ends. Affected creatures also retain their normal mode of respiration.
% Water Walk
% 3rd-level transmutation
% Casting Time: 1 action
% Range: 30 feet
% Components: V, S, M
% Duration: 1 hour
% 
% This spell grants the ability to move across any liquid surface--such as water, acid, mud, snow, quicksand, or lava--as if it were harmless solid ground (creatures crossing molten lava can still take damage from the heat). Up to ten willing creatures you can see within range gain this ability for the duration.
% 
% If you target a creature submerged in a liquid, the spell carries the target to the surface of the liquid at a rate of 60 feet per round.
% Wind Wall
% 3rd-level evocation
% Casting Time: 1 action
% Range: 120 feet
% Components: V, S, M
% Duration: Concentration, up to 1 minute
% 
% A wall of strong wind rises from the ground at a point you choose within range. You can make the wall up to 50 feet long, 15 feet high, and 1 foot thick. You can shape the wall in any way you choose so long as it makes one continuous path along the ground. The wall lasts for the duration.
% 
% When the wall appears, each creature within its area must make a Strength saving throw. A creature takes 3d8 bludgeoning damage on a failed save, or half as much damage on a successful one.
% 
% The strong wind keeps fog, smoke, and other gases at bay. Small or smaller flying creatures or objects can't pass through the wall. Loose, lightweight materials brought into the wall fly upward. Arrows, bolts, and other ordinary projectiles launched at targets behind the wall are deflected upward and automatically miss. (Boulders hurled by giants or siege engines, and similar projectiles, are unaffected.) Creatures in gaseous form can't pass through it.
% Blight
% 4th-level necromancy
% Casting Time: 1 action
% Range: 30 feet
% Components: V, S
% Duration: Instantaneous
% 
% Necromantic energy washes over a creature of your choice that you can see within range, draining moisture and vitality from it. The target must make a Constitution saving throw. The target takes 8d8 necrotic damage on a failed save, or half as much damage on a successful one. This spell has no effect on undead or constructs.
% 
% If you target a plant creature or a magical plant, it makes the saving throw with disadvantage, and the spell deals maximum damage to it.
% 
% If you target a nonmagical plant that isn't a creature, such as a tree or shrub, it doesn't make a saving throw; it simply withers and dies.
% 
% At higher levels:
% 
% When you cast this spell using a spell slot of 5th level or higher, the damage increases by 1d8 for each slot level above 4th.
% Confusion
% 4th-level enchantment
% Casting Time: 1 action
% Range: 90 feet
% Components: V, S, M
% Duration: Concentration, up to 1 minute
% 
% This spell assaults and twists creatures' minds, spawning delusions and provoking uncontrolled action. Each creature in a 10-foot-radius sphere centered on a point you choose within range must succeed on a Wisdom saving throw when you cast this spell or be affected by it.
% 
% An affected target can't take reactions and must roll a d10 at the start of each of its turns to determine its behavior for that turn.
% 
% At the end of each of its turns, an affected target can make a Wisdom saving throw. If it succeeds, this effect ends for that target.
% 
% At higher levels:
% 
% When you cast this spell using a spell slot of 5th level or higher, the radius of the sphere increases by 5 feet for each slot level above 4th.
% Conjure Minor Elementals
% 4th-level conjuration
% Casting Time: 1 minute
% Range: 90 feet
% Components: V, S
% Duration: Concentration, up to 1 hour
% 
% You summon elementals that appear in unoccupied spaces that you can see within range. You choose one the following options for what appears:
% 
%     One elemental of challenge rating 2 or lower
%     Two elementals of challenge rating 1 or lower
%     Four elementals of challenge rating 1/2 or lower
%     Eight elementals of challenge rating 1/4 or lower.
% 
% 
% An elemental summoned by this spell disappears when it drops to 0 hit points or when the spell ends.
% 
% The summoned creatures are friendly to you and your companions. Roll initiative for the summoned creatures as a group, which has its own turns. They obey any verbal commands that you issue to them (no action required by you). If you don't issue any commands to them, they defend themselves from hostile creatures, but otherwise take no actions.
% 
% The GM has the creatures' statistics.
% 
% At higher levels:
% 
% When you cast this spell using certain higher-level spell slots, you choose one of the summoning options above, and more creatures appear: twice as many with a 6th-level slot and three times as many with an 8th-level slot.
% 
% Sample Elementals
% Conjure Woodland Beings
% 4th-level conjuration
% Casting Time: 1 action
% Range: 60 feet
% Components: V, S, M
% Duration: Concentration, up to 1 hour
% 
% You summon fey creatures that appear in unoccupied spaces that you can see within range. Choose one of the following options for what appears:
% 
%     One fey creature of challenge rating 2 or lower
%     Two fey creatures of challenge rating 1 or lower
%     Four fey creatures of challenge rating 1/2 or lower
%     Eight fey creatures of challenge rating 1/4 or lower
% 
% 
% A summoned creature disappears when it drops to 0 hit points or when the spell ends.
% 
% The summoned creatures are friendly to you and your companions. Roll initiative for the summoned creatures as a group, which have their own turns. They obey any verbal commands that you issue to them (no action required by you). If you don't issue any commands to them, they defend themselves from hostile creatures, but otherwise take no actions.
% 
% The GM has the creatures' statistics. You can see some sample creatures below.
% 
% At higher levels:
% 
% When you cast this spell using certain higher-level spell slots, you choose one of the summoning options above, and more creatures appear: twice as many with a 6th-level slot and three times as many with an 8th-level slot.
% 
% Sample Creatures
% 
% Control Water
% 4th-level transmutation
% Casting Time: 1 action
% Range: 300 feet
% Components: V, S, M
% Duration: Concentration, up to 10 minutes
% 
% Until the spell ends, you control any freestanding water inside an area you choose that is a cube up to 100 feet on a side. You can choose from any of the following effects when you cast this spell. As an action on your turn, you can repeat the same effect or choose a different one.
% 
% Flood. You cause the water level of all standing water in the area to rise by as much as 20 feet. If the area includes a shore, the flooding water spills over onto dry land.
% 
% If you choose an area in a large body of water, you instead create a 20-foot tall wave that travels from one side of the area to the other and then crashes down. Any Huge or smaller vehicles in the wave's path are carried with it to the other side. Any Huge or smaller vehicles struck by the wave have a 25 percent chance of capsizing.
% 
% The water level remains elevated until the spell ends or you choose a different effect. If this effect produced a wave, the wave repeats on the start of your next turn while the flood effect lasts.
% 
% Part Water. You cause water in the area to move apart and create a trench. The trench extends across the spell's area, and the separated water forms a wall to either side. The trench remains until the spell ends or you choose a different effect. The water then slowly fills in the trench over the course of the next round until the normal water level is restored.
% 
% Redirect Flow. You cause flowing water in the area to move in a direction you choose, even if the water has to flow over obstacles, up walls, or in other unlikely directions. The water in the area moves as you direct it, but once it moves beyond the spell's area, it resumes its flow based on the terrain conditions. The water continues to move in the direction you chose until the spell ends or you choose a different effect.
% 
% Whirlpool. This effect requires a body of water at least 50 feet square and 25 feet deep. You cause a whirlpool to form in the center of the area. The whirlpool forms a vortex that is 5 feet wide at the base, up to 50 feet wide at the top, and 25 feet tall. Any creature or object in the water and within 25 feet of the vortex is pulled 10 feet toward it. A creature can swim away from the vortex by making a Strength (Athletics) check against your spell save DC.
% 
% When a creature enters the vortex for the first time on a turn or starts its turn there, it must make a Strength saving throw. On a failed save, the creature takes 2d8 bludgeoning damage and is caught in the vortex until the spell ends. On a successful save, the creature takes half damage, and isn't caught in the vortex. A creature caught in the vortex can use its action to try to swim away from the vortex as described above, but has disadvantage on the Strength (Athletics) check to do so.
% 
% The first time each turn that an object enters the vortex, the object takes 2d8 bludgeoning damage; this damage occurs each round it remains in the vortex.
% Dominate Beast
% 4th-level enchantment
% Casting Time: 1 action
% Range: 60 feet
% Components: V, S
% Duration: Concentration, up to 1 minute
% 
% You attempt to beguile a beast that you can see within range. It must succeed on a Wisdom saving throw or be charmed by you for the duration. If you or creatures that are friendly to you are fighting it, it has advantage on the saving throw.
% 
% While the beast is charmed, you have a telepathic link with it as long as the two of you are on the same plane of existence. You can use this telepathic link to issue commands to the creature while you are conscious (no action required), which it does its best to obey. You can specify a simple and general course of action, such as "Attack that creature," "Run over there," or "Fetch that object." If the creature completes the order and doesn't receive further direction from you, it defends and preserves itself to the best of its ability.
% 
% You can use your action to take total and precise control of the target. Until the end of your next turn, the creature takes only the actions you choose, and doesn't do anything that you don't allow it to do. During this time, you can also cause the creature to use a reaction, but this requires you to use your own reaction as well.
% 
% Each time the target takes damage, it makes a new Wisdom saving throw against the spell. If the saving throw succeeds, the spell ends.
% 
% At higher levels:
% 
% When you cast this spell with a 5th-level spell slot, the duration is concentration, up to 10 minutes. When you use a 6th-level spell slot, the duration is concentration, up to 1 hour. When you use a spell slot of 7th level or higher, the duration is concentration, up to 8 hours.
% Elemental Bane
% 4th-level transmutation
% Casting Time: 1 action
% Range: 90 feet
% Components: V, S
% Duration: Concentration, up to 1 minute
% 
% Choose one creature you can see within range, and choose one of the following damage types: acid, cold, fire, lightning, or thunder. The target must succeed on a Constitution saving throw or be affected by the spell for its duration. The first time each turn the affected target takes damage of the chosen type, the target takes an extra 2d6 damage of that type. Moreover, the target loses any resistance to that damage type until the spell ends.
% 
% At higher levels:
% 
% When you cast this spell using a spell slot of 5th level or higher, you can target one additional creature for each slot level above 4th. The creatures must be within 30 feet of each other when you target them.
% Freedom of Movement
% 4th-level abjuration
% Casting Time: 1 action
% Range: Touch
% Components: V, S, M
% Duration: 1 hour
% 
% You touch a willing creature. For the duration, the target's movement is unaffected by difficult terrain, and spells and other magical effects can neither reduce the target's speed nor cause the target to be paralyzed or restrained.
% 
% The target can also spend 5 feet of movement to automatically escape from nonmagical restraints, such as manacles or a creature that has it grappled. Finally, being underwater imposes no penalties on the target's movement or attacks.
% Giant Insect
% 4th-level transmutation
% Casting Time: 1 action
% Range: 30 feet
% Components: V, S
% Duration: Concentration, up to 10 minutes
% 
% You transform up to ten centipedes, three spiders, five wasps, or one scorpion within range into giant versions of their natural forms for the duration. A centipede becomes a giant centipede, a spider becomes a giant spider, a wasp becomes a giant wasp, and a scorpion becomes a giant scorpion.
% 
% Each creature obeys your verbal commands, and in combat, they act on your turn each round. The GM has the statistics for these creatures and resolves their actions and movement.
% 
% A creature remains in its giant size for the duration, until it drops to 0 hit points, or until you use an action to dismiss the effect on it.
% 
% The GM might allow you to choose different targets. For example, if you transform a bee, its giant version might have the same statistics as a giant wasp.
% Grasping Vine
% 4th-level conjuration
% Casting Time: 1 bonus action
% Range: 30 feet
% Components: V, S
% Duration: Concentration, up to 1 minute
% 
% You conjure a vine that sprouts from the ground in an unoccupied space of your choice that you can see within range. When you cast this spell, you can direct the vine to lash out at a creature within 30 feet of it that you can see. That creature must succeed on a dexterity saving throw or be pulled 20 feet directly toward the vine.
% 
% Until the spell ends, you can direct the vine to lash out at the same creature or another one as a bonus action on each of your turns.
% Hallucinatory Terrain
% 4th-level illusion
% Casting Time: 10 minutes
% Range: 300 feet
% Components: V, S, M
% Duration: 24 hours
% 
% You make natural terrain in a 150-foot cube in range look, sound, and smell like some other sort of natural terrain. Thus, open fields or a road can be made to resemble a swamp, hill, crevasse, or some other difficult or impassable terrain. A pond can be made to seem like a grassy meadow, a precipice like a gentle slope, or a rock-strewn gully like a wide and smooth road. Manufactured structures, equipment, and creatures within the area aren't changed in appearance.
% 
% The tactile characteristics of the terrain are unchanged, so creatures entering the area are likely to see through the illusion. If the difference isn't obvious by touch, a creature carefully examining the illusion can attempt an Intelligence (Investigation) check against your spell save DC to disbelieve it. A creature who discerns the illusion for what it is, sees it as a vague image superimposed on the terrain.
% Ice Storm
% 4th-level evocation
% Casting Time: 1 action
% Range: 300 feet
% Components: V, S, M
% Duration: Instantaneous
% 
% A hail of rock-hard ice pounds to the ground in a 20- foot-radius, 40-foot-high cylinder centered on a point within range. Each creature in the cylinder must make a Dexterity saving throw. A creature takes 2d8 bludgeoning damage and 4d6 cold damage on a failed save, or half as much damage on a successful one.
% 
% Hailstones turn the storm's area of effect into difficult terrain until the end of your next turn.
% 
% At higher levels:
% 
% When you cast this spell using a spell slot of 5th level or higher, the bludgeoning damage increases by 1d8 for each slot level above 4th.
% Locate Creature
% 4th-level divination
% Casting Time: 1 action
% Range: Self
% Components: V, S, M
% Duration: Concentration, up to 1 hour
% 
% Describe or name a creature that is familiar to you. You sense the direction to the creature's location, as long as that creature is within 1,000 feet of you. If the creature is moving, you know the direction of its movement.
% 
% The spell can locate a specific creature known to you, or the nearest creature of a specific kind (such as a human or a unicorn), so long as you have seen such a creature up close--within 30 feet--at least once. If the creature you described or named is in a different form, such as being under the effects of a polymorph spell, this spell doesn't locate the creature.
% 
% This spell can't locate a creature if running water at least 10 feet wide blocks a direct path between you and the creature.
% Polymorph
% 4th-level transmutation
% Casting Time: 1 action
% Range: 60 feet
% Components: V, S, M
% Duration: Concentration, up to 1 hour
% 
% This spell transforms a creature that you can see within range into a new form. An unwilling creature must make a Wisdom saving throw to avoid the effect. The spell has no effect on a shapechanger or a creature with 0 hit points.
% 
% The transformation lasts for the duration, or until the target drops to 0 hit points or dies. The new form can be any beast whose challenge rating is equal to or less than the target's (or the target's level, if it doesn't have a challenge rating). The target's game statistics, including mental ability scores, are replaced by the statistics of the chosen beast. It retains its alignment and personality.
% 
% The target assumes the hit points of its new form. When it reverts to its normal form, the creature returns to the number of hit points it had before it transformed. If it reverts as a result of dropping to 0 hit points, any excess damage carries over to its normal form. As long as the excess damage doesn't reduce the creature's normal form to 0 hit points, it isn't knocked unconscious.
% 
% The creature is limited in the actions it can perform by the nature of its new form, and it can't speak, cast spells, or take any other action that requires hands or speech.
% 
% The target's gear melds into the new form. The creature can't activate, use, wield, or otherwise benefit from any of its equipment.
% Stone Shape
% 4th-level transmutation
% Casting Time: 1 action
% Range: Touch
% Components: V, S, M
% Duration: Instantaneous
% 
% You touch a stone object of Medium size or smaller or a section of stone no more than 5 feet in any dimension and form it into any shape that suits your purpose. So, for example, you could shape a large rock into a weapon, idol, or coffer, or make a small passage through a wall, as long as the wall is less than 5 feet thick. You could also shape a stone door or its frame to seal the door shut. The object you create can have up to two hinges and a latch, but finer mechanical detail isn't possible.
% Stoneskin
% 4th-level abjuration
% Casting Time: 1 action
% Range: Touch
% Components: V, S, M (diamond dust worth 100 gp, which the spell consumes)
% Duration: Concentration, up to 1 hour
% 
% This spell turns the flesh of a willing creature you touch as hard as stone. Until the spell ends, the target has resistance to nonmagical bludgeoning, piercing, and slashing damage.
% Wall of Fire
% 4th-level evocation
% Casting Time: 1 action
% Range: 120 feet
% Components: V, S, M
% Duration: Concentration, up to 1 minute
% 
% You create a wall of fire on a solid surface within range. You can make the wall up to 60 feet long, 20 feet high, and 1 foot thick, or a ringed wall up to 20 feet in diameter, 20 feet high, and 1 foot thick. The wall is opaque and lasts for the duration.
% 
% When the wall appears, each creature within its area must make a Dexterity saving throw. On a failed save, a creature takes 5d8 fire damage, or half as much damage on a successful save.
% 
% One side of the wall, selected by you when you cast this spell, deals 5d8 fire damage to each creature that ends its turn within 10 feet of that side or inside the wall. A creature takes the same damage when it enters the wall for the first time on a turn or ends its turn there. The other side of the wall deals no damage.
% 
% At higher levels:
% 
% When you cast this spell using a spell slot of 5th level or higher, the damage increases by 1d8 for each slot level above 4th.
% Watery Sphere
% 4th-level conjuration
% Casting Time: 1 action
% Range: 90 feet
% Components: V, S, M
% Duration: Concentration, up to 1 minute
% 
% You conjure up a sphere of water with a 5-foot radius at a point you can see within range. The sphere can hover but no more than 10 feet off the ground. The sphere remains for the spell's duration.
% 
% Any creature in the sphere's space must make a Strength saving throw. On a successful save, a creature is ejected from that space to the nearest unoccupied space of the creature's choice outside the sphere. A Huge or larger creature succeeds on the saving throw automatically, and a Large or smaller creature can choose to fail it. On a failed save, a creature is restrained by the sphere and is engulfed by the water. At the end of each of its turns, a restrained target can repeat the saving throw, ending the effect on itself on a success.
% 
% The sphere can restrain as many as four Medium or smaller creatures or one Large creature. If the sphere restrains a creature that causes it to exceed this capacity, a random creature that was already restrained by the sphere falls out of it and lands prone in a space within 5 feet of it.
% 
% As an action, you can move the sphere up to 30 feet in a straight line. If it moves over a pit, a cliff, or other drop-off, it safely descends until it is hovering 10 feet above the ground. Any creature restrained by the sphere moves with it. You can ram the sphere into creatures, forcing them to make the saving throw.
% 
% When the spell ends, the sphere falls to the ground and extinguishes all normal flames within 30 feet of it. Any creature restrained by the sphere is knocked prone in the space where it falls. The water then vanishes.
% Antilife Shell
% 5th-level abjuration
% Casting Time: 1 action
% Range: Self
% Components: V, S
% Duration: Concentration, up to 1 hour
% 
% A shimmering barrier extends out from you in a 10- foot radius and moves with you, remaining centered on you and hedging out creatures other than undead and constructs. The barrier lasts for the duration.
% 
% The barrier prevents an affected creature from passing or reaching through. An affected creature can cast spells or make attacks with ranged or reach weapons through the barrier.
% 
% If you move so that an affected creature is forced to pass through the barrier, the spell ends.
% Awaken
% 5th-level transmutation
% Casting Time: 8 hours
% Range: Touch
% Components: V, S, M (an agate worth at least 1,000 gp, which the spell consumes)
% Duration: Instantaneous
% 
% After spending the casting time tracing magical pathways within a precious gemstone, you touch a Huge or smaller beast or plant. The target must have either no Intelligence score or an Intelligence of 3 or less. The target gains an Intelligence of 10. The target also gains the ability to speak one language you know. If the target is a plant, it gains the ability to move its limbs, roots, vines, creepers, and so forth, and it gains senses similar to a human's. Your GM chooses statistics appropriate for the awakened plant, such as the statistics for the awakened shrub or the awakened tree.
% 
% The awakened beast or plant is charmed by you for 30 days or until you or your companions do anything harmful to it. When the charmed condition ends, the awakened creature chooses whether to remain friendly to you, based on how you treated it while it was charmed.
% Commune with Nature
% 5th-level divination
% Casting Time: 1 minute
% Range: Self
% Components: V, S
% Duration: Instantaneous
% 
% You briefly become one with nature and gain knowledge of the surrounding territory. In the outdoors, the spell gives you knowledge of the land within 3 miles of you. In caves and other natural underground settings, the radius is limited to 300 feet. The spell doesn't function where nature has been replaced by construction, such as in dungeons and towns.
% 
% You instantly gain knowledge of up to three facts of your choice about any of the following subjects as they relate to the area:
% 
%     terrain and bodies of water
%     prevalent plants, minerals, animals, or peoples
%     powerful celestials, fey, fiends, elementals, or undead
%     influence from other planes of existence
%     buildings
% 
% 
% For example, you could determine the location of powerful undead in the area, the location of major sources of safe drinking water, and the location of any nearby towns.
% Conjure Elemental
% 5th-level conjuration
% Casting Time: 1 minute
% Range: 90 feet
% Components: V, S, M
% Duration: Concentration, up to 1 hour
% 
% You call forth an elemental servant. Choose an area of air, earth, fire, or water that fills a 10-foot cube within range. An elemental of challenge rating 5 or lower appropriate to the area you chose appears in an unoccupied space within 10 feet of it. For example, a fire elemental emerges from a bonfire, and an earth elemental rises up from the ground. The elemental disappears when it drops to 0 hit points or when the spell ends.
% 
% The elemental is friendly to you and your companions for the duration. Roll initiative for the elemental, which has its own turns. It obeys any verbal commands that you issue to it (no action required by you). If you don't issue any commands to the elemental, it defends itself from hostile creatures but otherwise takes no actions.
% 
% If your concentration is broken, the elemental doesn't disappear. Instead, you lose control of the elemental, it becomes hostile toward you and your companions, and it might attack. An uncontrolled elemental can't be dismissed by you, and it disappears 1 hour after you summoned it.
% 
% The GM has the elemental's statistics. Sample elementals can be found below.
% 
% At higher levels:
% 
% When you cast this spell using a spell slot of 6th level or higher, the challenge rating increases by 1 for each slot level above 5th.
% 
% Sample Elementals
% 
% Contagion
% 5th-level necromancy
% Casting Time: 1 action
% Range: Touch
% Components: V, S
% Duration: 7 days
% 
% Your touch inflicts disease. Make a melee spell attack against a creature within your reach. On a hit, you afflict the creature with a disease of your choice from any of the ones described below.
% 
% At the end of each of the target's turns, it must make a Constitution saving throw. After failing three of these saving throws, the disease's effects last for the duration, and the creature stops making these saves. After succeeding on three of these saving throws, the creature recovers from the disease, and the spell ends.
% 
% Since this spell induces a natural disease in its target, any effect that removes a disease or otherwise ameliorates a disease's effects apply to it.
% 
% Blinding Sickness. Pain grips the creature's mind, and its eyes turn milky white. The creature has disadvantage on Wisdom checks and Wisdom saving throws and is blinded.
% 
% Filth Fever. A raging fever sweeps through the creature's body. The creature has disadvantage on Strength checks, Strength saving throws, and attack rolls that use Strength.
% 
% Flesh Rot. The creature's flesh decays. The creature has disadvantage on Charisma checks and vulnerability to all damage.
% 
% Mindfire. The creature's mind becomes feverish. The creature has disadvantage on Intelligence checks and Intelligence saving throws, and the creature behaves as if under the effects of the confusion spell during combat.
% 
% Seizure. The creature is overcome with shaking. The creature has disadvantage on Dexterity checks, Dexterity saving throws, and attack rolls that use Dexterity.
% 
% Slimy Doom. The creature begins to bleed uncontrollably. The creature has disadvantage on Constitution checks and Constitution saving throws. In addition, whenever the creature takes damage, it is stunned until the end of its next turn.
% Control Winds
% 5th-level transmutation
% Casting Time: 1 action
% Range: 300 feet
% Components: V, S
% Duration: Concentration, up to 1 hour
% 
% You take control of the air in a 100-foot cube that you can see within range. Choose one of the following effects when you cast the spell. The effect lasts for the spell's duration, unless you use your action on a later turn to switch to a different effect. You can also use your action to temporarily halt the effect or to restart one you've halted.
% 
% Gusts. A wind picks up within the cube, continually blowing in a horizontal direction you designate. You choose the intensity of the wind: calm, moderate, or strong. If the wind is moderate or strong, ranged weapon attacks that enter or leave the cube or pass through it have disadvantage on their attack rolls. If the wind is strong, any creature moving against the wind must spend 1 extra foot of movement for each foot moved.
% 
% Downdraft. You cause a sustained blast of strong wind to blow downward from the top of the cube. Ranged weapon attacks that pass through the cube or that are made against targets within it have disadvantage on their attack rolls. A creature must make a Strength saving throw if it flies into the cube for the first time on a turn or starts its turn there flying. On a failed save, the creature is knocked prone.
% 
% Updraft. You cause a sustained updraft within the cube, rising upward from the cube's bottom side. Creatures that end a fall within the cube take only half damage from the fall. When a creature in the cube makes a vertical jump, the creature can jump up to 10 feet higher than normal.
% Geas
% 5th-level enchantment
% Casting Time: 1 minute
% Range: 60 feet
% Components: V
% Duration: 30 days
% 
% You place a magical command on a creature that you can see within range, forcing it to carry out some service or refrain from some action or course of activity as you decide. If the creature can understand you, it must succeed on a Wisdom saving throw or become charmed by you for the duration. While the creature is charmed by you, it takes 5d10 psychic damage each time it acts in a manner directly counter to your instructions, but no more than once each day. A creature that can't understand you is unaffected by the spell.
% 
% You can issue any command you choose, short of an activity that would result in certain death. Should you issue a suicidal command, the spell ends.
% 
% You can end the spell early by using an action to dismiss it. A remove curse, greater restoration, or wish spell also ends it.
% 
% At higher levels:
% 
% When you cast this spell using a spell slot of 7th or 8th level, the duration is 1 year. When you cast this spell using a spell slot of 9th level, the spell lasts until it is ended by one of the spells mentioned above.
% Greater Restoration
% 5th-level abjuration
% Casting Time: 1 action
% Range: Touch
% Components: V, S, M (diamond dust worth at least 100gp, which the spell consumes)
% Duration: Instantaneous
% 
% You imbue a creature you touch with positive energy to undo a debilitating effect. You can reduce the target's exhaustion level by one, or end one of the following effects on the target:
% 
%     One effect that charmed or petrified the target
%     One curse, including the target's attunement to a cursed magic item
%     Any reduction to one of the target's ability scores
%     One effect reducing the target's hit point maximum
% 
% Insect Plague
% 5th-level conjuration
% Casting Time: 1 action
% Range: 300 feet
% Components: V, S, M
% Duration: Concentration, up to 10 minutes
% 
% Swarming, biting locusts fill a 20-foot-radius sphere centered on a point you choose within range. The sphere spreads around corners. The sphere remains for the duration, and its area is lightly obscured. The sphere's area is difficult terrain.
% 
% When the area appears, each creature in it must make a Constitution saving throw. A creature takes 4d10 piercing damage on a failed save, or half as much damage on a successful one. A creature must also make this saving throw when it enters the spell's area for the first time on a turn or ends its turn there.
% 
% At higher levels:
% 
% When you cast this spell using a spell slot of 6th level or higher, the damage increases by 1d10 for each slot level above 5th.
% Maelstrom
% 5th-level evocation
% Casting Time: 1 action
% Range: 120 feet
% Components: V, S, M
% Duration: Concentration, up to 1 minute
% 
% A swirling mass of 5-foot-deep water appears in a 30-foot radius centered on a point you can see within range. The point must be on the ground or in a body of water. Until the spell ends, that area is difficult terrain, and any creature that starts its turn there must succeed on a Strength saving throw or take 6d6 bludgeoning damage and be pulled 10 feet toward the center.
% Mass Cure Wounds
% 5th-level conjuration
% Casting Time: 1 action
% Range: 60 feet
% Components: V, S
% Duration: Instantaneous
% 
% A wave of healing energy washes out from a point of your choice within range. Choose up to six creatures in a 30-foot-radius sphere centered on that point. Each target regains hit points equal to 3d8 + your spellcasting ability modifier. This spell has no effect on undead or constructs.
% 
% At higher levels:
% 
% When you cast this spell using a spell slot of 6th level or higher, the healing increases by 1d8 for each slot level above 5th.
% Planar Binding
% 5th-level abjuration
% Casting Time: 1 hour
% Range: 60 feet
% Components: V, S, M (a jewel worth at least 1,000 gp, which the spell consumes)
% Duration: 24 hours
% 
% With this spell, you attempt to bind a celestial, an elemental, a fey, or a fiend to your service. The creature must be within range for the entire casting of the spell. (Typically, the creature is first summoned into the center of an inverted magic circle in order to keep it trapped while this spell is cast.) At the completion of the casting, the target must make a Charisma saving throw. On a failed save, it is bound to serve you for the duration. If the creature was summoned or created by another spell, that spell's duration is extended to match the duration of this spell.
% 
% A bound creature must follow your instructions to the best of its ability. You might command the creature to accompany you on an adventure, to guard a location, or to deliver a message. The creature obeys the letter of your instructions, but if the creature is hostile to you, it strives to twist your words to achieve its own objectives. If the creature carries out your instructions completely before the spell ends, it travels to you to report this fact if you are on the same plane of existence. If you are on a different plane of existence, it returns to the place where you bound it and remains there until the spell ends.
% 
% At higher levels:
% 
% When you cast this spell using a spell slot of a higher level, the duration increases to 10 days with a 6th-level slot, to 30 days with a 7th- level slot, to 180 days with an 8th-level slot, and to a year and a day with a 9th-level spell slot.
% Reincarnate
% 5th-level transmutation
% Casting Time: 1 hour
% Range: Touch
% Components: V, S, M (rare oils and unguents worth at least 1,000 gp, which the spell consumes)
% Duration: Instantaneous
% 
% You touch a dead humanoid or a piece of a dead humanoid. Provided that the creature has been dead no longer than 10 days, the spell forms a new adult body for it and then calls the soul to enter that body. If the target's soul isn't free or willing to do so, the spell fails.
% 
% The magic fashions a new body for the creature to inhabit, which likely causes the creature's race to change. The GM rolls a d100 and consults the following table to determine what form the creature takes when restored to life, or the GM chooses a form.
% 
% The reincarnated creature recalls its former life and experiences. It retains the capabilities it had in its original form, except it exchanges its original race for the new one and changes its racial traits accordingly.
% Scrying
% 5th-level divination
% Casting Time: 10 minutes
% Range: Self
% Components: V, S, M (a focus worth at least 1,000 gp, such as a crystal ball, a silver mirror, or a font filled with holy water)
% Duration: Concentration, up to 10 minutes
% 
% You can see and hear a particular creature you choose that is on the same plane of existence as you. The target must make a Wisdom saving throw, which is modified by how well you know the target and the sort of physical connection you have to it. If a target knows you're casting this spell, it can fail the saving throw voluntarily if it wants to be observed.
% 
% On a successful save, the target isn't affected, and you can't use this spell against it again for 24 hours.
% 
% On a failed save, the spell creates an invisible sensor within 10 feet of the target. You can see and hear through the sensor as if you were there. The sensor moves with the target, remaining within 10 feet of it for the duration. A creature that can see invisible objects sees the sensor as a luminous orb about the size of your fist.
% 
% Instead of targeting a creature, you can choose a location you have seen before as the target of this spell. When you do, the sensor appears at that location and doesn't move.
% Transmute Rock
% 5th-level transmutation
% Casting Time: 1 action
% Range: 120 feet
% Components: V, S, M
% Duration: Instantaneous
% 
% You choose an area of stone or mud that you can see that fits within a 40-foot cube and is within range, and choose one of the following effects.
% 
% Transmute Rock to Mud. Nonmagical rock of any sort in the area becomes an equal volume of thick, flowing mud that remains for the spell's duration.
% 
% The ground in the spell's area becomes muddy enough that creatures can sink into it. Each foot that a creature moves through the mud costs 4 feet of movement, and any creature on the ground when you cast the spell must make a Strength saving throw. A creature must also make the saving throw when it moves into the area for the first time on a turn or ends its turn there. On a failed save, a creature sinks into the mud and is restrained, though it can use an action to end the restrained condition on itself by pulling itself free of the mud.
% 
% If you cast the spell on a ceiling, the mud falls. Any creature under the mud when it falls must make a Dexterity saving throw. A creature takes 4d8 bludgeoning damage on a failed save, or half as much damage on a successful one.
% 
% Transmute Mud to Rock. Nonmagical mud or quicksand in the area no more than 10 feet deep transforms into soft stone for the spell's duration. Any creature in the mud when it transforms must make a Dexterity saving throw. On a successful save, a creature is shunted safely to the surface in an unoccupied space. On a failed save, a creature becomes restrained by the rock. A restrained creature, or another creature within reach, can use an action to try to break the rock by succeeding on a DC 20 Strength check or by dealing damage to it. The rock has AC 15 and 25 hit points, and it is immune to poison and psychic damage.
% Tree Stride
% 5th-level conjuration
% Casting Time: 1 action
% Range: Self
% Components: V, S
% Duration: Concentration, up to 1 minute
% 
% You gain the ability to enter a tree and move from inside it to inside another tree of the same kind within 500 feet. Both trees must be living and at least the same size as you. You must use 5 feet of movement to enter a tree. You instantly know the location of all other trees of the same kind within 500 feet and, as part of the move used to enter the tree, can either pass into one of those trees or step out of the tree you're in. You appear in a spot of your choice within 5 feet of the destination tree, using another 5 feet of movement. If you have no movement left, you appear within 5 feet of the tree you entered.
% 
% You can use this transportation ability once per round for the duration. You must end each turn outside a tree.
% Wall of Stone
% 5th-level evocation
% Casting Time: 1 action
% Range: 120 feet
% Components: V, S, M
% Duration: Concentration, up to 10 minutes
% 
% A nonmagical wall of solid stone springs into existence at a point you choose within range. The wall is 6 inches thick and is composed of ten 10-foot- by-10-foot panels. Each panel must be contiguous with at least one other panel. Alternatively, you can create 10-foot-by-20-foot panels that are only 3 inches thick.
% 
% If the wall cuts through a creature's space when it appears, the creature is pushed to one side of the wall (your choice). If a creature would be surrounded on all sides by the wall (or the wall and another solid surface), that creature can make a Dexterity saving throw. On a success, it can use its reaction to move up to its speed so that it is no longer enclosed by the wall.
% 
% The wall can have any shape you desire, though it can't occupy the same space as a creature or object. The wall doesn't need to be vertical or rest on any firm foundation. It must, however, merge with and be solidly supported by existing stone. Thus, you can use this spell to bridge a chasm or create a ramp.
% 
% If you create a span greater than 20 feet in length, you must halve the size of each panel to create supports. You can crudely shape the wall to create crenellations, battlements, and so on.
% 
% The wall is an object made of stone that can be damaged and thus breached. Each panel has AC 15 and 30 hit points per inch of thickness. Reducing a panel to 0 hit points destroys it and might cause connected panels to collapse at the GM's discretion.
% 
% If you maintain your concentration on this spell for its whole duration, the wall becomes permanent and can't be dispelled. Otherwise, the wall disappears when the spell ends.
% Bones of the Earth
% 6th-level transmutation
% Casting Time: 1 action
% Range: 120 feet
% Components: V, S
% Duration: Instantaneous
% 
% You cause up to six pillars of stone to burst from places on the ground that you can see within range. Each pillar is a cylinder that has a diameter of 5 feet and a height of up to 30 feet. The ground where a pillar appears must be wide enough for its diameter, and you can target the ground under a creature if that creature is Medium or smaller. Each pillar has AC 5 and 30 hit points. When reduced to 0 hit points, a pillar crumbles into rubble, which creates an area of difficult terrain with a 10-foot radius that lasts until the rubble is cleared. Each 5-foot-diameter portion of the area requires at least 1 minute to clear by hand.
% 
% If a pillar is created under a creature, that creature must succeed on a Dexterity saving throw or be lifted by the pillar. A creature can choose to fail the save.
% 
% If a pillar is prevented from reaching its full height because of a ceiling or other obstacle, a creature on the pillar takes 6d6 bludgeoning damage and is restrained, pinched between the pillar and the obstacle. The restrained creature can use an action to make a Strength or Dexterity check (the creature's choice) against the spell's save DC. On a success, the creature is no longer restrained and must either move off the pillar or fall off it.
% 
% At higher levels:
% 
% When you cast this spell using a spell slot of 7th level or higher, you can create two additional pillars for each slot level above 6th.
% Conjure Fey
% 6th-level conjuration
% Casting Time: 1 minute
% Range: 90 feet
% Components: V, S
% Duration: Concentration, up to 1 hour
% 
% You summon a fey creature of challenge rating 6 or lower, or a fey spirit that takes the form of a beast of challenge rating 6 or lower. It appears in an unoccupied space that you can see within range. The fey creature disappears when it drops to 0 hit points or when the spell ends.
% 
% The fey creature is friendly to you and your companions for the duration. Roll initiative for the creature, which has its own turns. It obeys any verbal commands that you issue to it (no action required by you), as long as they don't violate its alignment. If you don't issue any commands to the fey creature, it defends itself from hostile creatures but otherwise takes no actions.
% 
% If your concentration is broken, the fey creature doesn't disappear. Instead, you lose control of the fey creature, it becomes hostile toward you and your companions, and it might attack. An uncontrolled fey creature can't be dismissed by you, and it disappears 1 hour after you summoned it.
% 
% The GM has the fey creature's statistics. Some sample creatures are listed below.
% 
% At higher levels:
% 
% When you cast this spell using a spell slot of 7th level or higher, the challenge rating increases by 1 for each slot level above 6th.
% 
% Sample Creatures
% 
% Find the Path
% 6th-level divination
% Casting Time: 1 minute
% Range: Self
% Components: V, S, M (a set of divinatory tools—such as bones, ivory sticks, cards, teeth, or carved runes—worth 100gp and an object from the location you wish to find)
% Duration: Concentration, up to 24 hours
% 
% This spell allows you to find the shortest, most direct physical route to a specific fixed location that you are familiar with on the same plane of existence. If you name a destination on another plane of existence, a destination that moves (such as a mobile fortress), or a destination that isn't specific (such as "a green dragon's lair"), the spell fails.
% 
% For the duration, as long as you are on the same plane of existence as the destination, you know how far it is and in what direction it lies. While you are traveling there, whenever you are presented with a choice of paths along the way, you automatically determine which path is the shortest and most direct route (but not necessarily the safest route) to the destination.
% Heal
% 6th-level evocation
% Casting Time: 1 action
% Range: 60 feet
% Components: V, S
% Duration: Instantaneous
% 
% Choose a creature that you can see within range. A surge of positive energy washes through the creature, causing it to regain 70 hit points. This spell also ends blindness, deafness, and any diseases affecting the target. This spell has no effect on constructs or undead.
% 
% At higher levels:
% 
% When you cast this spell using a spell slot of 7th level or higher, the amount of healing increases by 10 for each slot level above 6th.
% Heroes' Feast
% 6th-level conjuration
% Casting Time: 10 minutes
% Range: 30 feet
% Components: V, S, M (a gem-encrusted bowl worth at least 1,000gp, which the spell consumes)
% Duration: Instantaneous
% 
% You bring forth a great feast, including magnificent food and drink. The feast takes 1 hour to consume and disappears at the end of that time, and the beneficial effects don't set in until this hour is over. Up to twelve other creatures can partake of the feast.
% 
% A creature that partakes of the feast gains several benefits. The creature is cured of all diseases and poison, becomes immune to poison and being frightened, and makes all Wisdom saving throws with advantage. Its hit point maximum also increases by 2d10, and it gains the same number of hit points. These benefits last for 24 hours.
% Investiture of Flame
% 6th-level transmutation
% Casting Time: 1 action
% Range: Self
% Components: V, S
% Duration: Concentration, up to 10 minutes
% 
% Flames race across your body, shedding bright light in a 30-foot radius and dim light for an additional 30 feet for the spell's duration. The flames don't harm you. Until the spell ends, you gain the following benefits:
% 
%     You are immune to fire damage and have resistance to cold damage.
% 
%     Any creature that moves within 5 feet of you for the first time on a turn or ends its turn there takes 1d10 fire damage.
% 
%     You can use your action to create a line of fire 15 feet long and 5 feet wide extending from you in a direction you choose. Each creature in the line must make a Dexterity saving throw. A creature takes 4d8 fire damage on a failed save, or half as much damage on a successful one.
% 
% Investiture of Ice
% 6th-level transmutation
% Casting Time: 1 action
% Range: Self
% Components: V, S
% Duration: Concentration, up to 10 minutes
% 
% Until the spell ends, ice rimes your body, and you gain the following benefits:
% 
%     You are immune to cold damage and have resistance to fire damage.
% 
%     You can move across difficult terrain created by ice or snow without spending extra movement.
% 
%     The ground in a 10-foot radius around you is icy and is difficult terrain for creatures other than you. The radius moves with you.
% 
%     You can use your action to create a 15-foot cone of freezing wind extending from your outstretched hand in a direction you choose. Each creature in the cone must make a Constitution saving throw. A creature takes 4d6 cold damage on a failed save, or half as much damage on a successful one. A creature that fails its save against this effect has its speed halved until the start of your next turn.
% 
% Investiture of Stone
% 6th-level transmutation
% Casting Time: 1 action
% Range: Self
% Components: V, S
% Duration: Concentration, up to 10 minutes
% 
% Until the spell ends, bits of rock spread across your body, and you gain the following benefits:
% 
%     You have resistance to bludgeoning, piercing, and slashing damage from nonmagical attacks.
% 
%     You can use your action to create a small earthquake on the ground in a 15-foot radius centered on you. Other creatures on that ground must succeed on a Dexterity saving throw or be knocked prone.
% 
%     You can move across difficult terrain made of earth or stone without spending extra movement. You can move through solid earth or stone as if it was air and without destabilizing it, but you can't end your movement there. If you do so, you are ejected to the nearest unoccupied space, this spell ends, and you are stunned until the end of your next turn.
% 
% Investiture of Wind
% 6th-level transmutation
% Casting Time: 1 action
% Range: Self
% Components: V, S
% Duration: Concentration, up to 10 minutes
% 
% Until the spell ends, wind whirls around you, and you gain the following benefits:
% 
%     Ranged weapon attacks made against you have disadvantage on the attack roll.
% 
%     You gain a flying speed of 60 feet. If you are still flying when the spell ends, you fall, unless you can somehow prevent it.
% 
%     You can use your action to create a 15-foot cube of swirling wind centered on a point you can see within 60 feet of you. Each creature in that area must make a Constitution saving throw. A creature takes 2d10 bludgeoning damage on a failed save, or half as much damage on a successful one. If a Large or smaller creature fails the save, that creature is also pushed up to 10 feet away from the center of the cube.
% 
% Move Earth
% 6th-level transmutation
% Casting Time: 1 action
% Range: 120 feet
% Components: V, S, M
% Duration: Concentration, up to 2 hours
% 
% Choose an area of terrain no larger than 40 feet on a side within range. You can reshape dirt, sand, or clay in the area in any manner you choose for the duration. You can raise or lower the area's elevation, create or fill in a trench, erect or flatten a wall, or form a pillar. The extent of any such changes can't exceed half the area's largest dimension. So, if you affect a 40-foot square, you can create a pillar up to 20 feet high, raise or lower the square's elevation by up to 20 feet, dig a trench up to 20 feet deep, and so on. It takes 10 minutes for these changes to complete.
% 
% At the end of every 10 minutes you spend concentrating on the spell, you can choose a new area of terrain to affect.
% 
% Because the terrain's transformation occurs slowly, creatures in the area can't usually be trapped or injured by the ground's movement.
% 
% This spell can't manipulate natural stone or stone construction. Rocks and structures shift to accommodate the new terrain. If the way you shape the terrain would make a structure unstable, it might collapse.
% 
% Similarly, this spell doesn't directly affect plant growth. The moved earth carries any plants along with it.
% Primordial Ward
% 6th-level abjuration
% Casting Time: 1 action
% Range: Self
% Components: V, S
% Duration: Concentration, up to 1 minute
% 
% You have resistance to acid, cold, fire, lightning, and thunder damage for the spell's duration.
% 
% When you take damage of one of those types, you can use your reaction to gain immunity to that type of damage, including against the triggering damage.
% 
% If you do so, the resistances end, and you have the immunity until the end of your next turn, at which time the spell ends.
% Sunbeam
% 6th-level evocation
% Casting Time: 1 action
% Range: Self
% Components: V, S, M
% Duration: Concentration, up to 1 minute
% 
% A beam of brilliant light flashes out from your hand in a 5-foot-wide, 60-foot-long line. Each creature in the line must make a Constitution saving throw. On a failed save, a creature takes 6d8 radiant damage and is blinded until your next turn. On a successful save, it takes half as much damage and isn't blinded by this spell. Undead and oozes have disadvantage on this saving throw.
% 
% You can create a new line of radiance as your action on any turn until the spell ends.
% 
% For the duration, a mote of brilliant radiance shines in your hand. It sheds bright light in a 30-foot radius and dim light for an additional 30 feet. This light is sunlight.
% Transport via Plants
% 6th-level conjuration
% Casting Time: 1 action
% Range: 10 feet
% Components: V, S
% Duration: 1 round
% 
% This spell creates a magical link between a Large or larger inanimate plant within range and another plant, at any distance, on the same plane of existence. You must have seen or touched the destination plant at least once before. For the duration, any creature can step into the target plant and exit from the destination plant by using 5 feet of movement.
% Wall of Thorns
% 6th-level conjuration
% Casting Time: 1 action
% Range: 120 feet
% Components: V, S, M
% Duration: Concentration, up to 10 minutes
% 
% You create a wall of tough, pliable, tangled brush bristling with needle-sharp thorns. The wall appears within range on a solid surface and lasts for the duration. You choose to make the wall up to 60 feet long, 10 feet high, and 5 feet thick or a circle that has a 20-foot diameter and is up to 20 feet high and 5 feet thick. The wall blocks line of sight.
% 
% When the wall appears, each creature within its area must make a Dexterity saving throw. On a failed save, a creature takes 7d8 piercing damage, or half as much damage on a successful save.
% 
% A creature can move through the wall, albeit slowly and painfully. For every 1 foot a creature moves through the wall, it must spend 4 feet of movement. Furthermore, the first time a creature enters the wall on a turn or ends its turn there, the creature must make a Dexterity saving throw. It takes 7d8 slashing damage on a failed save, or half as much damage on a successful one.
% 
% At higher levels:
% 
% When you cast this spell using a spell slot of 7th level or higher, both types of damage increase by 1d8 for each slot level above 6th.
% Wind Walk
% 6th-level transmutation
% Casting Time: 1 minute
% Range: 30 feet
% Components: V, S, M
% Duration: 8 hours
% 
% You and up to ten willing creatures you can see within range assume a gaseous form for the duration, appearing as wisps of cloud. While in this cloud form, a creature has a flying speed of 300 feet and has resistance to damage from nonmagical weapons. The only actions a creature can take in this form are the Dash action or to revert to its normal form. Reverting takes 1 minute, during which time a creature is incapacitated and can't move. Until the spell ends, a creature can revert to cloud form, which also requires the 1-minute transformation.
% 
% If a creature is in cloud form and flying when the effect ends, the creature descends 60 feet per round for 1 minute until it lands, which it does safely. If it can't land after 1 minute, the creature falls the remaining distance.
% Fire Storm
% 7th-level evocation
% Casting Time: 1 action
% Range: 150 feet
% Components: V, S
% Duration: Instantaneous
% 
% A storm made up of sheets of roaring flame appears in a location you choose within range. The area of the storm consists of up to ten 10-foot cubes, which you can arrange as you wish. Each cube must have at least one face adjacent to the face of another cube. Each creature in the area must make a Dexterity saving throw. It takes 7d10 fire damage on a failed save, or half as much damage on a successful one.
% 
% The fire damages objects in the area and ignites flammable objects that aren't being worn or carried. If you choose, plant life in the area is unaffected by this spell.
% Mirage Arcane
% 7th-level illusion
% Casting Time: 10 minutes
% Range: Sight
% Components: V, S
% Duration: 10 days
% 
% You make terrain in an area up to 1 mile square look, sound, smell, and even feel like some other sort of terrain. The terrain's general shape remains the same, however. Open fields or a road could be made to resemble a swamp, hill, crevasse, or some other difficult or impassable terrain. A pond can be made to seem like a grassy meadow, a precipice like a gentle slope, or a rock-strewn gully like a wide and smooth road.
% 
% Similarly, you can alter the appearance of structures, or add them where none are present. The spell doesn't disguise, conceal, or add creatures.
% 
% The illusion includes audible, visual, tactile, and olfactory elements, so it can turn clear ground into difficult terrain (or vice versa) or otherwise impede movement through the area. Any piece of the illusory terrain (such as a rock or stick) that is removed from the spell's area disappears immediately.
% 
% Creatures with truesight can see through the illusion to the terrain's true form; however, all other elements of the illusion remain, so while the creature is aware of the illusion's presence, the creature can still physically interact with the illusion.
% Regenerate
% 7th-level transmutation
% Casting Time: 1 minute
% Range: Touch
% Components: V, S, M
% Duration: 1 hour
% 
% You touch a creature and stimulate its natural healing ability. The target regains 4d8 + 15 hit points. For the duration of the spell, the target regains 1 hit point at the start of each of its turns (10 hit points each minute).
% 
% The target's severed body members (fingers, legs, tails, and so on), if any, are restored after 2 minutes. If you have the severed part and hold it to the stump, the spell instantaneously causes the limb to knit to the stump.
% Reverse Gravity
% 7th-level transmutation
% Casting Time: 1 action
% Range: 100 feet
% Components: V, S, M
% Duration: Concentration, up to 1 minute
% 
% This spell reverses gravity in a 50-foot-radius, 100- foot high cylinder centered on a point within range. All creatures and objects that aren't somehow anchored to the ground in the area fall upward and reach the top of the area when you cast this spell. A creature can make a Dexterity saving throw to grab onto a fixed object it can reach, thus avoiding the fall.
% 
% If some solid object (such as a ceiling) is encountered in this fall, falling objects and creatures strike it just as they would during a normal downward fall. If an object or creature reaches the top of the area without striking anything, it remains there, oscillating slightly, for the duration.
% 
% At the end of the duration, affected objects and creatures fall back down.
% Whirlwind
% 7th-level evocation
% Casting Time: 1 action
% Range: 300 feet
% Components: V, M
% Duration: Concentration, up to 1 minute
% 
% A whirlwind howls down to a point that you can see on the ground within range. The whirlwind is a 10-foot-radius, 30-foot-high cylinder centered on that point. Until the spell ends, you can use your action to move the whirlwind up to 30 feet in any direction along the ground. The whirlwind sucks up any Medium or smaller objects that aren't secured to anything and that aren't worn or carried by anyone.
% 
% A creature must make a Dexterity saving throw the first time on a turn that it enters the whirlwind or that the whirlwind enters its space, including when the whirlwind first appears. A creature takes 10d6 bludgeoning damage on a failed save, or half as much damage on a successful one. In addition, a Large or smaller creature that fails the save must succeed on a Strength saving throw or become restrained in the whirlwind until the spell ends. When a creature starts its turn restrained by the whirlwind, the creature is pulled 5 feet higher inside it, unless the creature is at the top. A restrained creature moves with the whirlwind and falls when the spell ends, unless the creature has some means to stay aloft.
% 
% A restrained creature can use an action to make a Strength or Dexterity check against your spell save DC. If successful, the creature is no longer restrained by the whirlwind and is hurled 3d6 x 10 feet away from it in a random direction.
% Animal Shapes
% 8th-level transmutation
% Casting Time: 1 action
% Range: 30 feet
% Components: V, S
% Duration: Concentration, up to 24 hours
% 
% Your magic turns others into beasts. Choose any number of willing creatures that you can see within range. You transform each target into the form of a Large or smaller beast with a challenge rating of 4 or lower. On subsequent turns, you can use your action to transform affected creatures into new forms.
% 
% The transformation lasts for the duration for each target, or until the target drops to 0 hit points or dies. You can choose a different form for each target. A target's game statistics are replaced by the statistics of the chosen beast, though the target retains its alignment and Intelligence, Wisdom, and Charisma scores. The target assumes the hit points of its new form, and when it reverts to its normal form, it returns to the number of hit points it had before it transformed. If it reverts as a result of dropping to 0 hit points, any excess damage carries over to its normal form. As long as the excess damage doesn't reduce the creature's normal form to 0 hit points, it isn't knocked unconscious. The creature is limited in the actions it can perform by the nature of its new form, and it can't speak or cast spells.
% 
% The target's gear melds into the new form. The target can't activate, wield, or otherwise benefit from any of its equipment.
% Antipathy/Sympathy
% 8th-level enchantment
% Casting Time: 1 hour
% Range: 60 feet
% Components: V, S, M
% Duration: 10 days
% 
% This spell attracts or repels creatures of your choice. You target something within range, either a Huge or smaller object or creature or an area that is no larger than a 200-foot cube. Then specify a kind of intelligent creature, such as red dragons, goblins, or vampires. You invest the target with an aura that either attracts or repels the specified creatures for the duration. Choose antipathy or sympathy as the aura's effect.
% 
% Antipathy. The enchantment causes creatures of the kind you designated to feel an intense urge to leave the area and avoid the target. When such a creature can see the target or comes within 60 feet of it, the creature must succeed on a Wisdom saving throw or become frightened. The creature remains frightened while it can see the target or is within 60 feet of it. While frightened by the target, the creature must use its movement to move to the nearest safe spot from which it can't see the target. If the creature moves more than 60 feet from the target and can't see it, the creature is no longer frightened, but the creature becomes frightened again if it regains sight of the target or moves within 60 feet of it.
% 
% Sympathy. The enchantment causes the specified creatures to feel an intense urge to approach the target while within 60 feet of it or able to see it. When such a creature can see the target or comes within 60 feet of it, the creature must succeed on a Wisdom saving throw or use its movement on each of its turns to enter the area or move within reach of the target. When the creature has done so, it can't willingly move away from the target.
% 
% If the target damages or otherwise harms an affected creature, the affected creature can make a Wisdom saving throw to end the effect, as described below.
% 
% Ending the Effect. If an affected creature ends its turn while not within 60 feet of the target or able to see it, the creature makes a Wisdom saving throw. On a successful save, the creature is no longer affected by the target and recognizes the feeling of repugnance or attraction as magical. In addition, a creature affected by the spell is allowed another Wisdom saving throw every 24 hours while the spell persists.
% 
% A creature that successfully saves against this effect is immune to it for 1 minute, after which time it can be affected again.
% Control Weather
% 8th-level transmutation
% Casting Time: 10 minutes
% Range: Self
% Components: V, S, M
% Duration: Concentration, up to 8 hours
% 
% You take control of the weather within 5 miles of you for the duration. You must be outdoors to cast this spell. Moving to a place where you don't have a clear path to the sky ends the spell early.
% 
% When you cast the spell, you change the current weather conditions, which are determined by the DM based on the climate and season. You can change precipitation, temperature, and wind. It takes 1d4 x 10 minutes for the new conditions to take effect. Once they do so, you can change the conditions again. When the spell ends, the weather gradually returns to normal.
% 
% When you change the weather conditions, find a current condition on the following tables and change its stage by one, up or down. When changing the wind, you can change its direction.
% 
% Precipitation
% Earthquake
% 8th-level evocation
% Casting Time: 1 action
% Range: 500 feet
% Components: V, S, M
% Duration: Concentration, up to 1 minute
% 
% You create a seismic disturbance at a point on the ground that you can see within range. For the duration, an intense tremor rips through the ground in a 100-foot-radius circle centered on that point and shakes creatures and structures in contact with the ground in that area.
% 
% The ground in the area becomes difficult terrain. Each creature on the ground that is concentrating must make a Constitution saving throw. On a failed save, the creature's concentration is broken.
% 
% When you cast this spell and at the end of each turn you spend concentrating on it, each creature on the ground in the area must make a Dexterity saving throw. On a failed save, the creature is knocked prone.
% 
% This spell can have additional effects depending on the terrain in the area, as determined by the GM.
% 
% Fissures. Fissures open throughout the spell's area at the start of your next turn after you cast the spell. A total of 1d6 such fissures open in locations chosen by the GM. Each is 1d10 x 10 feet deep, 10 feet wide, and extends from one edge of the spell's area to the opposite side. A creature standing on a spot where a fissure opens must succeed on a Dexterity saving throw or fall in. A creature that successfully saves moves with the fissure's edge as it opens.
% 
% A fissure that opens beneath a structure causes it to automatically collapse (see below).
% 
% Structures. The tremor deals 50 bludgeoning damage to any structure in contact with the ground in the area when you cast the spell and at the start of each of your turns until the spell ends. If a structure drops to 0 hit points, it collapses and potentially damages nearby creatures. A creature within half the distance of a structure's height must make a Dexterity saving throw. On a failed save, the creature takes 5d6 bludgeoning damage, is knocked prone, and is buried in the rubble, requiring a DC 20 Strength (Athletics) check as an action to escape. The GM can adjust the DC higher or lower, depending on the nature of the rubble. On a successful save, the creature takes half as much damage and doesn't fall prone or become buried.
% Feeblemind
% 8th-level enchantment
% Casting Time: 1 action
% Range: 150 feet
% Components: V, S, M
% Duration: Instantaneous
% 
% You blast the mind of a creature that you can see within range, attempting to shatter its intellect and personality. The target takes 4d6 psychic damage and must make an Intelligence saving throw.
% 
% On a failed save, the creature's Intelligence and Charisma scores become 1. The creature can't cast spells, activate magic items, understand language, or communicate in any intelligible way. The creature can, however, identify its friends, follow them, and even protect them.
% 
% At the end of every 30 days, the creature can repeat its saving throw against this spell. If it succeeds on its saving throw, the spell ends.
% 
% The spell can also be ended by greater restoration, heal, or wish.
% Sunburst
% 8th-level evocation
% Casting Time: 1 action
% Range: 150 feet
% Components: V, S, M
% Duration: Instantaneous
% 
% Brilliant sunlight flashes in a 60-foot radius centered on a point you choose within range. Each creature in that light must make a Constitution saving throw. On a failed save, a creature takes 12d6 radiant damage and is blinded for 1 minute. On a successful save, it takes half as much damage and isn't blinded by this spell. Undead and oozes have disadvantage on this saving throw.
% 
% A creature blinded by this spell makes another Constitution saving throw at the end of each of its turns. On a successful save, it is no longer blinded.
% 
% This spell dispels any darkness in its area that was created by a spell.
% Tsunami
% 8th-level conjuration
% Casting Time: 1 minute
% Range: Sight
% Components: V, S
% Duration: Concentration, up to 6 rounds
% 
% A wall of water springs into existence at a point you choose within range. You can make the wall up to 300 feet long, 300 feet high, and 50 feet thick. The wall lasts for the duration.
% 
% When the wall appears, each creature within its area must make a strength saving throw. On a failed save, a creature takes 6 d10 bludgeoning damage, or half as much damage on a successful save.
% 
% At the start of each of your turns after the wall appears, the wall, along with any creatures in it, moves 50 feet away from you. Any Huge or smaller creature inside the wall or whose space the wall enters when it moves must succeed on a strength saving throw or take 5 d10 bludgeoning damage. A creature can take this damage only once per round. At the end of the turn, the wall’s height is reduced by 50 feet, and the damage creatures take from the spell on subsequent rounds is reduced by 1d10. When the wall reaches 0 feet in height, the spell ends.
% 
% A creature caught in the wall can move by swimming. Because of the force of the wave, though, the creature must make a successful Strength (Athletics) check against your spell save DC in order to move at all. If it fails the check, it can’t move. A creature that moves out of the area falls to the ground.
% Foresight
% 9th-level divination
% Casting Time: 1 minute
% Range: Touch
% Components: V, S, M
% Duration: 8 hours
% 
% You touch a willing creature and bestow a limited ability to see into the immediate future. For the duration, the target can't be surprised and has advantage on attack rolls, ability checks, and saving throws. Additionally, other creatures have disadvantage on attack rolls against the target for the duration.
% 
% This spell immediately ends if you cast it again before its duration ends.
% Shapechange
% 9th-level transmutation
% Casting Time: 1 action
% Range: Self
% Components: V, S, M (a jade circlet worth at least 1,500 gp, which you must place on your head before you cast the spell)
% Duration: Concentration, up to 1 hour
% 
% You assume the form of a different creature for the duration. The new form can be of any creature with a challenge rating equal to your level or lower. The creature can't be a construct or an undead, and you must have seen the sort of creature at least once. You transform into an average example of that creature, one without any class levels or the Spellcasting trait.
% 
% Your game statistics are replaced by the statistics of the chosen creature, though you retain your alignment and Intelligence, Wisdom, and Charisma scores. You also retain all of your skill and saving throw proficiencies, in addition to gaining those of the creature. If the creature has the same proficiency as you and the bonus listed in its statistics is higher than yours, use the creature's bonus in place of yours. You can't use any legendary actions or lair actions of the new form.
% 
% You assume the hit points and Hit Dice of the new form. When you revert to your normal form, you return to the number of hit points you had before you transformed. If you revert as a result of dropping to 0 hit points, any excess damage carries over to your normal form. As long as the excess damage doesn't reduce your normal form to 0 hit points, you aren't knocked unconscious.
% 
% You retain the benefit of any features from your class, race, or other source and can use them, provided that your new form is physically capable of doing so. You can't use any special senses you have (for example, darkvision) unless your new form also has that sense. You can only speak if the creature can normally speak.
% 
% When you transform, you choose whether your equipment falls to the ground, merges into the new form, or is worn by it. Worn equipment functions as normal. The GM determines whether it is practical for the new form to wear a piece of equipment, based on the creature's shape and size. Your equipment doesn't change shape or size to match the new form, and any equipment that the new form can't wear must either fall to the ground or merge into your new form. Equipment that merges has no effect in that state.
% 
% During this spell's duration, you can use your action to assume a different form following the same restrictions and rules for the original form, with one exception: if your new form has more hit points than your current one, your hit points remain at their current value.
% Storm of Vengeance
% 9th-level conjuration
% Casting Time: 1 action
% Range: Sight
% Components: V, S
% Duration: Concentration, up to 1 minute
% 
% A churning storm cloud forms, centered on a point you can see and spreading to a radius of 360 feet. Lightning flashes in the area, thunder booms, and strong winds roar. Each creature under the cloud (no more than 5,000 feet beneath the cloud) when it appears must make a Constitution saving throw. On a failed save, a creature takes 2d6 thunder damage and becomes deafened for 5 minutes.
% 
% Each round you maintain concentration on this spell, the storm produces additional effects on your turn.
% 
% Round 2. Acidic rain falls from the cloud. Each creature and object under the cloud takes 1d6 acid damage.
% 
% Round 3. You call six bolts of lightning from the cloud to strike six creatures or objects of your choice beneath the cloud. A given creature or object can't be struck by more than one bolt. A struck creature must make a Dexterity saving throw. The creature takes 10d6 lightning damage on a failed save, or half as much damage on a successful one.
% 
% Round 4. Hailstones rain down from the cloud. Each creature under the cloud takes 2d6 bludgeoning damage.
% 
% Round 5-10. Gusts and freezing rain assail the area under the cloud. The area becomes difficult terrain and is heavily obscured. Each creature there takes 1d6 cold damage. Ranged weapon attacks in the area are impossible. The wind and rain count as a severe distraction for the purposes of maintaining concentration on spells. Finally, gusts of strong wind (ranging from 20 to 50 miles per hour) automatically disperse fog, mists, and similar phenomena in the area, whether mundane or magical.
% True Resurrection
% 9th-level necromancy
% Casting Time: 1 hour
% Range: Touch
% Components: V, S, M (a sprinkle of holy water and diamonds worth at least 25,000gp, which the spell consumes)
% Duration: Instantaneous
% 
% You touch a creature that has been dead for no longer than 200 years and that died for any reason except old age. If the creature's soul is free and willing, the creature is restored to life with all its hit points.
% 
% This spell closes all wounds, neutralizes any poison, cures all diseases, and lifts any curses affecting the creature when it died. The spell replaces damaged or missing organs and limbs. If the creature was undead, it is restored to its non-undead form.
% 
% The spell can even provide a new body if the original no longer exists, in which case you must speak the creature's name. The creature then appears in an unoccupied space you choose within 10 feet of you.


% 
% \chapter{Chapter name}
% 
% % End document
\end{document}
