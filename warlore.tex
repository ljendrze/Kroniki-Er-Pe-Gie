\documentclass[10pt,twoside,twocolumn]{book}
\usepackage[bg-letter]{lib/rpg-book} % Options: bg-a4, bg-letter, bg-full, bg-print, bg-none.
\usepackage[polish]{babel}
\usepackage[utf8]{inputenc}
\usepackage{hyperref}
\usepackage{multicol}
\usepackage{multirow}
\usepackage{lipsum}
\usepackage{tabularx}

\title{Warlore}
\date{\today}
\author{The Dungeon Mistress}
\author{Khedre}
\author{Ampoldórëon Sailaquettë}
\author{Lander Stormwind}
\author{Randal Buckman}

% Start document
\begin{document}
\fontfamily{ppl}\selectfont % Set text font
\frontmatter

\maketitle
\begin{multicols}{2}
\tableofcontents
\end{multicols}

% Your content goes here
\mainmatter

\chapter{Ildrim - kontynent imperium}

\paragraph{}
Ciężko opisać kontynent jako całość, gdy nawet najbardziej odważni i potężni podróżnicy rzadko opuszczają jego wybrzeża.
Gdy nawet potężni mistrzowie magii z trudem sięgają umysłem poza jego granice. 
Wyobraźmy sobie masę lądu otoczoną ze wszystkich stron oceanem.
I to nie takim zwykłym, oceanem kierowanym starożytną magią z Wysokich Planów.
Oceanem chcącym pożreć, pochłonąć nieszczęsny ląd.
Taki jest nasz Ildrim, zwany też Kontynentem Imperium.
Jeśli nic się nie zmieni, bądź zmieni się zbyt wiele, zostaniemy wszyscy pogrążeni pod wodą, skała i magią.
Ponieważ Ildrimem władają moce poza naszą kontrolą.

\paragraph{}
Śmiercionośny ocean wpływa na naszą egzystencję w jeszcze jeden sposób: niemal kompletnie odcina nas od innych cywilizacji.
O ile pas wody w pobliżu wybrzeży w większości jest stosunkowo spokojny, głębokie wody są niemal zupełnie niedostępne.
Jedynie druidzi z \hyperref[Vicovaro]{Vicovaro}, elfowie z Vorath i „wolni żeglarze” z Wysp Kye są w stanie próbować swoich sił z żywiołem.

\paragraph{}
Kontynent zatem jest całym światem dla praktycznie wszystkich jego mieszkańców, zapewne z  wyjątkiem najprostszych rolników, których pojęcie obejmuje co najwyżej sąsiednie miasto.
I nie powinno dziwić, że przez milenia Ildrim zaznał niewiele spokoju.

\paragraph{}
Większość kontynentu zajmują \hyperref[SojuszniczeKrolestwa]{Sojusznicze Królestwa}: od górskich twierdz \hyperref[AnGammarna]{An’Gammarny} do spustoszonych iglic Yunait.
To te siedem państw jest odpowiedzialnych za ostatnie wojny i niepokoje na kontynencie.
Jednak po stuleciach walk \hyperref[SojuszniczeKrolestwa]{Sojusznicze Królestwa} utrzymywane są w pokoju przez \hyperref[GildiaKupcow]{Gildię Kupców}.
Z punktu widzenia \hyperref[GildiaKupcow]{Gildii} pokój zostanie zachowany.
Niezależnie do ceny.
Państwa te pozostają, przynajmniej we własnym mniemaniu, ostoją cywilizacji i dziedzicami Imperium.
Poza elfim państwem-miastem Vorath są jedyną cywilizacją na powierzchni Planu Materialnego nie pozostająca w ukryciu.

\paragraph{}
Prawda, jak to ma w zwyczaju, jest nieco bardziej skomplikowana.
Jeśli ktoś zdoła przebyć zdradzieckie północne góry \hyperref[AnGammarna]{An’Gammarny} i \hyperref[Vicovaro]{Vicovaro} dotrze do mroźnych pustkowi.
Tam na pokrytych wiecznym lodowcem wybrzeżach znajduje się owiane legendami państwo.
Niewielu wie o nim cokolwiek pewnego, chociaż większość dzieci w \hyperref[SojuszniczeKrolestwa]{Sojuszniczych Królestwach} słyszało przed snem baśnie o magicznych miastach pośród lodu.
Jeszcze dalej na zachód znajduje się sporych rozmiarów wyspa, ukryta pośród mgieł.
Ta wyspa to Vorath, ostatnie miasto elfów na Planie Materialnym, a w każdym razie ostatnie na tym kontynencie.

\paragraph{}
Od południowo-zachodniej strony Ildrimu znajduje się Platynowa Zatoka.
Stanowi ona wybrzeże \hyperref[AnGammarna]{An’Gammarny}, Croridu oraz Eneadoru.
Wody zatoki, są wyjątkowo zdradliwe, nawet jak na warunki kontynentu.
Powodem jest Latająca Wyspa Thor Ar Muria, gdzie znajduje się największa na kontynencie akademia magii arkanicznej.
Na obszarze Platynowej Zatoki znajdują się żeglowne przesmyki, dostępne dla doświadczonych śmiałków, jednak większość wypełniona jest wirami i wiatrami o nieprzewidywalnym, magicznym źródle.

\paragraph{}
Na wschodzie znajduje się niezbadana i niebezpieczna kraina zwana Pustkowiem Kataklizmu.
Nieliczne jego fragmenty są zamieszkane przez pionierów z Domen Krasnoludów.
Choć niewiele osób o tym wie, w północno-zachodniej części Pustkowia znajduje się Przystań: miasto uciekinierów z Podmroku.
Dawni mieszkańcy tego miasta przemierzają także te nieprzyjazne tereny jako nomadzi.
Niektórzy twierdzą, że wschodnie wybrzeża Ildrimu otaczają spokoje wody, którymi można przepłynąć do innych kontynentów. Nikomu jednak nie udało się do nich dotrzeć.

\paragraph{}
Nie są to jednak jedynie tereny na lądzie dotknięte niszczącą magią.
Na wschód od \hyperref[Vicovaro]{Vicovaro} i w \hyperref[AnGammarna]{An’Gammarnie} znajdują się ruiny dawnego Imperium.
Jakiekolwiek magiczne eksperymenty przeprowadzali ich mieszkańcy, ich moc przesiąknęła kamienie, ziemię i drzewa.
W samej \hyperref[AnGammarna]{An’Gammarnie} klerycy i paladyni zdołali opanować nienaturalny żywioł, ale Pomiędzy \hyperref[AnGammarna]{An’Gammarną}, a \hyperref[Vicovaro]{Vicovaro} pozostaje puszcza, kompletnie niestabilna i nienadająca się do zamieszkania zwana Doliną Konwalii.

\paragraph{}
Jest wiele teorii skąd wzięła się aż tak duża magiczna niestabilność występująca na kontynencie, a także czemu jest jedno miejsce praktycznie kompletnie od nich wolne: leżący na zachodzie archipelag Wysp Kye.
Wyspy stanowią przytań dla wielu, którzy przystani poszukują, ale niekoniecznie na nią zasługują.
Zamieszkujący je „wolni żeglarze” prowadzą nieskończone spory z Gildią Kupców chcących dostępu do spokojnych wód i dóbr archipelagu.
Z różnymi skutkami.
Wśród plotek i legend związanych z wyspami znajdują się labirynty podziemi wypełnionych skarbami i starożytnymi tajemnicami.
Niestety niewielu badaczy miało okazję poznać te wypełnione piratami i najemnikami wyspy.

\paragraph{}
Zanim powstały aktualne państwa większość północnej i zachodniej części kontynentu zajmowało Imperium, powszechnie uznawana za państwo elfów zanim opuściły one Plan Materialny.
Imperium najprawdopodobniej zostało zniszczone przez pierwszy na kontynencie magiczny kataklizm.
Możliwe nawet, że go wywołało, doprowadzając Ildrim do tego czym jest teraz.
Praktycznie nie ma organizacji, która obejmowałaby swoim zasięgiem cały kontynent.
Pewne powiązania ma tak zwana \hyperref[KonfrateriaMagow]{Konfrateria Magów}.
Nie jest to formalna organizacja, a raczej ogólne określenie na wszystkich formalnie wyszkolonych magów arkanicznych.
W ramach \hyperref[KonfrateriaMagow]{Konfraterii} funkcjonują Strażnicy Arkanum, stosunkowo niewielka grupa rozproszona po całym Ildrimie.
Tutaj można już mówić o zorganizowaniu, ale dotyczy to stosunkowo niewielkiej, dobrze ukrytej grupy magów.
Większość członków \hyperref[KonfrateriaMagow]{Konfraterii} nie ma pojęcia o ich istnieniu.
Na podobnych zasadach można traktować Kręgi Druidów -  o ile poszczególne Kręgi są rozproszone po całym kontynencie, to nie stanowią jednolitej organizacji.

\label{SojuszniczeKrolestwa}
\chapter{Sojusznicze Królestwa}

\paragraph{}
Można by napisać książkę, a nawet tuzin książek o różnych teoriach co do powstania państw, które stworzyły Sojusznicze Królestwa.
Jedyne co można z pewnością napisać to, że Imperium upadło i na kontynencie powstał chaos.
Na północy elfy zaczęły walczyć z ludźmi i ich sojusznikami, co zakończyło się zwycięstwem ludzi i opuszczeniem Planu Materialnego przez elfy.
Państwa na południu nagle straciły groźnego przeciwnika.

\paragraph{}
Ci mieszkańcy kontynentu którzy nie odsunęli się w trudnodostępne regiony i plany utworzyli siedem państw.
Od północy na południe są to: \hyperref[Vicovaro]{Vicovaro}, \hyperref[AnGammarna]{An’Gammarna}, Crorid, Sothedora, Eneador, Chacan i Yunait.
Eneador, Chacan i Yunait nigdy nie były częścią Imperium, chociaż Eneador był z nim w sojuszu od początku swojego istnienia, by uniknąć podboju (formalnie był ówcześnie częścią Domen Krasnoludów).
Pozostałe państwa powstały na ruinach po początkowym chaosie, w szczególności \hyperref[AnGammarna]{An’Gammarna} i Crorid.

\paragraph{}
O ile obecnie królestwa są mieszaniną różnych ras (z wyjątkiem Yunait, które pozostaje niemal całkowicie zamieszkane przez drakonów) pierwotnie były dużo bardziej podzielone.
\hyperref[Vicovaro]{Vicovaro} było państwem niziołków, \hyperref[AnGammarna]{An'Gammarna} i Sothedora były państwami ludzi z Imperium, Chacan zaś niezalezną cywilizacją ludzi spoza Imperium.
Crorid i Eneador były koloniami krasnoludów i gnomów, przy czym Crorid został wcześniej całkowicie podbity przez Imperium.

\paragraph{}
Mniejsze lub większe potyczki między królestwami rozpoczęły się niemal od razu po upadku Imperium.
Trwały stulecia, ograniczają rozwój i handel.
Żadne z państw nie mogło uzyskać dłuższej przewagi.
Wojny zapewne trwałyby nadal gdyby nie drugi kataklizm.
Nie ma pewności czy był echem zniszczenia Imperium, czy czymś zupełnie nowym.
Pewne jest to, że magiczna eksplozja wstrząsnęła pustkowiami przy wschodniej granicy Yunait, niemal całkowicie je niszcząc. 
Magia rozproszyła się po Ildrimie jak kręgi na wodzie.
W końcu dotarła do imperialnych ruin w Dolinie Konwalii i \hyperref[AnGammarna]{An’Gammarnie}.
Magia zawsze była w nich intensywna, ale to sprawiło, że rozpętała się magiczna burza.
I kompletny chaos.
Kiedy pierwsze skutki się uspokoiły nastąpiły te bardziej długotrwałe.
Cały kontynent objęła katastrofalna susza.
Powoli zdano sobie sprawę, że królestwa nie przetrwają pojedynczo i współpraca jest jedyną szansą na przeżycie.
Tak powstał Sojusz.

\paragraph{}
Wraz z Sojuszem powstała \hyperref[GildiaKupcow]{Gildia Kupców}.
Ta prosta nazwa nie oddaje ogromu organizacji która przez lata oplotła Królestwa siecią kontaktów, szpiegów i intryg.
I która praktycznie rządzi we wszystkich siedmiu państwach.
Załagodzenie kataklizmu i jego efektów zajęło lata, ale przyniosło dodatkowe profity: lepszą komunikację, szlaki handlowe, a co za tym idzie pieniądze.
Nie chcąc tracić zysków \hyperref[GildiaKupcow]{Gildia} skupiła się na swoim nowym najważniejszym celu - utrzymać pokój w Sojuszniczych Królestwach za wszelką cenę.

\paragraph{}
Przez stulecia pokój został utrzymany.
Jednak silne oddziaływanie \hyperref[GildiaKupcow]{Gildii} sprawia, że sytuacja polityczna staje się coraz bardziej niestabilna, a wysiłki \hyperref[GildiaKupcow]{Gildii} by utrzymać status quo coraz bardziej brutalne.
Obecnie \hyperref[Vicovaro]{Vicovaro} jest niemal całkowicie kontrolowane przez \hyperref[GildiaKupcow]{Gildię} (pomijajac atak przez wrogo nastawioną grupę druidów).
Udało im się także się w dużym stopniu zdestabilizować władzę w Croridzie.
Pozostałe państwa zachowuja względnie niezależną władzę, ale to także zaczyna sie powoli zmieniać.

\chapter{An'Gamarrna}

\paragraph{}
Państwo ludzi stworzone na ruinach dawnego Imperium, obecnie rządzone przez teokratyczny kult Bahamuta, do którego musi należeć nawet król.
Pańśtwo jest częścią Sojuszniczych Królestw.

\section{Geografia}
\paragraph{}
An'Gamarrna obejmuje zatokę pomiędzy Doliną Konwalii a Croridem i Eneaorem.
Platynowa zatoka nie umożliwia ze względu na niestabilne prądy przez istnienie latającej wyspy Muria, co powoduje bardzo napięte stosunki między An'Gamarrną a Croridem.
Obecną stolicą państwa jest święte miasto Glenrowan.
Innymi dużymi ośrodkami są dawna stolica Imperium, twierdza Bleihar (znajdująca się w górach na północy) oraz port Thelassa (około dzień drogi od Glenrowan nad brzegiem Platynowej Zatoki).
Klimat An'Gamarrny jest umiarkowany, z okazjolanymi przebłyskami niestabilnej pogody w miejscach gdzie kapłani nie kontrolują dobrze granic między planami.

\section{Polityka}
\paragraph{}
An'Gammarna rządzona jest przez króla, który aby odziedziczyć swoje królestwo musi zostać paladynem Bahamuta.
Obecnym królem jest Lorne III, który wziął za małżonkę Zephę, arcykapłankę Bahamuta.
Zepha jest młodszą od nieo o kilkanaście lat półelfką, dała mu dwóch synów.
Wiele osób uważa, że to ona naprawdę rządzi królestwem.
Wszelkie kulty czczące dobrych i neutralnych bogów cieszą się dużym poważaniem, kulty chaotyczne mogą być natomiast traktowane z pełną rezerwą.

\section{Historia}
\paragraph{}
An'Gammarna wg legend została założona przez zbuntowanych niewolników podczas schyłku Imperium.
Niewolnicy pierwotnie przejęli twierdzę Bleihar, a później objawił im się platynowy smok i polecił stworzyć święte miasto Glenrowan.
Ponieważ teren współczesnej An'Gammarny, poza twierdzą Bleihar, znajdował się ciągle w rękach elfów - wybuchła wojna.
Zakończyła się ona zwycięstwem ludzi i zniszczeniem elfiej stolicy Eleander.

\label{Vicovaro}
\chapter{Vicovaro}

\paragraph{}
Państwo niziołków stworzone w miejscu kręgu druidzkiego specjalizującego się z kontroli pogody oraz magii wody. 
Przez \hyperref[GildiaKupcow]{Vicovaro} przepływa większość handlu zewnętrznego w \hyperref[SojuszniczeKrolestwa]{Królestwach} i jest półoficjalną siedzibą \hyperref[GildiaKupcow]{Gildii Kupców}.
Państwo jest częścią \hyperref[SojuszniczeKrolestwa]{Sojuszniczych Królestw}.

\section{Geografia}
\paragraph{}
Vicovaro jest najbardziej na zachód wysuniętym państwem \hyperref[SojuszniczeKrolestwa]{Sojuszniczych Królestw}.
Od najbliższego wschodniego sąsiada, \hyperref[AnGammarna]{An'Gammarny}, oddzielona jest Doliną Konwalii. 
Poza miastem portowym Vicovaro w państwie nie było żadnych innych miast (co formalnie czyniło z Vicovaro miasto-państwo), do czasu kiedy 200 lat temu \hyperref[KultIoun]{kult Ioun} wybudował Sjorden. 
W Vicovaro znajduje się największy port w \hyperref[SojuszniczeKrolestwa]{Królestwach}, w którym druidzi utrzymują permanentnie korzystną dla żeglarzy pogodę.

\section{Polityka}
\paragraph{}
Miastem rządzi diuk z klanu niziołków Olkyn, potomek Adana Olkyna, oryginalnego założyciela miasta Vicovaro. 
Aktualny diuk Dregras Olkyn jest jedynie marionetką, faktyczna władza znajduje się w rękach Arcydruidki Eryn oraz Thundy, przywódcy \hyperref[GildiaKupcow]{Gildii Kupców}.
Oboje nie przepadają za sobą nawzajem, ale prowadzą niechętną współpracę. 
Krąg druidzki pozostaje niezmiennie znacząca siłą polityczną w Vicovaro, ze względu na zależność miasta od ich magii. 
Te powiązania sprawiają, że krąg Vicovaro jest traktowany z pewną wrogością przez inne kręgi.

\section{Historia}
\paragraph{}
Zanim powstało Vicovaro na wybrzeżu znajdował się krąg druidzki. 
Wiele stuleci temu przywódca kręgu: druid Adan ściągnął do nadmorskiego siedliska kręgu cały swój klan. 
Klan stworzył port handlowy Vicovaro. 
Stopniowo coraz więcej niziołków  z okolicznych dolin zaczęło ściągać do portu, miasto zaczęło się rozrastać i zyskiwać na znaczeniu.
Port stał się łakomym kąskiem dla okolicznych państw ze względu na bogactwo i strategiczną lokalizację: konflikty wybuchały głównie z \hyperref[AnGammarna]{An'Gammarną} i Wolnymi Wyspami Kye. 
Vicovaro zawsze jednak dało radę odeprzeć napastników dzięki swojej silnej flocie i magii druidzkej. 
Konflikt z \hyperref[AnGammarna]{An'Gammarna} udało się w znacznym stopniu załagodzić po tym jak utworzono \hyperref[SojuszniczeKrolestwa]{Sojusznicze Królestwa} i udostępniono ziemię na budowę Sjorden \hyperref[KultIoun]{kultowi Ioun}.
Ataki ze strony Wysp Kye ciągle się zdarzają, ale słabo zorganizowane wyspy nie maja szans ze współczesny Vicovaro.

\chapter{Organizacje}

\label{GildiaKupcow}
\section{Gildia Kupców}
\paragraph{}
Ściśle powiązana gigantyczna siatka kupców bankierów.
Gildia Kupców jest, rzecz jasna, najpotężniejszą organizacją obejmującą swoim zasięgiem \hyperref[SojuszniczeKrolestwa]{Sojusznicze Królestwa}.
Każdy kto chce zajmować się handlem na dłuższą metę musi zostac członkiem Gildii, albo spotkaja go “nieprzyjemnie konsekwencje” wszelakiego rodzaju.
Gildia (nieco mniej oficjalnie) zajmuje się także polityką i dyplomacją.
W praktyce to oni utrzymują pokój w królestwach dzięki gigantycznym wpływom na bardzo wysokich szczeblach oraz, oczywiście, gigantycznym pieniądzom.
Tradycyjną siedzibą Gildii jest \hyperref[Vicovaro]{Vicovaro}, największy port hyperref[SojuszniczeKrolestwa]{Sojuszniczych Królestw} i ich jedyne połaczenie z wyspami Kye (i dalej).

\label{KonfraterieMagow}
\section[Konfraterie Magów]{Konfraterie\\Magów}
\paragraph{}
Bardzo luźne zgrupowania magów w różnych miastach królestw.
Większość magów szkoliło się w jednym z ośrodków Konfraterii, ponieważ daje to dostęp do bibliotek, miejsc do bezpiecznych eksperymentów oraz potencjalnych nauczycieli.
Większość magów z dumą deklaruje przynależność, do którejś z konfraterii, jako dowód, że mają porządne wykształcenie.
Najbardziej prestiżową konfraterią jest Latająca Wyspa Muria, lewitująca nad wybrzeżem Croridu.

\label{Varjossa}
\section{Varjossa}
\paragraph{}
Niewielu poza wysoko postawionymi członkami \hyperref[GildiaKupcow]{Gildii} wie o innej organizacji mającej koneksje w całych \hyperref[SojuczniczeKrolestwa]{Królestwach}, a nawet poza nimi: Varjossa, zwana nieco błędnie Gildią Zabójców.
Gdziekolwiek podróżnicy stykają się z gildiami zabójców, złodziei itp. prawie na pewno stoi za tym Varjossa.
Nawet jeśli członkowie najniższej rangi sami o tym nie wiedzą.
Wyżej postawionych członków mozna rozpoznać po tatuażu w kształcie oka, najczęściej z tyłu szyi.
Varjossa ściśle współpracuje z \hyperref[GildiaKupcow]{Gildią Kupców}, ale obie organizacje podkreślaja swoją niezależność.

\label{KultIoun}
\section{Sekretny Kult Ioun}
\paragraph{}
O ile wiele mniejszych i większych organizacji religijnych działa w \hyperref[SojuszniczeKrolestwa]{Królestwach} na uwagę zasługuje Sekretny Kult Ioun.
Stanowi on ,,dyskretne'' ramie instytucji religijnych w \hyperref[SojuszniczeKrolestwa]{Królestwach}.
Zajmują się zbieraniem i zabezpieczaniem wiedzy, szczególnie tej niebezpiecznej.
Poza tym działają, kiedy potrzeba bardziej dyskretnego podejścia niż wysłanie typowego paladyna.
Posiadają nieco szemraną reputację i niezbyt "przyjacielskich" członków, ale koniec końców, posiadają szacunek większości władz religijnych w \hyperref[SojuszniczeKrolestwa]{Królestwach}.


\twocolumn
\normalsize
\chapter{Kwiatki z sesji}

\section*{3 lutego 2018}

\begin{rpg-quotebox}{Bracia jednak myślą podobnie}
   \begin{tabularx}{\columnwidth}{lX}
      \textbf{Lander:} & Mam na imię Lander.\\
      \textbf{Randal:} & Kurwa, ja jestem Randal.\\
      \textbf{Ampoldórëon:} &  Co to ma być?!\\
      \textbf{MG:} & Odezwał się...\\
   \end{tabularx}
\end{rpg-quotebox}

\begin{rpg-quotebox}{Szewc bez butów chodzi...}
   \begin{tabularx}{\columnwidth}{lX}
      \textbf{Lander:} & Jestem klerykiem, który ma -1 do religii.\\
   \end{tabularx}
\end{rpg-quotebox}

\begin{rpg-quotebox}{Jaką nazwę wybrać?}
   \textit{Przegląd klas w drużynie: Lore bard, Loremaster, War Cleric i Warlock. Nazwa drużyny: Warlore.}
\end{rpg-quotebox}

\begin{rpg-quotebox}{Vitofobia?}
   \begin{tabularx}{\columnwidth}{lX}
      \textbf{MG:} & Znajdujesz kilka żywych ciał.\\
      \textbf{Kedra:} & W nogi!\\
   \end{tabularx}
\end{rpg-quotebox}

\begin{rpg-quotebox}{Bez młota to nie robota!}
   \begin{tabularx}{\columnwidth}{lX}
      \textbf{Lander:} & Uderzam w kryształ \emph{delikatnie} młotkiem.\\
      \textbf{MG:} &  Czym?\\
      \textbf{Lander:} & No... młotem bojowym.\\
   \end{tabularx}
\end{rpg-quotebox}

\begin{rpg-quotebox}{Powalająca konsekwencja}
   \textit{W korytarzu zapadła się ziemia i Lander spadł na dół. Łoskot lecących kamieni zwabił grupkę koboldów.}\\

   \begin{tabularx}{\columnwidth}{lX}
      \textbf{Kedra:} & Rzucam iluzję podłogi w miejscu dziury.\\
      \multicolumn{2}{l}{\textit{Do Landera:}}\\
      \textbf{MG:} & Nad swoją podłogą widzisz sufit.\\
   \end{tabularx}
\end{rpg-quotebox}

\begin{rpg-quotebox}{Jakby tu drążyć temat}
   \begin{tabularx}{\columnwidth}{lX}
      \textbf{Randal:} & To co teraz?\\
      \textbf{MG:} & Lander i Kedra są w dziurze, a ty i Ampoldórëon jesteście po złej stronie dziury.\\
   \end{tabularx}
\end{rpg-quotebox}

\begin{rpg-quotebox}{Chaotyczny-neutralny...}
   \textit{Randal po próbie przeskoku nad dziurą zawisł na jej krawędzi i próbował się podciągnąć.}\\

   \begin{tabularx}{\columnwidth}{lX}
      \textbf{Kedra:} &  Ciągnę go za nogawkę, by wpadł do środka. Tak, o! Dla żartu.\\
      \textbf{MG:} &  Czujesz pociągnięcie za nogę i poddajesz sie upadkowi. Po gwałtownym lądowaniu otrzymujesz 1 pkt. obrażeń.\\
   \end{tabularx}
\end{rpg-quotebox}

\begin{rpg-quotebox}{Kryptokonwersacja}
   \begin{tabularx}{\columnwidth}{lX}
      \textbf{Randal:} &  Czy mogę z kimś rozmawiać z ukrycia?\\
   \end{tabularx}
\end{rpg-quotebox}

\begin{rpg-quotebox}{O potędze słowa}
   \begin{tabularx}{\columnwidth}{lX}
      \textbf{Randal:} &  Obrzucam go wyzwiskami i kobold otrzymuje 1 pkt. obrażeń.\\
      \textbf{MG:} &  Zwyzywałeś kobolda na śmierć.\\
   \end{tabularx}
\end{rpg-quotebox}

\begin{rpg-quotebox}{O tym jak to bard właściwie czaruje}
   \begin{tabularx}{\columnwidth}{lX}
      \textbf{Lander:} &  Jak bard właściwie czaruje?\\
      \textbf{Randal:} &  Tak moduluje głos, żeby tworzyć magię.\\
      \textbf{MG:} &  Teoretycznie powinien śpiewać, ale nie wiem czy tego chcemy.\\
      \textbf{Ampoldórëon:} &  Rapuj! A-to-sreb-erko-to-ze-skle-pu-wzią-łeś-śmie-ciu?\\
   \end{tabularx}
\end{rpg-quotebox}

\begin{rpg-quotebox}{O tym dlaczego dzieci nie powinno traktować się jak małych dorosłych}
   \textit{Drużyna napotkała grupę koboldzich dzieci i czujną pilnującą ich opiekunkę.}\\
  
   \begin{tabularx}{\columnwidth}{lX}
      \textbf{Ampoldórëon:} &  Wysuwam się na pierwszy front i rzucam Thunderwave.\\
      \multicolumn{2}{l}{\textit{MG nuci ,,And his name is John Cena''.}}\\
      \textbf{MG:} &  Grzmot zaklęcia wyrzuca koboldzie dzieci w powietrze, a ich krzyki i życia nikną w kolejnych nutach melodii. Opiekunka została ledwo draśnięta.\\
   \end{tabularx}
\end{rpg-quotebox}

\section*{25 lutego 2018}

\begin{rpg-quotebox}{}
   \begin{tabularx}{\columnwidth}{lX}
      \textbf{Ampoldórëon:} &  Jesteśmy poza Królestwami.\\
      \textbf{Kedra:} &  To będzie tu jakaś cywilizacja?\\
      \textbf{Lander:} &  My.\\
   \end{tabularx}
\end{rpg-quotebox}

\begin{rpg-quotebox}{}
   \begin{tabularx}{\columnwidth}{lX}
      \textbf{Ampoldórëon:} &  W kwestii tworzenia cywilizacji to jest nas 3 do 1, ale wolałbym w to nie wchodzić...\\
      \textbf{Ampoldórëon:} &  Okej. To źle zabrzmiało.\\
   \end{tabularx}
\end{rpg-quotebox}

\begin{rpg-quotebox}{}
   \begin{tabularx}{\columnwidth}{lX}
      \multicolumn{2}{l}{\textit{Do kostki, która Matheusowi wyrzuciła 20.}}\\
      \textbf{Łukasz:} &  Ty kurwo, zdrajco!\\
   \end{tabularx}
\end{rpg-quotebox}

\begin{rpg-quotebox}{}
   \textit{Randal podszedł w ukryciu do obozowiska w którym trwało jakieś zamieszanie. Niestety podczas nasłuchiwania odkryto go i ostrzelano.}\\

   \begin{tabularx}{\columnwidth}{lX}
      \textbf{Randal:} &  A bard twój, który jest w ukryciu - odda tobie...\\
   \end{tabularx}
\end{rpg-quotebox}

\begin{rpg-quotebox}{}
   \textit{Drużyna postanowiła podkraść się bliżej centrum wydarzeń.}\\

   \begin{tabularx}{\columnwidth}{lX}
      \multicolumn{2}{l}{\textit{Do Landera:}}\\
      \textbf{MG:} &  Ty ukrywasz się z utrudnieniem, bo masz zbroję.\\
      \textbf{MG:} &  Podsumowując idziecie ukryci, ale w towarzystwie "klang! klang!" wydawanego przez ekwipunek Landera.\\
      \textbf{Ampoldórëon:} &  Ja po prostu nie umiem się ukrywać, ale przy nim to jestem niewidzialny!\\
   \end{tabularx}
\end{rpg-quotebox}

\begin{rpg-quotebox}{}
   \begin{tabularx}{\columnwidth}{lX}
      \multicolumn{2}{l}{\textit{Do ledwo żywego Ampoldórëona:}}\\
      \textbf{MG:} &  Drow strzela w twoim kierunku z kuszy, ale nie trafia.\\
      \textbf{Kedra:} &  Twoja nieprzytomność przeleciała ci obok ucha.\\
   \end{tabularx}
\end{rpg-quotebox}

\begin{rpg-quotebox}{}
   \begin{tabularx}{\columnwidth}{lX}
      \textbf{Randal:} &  Obrzucam goblina wyzwiskami.\\
      \textbf{Ampoldórëon:} &  Co mówisz?\\
      \multicolumn{2}{l}{\textit{Chwila ciszy.}}\\
      \textbf{MG:} &  Może: ,,Twoja matka hodowała zgniłe grzyby?''\\
      \textbf{Randal:} &  Twoja matka dawała za zgniłe grzyby!\\
   \end{tabularx}
\end{rpg-quotebox}

\begin{rpg-quotebox}{}
   \textit{W poprzedniej kampanii Łukasz grał niziołkiem mnichem.}\\

   \begin{tabularx}{\columnwidth}{lX}
      \multicolumn{2}{X}{\textit{O zachowaniu postaci podczas rozmowy z~kobietą pół-orkiem:}}\\
      \textbf{Łukasz:} & Przepraszam, patrzę do góry.\\
      \textbf{Matheus:} & A nie powinieneś się już był do tego przyzwyczaić?\\
   \end{tabularx}
\end{rpg-quotebox}

\begin{rpg-quotebox}{}
   \textit{Randal zaproponował kwatermistrzyni wspólne ugotowanie zupy warzywnej aby dać się wszystkim zregenerować po walce.}\\

   \begin{tabularx}{\columnwidth}{lX}
      \multicolumn{2}{l}{\textit{Szeptem do Randala:}}\\
      \textbf{Lander:} &  Zrób \emph{grzybową}. Z tymi grzybkami z jaskiń.\\
   \end{tabularx}
   ~\newline~
   \textit{Gdy zupa była gotowa do podania drużyna chciała uniknąć naćpania się nią wraz z magami.}\\

   \begin{tabularx}{\columnwidth}{lX}
      \multicolumn{2}{X}{\textit{Rzucając po kryjomu czar oczyszczający jedzenie z negatywnych składników:}}\\
      \textbf{Lander:} &  To ja pomodlę się przed posiłkiem!\\
   \end{tabularx}
\end{rpg-quotebox}

\begin{rpg-quotebox}{}
   \textit{Magowie straszliwie się naćpali i doznali licznych halucynacji. Część z nich zaczęła tańczyć na stole, część walczyć z wyimaginowanymi potworami, a jedna bardzo młoda dama zaczęła się dobierać do Randala.}\\

   \begin{tabularx}{\columnwidth}{lX}
      \textbf{Lander:} &  Teraz możesz ją porwać!\\
      \textbf{Ampoldórëon:} &  Czy ty chcesz ją porywać?!\\
      \textbf{Randal:} &  Wstaw tam literę D.\\
      \textbf{MG:} &  Taaak, daj d...\\
   \end{tabularx}
\end{rpg-quotebox}

\begin{rpg-quotebox}{}
   \textit{Randal aktywnie flirtował z BN-em, więc MG w międzyczasie wstała i pomalowała sobie usta pomadką ochronną.}\\

   \begin{tabularx}{\columnwidth}{lX}
      \multicolumn{2}{l}{\textit{W szoku:}}\\
      \textbf{Randal:} &  Co się tu właśnie stanęło?\\
      \textbf{Ampoldórëon:} &  Ja tam nie wiem co ci stanęło.\\
   \end{tabularx}
\end{rpg-quotebox}

\begin{rpg-quotebox}{}
   \textit{Randal postanowił przyjrzeć się ciału goblina}\\

   \begin{tabularx}{\columnwidth}{lX}
      \textbf{MG:} &  Po śladach patrząc...\\
      \textbf{Ampoldórëon:} &  Gdzie ty mu tam patrzysz?!\\
   \end{tabularx}
\end{rpg-quotebox}

04/03/2018

1. Randal: czy jest szansa na posiłek?
Lander: rzuć kostką

2. DM: dużą część która dotyczy magii jesteś sobie w stanie po chwili przypomnieć
Kedra: masz wszystko w wewnętrznym archiwum
Ampoldórëon: proszę czekać: trwa indeksowanie

3. NPC: rozumiem, że potraficie walczyć
Lander: ścieram mózg wroga z mojego młota

4. Na propozycję oczyszczenia kopalni z magów Kedra wymyśliła, że można zawalić do niej wejście. Wtedy magowie przestaną zakłócać granicę gnomiego terytorium. W swym pomyśle była jednak sama i burzliwe dyskusje zakończyła decyzja że jednak drużyna oczyści ją własnoręcznie. Pozostawało zakupić odpowiednie zaopatrzenie.
Randal: co potrzebujemy do kopalni?
Kedra: dynamit!

5. Randal: nadsłuchuję pod drzwiami
K20: 3
DM: coś kapie, więc nie wchodzicie na sucho

6. Randal: rzucam Phantasmal Force - każdy, kto wierzy w ten obraz dostaje 1k6 obrażeń
Kedra: zadajesz obrażenia od placebo

7. DM: zbroja hide nie ma utrudnienia do stealth, because it's made of hide

8. Lander: jak wygląda sytuacja?
DM: jest korytarz, który rozwidla się na końcu, ale wcześniej wypełniony jest Goblinozą
(obrazek)

9. Ampoldórëon: wszyscy którzy nie przeszli testu zostają odrzuceni przez efekt Thunderwave
Randal: czyli lecą w kierunku Landera, który jest na ich tyłach
DM do Landera: tak, masz na jednego z lecących attack of opportunity
L (K20): 18
DM: trafiasz jednego w locie i odcinasz mu głowę

10. Drużyna szukała miejsca na odpoczynek i znalazła komnatę na jednym z końców korytarza
Ampoldórëon: stawiam alarm na rozwidleniu
DM: gdy podchodzisz do skrzyżowania kilka metrów dalej za nim widzisz magiczną barierę
Ampoldórëon: dobra - to potem, magiczna bariera nie spierdoli.

15/04/2018

1. DM: obserwujecie jak Randal penetruje barierę kijem od pochodni

2. Faceci skoczyli jeden za drugim w magiczną barierę i poobijali siebie nawzajem podczas lądowania. Randal dał znać Kedrze w odpowiednim odstępie czasu kiedy i ona może dołączyć do skoku.
Kedra po zgrabnym lądowaniu: SPUŚCIĆ facetów z oczu...

3. DM: czy wy pamiętacie, że macie blessa?
Kedra, patrząc na nieprzytomnego Landera: nasz bless padł

4. Randal: czy mogę jako akcja bonusową ukryć się podczas walki?
DM: nie jesteś łotrzykiem
Kedra: on BARDzo chciałby być

5. DM: w pokoju znajdujecie otwarte ciało i drzwi do następnego pomieszczenia
Drużyna zawołała Landera z innego pokoju, który ma problem ze skradaniem się
DM: słyszycie głosy zza drzwi
Kedra: ukrywam się pod stołem!
Randal: chowam się w cieniu
Ampoldórëon: chowam się za drzwiami
Zaalarmowani wrogowie wpadli do pokoju i zobaczyli tylko Landera.
6. Drużyna walczyła z wrogimi magami i jeden z nich użył zaklęcia Shatter, co w efekcie unieprzytomniło Randala. Lander go uleczył i Randal wrócił do bitwy, tylko po to by oberwać kolejnym Shatter'em od golemopodobnego potwora. Tym razem ocuciła go Kedra wyczarowaną przez siebie fiolką uzdrawiającej mikstury. Następnie Ampoldórëon odkrył, że golemopodobny potwór jest podatny na to samo zaklęcie i postanowił tę wiedzę wykorzystać w praktyce. Ponieważ czar jest obszarowy, a Randal był w zasięgu - potwór się wywinął, a Randal... padł.

7. Lander: czy topiąc się coś widzę?

8. DM: zauważacie, że Randal odzyskuje władzę w kończynach
Kedra: zastanawia mnie czy w penisie też odzyskał władzę
Ampoldórëon: możesz sprawdzić, ja wyjdę
Kedra: smyram
Randal: z trudem, bo z trudem pokazuję jej środkowy palec

9. Lander: walę młotem w runy kręgu w celu zniszczenia go
DM: po uderzeniu zauważasz na jednej z run maleńkie pęknięcie, ale twój młot też się odrobinę uszkodził
Kedra: teraz możesz przybijać wrogom stemple
Ampoldórëon: tak! w kształcie litery L jak Lamer

10. Ampoldórëon: a gdzie Randal?
Kedra:  maca swój flet
Ampoldórëon:...

11. Randal: Próbuję zagrać na znalezionym flecie
DM: otrzymujesz 5 pkt. obrażeń
Ampoldórëon: mama Ci nie mówiła, żeby nie brać wszystkiego do ust?

12.05.2018
1. Randal o szlamie: mógłbym go zwyzywać
Kedra: powiedz mu że jego matka była szmatą

2. Randal: idę do Landera i gram na flecie
Kedra: na jego flecie?

3. Randal: czy widzimy co ta pułapka robiła?
Ampoldórëon: myślę, że pułapkowała

4. DM: elf głębinowy podaje ci kiełbaskę
Randal: co to jest?
DM: elf w milczeniu wkłada ją do ogniska
Randal: robi mi się smutno na myśl o tym zwierzęciu, zwłaszcza że w tych warunkach polowych raczej nie mają czasu na robienie kiełbasek

5. Lander: Wychodzę z namiotu w idealnie wypolerowanej zbroi
Kedra, Randal, Ampoldórëon: aua! nie po oczach
Lander: ej, nie jestem pieprzonym discoballlem!

6. NPC: czy interesuje was zwykła porządnie wykonana kusza, czy coś z lekkim kopem?
Randal: hmmm, kusząca ta kusza

7. Lander: oddam moją poprzednią zbroję za zniżkę
Randal: spójrz Pani, jak olśniewająco błyszczy. Polerował ją całą noc nawet przez sen.
Kedra: tak, wtedy się o nią ocierał...
DM: krasnoludzica zasypana kwiecistymi zapewnieniami Randala sprzedaje ci zbroję z mieszanką zdegustowania i zachwytu

16/06/2018 r.
1. Drużyna w drodze do Podmroku decydowała o następnym kroku. Do wyboru był wypoczynek w karczmie i odwiedzenie biblioteki, która była najbliżej do niebezpiecznego tunelu.
Kedra: idźmy teraz do biblioteki zanim potwory zmiotą bibliotekę z... pod powierzchni ziemii

2. DM: dotarło do was, że przez następne kilka tygodni nie zobaczycie słońca
Kedra: jaki jest teraz czas w mieście? Skończyła się noc... ech nie nauczę się ... nocna warta?

3. Spotkanie i rozmowa z NPC-em w zagraconej bibliotece.
DM: generalnie nie wiecie jak on się nazywa, nie przedstawiliście się
Ampoldórëon: to trochę niegrzecznie odwiedzać kogoś w jego domu i potem się pytać o jego imię
Randal: dzień dobry, pan tak mieszka?? Co to za burdel?

4. Randal zrywał trujące rośliny i poparzył sobie palce, przez co nie mógł grać na flecie.
Randal: może mi ten krasnolud czymś je posmaruje...
DM: masz kleryka obok siebie
Randal do Landera: posmarujesz mi paluszki?
Lander: czym?
Kedra: miłością

5. Za każdym razem gdy drużyna ogarnęła przedstawianie się Kedra dodawała przydomek Landerowi. Drużyna siedziała w karczmie i przedstawiała się gościom:
Ampoldórëon
Kedra,
Randal
Lander
Kedra: ... Stormwind pogromca Umbrowych kolosów
Ampoldórëon okazał zdegustowanie
Lander już podkurwiony: zaraz będę pogromcą Kedry
6.  Ampoldórëon próbował otworzyć kamienne drzwi do pokoju. Niestety jako mag nie był zbyt umięśniony.
DM: rzuć na atletykę
K20: 1
DM: teraz rzucaj na astmę

7. Ampoldórëon nadal próbował otworzyć kamienne drzwi tym razem za pomocą rytuału
K20: 16
DM: przywołałeś niewidzialną siłę, która pomogła ci otworzyć drzwi. Weź 50 pkt. expa.
Zza winkla przyglądał się wszystkiemu zszokowany Lander, w końcu nie często ktoś rysuje pentagramy w korytarzu.

8. Drużyna spotkała nieumarłego Beholdera w łaźni w karczmie. Beholder rzucił się do ataku na Landera.
Ampoldórëon: uuu chłopie zginiesz marnie
DM: będąc ścisłym zginiesz w wannie

9.  Randalowi udało się przestraszyć Beholdera i ten zaczął uciekać.
Kedra: rzuć mu to mydło które mi wcześniej oferowałeś by się poślizgnął
DM: beholdery lewitują

10. Skończyła się playlista na youtubie w tle i rozgrywka odbywała się w ciszy, w tym momencie Beholder zabił Landera
DM: może jakaś muzyczka?
Kedra, Ampoldórëon: <nucą marsz pogrzebowy>

11. Lander: czy tylko ja mam nazwisko?
Ampoldórëon: ja się przedstawiałem
Randal: I'm Backman <głosem filmowego Batmana>

07/07/2018

1. Oglądaliśmy mecz Chorwacja-Rosja w trakcie sesji. Ampoldórëon rzucił Shattera w drzwi karczmy, by je wyważyć i zaskoczyć czyhających na zewnątrz wrogów. W momencie trafienia  w drzwi Chorwacja wbiła bramkę Rosji.
Drużyna chórem: goooooooooool!

2. DM: Duergary są odporne na zaklęcia i ich efekty
Lander wskazując na powalonych wrogów: jakoś tego po nich nie widać
DM: one są odporne na magię, nie na cegły

3. Drużyna rozmawiała z kapitanem straży o stanie pola walki i wspomniała m.in o okropnościach jakie napotkali w łaźni karczmy.
NPC: Tunel, który się otworzył wydaje się szerszy niż ostatnim razem
Randal: na tyle, by zmieścił się w nim beholder?
NPC: nie wnikam
Randal kaszląc: łaźnia.

4. Ampoldórëon: celuję w tego oświetlonego i zhexowanego
Kedra: rrrr.... on jest bardzo hexowny....

5. Kedra potrafi przenosić klątwę z wroga na wroga jako bonus action
Ampoldórëon z uznaniem: jebana...
Kedra: halfcaster, but successfull
Ampoldórëon: nie to co ja - fullcaster i chuj

6. Klątwa Kedry obniżyła zdolności wroga w zakresie rzutów obronnych na wisdom
Randal: okej, to ten grzyb który świeci i który ma problem z....
DM: nietrzymaniem moczu!
Randal:... z zachowaniem zdrowgo rozsądku.

7. Otoczona smrodem drużyna raz, po razie zwracała pod siebie zawartość swoich żołądków. Lander miał (nie)przyjemność stracić przytomność podczas walki i paść w morze rzygowin

8. Walka z grzybami zaczęła się od tego, że Ampoldórëon wypuścił w stronę ściany firebolta. Rozwścieczone Mykonidy zaatakowały w odwecie drużynę, która jakimś cudem wyszła z walki zwycięsko.
Ampoldórëon: proponuję wypalić z grzybów cały ten tunel
Randal udając naćpanego: mmmm dobry towar
Kedra: czy my mamy tyle racji by zajeść tą gigantyczną gastrofazę?
Ampoldórëon: dym może dolecieć aż do miasta krasnoludów jak dobrze pójdzie
Randal: Tak! Zrobimy z tego tunelu gigantyczne bongo!
9. DM: w pewnym momencie czujecie okropny smród pleśni
Ampoldórëon: staram się zidentyfikować źródło smrodu
Randal: to ja i moje nieświeże racje, sorry
Ampoldórëon i K20=4: Ok. Fakt. To Randal pierdzi

10. Drużyna zobaczyła dziwną kurtynę z grzybów, która wydawała z siebie ten obrzydliwy smród. Gdy Lander próbował ją odsłonić nie wytrzymał i zwymiotował.
Ampoldórëon: zawsze możemy to olać i wrócić złożyć raport
DM: Tak. W tunelu jest chujnia z grzybnią

11. DM: Lander, smród jest tak gryzący, że wymiotujesz drużynie pod nogi
Randal: O! Lander kiedy jadłeś spaghetti?
Lander: to tasiemiec

12. Grzyb ponownie zaatakował smrodem i znów zmusił drużynę do zwrotów z żołądków. Test na wytrzymałość zdał tylko Lander.
Randal: Lander już ma pusty żołądek

13. Lander: leczę się za... <rzut kością> Za kurwa mać.

14. Lander: to ja atakuję śmierdziuszka (dużego Mykonida atakującego z pięści)

15. DM do Ampoldórëona, który nie trafił swoim czarem: robisz coś jeszcze?
Ampoldórëon: płaczę nad swoim losem

16. DM: grzyb zaatakował Randala za 9 pkt. obrażeń
Randal: poprzestawiał mi cyferki! Miałem 21, a mam 12 HP

14/08/2018

1. DM: jesteście w Nowym Eidsvik
Ampoldórëon: aaa, tym zadupiu
Kedra: raczej poddupiu, bo to pod ziemią

2. DM: Piwo miło cię połechtało. Otrzymujesz 1 tymczasowy punkt życia.
Randal: Super! Przyda się do późniejszego lania się po mordach

3. DM: pijąc czujesz, że zaczynasz delikatnie unosić się nad ziemią
Kedra: wow, można z tym wygrać wszystkie konkursy taneczne!
Randal: oprócz stepowania

4. DM: słyszycie, że muzyka nagle się urwała, bo jeden z gości znokautował orkiestrę
Randal: należałoby ich zastąpić!
Alchemicus (npc): w stanie lekkiej nieważkości nie za bardzo możesz się ruszać. Jesteście na antresoli, a orkiestra jest pod wami.
Ampoldórëon: traktuję Randala mage-handem
Kedra: super! wyrzuć go na dół jak worek kartofli!
Randal: świetnie! Przeskakuję nad balustradą i niczym istota z niebios oferuję im muzykę swojego fletu!
DM: rzuć na performance
K20 + mod: 21

5. Alchemicus do Landera: ten trunek cię rozświetli, proszę wybierz kolor
Randal: czarny! Czarny się nie świeci!
Lander: niech będzie czerwony...


6. Alchemicus do Landera: po wypiciu świecisz się cały na czerwono. Jesteś jakby fluoresencyjny.
Kedra: czy twój wacek też się świeci?
Lander: co?!
Kedra: czy jak cię TAM pogłaszczę, to wskażesz punkt?
Lander: <gdyby wzrok mógł zabijać>

7. Alchemicus do Kedry na temat zboczonych żartów: ty czegoś potrzebujesz...
Kedra (cicho): mam ochotę na loda
DM: Alchemicus podaje Kedrze płyn o lekko białawym zabarwieniu...
Ranadal: pfffff!
8. DM: Randal patrzy na mydło i wspomina swoją matkę
Żarty o Żydach za 3... 2... 1...

9. Lander o planowanym użyciu czaru Invisibility: to jest na koncentrację, więc musicie spinać poślady.

10. DM: Alchemicus podarował wam granat wzmacniający efekty rzucanych czarów. W skrócie taka chmurka wpierdolu.

11. DM do Landera: rzuć na percepcję
Lander: mam 23
Kedra: all is cleric!

12. DM do Landera: macasz ścianę
Randal: ściana odmacuje

13. DM: wsypujący się piasek sięga wam już do ud, więc wszelki ruch jest utrudniony.
Drużyna ma zrezygnowane miny.
DM kontynuując: żywiołak jest w swoim żywiole :D

14. Randal: używam czaru Suggestion i sugeruję żywiołakowi, by zostawił nas w spokoju i otwarł nam drzwi
DM: żywiołak nie rozumie języka w którym mówisz
Randal: ale pokazałem mu drzwi!

15. Randal: czuję się jak w klepsydrze - coś mi się sypie na głowę i coś usypuje spod nóg
Kedra wskazując na piasek sięgający pasa: tak, i czas ci się kończy!

% \begin{rpg-quotebox}{}
%    \textit{}\\
%    
%    \begin{tabularx}{\columnwidth}{lX}
%       \multicolumn{2}{l}{\textit{}}\\
%       
%       \textbf{:} & \\
%    \end{tabularx}
% \end{rpg-quotebox}








% \begin{rpg-paperbox}{rpg-paperbox}
% 	you can add some text here
% \end{rpg-paperbox}
% 
% %\newpage % Acts as columbreak because of twocolumn option; for pagebreak use \clearpage
% 
% \section{Some tables}
% \header{default rpg-table (2 column)}
% \begin{rpg-table}
%    	\textbf{Table head 1}  & \textbf{Table head 2} \\
%    	Some value  & Some value \\
%    	Some value  & Some value \\
%    	Some value  & Some value
% \end{rpg-table}
% 
% % For more columns, you can say \begin{rpg-table}[your options here].
% % For instance, if you wanted three columns, you could say
% % \begin{rpg-table}[XXX]. The usual host of tabular parameters are
% % aailable as well.
% \header{rpg-table with more columns}
% \begin{rpg-table}[XXX]
%     \textbf{Table head 1}  & \textbf{Table head 2} & \textbf{Table head 3}\\
%    	Some value  & Some value & Some value\\
%    	Some value  & Some value & Some value\\
%    	Some value  & Some value & Some value
% \end{rpg-table}
% 
% \header{default rpg-table2 (2 column)}
% \begin{rpg-table2}
%    	\textbf{Table head 1}  & \textbf{Table head 2} \\
%    	Some value  & Some value \\
%    	Some value  & Some value \\
%    	Some value  & Some value
% \end{rpg-table2}
% 
% 
% \section{List}
% \begin{rpg-list}
%     \item first list item
%     \item second list item
% \end{rpg-list}
% 
% \section{Monster}
% % You can optionally not include the background by saying
% %\begin{rpg-monsterboxnobg}{monsterboxnob}
% \begin{rpg-monsterbox}{rpg-monsterbox}
% 	\textit{Small metasyntatic variable (golbinoid), neutral evil}\\
% 	\rpghline
% 	\basics[%
% 	armorclass = 12,
% 	hitpoints  = 16 (3d8 + 3),
% 	speed      = 50 t
% 	]
% 	\rpghline
% 	\stats[ % This stat command will autocomplete the modifier for you
%     STR = 12, 
%     DEX = 7
% 	]
% 	\rpghline
% 	\details[%
% 	% If you want to use commas in these sections, enclose the
% 	% description in braces.
% 	% I'm so sorry.
% 	languages = {Common Lisp, Erlang},
% 	]
% 	\rpghline \\[1mm]
% 	\begin{rpg-monsteraction}[rpg-monsteraction]
% 		This Monster has some serious superpowers!
% 	\end{rpg-monsteraction}
% 
% 	\rpgmonstersection{rpgmonstersection}
% 	\begin{rpg-monsteraction}[rpm-monsteraction]
% 		This one can generate tremendous amounts of text! Though only when it wants to.
% 	\end{rpg-monsteraction}
% 
% 	\begin{rpg-monsteraction}[rpg-monsteraction]
%     See, here he goes again! Yet more text.
% 	\end{rpg-monsteraction}
% \end{rpg-monsterbox}
% 
% \chapter{Chapter name}
% 
% % End document
\end{document}
