\documentclass[10pt,twoside,twocolumn]{book}
\usepackage[bg-letter]{lib/rpg-book} % Options: bg-a4, bg-letter, bg-full, bg-print, bg-none.
\usepackage[polish]{babel}
\usepackage[utf8]{inputenc}
\usepackage{hyperref}
\usepackage{multicol}
\usepackage{multirow}
\usepackage{lipsum}
\usepackage{tabularx}

\title{Warlore}
\date{\today}
\author{The Dungeon Mistress}
\author{Khedre}
\author{Ampoldórëon Sailaquettë}
\author{Lander Stormwind}
\author{Randal Buckman}

% Start document
\begin{document}
\fontfamily{ppl}\selectfont % Set text font
\frontmatter

\maketitle
\begin{multicols}{2}
\tableofcontents
\end{multicols}

% Your content goes here
\mainmatter

\chapter{Ildrim - kontynent imperium}

\paragraph{}
Ciężko opisać kontynent jako całość, gdy nawet najbardziej odważni i potężni podróżnicy rzadko opuszczają jego wybrzeża.
Gdy nawet potężni mistrzowie magii z trudem sięgają umysłem poza jego granice. 
Wyobraźmy sobie masę lądu otoczoną ze wszystkich stron oceanem.
I to nie takim zwykłym, oceanem kierowanym starożytną magią z Wysokich Planów.
Oceanem chcącym pożreć, pochłonąć nieszczęsny ląd.
Taki jest nasz Ildrim, zwany też Kontynentem Imperium.
Jeśli nic się nie zmieni, bądź zmieni się zbyt wiele, zostaniemy wszyscy pogrążeni pod wodą, skała i magią.
Ponieważ Ildrimem władają moce poza naszą kontrolą.

\paragraph{}
Śmiercionośny ocean wpływa na naszą egzystencję w jeszcze jeden sposób: niemal kompletnie odcina nas od innych cywilizacji.
O ile pas wody w pobliżu wybrzeży w większości jest stosunkowo spokojny, głębokie wody są niemal zupełnie niedostępne.
Jedynie druidzi z \hyperref[Vicovaro]{Vicovaro}, elfowie z Vorath i „wolni żeglarze” z Wysp Kye są w stanie próbować swoich sił z żywiołem.

\paragraph{}
Kontynent zatem jest całym światem dla praktycznie wszystkich jego mieszkańców, zapewne z  wyjątkiem najprostszych rolników, których pojęcie obejmuje co najwyżej sąsiednie miasto.
I nie powinno dziwić, że przez milenia Ildrim zaznał niewiele spokoju.

\paragraph{}
Większość kontynentu zajmują \hyperref[SojuszniczeKrolestwa]{Sojusznicze Królestwa}: od górskich twierdz \hyperref[AnGammarna]{An’Gammarny} do spustoszonych iglic Yunait.
To te siedem państw jest odpowiedzialnych za ostatnie wojny i niepokoje na kontynencie.
Jednak po stuleciach walk \hyperref[SojuszniczeKrolestwa]{Sojusznicze Królestwa} utrzymywane są w pokoju przez \hyperref[GildiaKupcow]{Gildię Kupców}.
Z punktu widzenia \hyperref[GildiaKupcow]{Gildii} pokój zostanie zachowany.
Niezależnie do ceny.
Państwa te pozostają, przynajmniej we własnym mniemaniu, ostoją cywilizacji i dziedzicami Imperium.
Poza elfim państwem-miastem Vorath są jedyną cywilizacją na powierzchni Planu Materialnego nie pozostająca w ukryciu.

\paragraph{}
Prawda, jak to ma w zwyczaju, jest nieco bardziej skomplikowana.
Jeśli ktoś zdoła przebyć zdradzieckie północne góry \hyperref[AnGammarna]{An’Gammarny} i \hyperref[Vicovaro]{Vicovaro} dotrze do mroźnych pustkowi.
Tam na pokrytych wiecznym lodowcem wybrzeżach znajduje się owiane legendami państwo.
Niewielu wie o nim cokolwiek pewnego, chociaż większość dzieci w \hyperref[SojuszniczeKrolestwa]{Sojuszniczych Królestwach} słyszało przed snem baśnie o magicznych miastach pośród lodu.
Jeszcze dalej na zachód znajduje się sporych rozmiarów wyspa, ukryta pośród mgieł.
Ta wyspa to Vorath, ostatnie miasto elfów na Planie Materialnym, a w każdym razie ostatnie na tym kontynencie.

\paragraph{}
Od południowo-zachodniej strony Ildrimu znajduje się Platynowa Zatoka.
Stanowi ona wybrzeże \hyperref[AnGammarna]{An’Gammarny}, Croridu oraz Eneadoru.
Wody zatoki, są wyjątkowo zdradliwe, nawet jak na warunki kontynentu.
Powodem jest Latająca Wyspa Thor Ar Muria, gdzie znajduje się największa na kontynencie akademia magii arkanicznej.
Na obszarze Platynowej Zatoki znajdują się żeglowne przesmyki, dostępne dla doświadczonych śmiałków, jednak większość wypełniona jest wirami i wiatrami o nieprzewidywalnym, magicznym źródle.

\paragraph{}
Na wschodzie znajduje się niezbadana i niebezpieczna kraina zwana Pustkowiem Kataklizmu.
Nieliczne jego fragmenty są zamieszkane przez pionierów z Domen Krasnoludów.
Choć niewiele osób o tym wie, w północno-zachodniej części Pustkowia znajduje się Przystań: miasto uciekinierów z Podmroku.
Dawni mieszkańcy tego miasta przemierzają także te nieprzyjazne tereny jako nomadzi.
Niektórzy twierdzą, że wschodnie wybrzeża Ildrimu otaczają spokoje wody, którymi można przepłynąć do innych kontynentów. Nikomu jednak nie udało się do nich dotrzeć.

\paragraph{}
Nie są to jednak jedynie tereny na lądzie dotknięte niszczącą magią.
Na wschód od \hyperref[Vicovaro]{Vicovaro} i w \hyperref[AnGammarna]{An’Gammarnie} znajdują się ruiny dawnego Imperium.
Jakiekolwiek magiczne eksperymenty przeprowadzali ich mieszkańcy, ich moc przesiąknęła kamienie, ziemię i drzewa.
W samej \hyperref[AnGammarna]{An’Gammarnie} klerycy i paladyni zdołali opanować nienaturalny żywioł, ale Pomiędzy \hyperref[AnGammarna]{An’Gammarną}, a \hyperref[Vicovaro]{Vicovaro} pozostaje puszcza, kompletnie niestabilna i nienadająca się do zamieszkania zwana Doliną Konwalii.

\paragraph{}
Jest wiele teorii skąd wzięła się aż tak duża magiczna niestabilność występująca na kontynencie, a także czemu jest jedno miejsce praktycznie kompletnie od nich wolne: leżący na zachodzie archipelag Wysp Kye.
Wyspy stanowią przytań dla wielu, którzy przystani poszukują, ale niekoniecznie na nią zasługują.
Zamieszkujący je „wolni żeglarze” prowadzą nieskończone spory z Gildią Kupców chcących dostępu do spokojnych wód i dóbr archipelagu.
Z różnymi skutkami.
Wśród plotek i legend związanych z wyspami znajdują się labirynty podziemi wypełnionych skarbami i starożytnymi tajemnicami.
Niestety niewielu badaczy miało okazję poznać te wypełnione piratami i najemnikami wyspy.

\paragraph{}
Zanim powstały aktualne państwa większość północnej i zachodniej części kontynentu zajmowało Imperium, powszechnie uznawana za państwo elfów zanim opuściły one Plan Materialny.
Imperium najprawdopodobniej zostało zniszczone przez pierwszy na kontynencie magiczny kataklizm.
Możliwe nawet, że go wywołało, doprowadzając Ildrim do tego czym jest teraz.
Praktycznie nie ma organizacji, która obejmowałaby swoim zasięgiem cały kontynent.
Pewne powiązania ma tak zwana \hyperref[KonfrateriaMagow]{Konfrateria Magów}.
Nie jest to formalna organizacja, a raczej ogólne określenie na wszystkich formalnie wyszkolonych magów arkanicznych.
W ramach \hyperref[KonfrateriaMagow]{Konfraterii} funkcjonują Strażnicy Arkanum, stosunkowo niewielka grupa rozproszona po całym Ildrimie.
Tutaj można już mówić o zorganizowaniu, ale dotyczy to stosunkowo niewielkiej, dobrze ukrytej grupy magów.
Większość członków \hyperref[KonfrateriaMagow]{Konfraterii} nie ma pojęcia o ich istnieniu.
Na podobnych zasadach można traktować Kręgi Druidów -  o ile poszczególne Kręgi są rozproszone po całym kontynencie, to nie stanowią jednolitej organizacji.

\label{SojuszniczeKrolestwa}
\chapter{Sojusznicze Królestwa}

\paragraph{}
Można by napisać książkę, a nawet tuzin książek o różnych teoriach co do powstania państw, które stworzyły Sojusznicze Królestwa.
Jedyne co można z pewnością napisać to, że Imperium upadło i na kontynencie powstał chaos.
Na północy elfy zaczęły walczyć z ludźmi i ich sojusznikami, co zakończyło się zwycięstwem ludzi i opuszczeniem Planu Materialnego przez elfy.
Państwa na południu nagle straciły groźnego przeciwnika.

\paragraph{}
Ci mieszkańcy kontynentu którzy nie odsunęli się w trudnodostępne regiony i plany utworzyli siedem państw.
Od północy na południe są to: \hyperref[Vicovaro]{Vicovaro}, \hyperref[AnGammarna]{An’Gammarna}, Crorid, Sothedora, Eneador, Chacan i Yunait.
Eneador, Chacan i Yunait nigdy nie były częścią Imperium, chociaż Eneador był z nim w sojuszu od początku swojego istnienia, by uniknąć podboju (formalnie był ówcześnie częścią Domen Krasnoludów).
Pozostałe państwa powstały na ruinach po początkowym chaosie, w szczególności \hyperref[AnGammarna]{An’Gammarna} i Crorid.

\paragraph{}
O ile obecnie królestwa są mieszaniną różnych ras (z wyjątkiem Yunait, które pozostaje niemal całkowicie zamieszkane przez drakonów) pierwotnie były dużo bardziej podzielone.
\hyperref[Vicovaro]{Vicovaro} było państwem niziołków, \hyperref[AnGammarna]{An'Gammarna} i Sothedora były państwami ludzi z Imperium, Chacan zaś niezalezną cywilizacją ludzi spoza Imperium.
Crorid i Eneador były koloniami krasnoludów i gnomów, przy czym Crorid został wcześniej całkowicie podbity przez Imperium.

\paragraph{}
Mniejsze lub większe potyczki między królestwami rozpoczęły się niemal od razu po upadku Imperium.
Trwały stulecia, ograniczają rozwój i handel.
Żadne z państw nie mogło uzyskać dłuższej przewagi.
Wojny zapewne trwałyby nadal gdyby nie drugi kataklizm.
Nie ma pewności czy był echem zniszczenia Imperium, czy czymś zupełnie nowym.
Pewne jest to, że magiczna eksplozja wstrząsnęła pustkowiami przy wschodniej granicy Yunait, niemal całkowicie je niszcząc. 
Magia rozproszyła się po Ildrimie jak kręgi na wodzie.
W końcu dotarła do imperialnych ruin w Dolinie Konwalii i \hyperref[AnGammarna]{An’Gammarnie}.
Magia zawsze była w nich intensywna, ale to sprawiło, że rozpętała się magiczna burza.
I kompletny chaos.
Kiedy pierwsze skutki się uspokoiły nastąpiły te bardziej długotrwałe.
Cały kontynent objęła katastrofalna susza.
Powoli zdano sobie sprawę, że królestwa nie przetrwają pojedynczo i współpraca jest jedyną szansą na przeżycie.
Tak powstał Sojusz.

\paragraph{}
Wraz z Sojuszem powstała \hyperref[GildiaKupcow]{Gildia Kupców}.
Ta prosta nazwa nie oddaje ogromu organizacji która przez lata oplotła Królestwa siecią kontaktów, szpiegów i intryg.
I która praktycznie rządzi we wszystkich siedmiu państwach.
Załagodzenie kataklizmu i jego efektów zajęło lata, ale przyniosło dodatkowe profity: lepszą komunikację, szlaki handlowe, a co za tym idzie pieniądze.
Nie chcąc tracić zysków \hyperref[GildiaKupcow]{Gildia} skupiła się na swoim nowym najważniejszym celu - utrzymać pokój w Sojuszniczych Królestwach za wszelką cenę.

\paragraph{}
Przez stulecia pokój został utrzymany.
Jednak silne oddziaływanie \hyperref[GildiaKupcow]{Gildii} sprawia, że sytuacja polityczna staje się coraz bardziej niestabilna, a wysiłki \hyperref[GildiaKupcow]{Gildii} by utrzymać status quo coraz bardziej brutalne.
Obecnie \hyperref[Vicovaro]{Vicovaro} jest niemal całkowicie kontrolowane przez \hyperref[GildiaKupcow]{Gildię} (pomijajac atak przez wrogo nastawioną grupę druidów).
Udało im się także się w dużym stopniu zdestabilizować władzę w Croridzie.
Pozostałe państwa zachowuja względnie niezależną władzę, ale to także zaczyna sie powoli zmieniać.

\chapter{An'Gamarrna}

\paragraph{}
Państwo ludzi stworzone na ruinach dawnego Imperium, obecnie rządzone przez teokratyczny kult Bahamuta, do którego musi należeć nawet król.
Pańśtwo jest częścią Sojuszniczych Królestw.

\section{Geografia}
\paragraph{}
An'Gamarrna obejmuje zatokę pomiędzy Doliną Konwalii a Croridem i Eneaorem.
Platynowa zatoka nie umożliwia ze względu na niestabilne prądy przez istnienie latającej wyspy Muria, co powoduje bardzo napięte stosunki między An'Gamarrną a Croridem.
Obecną stolicą państwa jest święte miasto Glenrowan.
Innymi dużymi ośrodkami są dawna stolica Imperium, twierdza Bleihar (znajdująca się w górach na północy) oraz port Thelassa (około dzień drogi od Glenrowan nad brzegiem Platynowej Zatoki).
Klimat An'Gamarrny jest umiarkowany, z okazjolanymi przebłyskami niestabilnej pogody w miejscach gdzie kapłani nie kontrolują dobrze granic między planami.

\section{Polityka}
\paragraph{}
An'Gammarna rządzona jest przez króla, który aby odziedziczyć swoje królestwo musi zostać paladynem Bahamuta.
Obecnym królem jest Lorne III, który wziął za małżonkę Zephę, arcykapłankę Bahamuta.
Zepha jest młodszą od nieo o kilkanaście lat półelfką, dała mu dwóch synów.
Wiele osób uważa, że to ona naprawdę rządzi królestwem.
Wszelkie kulty czczące dobrych i neutralnych bogów cieszą się dużym poważaniem, kulty chaotyczne mogą być natomiast traktowane z pełną rezerwą.

\section{Historia}
\paragraph{}
An'Gammarna wg legend została założona przez zbuntowanych niewolników podczas schyłku Imperium.
Niewolnicy pierwotnie przejęli twierdzę Bleihar, a później objawił im się platynowy smok i polecił stworzyć święte miasto Glenrowan.
Ponieważ teren współczesnej An'Gammarny, poza twierdzą Bleihar, znajdował się ciągle w rękach elfów - wybuchła wojna.
Zakończyła się ona zwycięstwem ludzi i zniszczeniem elfiej stolicy Eleander.

\label{Vicovaro}
\chapter{Vicovaro}

\paragraph{}
Państwo niziołków stworzone w miejscu kręgu druidzkiego specjalizującego się z kontroli pogody oraz magii wody. 
Przez \hyperref[GildiaKupcow]{Vicovaro} przepływa większość handlu zewnętrznego w \hyperref[SojuszniczeKrolestwa]{Królestwach} i jest półoficjalną siedzibą \hyperref[GildiaKupcow]{Gildii Kupców}.
Państwo jest częścią \hyperref[SojuszniczeKrolestwa]{Sojuszniczych Królestw}.

\section{Geografia}
\paragraph{}
Vicovaro jest najbardziej na zachód wysuniętym państwem \hyperref[SojuszniczeKrolestwa]{Sojuszniczych Królestw}.
Od najbliższego wschodniego sąsiada, \hyperref[AnGammarna]{An'Gammarny}, oddzielona jest Doliną Konwalii. 
Poza miastem portowym Vicovaro w państwie nie było żadnych innych miast (co formalnie czyniło z Vicovaro miasto-państwo), do czasu kiedy 200 lat temu \hyperref[KultIoun]{kult Ioun} wybudował Sjorden. 
W Vicovaro znajduje się największy port w \hyperref[SojuszniczeKrolestwa]{Królestwach}, w którym druidzi utrzymują permanentnie korzystną dla żeglarzy pogodę.

\section{Polityka}
\paragraph{}
Miastem rządzi diuk z klanu niziołków Olkyn, potomek Adana Olkyna, oryginalnego założyciela miasta Vicovaro. 
Aktualny diuk Dregras Olkyn jest jedynie marionetką, faktyczna władza znajduje się w rękach Arcydruidki Eryn oraz Thundy, przywódcy \hyperref[GildiaKupcow]{Gildii Kupców}.
Oboje nie przepadają za sobą nawzajem, ale prowadzą niechętną współpracę. 
Krąg druidzki pozostaje niezmiennie znacząca siłą polityczną w Vicovaro, ze względu na zależność miasta od ich magii. 
Te powiązania sprawiają, że krąg Vicovaro jest traktowany z pewną wrogością przez inne kręgi.

\section{Historia}
\paragraph{}
Zanim powstało Vicovaro na wybrzeżu znajdował się krąg druidzki. 
Wiele stuleci temu przywódca kręgu: druid Adan ściągnął do nadmorskiego siedliska kręgu cały swój klan. 
Klan stworzył port handlowy Vicovaro. 
Stopniowo coraz więcej niziołków  z okolicznych dolin zaczęło ściągać do portu, miasto zaczęło się rozrastać i zyskiwać na znaczeniu.
Port stał się łakomym kąskiem dla okolicznych państw ze względu na bogactwo i strategiczną lokalizację: konflikty wybuchały głównie z \hyperref[AnGammarna]{An'Gammarną} i Wolnymi Wyspami Kye. 
Vicovaro zawsze jednak dało radę odeprzeć napastników dzięki swojej silnej flocie i magii druidzkej. 
Konflikt z \hyperref[AnGammarna]{An'Gammarna} udało się w znacznym stopniu załagodzić po tym jak utworzono \hyperref[SojuszniczeKrolestwa]{Sojusznicze Królestwa} i udostępniono ziemię na budowę Sjorden \hyperref[KultIoun]{kultowi Ioun}.
Ataki ze strony Wysp Kye ciągle się zdarzają, ale słabo zorganizowane wyspy nie maja szans ze współczesny Vicovaro.

\chapter{Organizacje}

\label{GildiaKupcow}
\section{Gildia Kupców}
\paragraph{}
Ściśle powiązana gigantyczna siatka kupców bankierów.
Gildia Kupców jest, rzecz jasna, najpotężniejszą organizacją obejmującą swoim zasięgiem \hyperref[SojuszniczeKrolestwa]{Sojusznicze Królestwa}.
Każdy kto chce zajmować się handlem na dłuższą metę musi zostac członkiem Gildii, albo spotkaja go “nieprzyjemnie konsekwencje” wszelakiego rodzaju.
Gildia (nieco mniej oficjalnie) zajmuje się także polityką i dyplomacją.
W praktyce to oni utrzymują pokój w królestwach dzięki gigantycznym wpływom na bardzo wysokich szczeblach oraz, oczywiście, gigantycznym pieniądzom.
Tradycyjną siedzibą Gildii jest \hyperref[Vicovaro]{Vicovaro}, największy port hyperref[SojuszniczeKrolestwa]{Sojuszniczych Królestw} i ich jedyne połaczenie z wyspami Kye (i dalej).

\label{KonfraterieMagow}
\section[Konfraterie Magów]{Konfraterie\\Magów}
\paragraph{}
Bardzo luźne zgrupowania magów w różnych miastach królestw.
Większość magów szkoliło się w jednym z ośrodków Konfraterii, ponieważ daje to dostęp do bibliotek, miejsc do bezpiecznych eksperymentów oraz potencjalnych nauczycieli.
Większość magów z dumą deklaruje przynależność, do którejś z konfraterii, jako dowód, że mają porządne wykształcenie.
Najbardziej prestiżową konfraterią jest Latająca Wyspa Muria, lewitująca nad wybrzeżem Croridu.

\label{Varjossa}
\section{Varjossa}
\paragraph{}
Niewielu poza wysoko postawionymi członkami \hyperref[GildiaKupcow]{Gildii} wie o innej organizacji mającej koneksje w całych \hyperref[SojuczniczeKrolestwa]{Królestwach}, a nawet poza nimi: Varjossa, zwana nieco błędnie Gildią Zabójców.
Gdziekolwiek podróżnicy stykają się z gildiami zabójców, złodziei itp. prawie na pewno stoi za tym Varjossa.
Nawet jeśli członkowie najniższej rangi sami o tym nie wiedzą.
Wyżej postawionych członków mozna rozpoznać po tatuażu w kształcie oka, najczęściej z tyłu szyi.
Varjossa ściśle współpracuje z \hyperref[GildiaKupcow]{Gildią Kupców}, ale obie organizacje podkreślaja swoją niezależność.

\label{KultIoun}
\section{Sekretny Kult Ioun}
\paragraph{}
O ile wiele mniejszych i większych organizacji religijnych działa w \hyperref[SojuszniczeKrolestwa]{Królestwach} na uwagę zasługuje Sekretny Kult Ioun.
Stanowi on ,,dyskretne'' ramie instytucji religijnych w \hyperref[SojuszniczeKrolestwa]{Królestwach}.
Zajmują się zbieraniem i zabezpieczaniem wiedzy, szczególnie tej niebezpiecznej.
Poza tym działają, kiedy potrzeba bardziej dyskretnego podejścia niż wysłanie typowego paladyna.
Posiadają nieco szemraną reputację i niezbyt "przyjacielskich" członków, ale koniec końców, posiadają szacunek większości władz religijnych w \hyperref[SojuszniczeKrolestwa]{Królestwach}.


\twocolumn
\normalsize
\chapter{Kwiatki z sesji}

\section*{3 lutego 2018}

\begin{rpg-quotebox}{Bracia jednak myślą podobnie}
   \begin{tabularx}{\columnwidth}{lX}
      \textbf{Lander:} & Mam na imię Lander.\\
      \textbf{Randal:} & Kurwa, ja jestem Randal.\\
      \textbf{Ampoldórëon:} &  Co to ma być?!\\
      \textbf{MG:} & Odezwał się...\\
   \end{tabularx}
\end{rpg-quotebox}

\begin{rpg-quotebox}{Szewc bez butów chodzi...}
   \begin{tabularx}{\columnwidth}{lX}
      \textbf{Lander:} & Jestem klerykiem, który ma -1 do religii.\\
   \end{tabularx}
\end{rpg-quotebox}

\begin{rpg-quotebox}{Jaką nazwę wybrać?}
   \textit{Przegląd klas w drużynie: Lore bard, Loremaster, War Cleric i Warlock. Nazwa drużyny: Warlore.}
\end{rpg-quotebox}

\begin{rpg-quotebox}{Vitofobia?}
   \begin{tabularx}{\columnwidth}{lX}
      \textbf{MG:} & Znajdujesz kilka żywych ciał.\\
      \textbf{Kedra:} & W nogi!\\
   \end{tabularx}
\end{rpg-quotebox}

\begin{rpg-quotebox}{Bez młota to nie robota!}
   \begin{tabularx}{\columnwidth}{lX}
      \textbf{Lander:} & Uderzam w kryształ \emph{delikatnie} młotkiem.\\
      \textbf{MG:} &  Czym?\\
      \textbf{Lander:} & No... młotem bojowym.\\
   \end{tabularx}
\end{rpg-quotebox}

\begin{rpg-quotebox}{Powalająca konsekwencja}
   \textit{W korytarzu zapadła się ziemia i Lander spadł na dół. Łoskot lecących kamieni zwabił grupkę koboldów.}\\

   \begin{tabularx}{\columnwidth}{lX}
      \textbf{Kedra:} & Rzucam iluzję podłogi w miejscu dziury.\\
      \multicolumn{2}{l}{\textit{Do Landera:}}\\
      \textbf{MG:} & Nad swoją podłogą widzisz sufit.\\
   \end{tabularx}
\end{rpg-quotebox}

\begin{rpg-quotebox}{Jakby tu drążyć temat}
   \begin{tabularx}{\columnwidth}{lX}
      \textbf{Randal:} & To co teraz?\\
      \textbf{MG:} & Lander i Kedra są w dziurze, a ty i Ampoldórëon jesteście po złej stronie dziury.\\
   \end{tabularx}
\end{rpg-quotebox}

\begin{rpg-quotebox}{Chaotyczny-neutralny...}
   \textit{Randal po próbie przeskoku nad dziurą zawisł na jej krawędzi i próbował się podciągnąć.}\\

   \begin{tabularx}{\columnwidth}{lX}
      \textbf{Kedra:} &  Ciągnę go za nogawkę, by wpadł do środka. Tak, o! Dla żartu.\\
      \textbf{MG:} &  Czujesz pociągnięcie za nogę i poddajesz sie upadkowi. Po gwałtownym lądowaniu otrzymujesz 1 pkt. obrażeń.\\
   \end{tabularx}
\end{rpg-quotebox}

\begin{rpg-quotebox}{Kryptokonwersacja}
   \begin{tabularx}{\columnwidth}{lX}
      \textbf{Randal:} &  Czy mogę z kimś rozmawiać z ukrycia?\\
   \end{tabularx}
\end{rpg-quotebox}

\begin{rpg-quotebox}{O potędze słowa}
   \begin{tabularx}{\columnwidth}{lX}
      \textbf{Randal:} &  Obrzucam go wyzwiskami i kobold otrzymuje 1 pkt. obrażeń.\\
      \textbf{MG:} &  Zwyzywałeś kobolda na śmierć.\\
   \end{tabularx}
\end{rpg-quotebox}

\begin{rpg-quotebox}{O tym jak to bard właściwie czaruje}
   \begin{tabularx}{\columnwidth}{lX}
      \textbf{Lander:} &  Jak bard właściwie czaruje?\\
      \textbf{Randal:} &  Tak moduluje głos, żeby tworzyć magię.\\
      \textbf{MG:} &  Teoretycznie powinien śpiewać, ale nie wiem czy tego chcemy.\\
      \textbf{Ampoldórëon:} &  Rapuj! A-to-sreb-erko-to-ze-skle-pu-wzią-łeś-śmie-ciu?\\
   \end{tabularx}
\end{rpg-quotebox}

\begin{rpg-quotebox}{O tym dlaczego dzieci nie powinno traktować się jak małych dorosłych}
   \textit{Drużyna napotkała grupę koboldzich dzieci i czujną pilnującą ich opiekunkę.}\\
  
   \begin{tabularx}{\columnwidth}{lX}
      \textbf{Ampoldórëon:} &  Wysuwam się na pierwszy front i rzucam Thunderwave.\\
      \multicolumn{2}{l}{\textit{MG nuci ,,And his name is John Cena''.}}\\
      \textbf{MG:} &  Grzmot zaklęcia wyrzuca koboldzie dzieci w powietrze, a ich krzyki i życia nikną w kolejnych nutach melodii. Opiekunka została ledwo draśnięta.\\
   \end{tabularx}
\end{rpg-quotebox}

\section*{25 lutego 2018}

\begin{rpg-quotebox}{W dziczy są}
   \begin{tabularx}{\columnwidth}{lX}
      \textbf{Ampoldórëon:} & Jesteśmy poza Królestwami.\\
      \textbf{Kedra:} & To będzie tu jakaś cywilizacja?\\
      \textbf{Lander:} & My.\\
   \end{tabularx}
\end{rpg-quotebox}

\begin{rpg-quotebox}{Bo do tanga trzeba dwojga}
   \begin{tabularx}{\columnwidth}{lX}
      \textbf{Ampoldórëon:} & W kwestii tworzenia cywilizacji to jest nas 3 do 1, ale wolałbym w to nie wchodzić...\\
      \textbf{Ampoldórëon:} & Okej. To źle zabrzmiało.\\
   \end{tabularx}
\end{rpg-quotebox}

\begin{rpg-quotebox}{Plastik bywa zdradziecki}
   \begin{tabularx}{\columnwidth}{lX}
      \multicolumn{2}{l}{\textit{Do kostki, która Matheusowi wyrzuciła 20.}}\\
      \textbf{Łukasz:} & Ty kurwo, zdrajco!\\
   \end{tabularx}
\end{rpg-quotebox}

\begin{rpg-quotebox}{Czasem w ukryciu\, czasem mniej}
   \textit{Randal podszedł w ukryciu do obozowiska w którym trwało jakieś zamieszanie. Niestety podczas nasłuchiwania odkryto go i ostrzelano.}\\

   \begin{tabularx}{\columnwidth}{lX}
      \textbf{Randal:} & A bard twój, który jest w ukryciu - odda tobie...\\
   \end{tabularx}
\end{rpg-quotebox}

\begin{rpg-quotebox}{Ważny jest punkt widzenia}
   \textit{Drużyna postanowiła podkraść się bliżej centrum wydarzeń.}\\

   \begin{tabularx}{\columnwidth}{lX}
      \multicolumn{2}{l}{\textit{Do Landera:}}\\
      \textbf{MG:} & Ty ukrywasz się z utrudnieniem, bo masz zbroję.\\
      \textbf{MG:} & Podsumowując idziecie ukryci, ale w towarzystwie "klang! klang!" wydawanego przez ekwipunek Landera.\\
      \textbf{Ampoldórëon:} & Ja po prostu nie umiem się ukrywać, ale przy nim to jestem niewidzialny!\\
   \end{tabularx}
\end{rpg-quotebox}

\begin{rpg-quotebox}{O włos od utraty tchu}
   \begin{tabularx}{\columnwidth}{lX}
      \multicolumn{2}{l}{\textit{Do ledwo żywego Ampoldórëona:}}\\
      \textbf{MG:} & Drow strzela w twoim kierunku z kuszy, ale nie trafia.\\
      \textbf{Kedra:} & Twoja nieprzytomność przeleciała ci obok ucha.\\
   \end{tabularx}
\end{rpg-quotebox}

\begin{rpg-quotebox}{Nic tak nie cieszy jak żarty o Twojej Starej}
   \begin{tabularx}{\columnwidth}{lX}
      \textbf{Randal:} & Obrzucam goblina wyzwiskami.\\
      \textbf{Ampoldórëon:} & Co mówisz?\\
      \multicolumn{2}{l}{\textit{Chwila ciszy.}}\\
      \textbf{MG:} & Może: ,,Twoja matka hodowała zgniłe grzyby?''\\
      \textbf{Randal:} & Twoja matka dawała za zgniłe grzyby!\\
   \end{tabularx}
\end{rpg-quotebox}

\begin{rpg-quotebox}{O z góry przyjętych stereotypach}
   \textit{W poprzedniej kampanii Łukasz grał niziołkiem mnichem.}\\

   \begin{tabularx}{\columnwidth}{lX}
      \multicolumn{2}{X}{\textit{O zachowaniu postaci podczas rozmowy z~kobietą pół-orkiem:}}\\
      \textbf{Łukasz:} & Przepraszam, patrzę do góry.\\
      \textbf{Matheus:} & A nie powinieneś się już był do tego przyzwyczaić?\\
   \end{tabularx}
\end{rpg-quotebox}

\begin{rpg-quotebox}{Wspólne gotowanie zawsze łączy}
   \textit{Randal zaproponował kwatermistrzyni wspólne ugotowanie zupy warzywnej aby dać się wszystkim zregenerować po walce.}\\

   \begin{tabularx}{\columnwidth}{lX}
      \multicolumn{2}{l}{\textit{Szeptem do Randala:}}\\
      \textbf{Lander:} & Zrób \emph{grzybową}. Z tymi grzybkami z jaskiń.\\
   \end{tabularx}
   ~\newline~
   \textit{Gdy zupa była gotowa do podania drużyna chciała uniknąć naćpania się nią wraz z magami.}\\

   \begin{tabularx}{\columnwidth}{lX}
      \multicolumn{2}{X}{\textit{Rzucając po kryjomu czar oczyszczający jedzenie z negatywnych składników:}}\\
      \textbf{Lander:} & To ja pomodlę się przed posiłkiem!\\
   \end{tabularx}
\end{rpg-quotebox}

\begin{rpg-quotebox}{O tym\, że nic tak nie pomaga w kontaktach jak doprawienie trunku}
   \textit{Magowie straszliwie się naćpali i doznali licznych halucynacji. Część z nich zaczęła tańczyć na stole, część walczyć z wyimaginowanymi potworami, a jedna bardzo młoda dama zaczęła się dobierać do Randala.}\\

   \begin{tabularx}{\columnwidth}{lX}
      \textbf{Lander:} & Teraz możesz ją porwać!\\
      \textbf{Ampoldórëon:} & Czy ty chcesz ją porywać?!\\
      \textbf{Randal:} & Wstaw tam literę D.\\
      \textbf{MG:} & Taaak, daj d...\\
   \end{tabularx}
\end{rpg-quotebox}

\begin{rpg-quotebox}{O przypadkowych sygnałach}
   \textit{Randal aktywnie flirtował z BN-em. Całkowicie przypadkowo MG w międzyczasie wstała i pomalowała sobie usta pomadką ochronną.}\\

   \begin{tabularx}{\columnwidth}{lX}
      \multicolumn{2}{l}{\textit{W szoku:}}\\
      \textbf{Randal:} & Co się tu właśnie stanęło?\\
      \textbf{Ampoldórëon:} & Ja tam nie wiem co ci stanęło.\\
   \end{tabularx}
\end{rpg-quotebox}

\begin{rpg-quotebox}{O tym gdzie wzrok sięga}
   \textit{Randal postanowił przyjrzeć się ciału goblina}\\

   \begin{tabularx}{\columnwidth}{lX}
      \textbf{MG:} &  Po śladach patrząc...\\
      \textbf{Ampoldórëon:} &  Gdzie ty mu tam patrzysz?!\\
   \end{tabularx}
\end{rpg-quotebox}

\section*{4 marca 2018}

\begin{rpg-quotebox}{O tym\, że to los decyduje o wszystkim}
   \begin{tabularx}{\columnwidth}{lX}
      \textbf{Randal:} & Czy jest szansa na posiłek?\\
      \textbf{Lander:} & Rzuć kostką.\\
   \end{tabularx}
\end{rpg-quotebox}

\begin{rpg-quotebox}{O jakichś mitycznych technikach}
   \begin{tabularx}{\columnwidth}{lX}
      \textbf{MG:} & Dużą część która dotyczy magii jesteś sobie w stanie po chwili przypomnieć.\\
      \textbf{Kedra:} & Masz wszystko w wewnętrznym archiwum.\\
      \textbf{Ampoldórëon:} & Proszę czekać: trwa indeksowanie.\\
   \end{tabularx}
\end{rpg-quotebox}

\begin{rpg-quotebox}{O tym\, że czasem czyny przemawiają lepiej niż słowa}
   \begin{tabularx}{\columnwidth}{lX}
      \textbf{BN:} &  Rozumiem, że potraficie walczyć.\\
      \multicolumn{2}{l}{\textit{Lander ściera mózg wroga ze swojego młota.}}\\
   \end{tabularx}
\end{rpg-quotebox}

\begin{rpg-quotebox}{O tym\, jak czasem nie daje się za wygraną}
\textit{Na propozycję oczyszczenia kopalni z magów Kedra wymyśliła, że można zawalić do niej wejście. Wtedy magowie przestaną zakłócać granicę gnomiego terytorium. W swym pomyśle była jednak sama i burzliwe dyskusje zakończyła decyzja że jednak drużyna oczyści ją własnoręcznie. Pozostawało zakupić odpowiednie zaopatrzenie.}\\

   \begin{tabularx}{\columnwidth}{lX}
      \textbf{Randal:} & Co potrzebujemy do kopalni?\\
      \textbf{Kedra:} & Dynamit!\\
   \end{tabularx}
\end{rpg-quotebox}

\begin{rpg-quotebox}{Bez wilgoci ani rusz}
   \begin{tabularx}{\columnwidth}{lX}
      \textbf{Randal:} & Nadsłuchuję pod drzwiami.\\
      \includegraphics[scale=0.06]{img/d20.png}\textbf{:}& 3\\
      \textbf{MG:} & Coś kapie, więc nie wchodzicie na sucho.\\
   \end{tabularx}
\end{rpg-quotebox}

\begin{rpg-quotebox}{O tym jak bujna wyobraźnia powoduje ból}
   \begin{tabularx}{\columnwidth}{lX}
      \textbf{Randal:} & Rzucam Phantasmal Force - każdy, kto nabierze się na tą iluzję otrzymuje 1k6 obrażeń.\\
      \textbf{Kedra:} & Zadajesz obrażenia od placebo.\\
   \end{tabularx}
\end{rpg-quotebox}

\begin{rpg-quotebox}{To wszystko ma sens}
   \begin{tabularx}{\columnwidth}{lX}
      \textbf{MG:} & Zbroja typu ,,hide'' nie ma utrudnienia do skradania się, \emph{because it's made of hide}.\\
   \end{tabularx}
\end{rpg-quotebox}

\begin{rpg-quotebox}{Krótko i na temat}
   \begin{tabularx}{\columnwidth}{lX}
      \textbf{Lander:} & Jak wygląda sytuacja?\\
      \textbf{MG:} & Jest korytarz, który rozwidla się na końcu, ale wcześniej wypełniony jest Goblinozą\\
   \end{tabularx}
\end{rpg-quotebox}

\begin{rpg-quotebox}{O bojowej wersji baseballa}
   \begin{tabularx}{\columnwidth}{lX}
      \textbf{Ampoldórëon:} & Wszyscy którzy nie przeszli testu zostają odrzuceni przez efekt Thunderwave.\\
      \textbf{Randal:} & Czyli lecą w kierunku Landera, który jest na ich tyłach.\\
      \multicolumn{2}{l}{\textit{Do Landera:}}\\
      \textbf{MG:} & Tak, masz na jednego z lecących atak okazjonalny.\\
      \includegraphics[scale=0.06]{img/d20.png}\textbf{:}& 18\\
      \textbf{MG:} & Trafiasz jednego w locie i odcinasz mu głowę.\\
   \end{tabularx}
\end{rpg-quotebox}

\begin{rpg-quotebox}{O priorytetach}
   \textit{Drużyna szukała miejsca na odpoczynek i znalazła komnatę na jednym z końców korytarza.}\\

   \begin{tabularx}{\columnwidth}{lX}
      \textbf{Ampoldórëon:} & Stawiam alarm na rozwidleniu.\\
      \textbf{MG:} & Gdy podchodzisz do skrzyżowania kilka metrów dalej za nim widzisz magiczną barierę.\\
      \textbf{Ampoldórëon:} & Dobra - to potem, magiczna bariera nie spierdoli.\\
   \end{tabularx}
\end{rpg-quotebox}


\section*{15 kwietnia 2018}


\begin{rpg-quotebox}{Zawsze warto spróbować}
   \begin{tabularx}{\columnwidth}{lX}
      \textbf{MG:} & Obserwujecie jak Randal penetruje barierę kijem od pochodni.\\
   \end{tabularx}
\end{rpg-quotebox}


\begin{rpg-quotebox}{Może jeszcze telemark?}
\textit{Faceci skoczyli jeden za drugim w magiczną barierę i~poobijali siebie nawzajem podczas lądowania. Randal dał znać Kedrze w odpowiednim odstępie czasu kiedy i ona może dołączyć do skoku.}\\

   \begin{tabularx}{\columnwidth}{lX}
      \multicolumn{2}{l}{\textit{Po zgrabnym lądowaniu:}}\\
      \textbf{Kedra:} & SPUŚCIĆ facetów z oczu...\\
   \end{tabularx}
\end{rpg-quotebox}


\begin{rpg-quotebox}{Było i nie ma}
   \begin{tabularx}{\columnwidth}{lX}
      \textbf{MG:} & Czy wy pamiętacie, że macie blessa?\\
      \multicolumn{2}{l}{\textit{Patrząc na nieprzytomnego Landera:}}\\
      \textbf{Kedra:} & Nasz bless padł.\\
   \end{tabularx}
\end{rpg-quotebox}


\begin{rpg-quotebox}{Gdyby można było być kim się chce}
   \begin{tabularx}{\columnwidth}{lX}
      \textbf{Randal:} & Czy mogę jako akcja bonusową ukryć się podczas walki?\\
      \textbf{MG:} & Nie jesteś łotrzykiem.\\
      \textbf{Kedra:} & On BARDzo chciałby być.\\
   \end{tabularx}
\end{rpg-quotebox}


\begin{rpg-quotebox}{Prawdziwych przyjaciół poznaje się w biedzie}
   \begin{tabularx}{\columnwidth}{lX}
      \textbf{MG:} & W pokoju znajdujecie otwarte ciało i drzwi do~następnego pomieszczenia.\\
   \end{tabularx}
   ~\newline
   \textit{Drużyna zawołała niezbyt dyskretnego Landera, aby dołączył do pokoju w którym się znajdują.}\\

   \begin{tabularx}{\columnwidth}{lX}
      \textbf{MG:} & Słyszycie głosy zza drzwi.\\
      \textbf{Kedra:} & Ukrywam się pod stołem!\\
      \textbf{Randal:} & Chowam się w cieniu.\\
      \textbf{Ampoldórëon:} & Chowam się za drzwiami.\\
   \end{tabularx}
   ~\newline
   \textit{Zaalarmowani wrogowie wpadli do pokoju i zobaczyli tylko Landera.}\\
\end{rpg-quotebox}


\begin{rpg-quotebox}{Bańka-Wstańka}
   \textit{Drużyna walczyła z wrogimi magami i jeden z nich użył zaklęcia Shatter, co w efekcie doprowadziło Randala do nieprzytomności. Lander go uleczył i Randal wrócił do bitwy, tylko po to by oberwać kolejnym Shatter'em od golemopodobnego potwora. Tym razem ocuciła go Kedra wyczarowaną przez siebie fiolką uzdrawiającej mikstury. Następnie Ampoldórëon odkrył, że golemopodobny potwór jest podatny na to samo zaklęcie i postanowił tę wiedzę wykorzystać w praktyce. Ponieważ czar jest obszarowy, a Randal był w zasięgu - potwór się wywinął, a Randal... padł.}\\
\end{rpg-quotebox}


\begin{rpg-quotebox}{Widoki na walkę o życie}
   \begin{tabularx}{\columnwidth}{lX}
      \textbf{Lander:} & Czy topiąc się coś widzę?\\
   \end{tabularx}
\end{rpg-quotebox}


\begin{rpg-quotebox}{\text{Członki, kończyny - toż to synonimy.}}
   \begin{tabularx}{\columnwidth}{lX}
      \textbf{MG:} & Zauważacie, że Randal odzyskuje władzę w kończynach.\\
      \textbf{Kedra:} & Zastanawia mnie czy w penisie też odzyskał władzę.\\
      \textbf{Ampoldórëon:} & Możesz sprawdzić, ja wyjdę.\\
      \textbf{Kedra:} & Smyram.\\
      \textbf{Randal:} & Z trudem, bo z trudem, pokazuję jej środkowy palec.\\
   \end{tabularx}
\end{rpg-quotebox}


\begin{rpg-quotebox}{Pieczęć żałości}
   \begin{tabularx}{\columnwidth}{lX}
      \textbf{Lander:} & Walę młotem w runy kręgu w celu zniszczenia go.\\
      \textbf{MG:} & Po uderzeniu zauważasz na jednej z run maleńkie pęknięcie, ale twój młot też się odrobinę uszkodził.\\
      \textbf{Kedra:} & Teraz możesz przybijać wrogom stemple.\\
      \textbf{Ampoldórëon:} & Tak! W kształcie litery L jak Lamer.\\
   \end{tabularx}
\end{rpg-quotebox}


\begin{rpg-quotebox}{Duma z nowego nabytku}
   \begin{tabularx}{\columnwidth}{lX}
\textbf{Ampoldórëon:} & A gdzie Randal?\\
\textbf{Kedra:} & Maca swój flet.\\
\textbf{Ampoldórëon:} & ...\\
   \end{tabularx}
\end{rpg-quotebox}


\begin{rpg-quotebox}{Kara za łamanie zasady 5 sekund}
   \begin{tabularx}{\columnwidth}{lX}
      \textbf{Randal:} & Próbuję zagrać na znalezionym flecie.\\
      \textbf{MG:} & Otrzymujesz 5 pkt. obrażeń.\\
      \textbf{Ampoldórëon:} & Mama Ci nie mówiła, żeby nie brać wszystkiego do ust?\\
   \end{tabularx}
\end{rpg-quotebox}


\section*{12 maja 2018}


\begin{rpg-quotebox}{Śliska zabawa słowem}
   \begin{tabularx}{\columnwidth}{lX}
      \textit{O szlamie:}\\
      \textbf{Randal:} & Mógłbym go zwyzywać.\\
      \textbf{Kedra:} & Powiedz mu że jego matka była szmatą.\\
   \end{tabularx}
\end{rpg-quotebox}


\begin{rpg-quotebox}{Naigrywanki z grajka}
   \begin{tabularx}{\columnwidth}{lX}
      \textbf{Randal:} & Idę do Landera i gram na flecie.\\
      \textbf{Kedra:} & Na jego flecie?\\
   \end{tabularx}
\end{rpg-quotebox}


\begin{rpg-quotebox}{O istocie rzeczy}
   \begin{tabularx}{\columnwidth}{lX}
      \textbf{Randal:} & Czy widzimy co ta pułapka robiła?\\
      \textbf{Ampoldórëon:} & Myślę, że pułapkowała.\\
   \end{tabularx}
\end{rpg-quotebox}


\begin{rpg-quotebox}{Wątpliwy wyrób wędliniarski}
   \begin{tabularx}{\columnwidth}{lX}
      \textbf{MG:} & Elf głębinowy podaje ci kiełbaskę.\\
      \textbf{Randal:} & Co to jest?\\
      \textbf{MG:} & Elf w milczeniu wkłada ją do ogniska.\\
      \textbf{Randal:} & Robi mi się smutno na myśl o tym zwierzęciu, zwłaszcza że w tych warunkach polowych raczej nie mają czasu na robienie kiełbasek\\
   \end{tabularx}
\end{rpg-quotebox}


\begin{rpg-quotebox}{Wynglansowany na błysk}
   \begin{tabularx}{\columnwidth}{lX}
      \textbf{Lander:} & Wychodzę z namiotu w idealnie wypolerowanej zbroi.\\
      \textbf{Kedra:} & \multirow{3}{*}{Ała! nie po oczach!}\\
      \textbf{Randal:} & \\
      \textbf{Ampoldórëon:} & \\
      \textbf{Lander:} & Ej, nie jestem pieprzoną kulą disco!\\
   \end{tabularx}
\end{rpg-quotebox}


\begin{rpg-quotebox}{Kopa ma ta kusza}
   \begin{tabularx}{\columnwidth}{lX}
      \textbf{Kupiec:} & Czy interesuje was zwykła porządnie wykonana kusza, czy coś z lekkim kopem?\\
      \textbf{Randal:} & Hmmm, kusząca ta kusza.\\
   \end{tabularx}
\end{rpg-quotebox}


\begin{rpg-quotebox}{Złotouste negocjacje}
   \begin{tabularx}{\columnwidth}{lX}
      \textbf{Lander:} & Oddam moją poprzednią zbroję za zniżkę.\\
      \textbf{Randal:} & Spójrz Pani, jak olśniewająco błyszczy. Polerował ją całą noc nawet przez sen.\\
      \textbf{Kedra:} & Tak, wtedy się o nią ocierał...\\
      \textbf{MG:} & Krasnoludzica zasypana kwiecistymi zapewnieniami Randala sprzedaje ci zbroję z mieszanką zdegustowania i zachwytu.\\
   \end{tabularx}
\end{rpg-quotebox}


\section*{16 czerwca 2018}


\begin{rpg-quotebox}{O poprawnym doborze słów}
   \textit{Drużyna w drodze do Podmroku decydowała o następnym kroku. Do wyboru był wypoczynek w karczmie i odwiedzenie biblioteki, która była najbliżej do niebezpiecznego tunelu.}\\

   \begin{tabularx}{\columnwidth}{lX}
      \textbf{Kedra:} & Idźmy teraz do biblioteki zanim potwory zmiotą bibliotekę z... pod powierzchni ziemii\\
   \end{tabularx}
\end{rpg-quotebox}


\begin{rpg-quotebox}{O tęsknocie za światłem}
   \begin{tabularx}{\columnwidth}{lX}
      \textbf{MG:} & Dotarło do was, że przez następne kilka tygodni nie zobaczycie słońca.\\
      \textbf{Kedra:} & Jaki jest teraz czas w mieście? Skończyła się noc... Ech nie nauczę się... Nocna warta?\\
   \end{tabularx}
\end{rpg-quotebox}


\begin{rpg-quotebox}{Atak Aspergera}
   \textit{Spotkanie i rozmowa z BN-em w zagraconej bibliotece.}\\

   \begin{tabularx}{\columnwidth}{lX}
      \textbf{MG:} & Generalnie nie wiecie jak on się nazywa, nie przedstawiliście się.\\
      \textbf{Ampoldórëon:} & To trochę niegrzecznie odwiedzać kogoś w jego domu i potem się pytać o jego imię.\\
      \textbf{Randal:} & Dzień dobry, pan tak mieszka? Co to za burdel?\\
   \end{tabularx}
\end{rpg-quotebox}


\begin{rpg-quotebox}{Ta maść chyba nie pomoże}
   \textit{Randal zrywał trujące rośliny i poparzył sobie palce, przez co nie mógł grać na flecie.}\\

   \begin{tabularx}{\columnwidth}{lX}
      \textbf{Randal:} & Może mi ten krasnolud czymś je posmaruje...\\
      \textbf{MG:} & Masz kleryka obok siebie.\\
      \multicolumn{2}{l}{\textit{Do Landera:}}\\
      \textbf{Randal:} & Posmarujesz mi paluszki?\\
      \textbf{Lander:} & Czym?\\
      \textbf{Kedra:} & Miłością.\\
   \end{tabularx}
\end{rpg-quotebox}


\begin{rpg-quotebox}{O niepotrzebnym rozgłosie}
   \textit{Za każdym razem gdy drużyna ogarnęła przedstawianie się Kedra dodawała przydomek Landerowi. Drużyna siedziała w karczmie i przedstawiała się gościom. Po tym jak przedstawił się Lander:}\\

   \begin{tabularx}{\columnwidth}{lX}
      \textbf{Kedra:} & ... Stormwind, pogromca Umbrowych kolosów.\\
      \multicolumn{2}{l}{\textit{Ampoldórëon okazał zdegustowanie}}\\
      \multicolumn{2}{l}{\textit{Z wściekłością:}}\\
      \textbf{Lander:} & Zaraz będę pogromcą Kedry.\\
   \end{tabularx}
\end{rpg-quotebox}


\begin{rpg-quotebox}{Magia atletyki}
   \textit{Ampoldórëon próbował otworzyć kamienne drzwi do pokoju. Niestety jako mag nie był zbyt umięśniony.}\\

   \begin{tabularx}{\columnwidth}{lX}
      \textbf{MG:} & Rzuć na atletykę.\\
      \includegraphics[scale=0.055]{img/d20.png}\textbf{:}& 1.\\
      \textbf{MG:} & Teraz rzucaj na astmę.\\
   \end{tabularx}
\end{rpg-quotebox}


\begin{rpg-quotebox}{Nieczyste zagrywki}
   \textit{Ampoldórëon nadal próbował otworzyć kamienne drzwi tym razem za pomocą rytuału.}\\

   \begin{tabularx}{\columnwidth}{lX}
      \includegraphics[scale=0.055]{img/d20.png}\textbf{:}& 16.\\
      \textbf{MG:} & Przywołałeś niewidzialną siłę, która pomogła ci otworzyć drzwi. Weź 50 punktów doświadczenia.\\
   \end{tabularx}

   \textit{Zza winkla przyglądał się wszystkiemu zszokowany Lander, w końcu nie często ktoś rysuje pentagramy w korytarzu.}\\
\end{rpg-quotebox}


\begin{rpg-quotebox}{Taka tam gra słowna}
   \textit{Drużyna spotkała nieumarłego Beholdera w łaźni w karczmie. Beholder rzucił się do ataku na Landera.}\\

   \begin{tabularx}{\columnwidth}{lX}
      \textbf{Ampoldórëon:} & Uuu chłopie, zginiesz marnie.\\
      \textbf{MG:} & Będąc ścisłym - zginiesz w wannie.\\
   \end{tabularx}
\end{rpg-quotebox}


\begin{rpg-quotebox}{Pomysł było dobry}
   \textit{Randalowi udało się przestraszyć Beholdera i ten zaczął uciekać.}\\

   \begin{tabularx}{\columnwidth}{lX}
      \textbf{Kedra:} & Rzuć mu to mydło które mi wcześniej oferowałeś by się poślizgnął.\\
      \textbf{MG:} & Beholdery lewitują.\\
   \end{tabularx}
\end{rpg-quotebox}

\begin{rpg-quotebox}{Najważniejsze to myśleć pozytywnie}
   \textit{Skończyła się playlista na youtubie w tle i rozgrywka odbywała się w ciszy, w tym momencie Beholder zabił Landera.}\\

   \begin{tabularx}{\columnwidth}{lX}
      \textbf{MG:} & Może jakaś muzyczka?\\
   \end{tabularx}

\textit{Kedra i Ampoldórëon zaczynają nucić marsz pogrzebowy.}\\
\end{rpg-quotebox}


\begin{rpg-quotebox}{Dungeons and DC}
   \begin{tabularx}{\columnwidth}{lX}
      \textbf{Lander:} & Czy tylko ja mam nazwisko?\\
      \textbf{Ampoldórëon:} & Ja się przedstawiałem.\\
      \multicolumn{2}{l}{\textit{Głosem filmowego Batmana:}}\\
      \textbf{Randal:} & I'm Backman.\\
   \end{tabularx}
\end{rpg-quotebox}


\section*{7 lipca 2018}


\begin{rpg-quotebox}{Ciekawe zgranie wydarzeń}
   \textit{Oglądaliśmy mecz Chorwacja-Rosja w trakcie sesji. Ampoldórëon rzucił Shattera w drzwi karczmy, by je wyważyć i zaskoczyć czyhających na zewnątrz wrogów. W momencie trafienia  w drzwi Chorwacja wbiła bramkę Rosji, a cała drużyna zakrzyknęła chórem ,,Goool!''}\\
\end{rpg-quotebox}


\begin{rpg-quotebox}{\text{Nie magia, a też szkodzi na zdrowie}}
   \begin{tabularx}{\columnwidth}{lX}
      \textbf{MG:} & Duergary są odporne na zaklęcia i ich efekty.\\
      \multicolumn{2}{l}{\textit{Wskazując na powalonych wrogów:}}\\
      \textbf{Lander:} & Jakoś tego po nich nie widać.\\
      \textbf{MG:} & One są odporne na magię, nie na cegły.\\
   \end{tabularx}
\end{rpg-quotebox}


\begin{rpg-quotebox}{O tym jak nadmiar higieny może być ryzykowny}
   \textit{Drużyna rozmawiała z kapitanem straży o stanie pola walki i wspomniała m.in. o okropnościach jakie napotkali w łaźni karczmy.}\\

   \begin{tabularx}{\columnwidth}{lX}
      \textbf{BN:} & Tunel, który się otworzył wydaje się szerszy niż ostatnim razem.\\
      \textbf{Randal:} & Na tyle, by zmieścił się w nim beholder?\\
      \textbf{BN:} & Nie wnikam.\\
      \multicolumn{2}{l}{\textit{Kaszląc:}}\\
      \textbf{Randal:} & Łaźnia.\\
   \end{tabularx}
\end{rpg-quotebox}


\begin{rpg-quotebox}{I kolejne gierki słowne}
   \begin{tabularx}{\columnwidth}{lX}
      \textbf{Ampoldórëon:} & Celuję w tego oświetlonego i zhexowanego.\\
      \multicolumn{2}{l}{\textit{Mrucząc:}}\\
      \textbf{Kedra:} & On jest bardzo hexowny....\\
   \end{tabularx}
\end{rpg-quotebox}


\begin{rpg-quotebox}{\text{Lepszy rydz niż wszystko?}}
   \textit{Kedra potrafi przenosić klątwę z wroga na wroga w trakcie akcji bonusowej.}\\

   \begin{tabularx}{\columnwidth}{lX}
      \multicolumn{2}{l}{\textit{Z uznaniem:}}\\
      \textbf{Ampoldórëon:} & Jebana...\\
      \textbf{Kedra:} & Połowiczny mag, ale skuteczny.\\
      \textbf{Ampoldórëon:} & Nie to co ja - w pełni mag i chuj.\\
   \end{tabularx}
\end{rpg-quotebox}


\begin{rpg-quotebox}{Ech te wszędobylskie reklamy}
   \textit{Klątwa Kedry obniżyła zdolności wroga w zakresie rzutów obronnych na mądrość.}\\

   \begin{tabularx}{\columnwidth}{lX}
      \textbf{Randal:} & Okej, to ten grzyb który świeci i który ma problem z....\\
      \textbf{MG:} & Nietrzymaniem moczu!\\
      \textbf{Randal:} & ...z zachowaniem zdrowgo rozsądku...\\
   \end{tabularx}
\end{rpg-quotebox}


\begin{rpg-quotebox}{\text{I rzy(g)li sobie długo i szczęśliwie}}
   \textit{Otoczona smrodem drużyna raz, po razie zwracała pod siebie zawartość swoich żołądków. Lander miał (nie)przyjemność stracić przytomność podczas walki i paść w morze rzygowin.}\\
\end{rpg-quotebox}


\begin{rpg-quotebox}{}
   \textit{Walka z grzybami zaczęła się od tego, że Ampoldórëon wypuścił w stronę ściany firebolta. Rozwścieczone Mykonidy zaatakowały w odwecie drużynę, która jakimś cudem wyszła z walki zwycięsko.}\\

   \begin{tabularx}{\columnwidth}{lX}
      \textbf{Ampoldórëon:} & Proponuję wypalić z grzybów cały ten tunel.\\
      \multicolumn{2}{l}{\textit{Udając naćpanego:}}\\
      \textbf{Randal:} & Mmmm dobry towar.\\
      \textbf{Kedra:} & Czy my mamy tyle racji by zajeść tą gigantyczną gastrofazę?\\
      \textbf{Ampoldórëon:} & Dym może dolecieć aż do miasta krasnoludów jak dobrze pójdzie.\\
      \textbf{Randal:} & Tak! Zrobimy z tego tunelu gigantyczne bongo!\\
   \end{tabularx}
\end{rpg-quotebox}


\begin{rpg-quotebox}{Ech te wzdęcia}
   \begin{tabularx}{\columnwidth}{lX}
      \textbf{MG:} & W pewnym momencie czujecie okropny smród pleśni.\\
      \textbf{Ampoldórëon:} & Staram się zidentyfikować źródło smrodu.\\
      \textbf{Randal:} & To ja i moje nieświeże racje, sorry.\\
      \includegraphics[scale=0.055]{img/d20.png}\textbf{:}& 4.\\
      \textbf{Ampoldórëon:} & Ok. Fakt. To Randal pierdzi.\\
   \end{tabularx}
\end{rpg-quotebox}


\begin{rpg-quotebox}{O nowym znaczeniu powiedzenia.}
   \textit{Drużyna zobaczyła dziwną kurtynę z grzybów, która wydawała z siebie ten obrzydliwy smród. Gdy Lander próbował ją odsłonić nie wytrzymał i zwymiotował.}\\

   \begin{tabularx}{\columnwidth}{lX}
      \textbf{Ampoldórëon:} & Zawsze możemy to olać i wrócić złożyć raport.\\
      \textbf{MG:} & Tak. W tunelu jest chujnia z grzybnią.\\
   \end{tabularx}
\end{rpg-quotebox}


\begin{rpg-quotebox}{Spostrzegawczość i zbyt dużo informacji.}
   \begin{tabularx}{\columnwidth}{lX}
      \textbf{MG:} & Lander, smród jest tak gryzący, że wymiotujesz drużynie pod nogi.\\
      \textbf{Randal:} & O! Lander kiedy jadłeś spaghetti?\\
      \textbf{Lander:} & To tasiemiec.\\
   \end{tabularx}
\end{rpg-quotebox}


\begin{rpg-quotebox}{Zalety bycia na czczo.}
   \textit{Grzyb ponownie zaatakował smrodem i znów zmusił drużynę do zwrotów z żołądków. Test na wytrzymałość zdał tylko Lander.}\\

   \begin{tabularx}{\columnwidth}{lX}
      \textbf{Randal:} & Lander już ma pusty żołądek.\\
   \end{tabularx}
\end{rpg-quotebox}


\begin{rpg-quotebox}{O nowych liczbach}
   \begin{tabularx}{\columnwidth}{lX}
      \textbf{Lander:} & Leczę się za...\\
      \multicolumn{2}{l}{\textit{Rzuca kością.}}\\
      \textbf{Lander:} & ... za kurwa mać.\\
   \end{tabularx}
\end{rpg-quotebox}


\begin{rpg-quotebox}{Ach te eufemizmy}
   \textit{Komentarz na temat atakowania dużego Mykonida atakującego z pięści.}\\

   \begin{tabularx}{\columnwidth}{lX}
      \textbf{Lander:} & To ja atakuję śmierdziuszka.\\
   \end{tabularx}
\end{rpg-quotebox}


\begin{rpg-quotebox}{Rozpacz pełna rozbłysków}
   \begin{tabularx}{\columnwidth}{lX}
      \multicolumn{2}{l}{\textit{Do Ampoldórëona, który nie trafił swoim czarem:}}\\
      \textbf{MG:} & Robisz coś jeszcze?\\
      \textbf{Ampoldórëon:} & Płaczę nad swoim losem.\\
   \end{tabularx}
\end{rpg-quotebox}


\begin{rpg-quotebox}{Zaiste niezbędne spostrzeżenie}
   \begin{tabularx}{\columnwidth}{lX}
      \textbf{MG:} & Grzyb zaatakował Randala za 9 punktów obrażeń.\\
      \textbf{Randal:} & Poprzestawiał mi cyferki! Miałem 21, a mam 12 punktów życia.\\
   \end{tabularx}
\end{rpg-quotebox}


\section*{14 sierpnia 2018}


\begin{rpg-quotebox}{Cenne spostrzeżenie}
   \begin{tabularx}{\columnwidth}{lX}
      \textbf{MG:} & Jesteście w Nowym Eidsvik.\\
      \textbf{Ampoldórëon:} & Aaa, tym zadupiu.\\
      \textbf{Kedra:} & Raczej poddupiu, bo to pod ziemią.\\
   \end{tabularx}
\end{rpg-quotebox}


\begin{rpg-quotebox}{O niezbyt dobrym zapatrywaniu się na przyszłość}
   \begin{tabularx}{\columnwidth}{lX}
      \textbf{MG:} & Piwo miło cię połechtało. Otrzymujesz 1 tymczasowy punkt życia.\\
      \textbf{Randal:} & Super! Przyda się do późniejszego lania się po mordach.\\
   \end{tabularx}
\end{rpg-quotebox}


\begin{rpg-quotebox}{Uwaga o krok za daleko}
   \begin{tabularx}{\columnwidth}{lX}
      \textbf{MG:} & Pijąc czujesz, że zaczynasz delikatnie unosić się nad ziemią.\\
      \textbf{Kedra:} & Wow, można z tym wygrać wszystkie konkursy taneczne!\\
      \textbf{Randal:} & Oprócz stepowania.\\
   \end{tabularx}
\end{rpg-quotebox}


\begin{rpg-quotebox}{Grajek spadł jak z nieba}
   \begin{tabularx}{\columnwidth}{lX}
      \textbf{MG:} & Słyszycie, że muzyka nagle się urwała, bo jeden z gości znokautował orkiestrę.\\
      \textbf{Randal:} & Należałoby ich zastąpić!\\
      \textbf{Alchemik:} & W stanie lekkiej nieważkości nie za bardzo możesz się ruszać. Jesteście na antresoli, a orkiestra jest pod wami.\\
      \textbf{Ampoldórëon:} & Traktuję Randala Magiczną Dłonią.\\
      \textbf{Kedra:} & Super! Wyrzuć go na dół jak worek kartofli!\\
      \textbf{Randal:} & Świetnie! Przeskakuję nad balustradą i niczym istota z niebios oferuję im muzykę swojego fletu!\\
      \textbf{MG:} & Rzuć na Performance.\\
      \includegraphics[scale=0.055]{img/d20.png}\textbf{(Randal):}& 21.\\
   \end{tabularx}
\end{rpg-quotebox}


\begin{rpg-quotebox}{Z dwojga złego...}
   \begin{tabularx}{\columnwidth}{lX}
      \multicolumn{2}{l}{\textit{Do Landera:}}\\
      \textbf{Alchemik:} & Ten trunek cię rozświetli, proszę wybierz kolor.\\
      \textbf{Randal:} & Czarny! Czarny się nie świeci!\\
      \textbf{Lander:} & Niech będzie czerwony...\\
   \end{tabularx}
\end{rpg-quotebox}


\begin{rpg-quotebox}{Głodnemu chleb na myśli}
   \begin{tabularx}{\columnwidth}{lX}
      \multicolumn{2}{l}{\textit{Do Landera:}}\\
      \textbf{Alchemik:} & Po wypiciu świecisz się cały na czerwono. Jesteś jakby fluoresencyjny.\\
      \textbf{Kedra:} & Czy twój Wacek też się świeci?\\
      \textbf{Lander:} & Co?!\\
      \textbf{Kedra:} & Czy jak cię TAM pogłaszczę, to wskażesz punkt?\\
      \multicolumn{2}{l}{\textit{Lander spogląda z żądzą mordu w oczach.}}\\
   \end{tabularx}
\end{rpg-quotebox}


\begin{rpg-quotebox}{\text{Tym bardziej, chleb na myśli}}
   \begin{tabularx}{\columnwidth}{lX}
      \multicolumn{2}{l}{\textit{Do Kedry na temat zboczonych żartów:}}\\
      \textbf{Alchemik:} & Ty czegoś potrzebujesz...\\
      \multicolumn{2}{l}{\textit{Pod nosem:}}\\
      \textbf{Kedra:} & Mam ochotę na loda.\\
      \multicolumn{2}{l}{\textit{Alchemik podaje Kedrze płyn o białawym zabarwieniu.}}\\
      \multicolumn{2}{l}{\textit{Ranadal parska śmiechem.}}\\
   \end{tabularx}
\end{rpg-quotebox}


\begin{rpg-quotebox}{Podobno antysemityzm wyssany z mlekiem matki}
   \begin{tabularx}{\columnwidth}{lX}
      \textbf{MG:} & Randal patrzy na mydło i wspomina swoją matkę.\\
      \multicolumn{2}{l}{\textit{Żarty o Żydach za 3... 2... 1...}}\\
   \end{tabularx}
\end{rpg-quotebox}


\begin{rpg-quotebox}{Najwyższa forma koncentracji}
   \begin{tabularx}{\columnwidth}{lX}
      \multicolumn{2}{l}{\textit{O planowanym użyciu czaru Niewidzialność:}}\\
      \textbf{Lander:} & To jest na koncentrację, więc musicie spinać poślady.\\
   \end{tabularx}
\end{rpg-quotebox}


\begin{rpg-quotebox}{Tłumaczenie na dresowski}
   \begin{tabularx}{\columnwidth}{lX}
      \textbf{MG:} & Alchemik podarował wam granat wzmacniający efekty rzucanych czarów. W skrócie taka chmurka wpierdolu.\\
   \end{tabularx}
\end{rpg-quotebox}


\begin{rpg-quotebox}{}
   \begin{tabularx}{\columnwidth}{lX}
      \multicolumn{2}{l}{\textit{Do Landera:}}\\
      \textbf{MG:} & Rzuć na percepcję.\\
      \includegraphics[scale=0.055]{img/d20.png}\textbf{(Lander):}& 23.\\
      \textbf{Kedra:} & All is cleric!\\
   \end{tabularx}
\end{rpg-quotebox}


\begin{rpg-quotebox}{Macane należy do macanta}
   \begin{tabularx}{\columnwidth}{lX}
      \multicolumn{2}{l}{\textit{Do Landera:}}\\
      \textbf{MG:} & Macasz ścianę.\\
      \textbf{Randal:} & Ściana odmacuje.\\
   \end{tabularx}
\end{rpg-quotebox}


\begin{rpg-quotebox}{Taka tam gra słów.}
   \begin{tabularx}{\columnwidth}{lX}
      \textbf{MG:} & Wsypujący się piasek sięga wam już do ud, więc wszelki ruch jest utrudniony.\\
      \multicolumn{2}{l}{\textit{Drużyna ma zrezygnowane miny.}}\\
      \textbf{MG:} & Żywiołak jest w swoim żywiole.\\
   \end{tabularx}
\end{rpg-quotebox}


\begin{rpg-quotebox}{Mowa ciała nie wystarcza}
   \begin{tabularx}{\columnwidth}{lX}
      \textbf{Randal:} & Używam czaru Sugestia i sugeruję żywiołakowi, by zostawił nas w spokoju i otwarł nam drzwi.\\
      \textbf{MG:} & Żywiołak nie rozumie języka w którym mówisz.\\
      \textbf{Randal:} & Ale pokazałem mu drzwi!\\
   \end{tabularx}
\end{rpg-quotebox}


\begin{rpg-quotebox}{Czas sypie}
   \begin{tabularx}{\columnwidth}{lX}
      \textbf{Randal:} & Czuję się jak w klepsydrze - coś mi się sypie na głowę i coś usypuje spod nóg.\\
      \multicolumn{2}{l}{\textit{Wskazując na piasek sięgający pasa:}}\\
      \textbf{Kedra:} & Tak i czas ci się kończy!\\
   \end{tabularx}
\end{rpg-quotebox}


\section*{28 września 2018}


\begin{rpg-quotebox}{Jasno nakreślone priorytety}
   \begin{tabularx}{\columnwidth}{lX}
      \multicolumn{2}{l}{\textit{Do Randala:}}\\
      \textbf{Ampoldórëon:} & Gdybyś ty zginął albo, co gorsza, JA bym zginął...\\
   \end{tabularx}
\end{rpg-quotebox}


\begin{rpg-quotebox}{Taki tam lapsusik}
   \begin{tabularx}{\columnwidth}{lX}
      \textbf{MG:} & Piasek był w komnacie, w drzwiach i w was. Macie drzwi w butach, macie drzwi w bieliźnie...\\
   \end{tabularx}
\end{rpg-quotebox}


\begin{rpg-quotebox}{Gracze nie zapominają}
   \begin{tabularx}{\columnwidth}{lX}
      \textbf{Randal:} & Przyglądam się gruzowisku.\\
      \includegraphics[scale=0.055]{img/d20.png}\textbf{(Lander):}& 1.\\
      \textbf{MG:} & Masz w oczach piach.\\
      \textbf{Kedra:} & Gdybyś miał w oczach drzwi to byś widział.\\
   \end{tabularx}
\end{rpg-quotebox}


\begin{rpg-quotebox}{Jak to jest jak się gra z nerdami}
   \textit{Matheus dyskutował z Kingą na temat szkół zaklęć, z których wywodzą się np. zaklęcia leczące.}\\

   \begin{tabularx}{\columnwidth}{lX}
      \textbf{Kedra:} & No masz, on naprawdę studiował magię....\\
   \end{tabularx}
\end{rpg-quotebox}


\begin{rpg-quotebox}{Człowiek prądowy}
   \begin{tabularx}{\columnwidth}{lX}
      \multicolumn{2}{l}{\textit{Do MG:}}\\
      \textbf{Ampoldórëon:} & Od dziś jestem Amper.\\
      \textbf{Kedra:} & Stawiasz opór?\\
      \textbf{Ampoldórëon:} & Nie! Rzucam piorunami.\\
   \end{tabularx}
\end{rpg-quotebox}


\begin{rpg-quotebox}{Żarty z prądu}
   \textit{Komentarz na temat pięciu piorunów kulistych krążących wokół głowy Ampoldórëona.}\\

   \begin{tabularx}{\columnwidth}{lX}
      \textbf{Ampoldórëon:} & Poznaj moich przyjaciół: Amper, Volt, Farad, AC i DC.\\
   \end{tabularx}
\end{rpg-quotebox}


\begin{rpg-quotebox}{\text{Śmiech to zdrowie, ale nie zawsze}}
   \textit{Randal użył zaklęcia "Tasha's Hideous Laughter" które unieruchomiło ciężkiego przeciwnika i nie pozwoliło mu nic robić przez 6 tur. W tym czasie drużyna podleczyła nadszarpnięte zdrowie i przygotowała się do dalszej bitwy. Zanim jednak nadeszła jego kolej otrzymał cztery bełty i cios młota bojowego po czym padł martwy.}\\

   \begin{tabularx}{\columnwidth}{lX}
      \textbf{MG:} & Nienawidzę bardów.\\
   \end{tabularx}
\end{rpg-quotebox}


\begin{rpg-quotebox}{Cierpienia młodego Mistra Gry architekta}
   \begin{tabularx}{\columnwidth}{lX}
      \textbf{MG:} & Widzicie portal.\\
      \multicolumn{2}{l}{\textit{Po raz 58-y:}}\\
      \textbf{Ampoldórëon:} & Ale portal portal czy portal?\\
      \multicolumn{2}{l}{\textit{Z irytacją:}}\\
      \textbf{MG:} & prosty otwór drzwiowy!\\
   \end{tabularx}
\end{rpg-quotebox}


\begin{rpg-quotebox}{Skoro zwierzęta dają radę}
   \begin{tabularx}{\columnwidth}{lX}
\textbf{Randal:} & Czy tam jest bepiecznie?\\
\textbf{Ampoldórëon:} & Owa (ang. owl) przeżyła.\\
   \end{tabularx}
\end{rpg-quotebox}


\begin{rpg-quotebox}{I koniec wygód}
   \textit{Lander opatulił dziwny kamień swoim śpiworem i w ten sposób chciał zdjąć go z posągu.}\\

   \begin{tabularx}{\columnwidth}{lX}
      \textbf{MG:} & Po chwili twój śpiwór staje w płomieniach, a szmaragd upada na podłogę.\\
      \textbf{Drużyna:} & O nie! I w czym teraz będziesz spał?\\
   \end{tabularx}
\end{rpg-quotebox}


\begin{rpg-quotebox}{Logika}
   \begin{tabularx}{\columnwidth}{lX}
      \textbf{Randal:} & Wysyłam sztuczkę Wiadomość do kryształu z zapytaniem "Kim jesteś?".\\
      \textbf{MG:} & Bzzzz.\\
      \textbf{Randal:} & Aha. Wyciszony.\\
   \end{tabularx}
\end{rpg-quotebox}


\begin{rpg-quotebox}{Cięta pirosta}
   \begin{tabularx}{\columnwidth}{lX}
      \multicolumn{2}{l}{\textit{Zainspirowany czynem Randala:}}\\
      \textbf{Ampoldórëon:} & Używam Zrozumienia Języków by zrozumieć brzęczenie kryształu.\\
      \textbf{MG:} & Udało ci się przetłumaczyć "bzy" na "ybz".\\
   \end{tabularx}
\end{rpg-quotebox}


\begin{rpg-quotebox}{O stałości w kontaktach}
   \begin{tabularx}{\columnwidth}{lX}
      \textbf{Ampoldórëon:} & Czy można go zidentyfikować.\\
      \textbf{Kedra:} & Nie można go utrzymać przez dłuższą chwilę, bo razi prądem.\\
      \textbf{MG:} & Musisz się z nim ciągle stykać podczas identyfikowania.\\
      \multicolumn{2}{l}{\textit{Zamierzając się otulić wokół kryształu:}}\\
      \textbf{Randal:} & Ach, czyli stosunek przerywany odpada...\\
   \end{tabularx}
\end{rpg-quotebox}


\begin{rpg-quotebox}{O nieprzemyślanych decyzjach}
   \textit{Drużyna pytała MG o inne możliwości zneutralizowania bzyczącego kryształu.}\\

   \begin{tabularx}{\columnwidth}{lX}
      \textbf{MG:} & Można go było rozwalić bo miał tylko AC 18, ale Randal postanowił zastawić własną duszę do Kruczej Królowej by zrobiła to za was.\\
      \textbf{Lander:} & I w ten sposób mamy szmaragd do sprzedania!\\
      \multicolumn{2}{l}{\textit{O Landerze:}}\\
      \textbf{Kedra:} &  On jest klerykiem z domeny księgowości.\\
   \end{tabularx}
\end{rpg-quotebox}


\section*{12 października 2018}

1. Kedra: czy szmaragd jest dużo warty?
Ampoldórëon: funkel nówka nie śmigana

2. Drużyna nie wiedziała, czy ich sojuszniczy NPC żyje
MG: Alchemik Schrödingera

3. Drużyna odgruzowała dziurę w ścianie, w której Ampoldórëon wyczuwał magię. Mimo to nikt nie kwapił się by wejść pierwszemu do ciemnej jamy.
Lander: zapalam pochodnię Lightem i wrzucam do środka

4. Ampoldórëon: ja się nie pytam Randala tylko Landera i chuj wam w dupę za te imiona.

5. MG: jeśli chcesz próbować się przecisnąć musisz rzucić na Akrobatykę
Ampoldórëon: najpewniej utknę...
Kedra: jakich jesteś gabarytów pod tą zwiewną szatą?
Ampoldórëon: teoretycznie smukłym elfem, ale chuj wie co wypadnie

6. Kedra: chcę pomóc Ampoldórëonowi przeszukać pokój mimo że jakiś czas temu oślepił mnie ten Landerowy kij...
Randa: ale Braille'a przeczytasz

7. Drużyna znalazła miksturę pomniejszającą.
Lander z przekąsem po incydencie bolesnego utknięcia w szczelinie: przydałaby się 10 minut tem
[11:03] Vivien / Kedra: 8. Drużyna grubo imprezowała i powstało ryzyko Kaca Mordercy
MG do Randala: rzucaj na wytrzymałość
K20: 11
% \includegraphics[scale=0.055]{img/d20.png}\textbf{(Lander):}& 23.\\
MG: słyszysz, że kurz tańczy tańce irlandzkie

9. Randala na kacu bardzo irytowały dźwięki
Ampoldórëon: idę poszukać lekarstwa na to, ale bardzo przy tym hałasuję
Kedra: pomagam mu specjalnie doskakując do miejsca w którym mag stoi
Lander: rzucam na siebie Taumaturgy i wzmocnionym głosem inkantuję Lesser Restoration by wyleczyć Randala
MG: to było Louder Restoration

10. MG: gotowanie wynika z umiejętności Survival, więc Lander ugotował przepyszną jajecznicę.
Kedra: brawo! potrafisz przetrwać w kuchni!
Ampoldórëon: hmmm... ktoś tu znalazł kandydata na męża
Kedra: taa... przez żołądek do serca...

11. Skirfir (npc) czuł się bardzo źle po wczorajszej libacji i gdy mówił nie podnosił głowy tylko patrzył się w swój talerz jajecznicy: dziękuję Wam za wszystko co dla mnie zrobiliście
Randal: rzucam Minor Illusion na jego jajecznicę by uformować z niej twarz mówiącą "nie ma za co"

12. Drużyna planowała pominąć odwiedziny w stolicy podziemnych krasnoludów Artneyg i bezpośrednio czarem teleportacji przenieść się na docelowe Latające Wyspy. Niestety okazało się, że Ampoldórëon nie znał tego zaklęcia.
Kedra: ale zaraz, Skirfir ma zwój, a podobno magowie potrafią uczyć się czarów ze zwojów...
Ampoldórëon: to nie tak, że to za wysoki poziom zaklęć...
Kedra: ile ci to zajmie?


% \begin{rpg-quotebox}{}
%    \begin{tabularx}{\columnwidth}{lX}
%    \end{tabularx}
% \end{rpg-quotebox}

% \begin{rpg-quotebox}{}
%    \textit{}\\
%    
%    \begin{tabularx}{\columnwidth}{lX}
%       \multicolumn{2}{l}{\textit{}}\\
%       
%       \textbf{:} & \\
%    \end{tabularx}
% \end{rpg-quotebox}








% \begin{rpg-paperbox}{rpg-paperbox}
% 	you can add some text here
% \end{rpg-paperbox}
% 
% %\newpage % Acts as columbreak because of twocolumn option; for pagebreak use \clearpage
% 
% \section{Some tables}
% \header{default rpg-table (2 column)}
% \begin{rpg-table}
%    	\textbf{Table head 1}  & \textbf{Table head 2} \\
%    	Some value  & Some value \\
%    	Some value  & Some value \\
%    	Some value  & Some value
% \end{rpg-table}
% 
% % For more columns, you can say \begin{rpg-table}[your options here].
% % For instance, if you wanted three columns, you could say
% % \begin{rpg-table}[XXX]. The usual host of tabular parameters are
% % aailable as well.
% \header{rpg-table with more columns}
% \begin{rpg-table}[XXX]
%     \textbf{Table head 1}  & \textbf{Table head 2} & \textbf{Table head 3}\\
%    	Some value  & Some value & Some value\\
%    	Some value  & Some value & Some value\\
%    	Some value  & Some value & Some value
% \end{rpg-table}
% 
% \header{default rpg-table2 (2 column)}
% \begin{rpg-table2}
%    	\textbf{Table head 1}  & \textbf{Table head 2} \\
%    	Some value  & Some value \\
%    	Some value  & Some value \\
%    	Some value  & Some value
% \end{rpg-table2}
% 
% 
% \section{List}
% \begin{rpg-list}
%     \item first list item
%     \item second list item
% \end{rpg-list}
% 
% \section{Monster}
% % You can optionally not include the background by saying
% %\begin{rpg-monsterboxnobg}{monsterboxnob}
% \begin{rpg-monsterbox}{rpg-monsterbox}
% 	\textit{Small metasyntatic variable (golbinoid), neutral evil}\\
% 	\rpghline
% 	\basics[%
% 	armorclass = 12,
% 	hitpoints  = 16 (3d8 + 3),
% 	speed      = 50 t
% 	]
% 	\rpghline
% 	\stats[ % This stat command will autocomplete the modifier for you
%     STR = 12, 
%     DEX = 7
% 	]
% 	\rpghline
% 	\details[%
% 	% If you want to use commas in these sections, enclose the
% 	% description in braces.
% 	% I'm so sorry.
% 	languages = {Common Lisp, Erlang},
% 	]
% 	\rpghline \\[1mm]
% 	\begin{rpg-monsteraction}[rpg-monsteraction]
% 		This Monster has some serious superpowers!
% 	\end{rpg-monsteraction}
% 
% 	\rpgmonstersection{rpgmonstersection}
% 	\begin{rpg-monsteraction}[rpm-monsteraction]
% 		This one can generate tremendous amounts of text! Though only when it wants to.
% 	\end{rpg-monsteraction}
% 
% 	\begin{rpg-monsteraction}[rpg-monsteraction]
%     See, here he goes again! Yet more text.
% 	\end{rpg-monsteraction}
% \end{rpg-monsterbox}
% 
% \chapter{Chapter name}
% 
% % End document
\end{document}
