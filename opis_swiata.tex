\chapter{Ildrim - kontynent imperium}

\paragraph{}
Ciężko opisać kontynent jako całość, gdy nawet najbardziej odważni i potężni podróżnicy rzadko opuszczają jego wybrzeża.
Gdy nawet potężni mistrzowie magii z trudem sięgają umysłem poza jego granice. 
Wyobraźmy sobie masę lądu otoczoną ze wszystkich stron oceanem.
I to nie takim zwykłym, oceanem kierowanym starożytną magią z Wysokich Planów.
Oceanem chcącym pożreć, pochłonąć nieszczęsny ląd.
Taki jest nasz Ildrim, zwany też Kontynentem Imperium.
Jeśli nic się nie zmieni, bądź zmieni się zbyt wiele, zostaniemy wszyscy pogrążeni pod wodą, skała i magią.
Ponieważ Ildrimem władają moce poza naszą kontrolą.

\paragraph{}
Śmiercionośny ocean wpływa na naszą egzystencję w jeszcze jeden sposób: niemal kompletnie odcina nas od innych cywilizacji.
O ile pas wody w pobliżu wybrzeży w większości jest stosunkowo spokojny, głębokie wody są niemal zupełnie niedostępne.
Jedynie druidzi z \hyperref[Vicovaro]{Vicovaro}, elfowie z Vorath i „wolni żeglarze” z Wysp Kye są w stanie próbować swoich sił z żywiołem.

\paragraph{}
Kontynent zatem jest całym światem dla praktycznie wszystkich jego mieszkańców, zapewne z  wyjątkiem najprostszych rolników, których pojęcie obejmuje co najwyżej sąsiednie miasto.
I nie powinno dziwić, że przez milenia Ildrim zaznał niewiele spokoju.

\paragraph{}
Większość kontynentu zajmują \hyperref[SojuszniczeKrolestwa]{Sojusznicze Królestwa}: od górskich twierdz \hyperref[AnGammarna]{An’Gammarny} do spustoszonych iglic Yunait.
To te siedem państw jest odpowiedzialnych za ostatnie wojny i niepokoje na kontynencie.
Jednak po stuleciach walk \hyperref[SojuszniczeKrolestwa]{Sojusznicze Królestwa} utrzymywane są w pokoju przez \hyperref[GildiaKupcow]{Gildię Kupców}.
Z punktu widzenia \hyperref[GildiaKupcow]{Gildii} pokój zostanie zachowany.
Niezależnie do ceny.
Państwa te pozostają, przynajmniej we własnym mniemaniu, ostoją cywilizacji i dziedzicami Imperium.
Poza elfim państwem-miastem Vorath są jedyną cywilizacją na powierzchni Planu Materialnego nie pozostająca w ukryciu.

\paragraph{}
Prawda, jak to ma w zwyczaju, jest nieco bardziej skomplikowana.
Jeśli ktoś zdoła przebyć zdradzieckie północne góry \hyperref[AnGammarna]{An’Gammarny} i \hyperref[Vicovaro]{Vicovaro} dotrze do mroźnych pustkowi.
Tam na pokrytych wiecznym lodowcem wybrzeżach znajduje się owiane legendami państwo.
Niewielu wie o nim cokolwiek pewnego, chociaż większość dzieci w \hyperref[SojuszniczeKrolestwa]{Sojuszniczych Królestwach} słyszało przed snem baśnie o magicznych miastach pośród lodu.
Jeszcze dalej na zachód znajduje się sporych rozmiarów wyspa, ukryta pośród mgieł.
Ta wyspa to Vorath, ostatnie miasto elfów na Planie Materialnym, a w każdym razie ostatnie na tym kontynencie.

\paragraph{}
Od południowo-zachodniej strony Ildrimu znajduje się Platynowa Zatoka.
Stanowi ona wybrzeże \hyperref[AnGammarna]{An’Gammarny}, Croridu oraz Eneadoru.
Wody zatoki, są wyjątkowo zdradliwe, nawet jak na warunki kontynentu.
Powodem jest Latająca Wyspa Thor Ar Muria, gdzie znajduje się największa na kontynencie akademia magii arkanicznej.
Na obszarze Platynowej Zatoki znajdują się żeglowne przesmyki, dostępne dla doświadczonych śmiałków, jednak większość wypełniona jest wirami i wiatrami o nieprzewidywalnym, magicznym źródle.

\paragraph{}
Na wschodzie znajduje się niezbadana i niebezpieczna kraina zwana Pustkowiem Kataklizmu.
Nieliczne jego fragmenty są zamieszkane przez pionierów z Domen Krasnoludów.
Choć niewiele osób o tym wie, w północno-zachodniej części Pustkowia znajduje się Przystań: miasto uciekinierów z Podmroku.
Dawni mieszkańcy tego miasta przemierzają także te nieprzyjazne tereny jako nomadzi.
Niektórzy twierdzą, że wschodnie wybrzeża Ildrimu otaczają spokoje wody, którymi można przepłynąć do innych kontynentów. Nikomu jednak nie udało się do nich dotrzeć.

\paragraph{}
Nie są to jednak jedynie tereny na lądzie dotknięte niszczącą magią.
Na wschód od \hyperref[Vicovaro]{Vicovaro} i w \hyperref[AnGammarna]{An’Gammarnie} znajdują się ruiny dawnego Imperium.
Jakiekolwiek magiczne eksperymenty przeprowadzali ich mieszkańcy, ich moc przesiąknęła kamienie, ziemię i drzewa.
W samej \hyperref[AnGammarna]{An’Gammarnie} klerycy i paladyni zdołali opanować nienaturalny żywioł, ale Pomiędzy \hyperref[AnGammarna]{An’Gammarną}, a \hyperref[Vicovaro]{Vicovaro} pozostaje puszcza, kompletnie niestabilna i nienadająca się do zamieszkania zwana Doliną Konwalii.

\paragraph{}
Jest wiele teorii skąd wzięła się aż tak duża magiczna niestabilność występująca na kontynencie, a także czemu jest jedno miejsce praktycznie kompletnie od nich wolne: leżący na zachodzie archipelag Wysp Kye.
Wyspy stanowią przytań dla wielu, którzy przystani poszukują, ale niekoniecznie na nią zasługują.
Zamieszkujący je „wolni żeglarze” prowadzą nieskończone spory z Gildią Kupców chcących dostępu do spokojnych wód i dóbr archipelagu.
Z różnymi skutkami.
Wśród plotek i legend związanych z wyspami znajdują się labirynty podziemi wypełnionych skarbami i starożytnymi tajemnicami.
Niestety niewielu badaczy miało okazję poznać te wypełnione piratami i najemnikami wyspy.

\paragraph{}
Zanim powstały aktualne państwa większość północnej i zachodniej części kontynentu zajmowało Imperium, powszechnie uznawana za państwo elfów zanim opuściły one Plan Materialny.
Imperium najprawdopodobniej zostało zniszczone przez pierwszy na kontynencie magiczny kataklizm.
Możliwe nawet, że go wywołało, doprowadzając Ildrim do tego czym jest teraz.
Praktycznie nie ma organizacji, która obejmowałaby swoim zasięgiem cały kontynent.
Pewne powiązania ma tak zwana \hyperref[KonfrateriaMagow]{Konfrateria Magów}.
Nie jest to formalna organizacja, a raczej ogólne określenie na wszystkich formalnie wyszkolonych magów arkanicznych.
W ramach \hyperref[KonfrateriaMagow]{Konfraterii} funkcjonują Strażnicy Arkanum, stosunkowo niewielka grupa rozproszona po całym Ildrimie.
Tutaj można już mówić o zorganizowaniu, ale dotyczy to stosunkowo niewielkiej, dobrze ukrytej grupy magów.
Większość członków \hyperref[KonfrateriaMagow]{Konfraterii} nie ma pojęcia o ich istnieniu.
Na podobnych zasadach można traktować Kręgi Druidów -  o ile poszczególne Kręgi są rozproszone po całym kontynencie, to nie stanowią jednolitej organizacji.

\label{SojuszniczeKrolestwa}
\chapter{Sojusznicze Królestwa}

\paragraph{}
Można by napisać książkę, a nawet tuzin książek o różnych teoriach co do powstania państw, które stworzyły Sojusznicze Królestwa.
Jedyne co można z pewnością napisać to, że Imperium upadło i na kontynencie powstał chaos.
Na północy elfy zaczęły walczyć z ludźmi i ich sojusznikami, co zakończyło się zwycięstwem ludzi i opuszczeniem Planu Materialnego przez elfy.
Państwa na południu nagle straciły groźnego przeciwnika.

\paragraph{}
Ci mieszkańcy kontynentu którzy nie odsunęli się w trudnodostępne regiony i plany utworzyli siedem państw.
Od północy na południe są to: \hyperref[Vicovaro]{Vicovaro}, \hyperref[AnGammarna]{An’Gammarna}, Crorid, Sothedora, Eneador, Chacan i Yunait.
Eneador, Chacan i Yunait nigdy nie były częścią Imperium, chociaż Eneador był z nim w sojuszu od początku swojego istnienia, by uniknąć podboju (formalnie był ówcześnie częścią Domen Krasnoludów).
Pozostałe państwa powstały na ruinach po początkowym chaosie, w szczególności \hyperref[AnGammarna]{An’Gammarna} i Crorid.

\paragraph{}
O ile obecnie królestwa są mieszaniną różnych ras (z wyjątkiem Yunait, które pozostaje niemal całkowicie zamieszkane przez drakonów) pierwotnie były dużo bardziej podzielone.
\hyperref[Vicovaro]{Vicovaro} było państwem niziołków, \hyperref[AnGammarna]{An'Gammarna} i Sothedora były państwami ludzi z Imperium, Chacan zaś niezalezną cywilizacją ludzi spoza Imperium.
Crorid i Eneador były koloniami krasnoludów i gnomów, przy czym Crorid został wcześniej całkowicie podbity przez Imperium.

\paragraph{}
Mniejsze lub większe potyczki między królestwami rozpoczęły się niemal od razu po upadku Imperium.
Trwały stulecia, ograniczają rozwój i handel.
Żadne z państw nie mogło uzyskać dłuższej przewagi.
Wojny zapewne trwałyby nadal gdyby nie drugi kataklizm.
Nie ma pewności czy był echem zniszczenia Imperium, czy czymś zupełnie nowym.
Pewne jest to, że magiczna eksplozja wstrząsnęła pustkowiami przy wschodniej granicy Yunait, niemal całkowicie je niszcząc. 
Magia rozproszyła się po Ildrimie jak kręgi na wodzie.
W końcu dotarła do imperialnych ruin w Dolinie Konwalii i \hyperref[AnGammarna]{An’Gammarnie}.
Magia zawsze była w nich intensywna, ale to sprawiło, że rozpętała się magiczna burza.
I kompletny chaos.
Kiedy pierwsze skutki się uspokoiły nastąpiły te bardziej długotrwałe.
Cały kontynent objęła katastrofalna susza.
Powoli zdano sobie sprawę, że królestwa nie przetrwają pojedynczo i współpraca jest jedyną szansą na przeżycie.
Tak powstał Sojusz.

\paragraph{}
Wraz z Sojuszem powstała \hyperref[GildiaKupcow]{Gildia Kupców}.
Ta prosta nazwa nie oddaje ogromu organizacji która przez lata oplotła Królestwa siecią kontaktów, szpiegów i intryg.
I która praktycznie rządzi we wszystkich siedmiu państwach.
Załagodzenie kataklizmu i jego efektów zajęło lata, ale przyniosło dodatkowe profity: lepszą komunikację, szlaki handlowe, a co za tym idzie pieniądze.
Nie chcąc tracić zysków \hyperref[GildiaKupcow]{Gildia} skupiła się na swoim nowym najważniejszym celu - utrzymać pokój w Sojuszniczych Królestwach za wszelką cenę.

\paragraph{}
Przez stulecia pokój został utrzymany.
Jednak silne oddziaływanie \hyperref[GildiaKupcow]{Gildii} sprawia, że sytuacja polityczna staje się coraz bardziej niestabilna, a wysiłki \hyperref[GildiaKupcow]{Gildii} by utrzymać status quo coraz bardziej brutalne.
Obecnie \hyperref[Vicovaro]{Vicovaro} jest niemal całkowicie kontrolowane przez \hyperref[GildiaKupcow]{Gildię} (pomijajac atak przez wrogo nastawioną grupę druidów).
Udało im się także się w dużym stopniu zdestabilizować władzę w Croridzie.
Pozostałe państwa zachowuja względnie niezależną władzę, ale to także zaczyna sie powoli zmieniać.

\chapter{An'Gamarrna}

\paragraph{}
Państwo ludzi stworzone na ruinach dawnego Imperium, obecnie rządzone przez teokratyczny kult Bahamuta, do którego musi należeć nawet król.
Pańśtwo jest częścią Sojuszniczych Królestw.

\section{Geografia}
\paragraph{}
An'Gamarrna obejmuje zatokę pomiędzy Doliną Konwalii a Croridem i Eneaorem.
Platynowa zatoka nie umożliwia ze względu na niestabilne prądy przez istnienie latającej wyspy Muria, co powoduje bardzo napięte stosunki między An'Gamarrną a Croridem.
Obecną stolicą państwa jest święte miasto Glenrowan.
Innymi dużymi ośrodkami są dawna stolica Imperium, twierdza Bleihar (znajdująca się w górach na północy) oraz port Thelassa (około dzień drogi od Glenrowan nad brzegiem Platynowej Zatoki).
Klimat An'Gamarrny jest umiarkowany, z okazjolanymi przebłyskami niestabilnej pogody w miejscach gdzie kapłani nie kontrolują dobrze granic między planami.

\section{Polityka}
\paragraph{}
An'Gammarna rządzona jest przez króla, który aby odziedziczyć swoje królestwo musi zostać paladynem Bahamuta.
Obecnym królem jest Lorne III, który wziął za małżonkę Zephę, arcykapłankę Bahamuta.
Zepha jest młodszą od nieo o kilkanaście lat półelfką, dała mu dwóch synów.
Wiele osób uważa, że to ona naprawdę rządzi królestwem.
Wszelkie kulty czczące dobrych i neutralnych bogów cieszą się dużym poważaniem, kulty chaotyczne mogą być natomiast traktowane z pełną rezerwą.

\section{Historia}
\paragraph{}
An'Gammarna wg legend została założona przez zbuntowanych niewolników podczas schyłku Imperium.
Niewolnicy pierwotnie przejęli twierdzę Bleihar, a później objawił im się platynowy smok i polecił stworzyć święte miasto Glenrowan.
Ponieważ teren współczesnej An'Gammarny, poza twierdzą Bleihar, znajdował się ciągle w rękach elfów - wybuchła wojna.
Zakończyła się ona zwycięstwem ludzi i zniszczeniem elfiej stolicy Eleander.

\label{Vicovaro}
\chapter{Vicovaro}

\paragraph{}
Państwo niziołków stworzone w miejscu kręgu druidzkiego specjalizującego się z kontroli pogody oraz magii wody. 
Przez \hyperref[GildiaKupcow]{Vicovaro} przepływa większość handlu zewnętrznego w \hyperref[SojuszniczeKrolestwa]{Królestwach} i jest półoficjalną siedzibą \hyperref[GildiaKupcow]{Gildii Kupców}.
Państwo jest częścią \hyperref[SojuszniczeKrolestwa]{Sojuszniczych Królestw}.

\section{Geografia}
\paragraph{}
Vicovaro jest najbardziej na zachód wysuniętym państwem \hyperref[SojuszniczeKrolestwa]{Sojuszniczych Królestw}.
Od najbliższego wschodniego sąsiada, \hyperref[AnGammarna]{An'Gammarny}, oddzielona jest Doliną Konwalii. 
Poza miastem portowym Vicovaro w państwie nie było żadnych innych miast (co formalnie czyniło z Vicovaro miasto-państwo), do czasu kiedy 200 lat temu \hyperref[KultIoun]{kult Ioun} wybudował Sjorden. 
W Vicovaro znajduje się największy port w \hyperref[SojuszniczeKrolestwa]{Królestwach}, w którym druidzi utrzymują permanentnie korzystną dla żeglarzy pogodę.

\section{Polityka}
\paragraph{}
Miastem rządzi diuk z klanu niziołków Olkyn, potomek Adana Olkyna, oryginalnego założyciela miasta Vicovaro. 
Aktualny diuk Dregras Olkyn jest jedynie marionetką, faktyczna władza znajduje się w rękach Arcydruidki Eryn oraz Thundy, przywódcy \hyperref[GildiaKupcow]{Gildii Kupców}.
Oboje nie przepadają za sobą nawzajem, ale prowadzą niechętną współpracę. 
Krąg druidzki pozostaje niezmiennie znacząca siłą polityczną w Vicovaro, ze względu na zależność miasta od ich magii. 
Te powiązania sprawiają, że krąg Vicovaro jest traktowany z pewną wrogością przez inne kręgi.

\section{Historia}
\paragraph{}
Zanim powstało Vicovaro na wybrzeżu znajdował się krąg druidzki. 
Wiele stuleci temu przywódca kręgu: druid Adan ściągnął do nadmorskiego siedliska kręgu cały swój klan. 
Klan stworzył port handlowy Vicovaro. 
Stopniowo coraz więcej niziołków  z okolicznych dolin zaczęło ściągać do portu, miasto zaczęło się rozrastać i zyskiwać na znaczeniu.
Port stał się łakomym kąskiem dla okolicznych państw ze względu na bogactwo i strategiczną lokalizację: konflikty wybuchały głównie z \hyperref[AnGammarna]{An'Gammarną} i Wolnymi Wyspami Kye. 
Vicovaro zawsze jednak dało radę odeprzeć napastników dzięki swojej silnej flocie i magii druidzkej. 
Konflikt z \hyperref[AnGammarna]{An'Gammarna} udało się w znacznym stopniu załagodzić po tym jak utworzono \hyperref[SojuszniczeKrolestwa]{Sojusznicze Królestwa} i udostępniono ziemię na budowę Sjorden \hyperref[KultIoun]{kultowi Ioun}.
Ataki ze strony Wysp Kye ciągle się zdarzają, ale słabo zorganizowane wyspy nie maja szans ze współczesny Vicovaro.

\chapter{Organizacje}

\label{GildiaKupcow}
\section{Gildia Kupców}
\paragraph{}
Ściśle powiązana gigantyczna siatka kupców bankierów.
Gildia Kupców jest, rzecz jasna, najpotężniejszą organizacją obejmującą swoim zasięgiem \hyperref[SojuszniczeKrolestwa]{Sojusznicze Królestwa}.
Każdy kto chce zajmować się handlem na dłuższą metę musi zostac członkiem Gildii, albo spotkaja go “nieprzyjemnie konsekwencje” wszelakiego rodzaju.
Gildia (nieco mniej oficjalnie) zajmuje się także polityką i dyplomacją.
W praktyce to oni utrzymują pokój w królestwach dzięki gigantycznym wpływom na bardzo wysokich szczeblach oraz, oczywiście, gigantycznym pieniądzom.
Tradycyjną siedzibą Gildii jest \hyperref[Vicovaro]{Vicovaro}, największy port hyperref[SojuszniczeKrolestwa]{Sojuszniczych Królestw} i ich jedyne połaczenie z wyspami Kye (i dalej).

\label{KonfraterieMagow}
\section[Konfraterie Magów]{Konfraterie\\Magów}
\paragraph{}
Bardzo luźne zgrupowania magów w różnych miastach królestw.
Większość magów szkoliło się w jednym z ośrodków Konfraterii, ponieważ daje to dostęp do bibliotek, miejsc do bezpiecznych eksperymentów oraz potencjalnych nauczycieli.
Większość magów z dumą deklaruje przynależność, do którejś z konfraterii, jako dowód, że mają porządne wykształcenie.
Najbardziej prestiżową konfraterią jest Latająca Wyspa Muria, lewitująca nad wybrzeżem Croridu.

\label{Varjossa}
\section{Varjossa}
\paragraph{}
Niewielu poza wysoko postawionymi członkami \hyperref[GildiaKupcow]{Gildii} wie o innej organizacji mającej koneksje w całych \hyperref[SojuczniczeKrolestwa]{Królestwach}, a nawet poza nimi: Varjossa, zwana nieco błędnie Gildią Zabójców.
Gdziekolwiek podróżnicy stykają się z gildiami zabójców, złodziei itp. prawie na pewno stoi za tym Varjossa.
Nawet jeśli członkowie najniższej rangi sami o tym nie wiedzą.
Wyżej postawionych członków mozna rozpoznać po tatuażu w kształcie oka, najczęściej z tyłu szyi.
Varjossa ściśle współpracuje z \hyperref[GildiaKupcow]{Gildią Kupców}, ale obie organizacje podkreślaja swoją niezależność.

\label{KultIoun}
\section{Sekretny Kult Ioun}
\paragraph{}
O ile wiele mniejszych i większych organizacji religijnych działa w \hyperref[SojuszniczeKrolestwa]{Królestwach} na uwagę zasługuje Sekretny Kult Ioun.
Stanowi on ,,dyskretne'' ramie instytucji religijnych w \hyperref[SojuszniczeKrolestwa]{Królestwach}.
Zajmują się zbieraniem i zabezpieczaniem wiedzy, szczególnie tej niebezpiecznej.
Poza tym działają, kiedy potrzeba bardziej dyskretnego podejścia niż wysłanie typowego paladyna.
Posiadają nieco szemraną reputację i niezbyt "przyjacielskich" członków, ale koniec końców, posiadają szacunek większości władz religijnych w \hyperref[SojuszniczeKrolestwa]{Królestwach}.
